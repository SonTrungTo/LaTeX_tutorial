\documentclass[a4paper,11pt]{article}
\usepackage[finnish]{babel}
\usepackage[utf8]{inputenc}
\usepackage[affil-it]{authblk}
\usepackage{microtype,fancyhdr}
\usepackage[pdftex,hyperindex=false,%
colorlinks,unicode,hidelinks]{hyperref}
\pagestyle{fancy}
\newcommand{\note}[1]{\textbf{#1}}

\begin{document}

  \title{Collection of Finnish articles on Internet}
  \author{Son To\\
  $<$\href{mailto:son.trung.to@gmail.com}%
  {son.trung.to@gmail.com}$>$}
  \affil{Ravintola Kiltakellari}
  \date{$4.$ Elokuutata, $2017$}

  \maketitle
  \tableofcontents

  \section{\href{http://www.sodexo.fi/yritys/uutiset/2017-06-20-nuoret-paasevat-kesasetelin-turvin-kokeilemaan-tyontekoa}%
  {Nuoret pääsevät Kesäsetelin turvin kokeilemaan työntekoa}}
  $10.7.2017$
  \par
  Kesäseteli on nuorten, yrittäjien ja Helsingin kaupungin
  yhteinen hanke, jolla tuetaan nuorten kesätyöllistymistä.
  Kesäsetelin saivat tänä vuonna kaikki Helsingin koulujen
  $9.$-luokkalaiset. Kesäsetelin arvo oli $300$ euroa, ja
  Helsingin kaupunki korvasi summan palkkakuluista
  työnantajalle.
  \par
  Sodexon Vaisalan Vantaalla toimipisteessä Kesäsetelillä
  työskenteli tänä kesänä kahden viikon ajan 16-vuotias
  \note{Saara Laustela}.
  \begin{itemize}
    \item[-] Työnhaku setelin kanssa sujui tosi helposti. Hain
    kolmeen paikkaan ja pääsin kaikkiin. Työnantajat suhtautuivat
    kesäseteilillä töitä hakevaan tosi positiivisesti, Laustela
    kertoo.
  \end{itemize}
  Ysiluokalta päässyt Laustela valitsi Sodexon Vaisalan, koska
  hän suoritti samassa ravintolassa $7.$-luokan TET-harjoittelun
  ja viihtyi hyvin.
  \begin{itemize}
    \item[-] Olen harkinnut kondiittorialaa, mutta ensin käyn
    lukion.
    \item[-] Kesäseteli on hieno juttu. Meistä on kiva antaa
    nuorille mahdollisuus tutustua työelämään ja kokeilla
    siipiään. Saaran tunsimme entuudestaan ja tiesimme, että
    hän ei pelkää asiakaspalvelua ja on motivoitunut, kertoo
    Vaisalan asiakaspalveluvastaava \note{Maiju Nikkonen}.
  \end{itemize}
  Saara on Vaisalan ensimmäinen kesäsetelillä työskentelevä,
  mutta ammattikoulutuksen harjoittelijoita ravintolassa
  työskentelee säännöllisesti.
  \begin{itemize}
    \item[-] Harjoittelu on ollut monille väylä
    vakituisempaankin työhön. Hyvät harjoittelijamme ovat
    ilmoittautuneet vuokratyöfirman palkkalistoille ja
    käytämme heitä tuuraajina, Nikkonen kertoo.
  \end{itemize}
  \subsection{Monipuolisia työtehtäviä}
  Saara perehdytettiin hyvin, varsinkin turvallisuusasioihin.
  \begin{itemize}
    \item[-] Nuorten kohdalla on erityisen tärkeää, että
    teemme asiat turvallisesti, ettei satu mitään vahinkoja.
    Heillä on niin vähän kokemusta työelämästä ja varsinkaan
    ravintola-alasta, että oikeat toimintatavat pitää opettaa
    hyvin.
  \end{itemize}
  Laustelan työaika oli viisi tuntia päivässä, yhdeksästä
  kahteen. Hän auttoi laittamaan lounaita esiin, huolehti
  lounaan ajan leipäpöydästä, täydensi salaattipöytää ja
  toi saliin puhtaita astioita.
  \begin{itemize}
    \item[-] Työt Saaran kanssa sujuivat tosi hyvin. Hän
    teko reippaasti kaiken ja hänestä oli valtavasti apua,
    Nikkonen kertoo.
  \end{itemize}
  Nikkosen mielestä Kesäsetelistä tiedottaminen on ollut
  kovin vähäistä ja siksi yrityksetkin tuntevat sen
  huonosti.
  \begin{itemize}
    \item[-] Itse kuulin Kesäsetelistä vasta Saaralta. Tämähän
    on kaupungilta loistava kädenojennus nuorille ja olisi
    tärkeää, että vastaavia toimintamalleja olisi kaikissa
    kaupungeissa. Kesäseteli on loistava idea ja auttaa
    varmasti nuoria työllistymään jatkossa, Nikkonen toteaa.
  \end{itemize}
  \subsection{Hyvä tilaisuus}
  Kesäseteliä käytettäessä työnantaja tekee kesätyöntekijän
  kanssa työsopimuksen ja hoitaa normaalit työnantajan
  velvollisuudet. Sodexolla oli tänä kesänä muutama
  Kesäsetelillä työskentelevä nuori.
  \par
  Sodexon HR-osasto uskoo, että Kesäsetelillä työskentelevät
  nuoret oppivat paljon hyödyllisiä asioita työelämästä: miten
  työtä haetaan, millaisia ovat työelämän pelisäännöt sekä
  miten töissä kuuluu olla ja käyttäytyä.
  \begin{itemize}
    \item[-] Nuoret ovat tulevaisuuden rakentajia ja siksi
    haluamme heidät mukaan. Yläkouluikäiselle nuorelle
    kesätyöpaikan löytäminen ja saaminen voivat olla
    vaikeaa ja Kesäseteli auttaa tässä. Kesäsetelillä
    työskentely on nuoren työelämänpolun ensimmäinen askel,
    joka auttaa seuraavien töiden saamista. Kesäseteli avulla
    nuoret voivat selvittää omia vahvuuksiaan ja oppia
    ymmärtämään työelämän vaatimuksia, toteavat Sodexon HR-%
    osaston \note{Arja Ahola} ja \note{Minna Brace}.
  \end{itemize}
\end{document}
