\documentclass[a4paper,11pt]{article}
\usepackage[finnish]{babel}
\usepackage[utf8]{inputenc}
\usepackage[affil-it]{authblk}
\usepackage{microtype,fancyhdr}
\usepackage[pdftex,hyperindex=false,%
colorlinks,unicode,hidelinks]{hyperref}
\pagestyle{fancy}
\newcommand{\note}[1]{\textbf{#1}}

\begin{document}

  \title{Collection of Finnish articles on Internet}
  \author{Son To\\
  $<$\href{mailto:son.trung.to@gmail.com}%
  {son.trung.to@gmail.com}$>$}
  \affil{Ravintola Kiltakellari}
  \date{$4.$ Elokuutata, $2017$}

  \maketitle
  \tableofcontents

  \section{\href{http://www.sodexo.fi/yritys/uutiset/2017-06-20-nuoret-paasevat-kesasetelin-turvin-kokeilemaan-tyontekoa}%
  {Nuoret pääsevät Kesäsetelin turvin kokeilemaan työntekoa}}
  $10.7.2017$
  \par
  Kesäseteli on nuorten, yrittäjien ja Helsingin kaupungin
  yhteinen hanke, jolla tuetaan nuorten kesätyöllistymistä.
  Kesäsetelin saivat tänä vuonna kaikki Helsingin koulujen
  $9.$-luokkalaiset. Kesäsetelin arvo oli $300$ euroa, ja
  Helsingin kaupunki korvasi summan palkkakuluista
  työnantajalle.
  \par
  Sodexon Vaisalan Vantaalla toimipisteessä Kesäsetelillä
  työskenteli tänä kesänä kahden viikon ajan 16-vuotias
  \note{Saara Laustela}.
  \begin{itemize}
    \item[-] Työnhaku setelin kanssa sujui tosi helposti. Hain
    kolmeen paikkaan ja pääsin kaikkiin. Työnantajat suhtautuivat
    kesäseteilillä töitä hakevaan tosi positiivisesti, Laustela
    kertoo.
  \end{itemize}
  Ysiluokalta päässyt Laustela valitsi Sodexon Vaisalan, koska
  hän suoritti samassa ravintolassa $7.$-luokan TET-harjoittelun
  ja viihtyi hyvin.
  \begin{itemize}
    \item[-] Olen harkinnut kondiittorialaa, mutta ensin käyn
    lukion.
    \item[-] Kesäseteli on hieno juttu. Meistä on kiva antaa
    nuorille mahdollisuus tutustua työelämään ja kokeilla
    siipiään. Saaran tunsimme entuudestaan ja tiesimme, että
    hän ei pelkää asiakaspalvelua ja on motivoitunut, kertoo
    Vaisalan asiakaspalveluvastaava \note{Maiju Nikkonen}.
  \end{itemize}
  Saara on Vaisalan ensimmäinen kesäsetelillä työskentelevä,
  mutta ammattikoulutuksen harjoittelijoita ravintolassa
  työskentelee säännöllisesti.
  \begin{itemize}
    \item[-] Harjoittelu on ollut monille väylä
    vakituisempaankin työhön. Hyvät harjoittelijamme ovat
    ilmoittautuneet vuokratyöfirman palkkalistoille ja
    käytämme heitä tuuraajina, Nikkonen kertoo.
  \end{itemize}
  \subsection{Monipuolisia työtehtäviä}
  Saara perehdytettiin hyvin, varsinkin turvallisuusasioihin.
  \begin{itemize}
    \item[-] Nuorten kohdalla on erityisen tärkeää, että
    teemme asiat turvallisesti, ettei satu mitään vahinkoja.
    Heillä on niin vähän kokemusta työelämästä ja varsinkaan
    ravintola-alasta, että oikeat toimintatavat pitää opettaa
    hyvin.
  \end{itemize}
  Laustelan työaika oli viisi tuntia päivässä, yhdeksästä
  kahteen. Hän auttoi laittamaan lounaita esiin, huolehti
  lounaan ajan leipäpöydästä, täydensi salaattipöytää ja
  toi saliin puhtaita astioita.
  \begin{itemize}
    \item[-] Työt Saaran kanssa sujuivat tosi hyvin. Hän
    teko reippaasti kaiken ja hänestä oli valtavasti apua,
    Nikkonen kertoo.
  \end{itemize}
  Nikkosen mielestä Kesäsetelistä tiedottaminen on ollut
  kovin vähäistä ja siksi yrityksetkin tuntevat sen
  huonosti.
  \begin{itemize}
    \item[-] Itse kuulin Kesäsetelistä vasta Saaralta. Tämähän
    on kaupungilta loistava kädenojennus nuorille ja olisi
    tärkeää, että vastaavia toimintamalleja olisi kaikissa
    kaupungeissa. Kesäseteli on loistava idea ja auttaa
    varmasti nuoria työllistymään jatkossa, Nikkonen toteaa.
  \end{itemize}
  \subsection{Hyvä tilaisuus}
  Kesäseteliä käytettäessä työnantaja tekee kesätyöntekijän
  kanssa työsopimuksen ja hoitaa normaalit työnantajan
  velvollisuudet. Sodexolla oli tänä kesänä muutama
  Kesäsetelillä työskentelevä nuori.
  \par
  Sodexon HR-osasto uskoo, että Kesäsetelillä työskentelevät
  nuoret oppivat paljon hyödyllisiä asioita työelämästä: miten
  työtä haetaan, millaisia ovat työelämän pelisäännöt sekä
  miten töissä kuuluu olla ja käyttäytyä.
  \begin{itemize}
    \item[-] Nuoret ovat tulevaisuuden rakentajia ja siksi
    haluamme heidät mukaan. Yläkouluikäiselle nuorelle
    kesätyöpaikan löytäminen ja saaminen voivat olla
    vaikeaa ja Kesäseteli auttaa tässä. Kesäsetelillä
    työskentely on nuoren työelämänpolun ensimmäinen askel,
    joka auttaa seuraavien töiden saamista. Kesäseteli avulla
    nuoret voivat selvittää omia vahvuuksiaan ja oppia
    ymmärtämään työelämän vaatimuksia, toteavat Sodexon HR-%
    osaston \note{Arja Ahola} ja \note{Minna Brace}.
  \end{itemize}

  \section{Donald Trump aloittaa kesälomansa
  omalla golfklubillaan - Näin Yhdysvaltain
  presidentit ovat rentoutuneet}
  Presidentti Donald Trump lähtee tänään 17 päivää
  kestävälle lomalle.
  \par
  Yhdysvaltojen presidentti Donald Trump aloittaa tänään
  kesälomansa. Trump aikoo viettää 17 päivää yksityisellä
  golfkerhollaan New Jerseyn Bedminsterissä, kertoo
  uutistoimisto Reuters.
  \par
  Amerikkalaisittain pitkää lomaa selittävät lainsäädäntötyön
  pysäyttävä kongressin istuntotauko sekä Valkoisen talon
  ilmastointiremontti, kertoo sanomalehti The Guardian.
  \par
  Presidentin hallinto ei ole kertonut tarkemmin Trumpin
  lomasuunnitelmista. Oletettavasti hän ainakin harrastaa
  golfia ja tekee töitä. Presidenttien mukana on yleensä
  matkustanut noin $200$ hengen seurue avustajia ja
  turvämiehiä.
  \par
  Myös monille Trumpin edeltäjille golf on ollut rakas
  harrastus. Myös kalastus ja ulkoilu ovat olleet suosittuja.
  \subsection{Obama pyöräili ja golffasi}
  Yhdisvaltojen edellinen presidentti Barack Obama vietti
  seitsemän kahdeksasta kesälomastaaan Martha's Vineyardin
  saarella Massachusettsin edustalla. Sanomalehti Boston
  Globen mukaan Obama palaavat saarelle myös tänä kesänä.
  \par
  Barack Obaman lomaharrastuksiin kuuluu etenkin golf.
  Massachusettsin ohella Obama viettivät lomiaan myös
  Havaijilla presidentin lapsuuden maisemissa.
  \par
  Donald Trump haukkui toistuvasti Obamaa tiheästä lomailusta
  ja väitti, ettei itse ehtisi presidenttinä pelata golfia.
  Presidenttinä Trump on kuitenkin viettänyt jo $13$
  viikonluppua poissa Washingtonista, enimmäkseen omilla
  golfkerhoillaan.
  \subsection{George W.Bush lomaili paljon}
  Presidentti George W. Bush kävi presidenttikausiensa aikana
  peräti $77$ kertaa karjatilallaan Crawfordissa Texasissa.
  Bushia arvosteltiin ahkerasta lomailusta: hän piti peräti
  $879$ lomapäivää.
  \par
  Bushille Crawfordin tila oli kuitenkin myös tärkeä
  neuvottelupaikka, jossa valtionpäämiesten oli mahdollista
  keskustella rennommassa ympäristössä. Bush kutsui osan
  vieraistaan pyöräretkille tai kalastamaan laajoilla
  tiluksillan, kertoo Architecture Digest-lehti.
  \par
  George W. Bushin isä, presidentti H.W.Bush oli puolestaan
  intohimoinen kalastaja.
  \subsection{Bill Clinton inhosi retkeilyä}
  Obaman tapaan myös Bill Clintonilla oli tapana lomailla
  perheineen useimmiten Martha's Vineyardin saarella, jossa
  presidentti Clinton harrasti innokkaasti golfia.
  \par
  Martha's Vineyardilla on maine elitistisenä
  lomanvietttopaikkana. Vuonna $2014$ saari oli Trip Advisor%
  -matkailusivuston listaamista suosituista kesälomakohteista
  kallein.
  \par
  Vuosina $1995$ ja $1996$ Clintonit lomailivatkin Grand
  Tetonin kansallispuistossa Wyomingissa, vaikka presidentti
  inhosi telttailua ja kalastusta. Hän halusi kuitenkin
  välittää itsestään maanläheisen kuvan.
  \subsection{Ronald Reagan ratsasti Kalifornian
  kukkuloilla}
  Ronald Reagan lomaili usein maatilallaan Kalifornian Santa
  Ynezissa. Reaganit antoivat vuoristossa sijaitsevalle tilalle
  nimen Rancho del Cielo, Taivaan maatila.
  \par
  Reaganit harrastivat maatilallaan muun muassa ratsastusta.
  \subsection{Jimmy Carter ei juuri lomaillut}
  Presidentti Jimmy Carter ehti lomailla nykypresidenteistä
  vähiten, vain $79$ päivää nelivuotisen kautensa aikana.
  Hän kalasti mielellään presidentin virallisella Camp
  Davidin kesäasunnolla Marylandissä.
  \par
  Camp Davidissä tehtiin myös politiikkaa. Carter johti
  siellä neuvotteluja, joiden tuloksena Egypti ja Israel
  solmivat rauhansopimuksen vuonna $1978$.
  \subsection{Richard Nixon eristäytyi kartanoihinnsa}
  Presidentti Richard Nixon vietti lomiaan Floridassa
  ja Kaliforniassa sijaitsevilla kartanoillaan. Kun
  Watergate-skandaali syveni Nixonin presidenttikauden
  lopussa, hän pysytteli toisinaan eristyksissä loma-%
  asunnoissaan.
  \subsection{Lyndon B. Johnson viihtyi
  kotitilallaan}
  Presidentti Lyndon B. Johnson vietti aikaa Texasissa
  sijaitsevalla maatilallaan, jossa hän oli syntynyt
  ja jonne hänet lopulta haudattiin.
  \subsection{John F. Kennedy lomaili vaimonsa kotona}
  Presidentti John F. Kennedy esitteli perhettään julkisuudessa
  edeltäjiään enemmän. Kennedyt lomailivat Jacqueline
  Kennedyn lapsuudenkodissa Rhode Islandissa.
  Hammersmitth Farmin kartanossa on 28 huonetta.
\end{document}
