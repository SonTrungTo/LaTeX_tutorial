\section{Chapter 1}
\subsection{Deductive reasoning and logical connectives}
\begin{exx}
  \begin{enumerate}[label=(\alph*)]
    \item $(R\lor H) \land \lnot(H \land T)$
    \item $S = \text{``You go skiing''}$, $N = \text{``There is snow''}$.\\
    $\lnot S \lor (S \land \lnot N)$
    \item $\lnot((\sqrt{7}=2)\lor (\sqrt{7}<2))$
  \end{enumerate}
\end{exx}

\begin{exx}
  \begin{enumerate}[label=(\alph*)]
    \item $J=\text{John is telling the truth}$, $B=\text{Bill is telling the truth}$ \\
    $(J\land B)\lor (\lnot J \land \lnot B)$
    \item $F=\text{``I have fish''}$, $C=\text{``I have chicken''}$, $M=\text{``I have mashed potatoes''}$ \\
    $(F \lor C) \land \lnot(F \land M)$
    \item $(6 \vdots 3) \land (9\vdots 3) \land (15 \vdots 3)$
  \end{enumerate}
\end{exx}

\begin{exx}
  $A=\text{Alice is in the room}$, $B=\text{Bob is in the room}$
  \begin{enumerate}[label=(\alph*)]
    \item $\lnot(A \land B)$
    \item $\lnot A \land \lnot B$
    \item $\lnot A \lor \lnot B$
    \item $\lnot A \land \lnot B$
  \end{enumerate}
\end{exx}

\begin{exx}
  $a)$ and $c)$
\end{exx}

\begin{exx}
  \begin{enumerate}[label=(\alph*)]
    \item I will not buy the pants without the shirt.
    \item I will buy neither the pants nor the shirt.
    \item Either I will not buy the pants or I will not buy the shirt.
  \end{enumerate}
\end{exx}

\begin{exx}
  \begin{enumerate}[label=(\alph*)]
    \item At least one of them is happy and at least one of them is not happy.
    \item Either at least one of Steve and George is happy or both are unhappy.
    \item Either Steve is happy or George, not Steve, is happy.
  \end{enumerate}
\end{exx}

\begin{exx}
  \begin{cd}
    An argument is \emph{valid} if the premises cannot all be true without the conclusion
    being true as well.
  \end{cd}
  \begin{enumerate}[label=(\alph*)]
    \item Valid.
    \item If Beef and Peas were served, clearly the first two premises were satisfied.
    The third premise was also satisfied because both Fish and Corn were not served. Hence,
    the argument is invalid.
    \item We base our argument on Bill,
    \begin{enumerate}
      \item[-] If Bill is lying, then John must be telling the truth.
      \item[-] If Bill is telling the truth, then Sam must be lying.
    \end{enumerate}
    The conclusion is valid. \par
    \emph{Another approach:} Suppose John is lying and Sam is telling the truth, then Bill is
    telling the truth and Sam must be lying: premise is not satisfied. Hence, the conclusion
    is valid.
    \item Suppose sales and expenses go up. Then the premise is satisfied. Hence,
    the conclusion is invalid.
  \end{enumerate}
\end{exx}

\subsection{Sentential Logic}
\begin{exx} \label{2.2.1}
  \begin{tabular}{c c c}
    $\lnot P$ & $Q$ & $\lnot P \lor Q$ \\
    \hline
    TF & T & T \\
    FT & F & F\\
    TF & F & T\\
    FT & T & T
  \end{tabular}
  \begin{tabular}{c c c c c}
    $\lnot S$ & $\lnot G$ & $S\lor G$ & $\lnot S\lor \lnot G$ & $(S\lor G)\land(\lnot S\lor \lnot G)$ \\
    \hline
    TF & FT & T & T & T \\
    FT & TF & T & T & T \\
    TF & TF & F & T & F \\
    FT & FT & T & F & F
  \end{tabular}
\end{exx}

\begin{exx}
  First part is equivalent to $\lnot P\lor \lnot Q$. The truth table is easy to make.
\end{exx}

\begin{exx}
    $P+Q = (P\lor Q)\land\lnot(P\land Q) = (P\lor Q)\land(\lnot P\lor \lnot Q)$.
    Look at \autoref{2.2.1}.
\end{exx}

\begin{exx}
  $P\lor Q=\lnot(\lnot P\land \lnot Q)$.
\end{exx}
