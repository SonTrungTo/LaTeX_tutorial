\documentclass[a4paper,11pt]{article}
%\usepackage{syntonly}
\usepackage[affil-it]{authblk}
\usepackage[T1]{fontenc}
\usepackage[utf8]{inputenc}
\usepackage{microtype,amsmath,amsthm,amssymb,fancyhdr,enumitem,%
pgfplots,tikz}
\usepackage[pdftex, colorlinks, urlcolor = {red}, unicode]{hyperref}
\usepackage[retainorgcmds]{IEEEtrantools}
\pgfplotsset{width=7cm,compat=1.9}

\theoremstyle{plain} \newtheorem{id}{Lemma}[section]
                     \newtheorem{thm}{Theorem}[section]
\theoremstyle{definition} \newtheorem{ex}{Exercise}[section]
\theoremstyle{remark}     \newtheorem{ab}{Remark}[section]

\newcommand{\idautorefname}{Lemma}
\newcommand{\thmautorefname}{Theorem}
\newcommand{\exautorefname}{Exercise}
\newcommand{\abautorefname}{Remark}

\renewcommand{\familydefault}{cmtt}

\author{Son~To\\
$<$\href{mailto:son.trung.to@gmail.com}{son.trung.to@gmail.com}$>$}
\affil{StaffPoint Oy}
\title{Answers to exercises in How To Prove It}

\begin{document}
  \maketitle
  This is to answer all the questions in the books ``How to prove it'' by Velleman.
  Comments are appreciated!

  \clearpage
  \tableofcontents
  \clearpage

  \section{Introduction}
  \begin{ex}
    \begin{enumerate}[label=(\alph*)]
      \item\label{1.a} $a=3$, $b=5$ $\Rightarrow$ $x=2^5-1=31$, $y=1+2^5+2^{10}=1057$
      \item Since $32,767$ is not a prime, $2^{32,767} - 1$ is not a prime either.
      Therefore, there exists a positive integer $0<x<2^{32,767}-1$ such that
      $2^{32,767}-1$ is divisible by $x$. Hence, by \ref{1.a}, $x=2^{31}-1$ satisfies this.
    \end{enumerate}
  \end{ex}

  \begin{ex}
    $ $\newline
    \begin{tabular}{c c c}
      $n$ & $3^n - 1$ & $3^n - 2^n$ \\
      \hline
      $2$ & $8$, not prime & $5$, prime
    \end{tabular}
  \end{ex}
  \section{Chapter $1$}
  \begin{ex}
    OK MAN
  \end{ex}


\end{document}
