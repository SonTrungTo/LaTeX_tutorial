\documentclass[a4paper,11pt]{memoir}
\usepackage[affil-it]{authblk}
\usepackage{amsmath,amssymb,amsthm,fancyhdr,enumitem,%
microtype,pgfplots,tikz}
\usepackage[pdftex,hyperindex=false,%
colorlinks,unicode,hidelinks]{hyperref}
\pgfplotsset{width=7cm,compat=1.9}
\usepackage[retainorgcmds]{IEEEtrantools}
\newcommand{\note}[1]{\emph{#1}}
\theoremstyle{plain} \newtheorem{id}{Lemma}
                     \newtheorem{thm}{Theorem}
\theoremstyle{definition} \newtheorem{pr}{Problem}[chapter]
\theoremstyle{remark}\newtheorem*{ab}{Remark}
\newcommand{\idautorefname}{Lemma}
\newcommand{\abautorefname}{Remark}
\newcommand{\thmautorefname}{Theorem}
\newcommand{\prautorefname}{Problem}
\newenvironment{solution}%
{\renewcommand{\qedsymbol}{$\blacksquare$}%
\begin{proof}[Solution]}%
{\end{proof}}
\pagestyle{fancy}
\fancyhf{}
\fancyhead[LO,RE]{\textsl{\leftmark}}
\fancyhead[RO,LE]{Page \thepage\ of \thelastpage}
\fancyfoot[CE]{Son To}
\fancyfoot[CO]{\textsl{Solutions to Michael Spivak's Calculus}}
\renewcommand{\headrulewidth}{0.4pt}
\renewcommand{\footrulewidth}{0.4pt}

\begin{document}

  \title{Solutions to Michael Spivak's Calculus}
  \author{Son To\\
  $<$\href{mailto:son.trung.to@gmail.com}%
  {son.trung.to@gmail.com}$>$}
  \affil{Ravintola Kiltakellari
  \thanks{I thank my employer!
  }}
  \date{25th June, 2017}

  \maketitle
  \makeatletter
  \def\cleardoublepage{\clearpage\if@twoside
  \ifodd\c@page\else
    \hbox{}
    \vspace*{\fill}
    \begin{center}
      \doublenote
    \end{center}
    \vspace*{\fill}
    \thispagestyle{empty}
    \newpage
    \if@twocolumn\hbox{}\newpage\fi\fi\fi
}
\newcommand{\mylabel}[2]{#2\def\@currentlabel{#2}\label{#1}}
\def\@endpart{\vfill\newpage
              \if@twoside
                \if@openright
                  \null
                  \thispagestyle{empty}%
%-------------------------------
\vspace*{\fill}%
\begin{quote}%
  \partnote
\end{quote}%
\vspace*{\fill}%
%-------------------------------
                  \newpage
                \fi
              \fi
              \if@tempswa
                \twocolumn
              \fi
}
\newcommand{\partnote}{}
\newcommand{\doublenote}{}

%----This is the TeX for Pascal triangle
%----using binomialCoefficient
\newcommand\binomialCoefficient[2]{%
  %Store values
  \c@pgf@counta=#1 % n
  \c@pgf@countb=#2 % k
  %
  % Take advantage of the symmetry if k > n - k
  \c@pgf@countc=\c@pgf@counta%
  \advance \c@pgf@countc by-\c@pgf@countb%
  \ifnum\c@pgf@countb>\c@pgf@countc%
    \c@pgf@countb=\c@pgf@countc%
  \fi%
  %
  % Recursively compute the coefficients
  \c@pgf@countc=1 % hold the result
  \c@pgf@countd=0 % counter
  \pgfmathloop % c -> c*(n-i)/(i+1), i=0,1,...,k-1
    \ifnum\c@pgf@countd<\c@pgf@countb%
    \multiply \c@pgf@countc by\c@pgf@counta%
    \advance  \c@pgf@counta by-1%
    \advance  \c@pgf@countd by1%
    \divide   \c@pgf@countc by\c@pgf@countd%
  \repeatpgfmathloop%
  \the\c@pgf@countc%
}
%-------binomialCoefficient ends here!
\makeatother

\renewcommand{\doublenote}{%
\note{To my mother, friends and all those who influence me.}}
\frontmatter
  \chapter{Preface}
    This is my own solutions to Michael Spivak's Calculus
    textbook.

    \flushright
    Vantaa, Finland \\
    $25^{th}$ June, $2017$.
  \clearpage
  \tableofcontents

\mainmatter
  \renewcommand{\partnote}{
  This page is intentionally left blank.
  }
  \part{Prologue}
    \chapter{Basic properties of number}
\begin{pr} \label{1.1}
  Prove the following:
  \begin{enumerate}[label=(\roman*)]
    \item If $ax=a$ for some number $a\neq0$, then
    $x=1$
    \item $x^2-y^2=(x-y)(x+y)$
    \item If $x^2=y^2$, then $x=y$ or $x=-y$
    \item \label{1:iv}
    $x^3-y^3=(x-y)(x^2+xy+y^2)$
    \item \label{1.1:v}
    $x^n-y^n=(x-y)(x^{n-1}+x^{n-2}y+\dots+xy^{n-2}
    +y^{n-1})$
    \item \label{1.1:vi}
    $x^3+y^3=(x+y)(x^2-xy+y^2)$ (There is a particularly
    easy way to do this using \ref{1:iv}, and it will show
    you how to find a factorization for $x^n+y^n$ whenever
    n is odd.)
  \end{enumerate}
\end{pr}

\begin{solution}
  \begin{enumerate}[label=(\roman*)]
    \item By (P7)(Existence of multiplicative inverses),
    there exists $a^{-1}$ such that,
    \begin{IEEEeqnarray*}{rCl}
      (a^{-1}\cdot a)x & = & (a^{-1}\cdot a) \\
      x & = & 1 \\
    \end{IEEEeqnarray*}
    \item \label{(ii)} By (P9) for 2 times,
    \begin{IEEEeqnarray*}{+rCl+x*}
      (x-y)(x+y) & \stackrel{1}{=} & x\cdot(x+y)+(-y)\cdot(x+y)\\
      & \stackrel{2}{=} & x\cdot x+x\cdot y+(-y)\cdot x
      +(-y)\cdot y \\
      & = & x^2 +x\cdot y+[-(x\cdot y)]+[-(y^2)] \\
      & = & x^2-y^2 \\
    \end{IEEEeqnarray*}
    \item From \ref{(ii)} and since $x^2=y^2$,
    \begin{equation*}
      x^2-y^2=(x-y)(x+y)=0
    \end{equation*}
    This means $(x-y)=0 \lor (x+y)=0$, which is $x=y \lor x=-y$
    \item Starting with the right-hand side,
    \begin{IEEEeqnarray*}{rCl}
      (x-y)(x^2+xy+y^2) & = &
      x\cdot(x^2+xy+y^2)+(-y)\cdot(x^2+xy+y^2)\\
      & = &
      x^3+x^2y+xy^2+[-(x^2y)]+[-(xy^2)]+[-(y)^3] \\
      & = &
      x^3-y^3 \\
    \end{IEEEeqnarray*}
    \item I propose two solutions for this problem.
    The first one is the direct right-hand side manipulation,
    while the latter is done by induction.
    \renewcommand{\qedsymbol}{\textsl Q.E.D}
    \begin{proof}[The first solution]
      \begin{IEEEeqnarray*}{+l+x*}
        (x-y)(x^{n-1}+x^{n-2}y+\cdots+xy^{n-2}+y^{n-1}) \\=
        x^n+x^{n-1}y+\cdots+x^2y^{n-2}+xy^{n-1} \\
        +[-(x^{n-1}y)]+[-(x^{n-2}y^2)]+\cdots+
        [-(xy^{n-1})]+[-(y^n)] \\
        =x^n-y^n \\
        & \qedhere
      \end{IEEEeqnarray*}
    \end{proof}
    \begin{proof}[The second solution]
      Let n=1, then indeed $x-y=x-y$. Suppose the statement
      holds true for $n=k$ with $k\in \mathbb{N}$, that is
      \begin{equation*}
        x^k-y^k=(x-y)(x^{k-1}+x^{k-2}y+\cdots+xy^{k-2}+y^{k-1})
      \end{equation*}
    is true. To finish the proof, we need to prove
    \begin{equation*}
      x^{k+1}-y^{k+1}=(x-y)(x^k+x^{k-1}y+\cdots+xy^{k-1}+y^k)
    \end{equation*}
    That is, the statement holds for $n=k$. Starting from
    the left hand side,
    \begin{IEEEeqnarray*}{*x+C+x*}
      & x^{k+1}-y^{k+1} \\
      = & x^{k+1}-x^ky+x^ky-y^{k+1} \\
      = & x^k(x-y)+y(x^k-y^k) \\
      = & x^k(x-y)+y(x-y)(x^{k-1}+x^{k-2}y+\cdots+
      xy^{k-2}+y^{k-1})\\
      = & (x-y)[x^k+y(x^{k-1}+x^{k-2}y+\cdots+xy^{k-2}+y^{k-1})]
      \\
      = & (x-y)(x^k+x^{k-1}y+x^{k-2}y^2+\cdots+xy^{k-1}+
      y^k) \\
      & & \qedhere
    \end{IEEEeqnarray*}
    \end{proof}
    \item We will use \ref{1:iv} in our proof,
    \begin{IEEEeqnarray*}{*x+C+x*}
      & x^3+y^3 \\
    = & x^3-y^3+2y^3 \\
    = & (x-y)(x^2+xy+y^2)+2y[(x^2+xy+y^2)+(-x)(x+y)] \\
    = & (x+y)(x^2+xy+y^2)+2[-(xy)](x+y) \\
    = & (x+y)(x^2-xy+y^2) \\
    \end{IEEEeqnarray*}
  \end{enumerate}
\end{solution}

\begin{pr}
  What is wrong with the following ``proof''? Let $x=y$. Then
  \begin{IEEEeqnarray*}{rCl}
    x^2 & = & xy, \\
    x^2-y^2 & = & xy -y^2, \\
    (x+y)(x-y) & = & y(x-y), \\
    x+y & = & y, \\
    2y & = & y, \\
    2 & = & 1.
  \end{IEEEeqnarray*}
\end{pr}

\begin{solution}
  Note that in the transition from line 3 to line 4,
  the author ``simplifies'' $(x-y)$ by
  dividing $(x-y)$ on both sides. This is wrong since $x-y=0$,
  and hence $1/0$ is undefined as implied by (P7)
  in the textbook.
\end{solution}

\begin{pr}
  Prove the following:
  \begin{enumerate}[label=(\roman*)]
    \item $\displaystyle{\frac{a}{b} = \frac{ac}{bc}}$,
    if $b,c\neq0$.
    \item
      $\displaystyle{\frac{a}{b}+\frac{c}{d}=\frac{ad+bc}{bd}}$,
    if $b,d\neq0$.
    \item
    $(ab)^{-1}=a^{-1}b^{-1},\text{ if }a,b\neq0$. (To do this
    you must remember the defining property of $(ab)^{-1}$.)
    \item
    $\dfrac{a}{b}\cdot\dfrac{c}{d}=\dfrac{ac}{db}$, if
    $b,d\neq0$.
    \item
    $\dfrac{a}{b}\Bigg/\dfrac{c}{d}=\dfrac{ad}{bc}$, if
    $b,c,d\neq0$.
    \item
    If $b,d\neq0$, then $\dfrac{a}{b}=\dfrac{c}{d}$ if and
    only if $ad=bc$. Also determine when $\dfrac{a}{b}=
    \dfrac{b}{a}$.
  \end{enumerate}
\end{pr}

\begin{solution}
  \begin{enumerate}[label=(\roman*)]
    \item \label{2:i}
    Until \ref{2:iii} is proved, the solution is to
    test the equality between two sides.
    \begin{IEEEeqnarray*}{rCl}
      a(b)^{-1} & = & (ac)(bc)^{-1}\\
      a[(b)^{-1}b] & = & (ac)(bc)^{-1}b \\
      (a^{-1}a) & = & (a^{-1}a)c(bc)^{-1}b \\
      1 & = & (bc)(bc)^{-1} = 1\\
    \end{IEEEeqnarray*}
    \item\label{p13:ii} Similar to the above,
    \begin{IEEEeqnarray*}{rCl}
      a(b)^{-1}+c(d)^{-1} & = & (ad+bc)(bd)^{-1} \\
      a(b)^{-1}bd+c(d)^{-1}bd & = & (ad+bc)[(bd)^{-1}(bd)]\\
      ad(b^{-1}b)+bc(d^{-1}d) & = & (ad+bc)\\
      ad+bc & = & ad+bc
    \end{IEEEeqnarray*}
    \item \label{2:iii}
    Since $a,b\neq0$, there exists $(ab)^{-1},a^{-1},
    b^{-1}$ such that,
    \begin{IEEEeqnarray*}{rCl}
      ab & = & ab \\
      (ab)^{-1}(ab) & = & (ab)^{-1}(ab)=1 \\
      (ab)^{-1}a(bb^{-1}) & = & b^{-1} \\
      (ab)^{-1}(aa^{-1}) & = & b^{-1}a^{-1} \\
      (ab)^{-1} & = & a^{-1}b^{-1} \\
    \end{IEEEeqnarray*}
    \item For $b,d\neq0$,
    \begin{equation*}
      \frac{a}{b}\cdot\frac{c}{d}=ab^{-1}cd^{-1}=ac(d^{-1}b^{-1})
      =ac(db)^{-1}=\frac{ac}{db}
    \end{equation*}
    where the next-to-last equality follows from \ref{2:iii}.
    \item I first establish for any number $a\neq0$,
    \begin{equation*}
      (a^{-1})^{-1}=a
    \end{equation*}
    Let $t=a^{-1}$, we want to prove $t^{-1}=a$. Observe that
    \begin{IEEEeqnarray*}{rCl}
      t & = & a^{-1} \\
      t\cdot(t)^{-1} & = & a^{-1}\cdot (t)^{-1} \\
      a\cdot 1 & = & (a\cdot a^{-1})\cdot (t)^{-1} \\
      a & = & (t)^{-1}
    \end{IEEEeqnarray*}
    From the left hand side of the statement,
    \begin{equation*}
      \frac{a}{b}\Bigg/\frac{c}{d}=a(b)^{-1}[c(d)^{-1}]^{-1}
      =a(b)^{-1}(c)^{-1}[(d)^{-1}]^{-1}
      =(ad)(bc)^{-1}=\frac{ad}{bc}
    \end{equation*}
    where the second and third equality follows both
    from \ref{2:iii}
    and the proof above.
    \item Using \ref{p13:ii},
    \begin{IEEEeqnarray*}{rCl}
      \frac{a}{b} & = & \frac{c}{d} \\
      \frac{a}{b}+(-\frac{c}{d}) & = & 0 \\
      \frac{ad-bc}{bd} & = & 0 \\
      ad & = & bc \\
    \end{IEEEeqnarray*}
    Now, put $c=b \land d=a$. It follows that
    $\dfrac{a}{b}=\dfrac{b}{a}$ if and only if
    $a^2=b^2$. It follows $(a-b)(a+b)=0$, or
    $a=b \lor a=-b$.
  \end{enumerate}
\end{solution}

\begin{pr}
  Find all numbers x for which
  \begin{enumerate}[label=(\roman*)]
    \item $4-x<3-2x$
    \item $5-x^2<8$
    \item $5-x^2<-2$
    \item $(x-1)(x-3)>0$ (When is a product of two numbers
    positive?)
    \item $x^2-2x+2>0$
    \item $x^2+x+1>2$
    \item $x^2-x+10>16$
    \item $x^2+x+1>0$
    \item $(x-\pi)(x+5)(x-3)>0$
    \item $(x-\sqrt[3]{2})(x-\sqrt{2})>0$
    \item $2^x<8$
    \item $x+3^x<4$
    \item $\dfrac{1}{x}+\dfrac{1}{1-x}>0$
    \nopagebreak[3]
    \item $\dfrac{x-1}{x+1}>0$
  \end{enumerate}
\end{pr}
\pagebreak
\begin{solution}
  \begin{enumerate}[label=(\roman*)]
    \item
    \begin{IEEEeqnarray*}{rCl}
      4-x & < & 3-2x \\
      4+(-x+2x) & < & 3+(-2x+2x) \\
      (-4+4)+x & < & -4+3 \\
      x & < & -1 \\
    \end{IEEEeqnarray*}
    \item
    \begin{IEEEeqnarray*}{rCl}
      5-x^2 & < & 8 \\
      5-8 & < & x^2 \\
      -3 & < & x^2
    \end{IEEEeqnarray*}
    Since $x^2\geq0\ \forall x\in\mathbb{R}$, the inequality
    holds $\forall x$.
    \item
    \begin{IEEEeqnarray*}{rCl}
      5-x^2 & < & -2 \\
      7 & < & x^2 \\
      0 & < & x^2-7=(x-\sqrt{7})(x+\sqrt{7})
    \end{IEEEeqnarray*}
    Hence, either $x>\sqrt{7}~\land\ x>-\sqrt{7}$
    or $x<\sqrt{7}~\land\ x<-\sqrt{7}$, which is
    $x>\sqrt{7}~\lor\ x<-\sqrt{7}$.
    \item
    \begin{IEEEeqnarray*}{rCl}
      (x-1)(x-3) & > & 0 \\
      (x>1~\land\ x>3) & \lor & (x<1~\land\ x<3)\\
      x>3 & \lor & x<1
    \end{IEEEeqnarray*}
    \item
    \begin{IEEEeqnarray*}{rCl}
      x^2-2x+2 & > & 0 \\
      (x^2-2x+1)+1 & > & 0 \\
      (x-1)^2+1 & > & 0
    \end{IEEEeqnarray*}
    Hence the inequality is satisfied $\forall\ x$.
    \item
    \begin{IEEEeqnarray*}{rCl}
      x^2+x+1 & > & 2 \\
      x^2+x-1 & > & 0 \\
      x^2+\left(\frac{1+\sqrt{5}}{2}\right)x
      +\left(\frac{1-\sqrt{5}}{2}\right)x
      +\left(\frac{(1-\sqrt{5})(1+\sqrt{5})}{4}\right)
      & > & 0 \\
      \left(x+\frac{1+\sqrt{5}}{2}\right)
      \left(x+\frac{1-\sqrt{5}}{2}\right) & > & 0 \\
      x>\left(\frac{\sqrt{5}-1}{2}\right) \lor
      x<\left(\frac{-(\sqrt{5}+1)}{2}\right)
    \end{IEEEeqnarray*}
    \item
    \begin{IEEEeqnarray*}{rCl}
      x^2-x+10 & > & 16 \\
      x^2-x-6 & > & 0 \\
      x^2-3x+2x-6 & > & 0 \\
      x(x-3)+2(x-3) & > & 0 \\
      (x+2)(x-3) & > & 0 \\
      x>3 & \lor & x<-2
    \end{IEEEeqnarray*}
    \item
    \begin{IEEEeqnarray*}{rCl}
      x^2+x+1 & > & 0 \\
      x^2+x+\frac{1}{4}-\frac{1}{4}+1 & > & 0 \\
      (x+\frac{1}{2})^2+\frac{3}{4} & > & 0
    \end{IEEEeqnarray*}
    which is true for all $x$.
    \item Divide the problem into two cases: $x>\pi$
    and $x<\pi$.
    \par
    \note{Case 1: }$x>\pi$\\
    Then $(x+5)(x-3)>0$, which is $x>3~\lor\ x<-5$.
    \par
    \note{Case 2: }$x<\pi$\\
    Then $(x+5)(x-3)<0$, which is $-5<x<3$.
    \item
    \begin{IEEEeqnarray*}{rCl}
      (x-\sqrt[3]{2})(x-\sqrt{2}) & > & 0 \\
      x>\sqrt{2} & \lor & x<\sqrt[3]{2}
    \end{IEEEeqnarray*}
    \item (Sometimes, to solve a problem, intuition
    is a necessity.)
    \begin{IEEEeqnarray*}{rCl}
      2^x & < & 8 \\
      2^x & < & 2^3 \\
      x & < & 3
    \end{IEEEeqnarray*}
    \item
    \begin{IEEEeqnarray*}{rCl}
      x+3^x & < & 4 \\
      x+3^x & < & 1+3^1 \\
      x & < & 1
    \end{IEEEeqnarray*}
    \item
    \begin{IEEEeqnarray*}{rCl}
      \frac{1}{x}+\frac{1}{1-x} & > & 0 \\
      \frac{1}{x(1-x)} & > & 0 \\
    \end{IEEEeqnarray*}
    Hence, $x(1-x)>0$. This means $0<x<1$.
    \item
    \begin{IEEEeqnarray*}{rCl}
      \frac{x-1}{x+1} & > & 0 \\
    \end{IEEEeqnarray*}
    Hence, $(x-1)(x+1)>0$, or $x>1~\lor\ x<-1$.
  \end{enumerate}
\end{solution}

\begin{pr} \label{1.5}
  Prove the following:
  \begin{enumerate}[label=(\roman*)]
    \item \label{1.5:i}
    If $a<b$ and $c<d$, then $a+c<b+d$
    \item If $a<b$, then $-b<-a$
    \item If $a<b$ and $c>d$, then $a-c<b-d$
    \item If $a<b$ and $c>0$, then $ac<bc$
    \item If $a<b$ and $c<0$, then $ac>bc$
    \item If $a>1$, then $a^2>a$
    \item If $0<a<1$, then $a^2<a$
    \item \label{1.5:viii}
    If $0\leq a<b$ and $0\leq c<d$, then $ac<bd$
    \nopagebreak[3]
    \item \label{1.5:ix}
    If $0\leq a<b$, then $a^2<b^2$. (Use \ref{1.5:viii}.)
    \item If $a,b\geq0$ and $a^2<b^2$, then $a<b$.%
    (Use \ref{1.5:ix}, backwards.)
  \end{enumerate}
\end{pr}
\pagebreak
\begin{solution}
  Let P be the set of all positive numbers.
  \begin{enumerate}[label=(\roman*)]
    \item To prove this, we apply (P11): If $a<b~\land\ c<d$,
    then $(b-a\in P)~\land\ (d-c\in P)$. Then $(b-a)+(d-c)=
    (b+d)-(a+c)\in P$. Therefore, $a+c<b+d$.
    \item \label{1.5:ii}
    We provide two solutions: The first one is
    by Trichotomy Law (P10), and the second one is
    by adding $[(-a)+(-b)]$ to both sides.
    \renewcommand{\qedsymbol}{\textsl Q.E.D}
    \begin{proof}[Proof by Trichotomy Law]
      If $a<b$, then $b-a\in P$. By Trichotomy Law,
      $a-b\notin P$ and $a-b\neq 0$. Therefore, $a-b<0$,
      which is $-b<-a$.
    \end{proof}
    \begin{proof}[Proof by adding]
      \begin{IEEEeqnarray*}{+rCl+x*}
        a & < & b \\
        a+[(-a)+(-b)] & < & b+[(-a)+(-b)] \\
        \left[a+(-a)\right]+(-b) & < & [b+(-b)]+(-a) \\
        -b & < & -a \\
        & & & \qedhere
      \end{IEEEeqnarray*}
    \end{proof}
    \item Using (P11), we have $b-a\in P~\land\ c-d\in P$.
    Then $(b-a)+(c-d)\in P$. Hence, $a-c<b-d$.
    \item \label{1.5:iv}
    Using (P12), note that $b-a\in P$. Since $c>0$,
    $c(b-a)\in P$, which means $bc-ac>0$, or $ac<bc$.
    \item By Trichotomy law(P10), $-c\in P$. Then
    by \ref{1.5:iv}, $-(ac)<-(bc)$. By \ref{1.5:ii},
    $ac>bc$.
    \item
    Since $a>1>0$, by \ref{1.5:iv}, $a^2>a$.
    \item Since $a>0$, by \ref{1.5:iv}, $a^2<a$.
    \item Because $0<b$, $bc<bd$. Furthermore, if $c\geq0$,
    $ac\leq bc$ (equality occurs if $c=0$), by \ref{1.5:iv}.
    Therefore, $ac\leq bc<bd$. Hence, $ac<bd$.
    \item From \ref{1.5:viii}, let $c=a$ and $d=b$, then
    the result follows.
    \item
    Suppose $a\geq b$. Then $a\geq b\geq 0$. By \ref{1.5:ix}
    and (P9), $a^{2}\geq b^2$. This contradicts $a^2<b^2$.
  \end{enumerate}
\end{solution}

\begin{pr} \label{1.6}
  \begin{enumerate}[label=(\alph*)]
    \item \label{1.6:a}
    Prove that if $0\leq x<y$, then $x^n<y^n$,
    $n=1,2,3,\dots$.
    \item \label{1.6:b}
    Prove that if $x<y$ and $n$ is odd,
    then $x^n<y^n$.
    \item \label{1.6:c}
    Prove that if $x^n=y^n$ and $n$ is odd,
    then $x=y$.
    \item \label{1.6:d}
    Prove that if $x^n=y^n$ and $n$ is even,
    then $x=y$ or $x=-y$.
  \end{enumerate}
\end{pr}
\pagebreak
\begin{solution}
  \begin{enumerate}[label=(\alph*)]
    \item
    Repeatedly apply problem \ref{1.5}\ref{1.5:viii} for
    $0\leq x<y$, we have $x^n<y^n$ with $n=1,2,3,\ldots$
    \item
    The statement is true for the case $0\leq x<y$. In
    the case $x<y\leq 0$, by \ref{1.5}\ref{1.5:ii},
    $(-x)>(-y)\geq 0$. By \ref{1.6:a}, $(-x)^n>(-y)^n$
    for all odd $n$. Since $n$ is odd, $-(x^n)>-(y^n)$.
    Hence, by \ref{1.5}\ref{1.5:ii}, $x^n<y^n$. In the
    case $x\leq 0<y$, since $n$ is odd, $x^n<y^n$.
    \item
    Suppose that either $x\neq y$. W.l.o.g, let $x<y$,
    by \ref{1.6:b}, $x^n<y^n$ for all odd $n$, contradicting
    $x^n=y^n$ for all odd $n$.
    \item
    Suppose that both $x\neq y$ and $x\neq -y$. Then
    $x^2-y^2\neq 0$. W.l.o.g, suppose $x^2>y^2\geq0$.
    Applying \ref{1.6:a}, this generalizes to
    $x^n>y^n$ for all even $n$, contradicting our assumption.
    Therefore, $x=y$ or $x=-y$.
    \renewcommand{\qedsymbol}{}
    \begin{proof}[The direct proof]
      In the case $x,y \geq 0$; by \ref{1.6:a}, if $x^n=y^n$
      for all even $n$, then $x=y$. In the case $x,y\leq 0$;
      if $x^n=y^n$ for all even $n$, then
      $(-x),(-y)\geq 0$ and $(-x)^n=(-y)^n$, so $-x=-y$
      and hence $x=y$.
      In the case of $x$ and $y$ have different signs, then
      $x$ and $-y$ are either two positive or two negative
      numbers. In either subcase, if $x^n=y^n$ for all even $n$,
      then $x^n=(-y)^n$, and it follows $x=-y$ from the
      previous case.
    \end{proof}
  \end{enumerate}
\end{solution}

\begin{pr} \label{1.7}
  Prove that if $0<a<b$, then
  \begin{equation*}
    a<\sqrt{ab}<\frac{a+b}{2}<b
  \end{equation*}
  Notice that the inequality $\sqrt{ab}\leq(a+b)/2$ holds
  for all $a,b\geq0$. A generalization of this fact
  occurs in Problem 2.22. % to be \ref{2.22}.
\end{pr}
\begin{solution}
  Let us first establish that $a<\dfrac{a+b}{2}<b$. Note
  that,
  \begin{equation*}
    a+a<a+b<b+b
  \end{equation*}
  and therefore, $a<\dfrac{a+b}{2}<b$. To finish the proof,
  we need to prove ${a<\sqrt{ab}<\dfrac{a+b}{2}}$. To do this,
  let us prove that if $0<a<b$, then ${0<\sqrt{a}<\sqrt{b}}$.
  Note that since $b-a>0$,
  \begin{equation*}
    b-a=(\sqrt{b}-\sqrt{a})(\sqrt{b}+\sqrt{a})>0
  \end{equation*}
  Therefore, $\sqrt{b}>\sqrt{a}>0$. We rewrite the inequality
  as follows,
  \begin{equation*}
    \sqrt{a}\cdot(\sqrt{b}-\sqrt{a})>0
  \end{equation*}
  Then
  \begin{equation}
    a<\sqrt{ab} \label{1.7:eq1}
  \end{equation}
  We next notice that since ${\sqrt{b}-\sqrt{a}>0}$, it follows
  that \\$(\sqrt{b}-\sqrt{a})\cdot(\sqrt{b}-\sqrt{a})
  =(\sqrt{b}-\sqrt{a})^2>0$. Expand the left hand side,
  \begin{equation*}
    (\sqrt{b}-\sqrt{a})^2=a+b-2\sqrt{ab}>0
  \end{equation*}
  which implies,
  \begin{equation}
    \sqrt{ab}<\frac{a+b}{2} \label{1.7:eq2}
  \end{equation}
  From (\ref{1.7:eq1}) and (\ref{1.7:eq2}), we have
  $a<\sqrt{ab}<\dfrac{a+b}{2}$.
\end{solution}

\begin{pr}[*] \label{1.8}
  Although the basic properties of inequalities were stated
  in terms of the collection P of all positive numbers,
  and $<$ was defined in terms of $P$, this procedure
  can be reversed. Suppose that P10--P12 are replaced by
  \begin{enumerate}
    \item[\mylabel{P'10}{(P'10)}]
    For any numbers $a$ and $b$ one, and only one, of the
    following holds:
    \begin{enumerate}[label=(\roman*)]
      \item $a=b$,
      \item $a<b$,
      \item $b<a$.
    \end{enumerate}
    \item[\mylabel{P'11}{(P'11)}]
    For any numbers $a$, $b$, and $c$, if $a<b$ and $b<c$,
    then $a<c$.
    \item[\mylabel{P'12}{(P'12)}]
    For any numbers $a$, $b$, and $c$, if $a<b$, then
    $a+c<b+c$.
    \item[\mylabel{P'13}{(P'13)}]
    For any numbers $a$, $b$, and $c$, if $a<b$ and
    $0<c$, then $ac<bc$.
  \end{enumerate}
  Show that P10--P12 can then be deduced as theorems.
\end{pr}
\begin{solution}
  Let $P$ be the set of all positive numbers.
  \begin{itemize}
    \item To prove P10, let $c=a-b$, from \ref{P'10}, P10
    follows.
    \item To prove P11, let $a,b\in P$; it is
    sufficient to prove that $a+b>0$. From \ref{P'10},
    we divide the proof into three subscases:\par
    \note{Case 1: }$a=b$ \\
    Then $a+b=b+b>0+b>0$, where the first inequality follows
    from \ref{P'12}. By \ref{P'11}, $a+b>0$.\par
    \note{Case 2: }$a<b$ \\
    Then $a+b>a+a>0+a>0$, where the first and second inequality
    follow from \ref{P'12}. By applying \ref{P'11} twice,
    $a+b>0$. \par
    \note{Case 3: }$a>b$ \\
    Interchanging the role of $a$ and $b$, we have the result.
    \item To prove P12, let $a,b\in P$; it is
    sufficient to prove that $a\cdot b>0$. From \ref{P'10},
    we divide the proof into three subcases:\par
    \note{Case 1: }$a=b$ \\
    Then $a\cdot b=b\cdot b>0\cdot b=0$, where the first
    inequality follows from \ref{P'13} and the equality after
    which is from (P9).\par
    \note{Case 2: }$a<b$ \\
    Then $b\cdot a>a\cdot a>0\cdot a=0$, where the first
    and second inequality is from \ref{P'13}. By \ref{P'11},
    $a\cdot b>0$. \par
    \note{Case 3: }$a>b$ \\
    Interchanging $a$ and $b$ returns us to case 2, which
    yields the result.
  \end{itemize}
\end{solution}

\begin{pr}
  Express each of the following with at least one less pair
  of absolute value signs.
  \begin{enumerate}[label=(\roman*)]
    \item $|\sqrt{2}+\sqrt{3}-\sqrt{5}+\sqrt{7}|$
    \item $|(|a+b|-|a|-|b|)|$
    \item $|(|a+b|+|c|-|a+b+c|)|$
    \item $|(|\sqrt{2}+\sqrt{3}|-|\sqrt{5}-\sqrt{7}|)|$
  \end{enumerate}
\end{pr}

\begin{solution}
  \begin{enumerate}[label=(\roman*)]
    \item Note $\sqrt{7}-\sqrt{5}>0$, hence
    \begin{equation*}
      |\sqrt{2}+\sqrt{3}-\sqrt{5}+\sqrt{7}|=
      \sqrt{2}+\sqrt{3}-\sqrt{5}+\sqrt{7}
    \end{equation*}
    \item Since $|a+b|-|a|-|b|\leq 0$,
    \begin{equation*}
      |(|a+b|-|a|-|b|)|=|a|+|b|-|a+b|
    \end{equation*}
    \item Since $|a+b+c|\leq|a+b|+|c|$,
    \begin{equation*}
      |(|a+b|+|c|-|a+b+c|)|=|a+b|+|c|-|a+b+c|
    \end{equation*}
    \item
    \begin{equation*}
      |(|\sqrt{2}+\sqrt{3}|-|\sqrt{5}-\sqrt{7}|)|
      =|\sqrt{2}+\sqrt{3}-\sqrt{7}+\sqrt{5}|
    \end{equation*}
  \end{enumerate}
\end{solution}

\begin{pr}
  Express each of the following without absolute value
  signs, treating various cases separately when necessary.
  \begin{enumerate}[label=(\roman*)]
    \item $|a+b|-|b|$
    \item $|(|x|-1)|$
    \item $|x|-|x^2|$
    \item $a-|(a-|a|)|$
  \end{enumerate}
\end{pr}
\pagebreak
\begin{solution}
  \begin{enumerate}[label=(\roman*)]
    \item We divide into four cases:
    \begin{align}
      a\geq0 \quad \text{and} \quad b\geq0 \tag{Case 1} \\
      a\leq0 \quad \text{and} \quad b\leq0 \tag{Case 2} \\
      a\geq0 \quad \text{and} \quad b\leq0 \tag{Case 3} \\
      a\leq0 \quad \text{and} \quad b\geq0 \tag{Case 4}
    \end{align}
    In Case 1 and Case 2, we have $|a+b|-|b|=a$
    since $|a+b|\leq|a|+|b|$.\par
    In Case 3,
    if $a+b\geq0$, then
    \begin{equation*}
      |a+b|-|b|=(a+b)-(-b)=a+b+b=2b
    \end{equation*}
    If $a+b\leq0$, then
    \begin{equation*}
      |a+b|-|b|=(-a-b)-(-b)=-a+(-b)+b=-a
    \end{equation*}
    In Case 4, if $a+b\geq0$, then
    \begin{equation*}
      |a+b|-|b|=(a+b)-(b)=a
    \end{equation*}
    If $a+b\leq0$, then
    \begin{equation*}
      |a+b|-|b|=-a+(-b)+(-b)=-a-2b
    \end{equation*}
    \item We make the problem into 4 cases.
    \begin{align}
      x\geq1 \tag{Case 1} \\
      0\leq x\leq 1 \tag{Case 2} \\
      -1\leq x\leq 0 \tag{Case 3} \\
      x\leq -1 \tag{Case 4}
    \end{align}
    In Case 1, $|(|x|-1)|=x-1$. \par
    In Case 2, $|(|x|-1)|=1-x$. \par
    In Case 3, $|(|x|-1)|=x+1$. \par
    In Case 4, $|(|x|-1)|=-(x+1)$.
    \item Since $x^2\geq0$, $|x|-|x^2|=|x|-x^2$. \par
    If $x\geq0$, then $|x|-x^2=x(1-x)$. If $x\leq0$,
    then\\ $|x|-x^2=-x+(-x^2)=-x(1+x)$.
    \item Note that $|a|\geq a$. Hence,
    \begin{equation*}
      a-|(a-|a|)|=a+a-|a|=2a-|a|
    \end{equation*}
    We have two cases,
    \par
    \note{Case 1: }$a\geq0$
    \begin{equation*}
      2a-|a|=2a-a=a
    \end{equation*}
    \note{Case 2: }$a\leq0$
    \begin{equation*}
      2a-|a|=2a+a=3a
    \end{equation*}
  \end{enumerate}
\end{solution}

\begin{pr}
  Find all numbers $x$ for which
  \begin{enumerate}[label=(\roman*)]
    \item $|x-3|=8$
    \item $|x-3|<8$
    \item $|x+4|<2$
    \item $|x-1|+|x-2|>1$
    \item $|x-1|+|x+1|<2$
    \item $|x-1|+|x+1|<1$
    \item $|x-1|\cdot|x+1|=0$
    \item $|x-1|\cdot|x+2|=3$
  \end{enumerate}
\end{pr}

\begin{solution}
  \begin{enumerate}[label=(\roman*)]
    \item \begin{IEEEeqnarray*}{rCl}
      x-3=8 & \lor & x-3=-8 \\
      x=11 & \lor & x=-5
    \end{IEEEeqnarray*}
    \item Then $-8<x-3<8$. Hence, $-5<x<11$.
    \item Then $-2<x+4<2$. Hence, $-6<x<-2$.
    \item If $1\leq x\leq 2$, then the inequality becomes
    $(x-1)+(2-x)=1$. If $x>2$, then $2x-3>1$, which is
    $x>2$. If $x<1$, then $-2x+3>1$, which is $x<1$.
    Therefore, either $x>2$ or $x<1$ satisfies the inequality.
    \item If $-1\leq x\leq 1$, then $(1-x)+(x+1)=2$. If
    $x>1$, then $x<1$, which is contradictory. If $x<-1$,
    then $(1-x)+(-x-1)=-2x<2$ only if $x>-1$, which is
    contradictory. Hence, there is no $x$ to satisfy the
    inequality.
    \item It is implied from above that
    \begin{equation*}
      |x-1|+|x+1| \geq 2
    \end{equation*}
    Therefore, there is no $x$ satisfying the inequality.
    \item Either $x=1$ or $x=-1$.
    \item If $-2\leq x\leq 1$, then we obtain $x^2+x+1>0$.
    Hence, in either $x<-2$ or $x>1$, we have to solve
    the equation $x^2+x-5=0$, whose solution is either
    $x=\dfrac{-1+\sqrt{21}}{2}$ or
    $x=\dfrac{-1-\sqrt{21}}{2}$.
  \end{enumerate}
\end{solution}

\begin{pr} \label{1.12}% Prob12
  Prove the following:
  \begin{enumerate}[label=(\roman*)]
    \item \label{1.12:i}
    $|xy|=|x|\cdot|y|$
    \item \label{1.12:ii}
    $\left|\dfrac{1}{x}\right|=\dfrac{1}{|x|}$,
    if $x\neq0$. (The
    best way to do this is to remember what\\ $|x|^{-1}$ is.)
    \item $\dfrac{|x|}{|y|}=\left|\dfrac{x}{y}\right|$,
    if $y\neq0$.
    \item $|x-y|\leq|x|+|y|$ (Give a very short proof.)
    \item \label{1.12:v}
    $|x|-|y|\leq|x-y|$ (A very short proof is possible,
    if you write things in the right way.)
    \item $|(|x|-|y|)|\leq|x-y|$ (Why does this follow
    immediately from \ref{1.12:v}?)
    \item $|x+y+z|\leq|x|+|y|+|z|$. Indicate when equality
    holds, and prove your statement.
  \end{enumerate}
\end{pr}

\begin{solution}
  \begin{enumerate}[label=(\roman*)]
    \item We have 4 cases,
    \begin{align*}
      x\geq0 \quad y\geq0 \tag{1} \\
      x\geq0 \quad y\leq0 \tag{2} \\
      x\leq0 \quad y\geq0 \tag{3} \\
      x\leq0 \quad y\leq0 \tag{4}
    \end{align*}
    In (1), $|x|\cdot|y|=xy=|xy|$ \par
    In (4), $|x|\cdot|y|=(-x)(-y)=xy=|xy|$ \par
    In (3), $|x|\cdot|y|=(-x)(y)=-(xy)=|xy|$ \par
    In (2), interchanging $x$ and $y$ leads to (3).
    \item Since $x\neq0$, there exists $|x|^{-1}$ such that
    \begin{equation*}
      |x||x|^{-1}=1=|x|\left|\frac{1}{x}\right|
    \end{equation*}
    where the second equality is by \ref{1.12:i}. Dividing
    both sides by $|x|$, we have the result.
    \item Since $y\neq0$, from \ref{1.12:ii}, we immediately
    have
    \begin{equation*}
      \left|\frac{1}{y}\right|=\frac{1}{|y|}
    \end{equation*}
    Hence, applying \ref{1.12:ii} once more,
    \begin{equation*}
      \left|\frac{x}{y}\right|=|x|\left|\frac{1}{y}\right|
      =\frac{|x|}{|y|}
    \end{equation*}
    \item Note that,
    \begin{equation*}
      |x-y|=|x+(-y)|\leq|x|+|-y|=|x|+|y|
    \end{equation*}
    where the last equality follows from \ref{1.12:i}.
    \item Note that,
    \begin{equation*}
      |x-y+y|\leq|x-y|+|y|
    \end{equation*}
    Therefore, $|x|-|y|\leq|x-y|$.
    \item Let the first term be $y$ and the second term
    be $y-x$. Applying \ref{1.12:v}, we have
    \begin{equation*}
      |y|-|y-x|\leq|x|
    \end{equation*}
    Hence, $-|x-y|\leq|x|-|y|$. Combining with \ref{1.12:v}
    gives $|(|x|-|y|)|\leq|x-y|$.
    \item Notice the pattern,
    \begin{equation*}
      |x+y+z|\leq|x+y|+|z|\leq|x|+|y|+|z|
    \end{equation*}
    the equality holds only if either $x,y,z$ have the same
    sign or at least two of them must be equal to $0$.
    It is easy to verify this.\\
    Suppose not, then both $x,y,z$ have different signs and
    at most one of them is $0$. If the latter is true,
    then, w.l.o.g, suppose $z=0$, then $x,y$ have different
    sign, and we are done. If none of them is $0$, then,
    w.l.o.g, suppose $z<0$ and pick $z$ such that
    $x+y<-z$. Then,
    \begin{equation*}
      |x+y+z|=-(x+y+z)=-x-y-z<|x|+|y|+|z|
    \end{equation*}
    where inequality must follow since $x,y\neq0$.
  \end{enumerate}
\end{solution}

\begin{pr}
  The maximum of two numbers $x$ and $y$ is denoted by
  $max(x,y)$. Thus $max(-1,3)=max(3,3)=3$ and
  $max(-1,-4)=max(-4,-1)=-1$. The minimum of $x$ and $y$
  is denoted by $min(x,y)$. Prove that
  \begin{equation*}
    max(x,y)=\frac{x+y+|y-x|}{2},
  \end{equation*}
  \begin{equation*}
    min(x,y)=\frac{x+y-|y-x|}{2}.
  \end{equation*}
  Derive the formula for $max(x,y,z)$ and $min(x,y,z)$,
  using, for example
  \begin{equation*}
    max(x,y,z)=max(x,max(y,z)).
  \end{equation*}
\end{pr}
\pagebreak
\begin{solution}
  Assume that $x\geq y$, we want to prove that $max(x,y)=x$.
  \begin{equation*}
    max(x,y)=\frac{x+y+|y-x|}{2}=\frac{x+y+x-y}{2}
    =\frac{2x}{2}=x
  \end{equation*}
  Similarly, we need $min(x,y)=y$.
  \begin{equation*}
    min(x,y)=\frac{x+y-|y-x|}{2}=\frac{x+y-(x-y)}{2}
    =\frac{x+y-x+y}{2}=\frac{2y}{2}=y
  \end{equation*}
  Let $max(x,y,z)=max(x,max(y,z))$. Then
  \begin{IEEEeqnarray*}{rCl}
    max(x,y,z) & = & \frac{x+max(y,z)+|max(y,z)-x|}{2} \\
               & = & \frac{x+\dfrac{y+z+|z-y|}{2}+\left|
    \dfrac{y+z+|z-y|}{2}-x\right|}{2} \\
               & = & \frac{2x+y+z+|z-y|+\left|
    y+z+|z-y|-2x\right|}{4}
  \end{IEEEeqnarray*}
  Similarly,
  \begin{IEEEeqnarray*}{rCl}
    min(x,y,z) & = & \frac{2x+y+z-|z-y|-\left|
    y+z-|z-y|-2x\right|}{4}
  \end{IEEEeqnarray*}
\end{solution}

\begin{pr} \label{1.14}%1.14
  \begin{enumerate}[label=(\alph*)]
    \item \label{1.14:i}
    Prove that $|a|=|-a|$. (The trick is not to become
    confused by too many cases. First prove the statement for
    $a\geq0$. Why is it then obvious for $a\leq0$?)
    \item \label{1.14:ii}
    Prove that $-b\leq a\leq b$ if and only if
    $|a|\leq b$. In particular, it follows that
    $-|a|\leq a\leq|a|$.
    \item Use this fact to give a new proof that
    $|a+b|\leq|a|+|b|$.
  \end{enumerate}
\end{pr}

\begin{solution} %1.14 solution
  \begin{enumerate}[label=(\alph*)]
    \item \autoref{1.12}\ref{1.12:i} easily tells us that
    \begin{equation*}
      |-a|=|(-1)a|=|-1||a|=1|a|=|a|
    \end{equation*}
    \item
    If $a\geq0$, then $a\leq b$. If $a\leq0$,
    $-a\leq b$ follows from $a\geq -b$.
    Therefore, $|a|\leq b$. Conversely,
    suppose $|a|\leq b$. Then it is certain $a\leq b$
    since $a\leq|a|\leq b$. From \ref{1.14:i}, $|-a|\leq b$,
    and hence $a\geq -b$. We conclude that $-b\leq a\leq b$.
    Note that since $|a|\leq|a|$, $-|a|\leq a\leq|a|$.
    \item
    Because we have $-|a|\leq a\leq|a|$
    and $-|b|\leq b\leq|b|$, by \autoref{1.5}\ref{1.5:i},
    we obtain $-(|a|+|b|)\leq a+b\leq|a|+|b|$. From
    \ref{1.14:ii}, we arrive at the
    conclusion $|a+b|\leq|a|+|b|$.
  \end{enumerate}
\end{solution}

\pagebreak

\begin{pr}[*] %1.15
  Prove that if $x$ and $y$ are not both $0$, then
  \begin{IEEEeqnarray*}{rCl}
    x^2 + xy + y^2 & > & 0 \\
    x^4+x^3y+x^2y^2+xy^3+y^4 & > & 0
  \end{IEEEeqnarray*}
  Hint: Use problem 1.
\end{pr}

\begin{solution} %1.15 solution
  For the first part, note that
  \begin{equation*}
    x^2+xy+y^2=x^2+2\cdot x\cdot \frac{1}{2}y
    +\frac{1}{4}y^2-\frac{1}{4}y^2
    +y^2=\left(x+\frac{1}{2}y\right)^2+\frac{3}{4}y^2 > 0
  \end{equation*}
  For the second part, if $x=y$, then the left-hand side
  is $5x^4>0$. Hence, suppose $x\neq y$.
  From \autoref{1.1}\ref{1.1:v},
  \begin{equation*}
    x^5-y^5=(x-y)(x^4+x^3y+x^2y^2+xy^3+y^4)\neq0
  \end{equation*}
  If $x>y$, then $x^5>y^5$ by \autoref{1.6}\ref{1.6:b}.
  This implies that the second term must be greater than $0$.
  Conversely, $x<y\Rightarrow x^5<y^5$ implies that
  it must be greater than $0$.
\end{solution}

\begin{pr}[*] \label{1.16}%Problem 1.16
  \begin{enumerate}[label=(\alph*)]
    \item \label{1.16:a}
    Show that
    \begin{IEEEeqnarray*}{rCl}
      (x+y)^2 & = & x^2+y^2 \ \text{ only when }
      x=0 \text{ or } y=0,\\
      (x+y)^3 & = & x^3+y^3 \ \text{ only when }
      x=0 \text{ or } y=0 \text{ or } x=-y.
    \end{IEEEeqnarray*}
    \item \label{1.16:b}
    Using the fact that
    \begin{equation*}
      x^2 + 2xy + y^2 = (x+y)^2\geq0,
    \end{equation*}
    show that $4x^2+6xy+4y^2>0$ unless $x$ and $y$ are
    both $0$.
    \item \label{1.16:c}
    Use part \ref{1.16:b} to find out when
    $(x+y)^4 = x^4 + y^4$.
    \item \label{1.16:d}
    Find out when $(x+y)^5=x^5+y^5$. Hint: From the assumption
    $(x+y)^5=x^5+y^5$ you should be able to derive the
    equation $x^3+2x^2y+2xy^2+y^3=0$, if $xy\neq0$. This
    implies that $(x+y)^3=x^2y+xy^2=xy(x+y)$.
  \end{enumerate}
  You should know be able to make a good guess as to when
  $(x+y)^n=x^n+y^n$; the proof is contained in Problem
  11.57 % \ref{11.57}
\end{pr}

\begin{solution}
  \begin{enumerate}[label=(\alph*)]
    \item For the first part,
    \begin{equation*}
      (x+y)^2=x^2+2xy+y^2
    \end{equation*}
    Hence, $(x+y)^2=x^2+y^2$ only when $x=0$ or $y=0$.
    For the second part, from \autoref{1.1}\ref{1.1:vi},
    \begin{IEEEeqnarray*}{rCl}
      (x+y)^3-(x+y)(x^2-xy+y^2) & = & 0 \\
      (x+y)(xy) & = & 0
    \end{IEEEeqnarray*}
    which is true only when $x=0$ or $y=0$ or $x=-y$.
    \item Note that $4x^2+6xy+4y^2=
    \underbrace{3(x+y)^2}_\text{$\geq0$}
    +\underbrace{x^2+y^2}_\text{$>0$}>0$ unless
    $x=0$ and $y=0$.
    \item Let us expand $(x+y)^4$.
    \begin{IEEEeqnarray*}{rCl}
      (x+y)^2(x+y)^2 & = & (x^2+2xy+y^2)(x^2+2xy+y^2) \\
      & = & x^4 + 4x^3y+ 6x^2y^2+ 4xy^3 + y^4 \\
      & = & x^4 + y^4 + xy(4x^2 + 6xy + 4y^2)
    \end{IEEEeqnarray*}
    Hence, $(x+y)^4=x^4+y^4$ only when $x=0$ or $y=0$,
    by part \ref{1.16:b}.
    \item Let us expand $(x+y)^5$.
    \begin{IEEEeqnarray*}{rCl}
      (x+y)^4(x+y) & = & x^5 + y^5 + xy(x+y)(4x^2 + 6xy + 4y^2)
      +xy(x^3 + y^3) \\
      & = & x^5 + y^5 + 5xy(x+y)(x^2 - xy + y^2)
    \end{IEEEeqnarray*}
    If $xy\neq0$ and $x+y\neq0$, let $z=-y$, by \autoref{1.6}%
    \ref{1.6:b}, $x^3\neq z^3$. Hence, $x^2 - xy + y^2\neq0$.
    Therefore, $(x+y)^5=x^5+y^5$ only when $x=0$ or $y=0$
    or $x=-y$.
    \begin{ab}
      Hence, for $(x+y)^n = x^n + y^n$, if $n$ is even,
      then $x=0$ or $y=0$. If $n$ is odd, then $x=0$ or
      $y=0$ or $x=-y$.
    \end{ab}
  \end{enumerate}
\end{solution}

\begin{pr} %Problem 1.17
  \begin{enumerate}[label=(\alph*)]
    \item
    Find the smallest possible value of $2x^2 - 3x + 4$.
    Hint: ``Complete the square'', i.e., write
    $2x^2 - 3x + 4 = 2(x-3/4)^2 + \text{?}$
    \item
    Find the smallest possible value of
    $x^2 - 3x + 2y^2 + 4y + 2$.
    \item
    Find the smallest possible value of
    $x^2 + 4xy + 5y^2 - 4x - 6y + 7$.
  \end{enumerate}
\end{pr}

\begin{solution} %Solution 1.17
  \begin{enumerate}[label=(\alph*)]
    \item Since $2x^2 - 3x + 4 = 2(x^2 - \dfrac{3}{2}x + 2)$,
    \begin{IEEEeqnarray*}{rCl}
      2(x^2 - 2\cdot x\frac{3}{4} + \frac{9}{16} - \frac{9}{16}
      + 2) & = & 2(x-\frac{3}{4})^2 + \frac{23}{8}
    \end{IEEEeqnarray*}
    Hence the minimum value is $\dfrac{23}{8}$ when
    $x=\dfrac{3}{4}$.
    \item \begin{equation*}
      x^2 - 3x + \frac{9}{4} - \frac{9}{4}
      + 2(y^2 + 2y + 1)
      = \left(x - \frac{3}{2}\right)^2 + 2(y + 1)^2
      - \frac{9}{4}
  \end{equation*}
  The minimum value is $-\dfrac{9}{4}$ when $x=\dfrac{3}{2}$
  and $y=-1$.
  \item
  \begin{IEEEeqnarray*}{rC}
    & \frac{1}{2}x^2 + 4xy + 8y^2 - 3y^2 - 6y + 7
    +\frac{1}{2}x^2 - 4x \\
    = & \frac{1}{2}(x^2 + 8xy + 16y^2) - 3(y^2 + 2y +1)
    + \frac{1}{2}(x^2 - 8x + 16) + 2 \\
    = & \frac{1}{2}(x + 4y)^2 - 3(y + 1)^2
    +\frac{1}{2}(x - 4)^2 + 2
  \end{IEEEeqnarray*}
  Therefore, the minimum value is $2$ when $x=4$ and
  $y=-1$.
  \end{enumerate}
\end{solution}


\begin{pr} \label{1.18}%Problem 1.18
  \begin{enumerate}[label=(\alph*)]
    \item Suppose that $b^2 - 4c \geq 0$. Show that
    the numbers
    \begin{align*}
      \frac{-b+\sqrt{b^2-4c}}{2},\quad
      \frac{-b-\sqrt{b^2-4c}}{2}
    \end{align*}
    both satisfy the equation $x^2 + bx + c = 0$.
    \item \label{1.18:b}
    Suppose that $b^2 - 4c < 0$. Show that there
    are no numbers $x$ satisfying $x^2 + bx + c = 0$;
    in fact, $x^2 + bx + c > 0$ for all $x$. Hint:
    Complete the square.
    \item Use this fact to give another proof that
    if $x$ and $y$ are not both $0$, then
    $x^2 + xy + y^2 > 0$.
    \item For which number $\alpha$ is it true that
    $x^2 + \alpha xy + y^2 > 0$ whenever $x$ and $y$
    are not both $0$?
    \item Find the smallest possible value of
    $x^2 + bx + c$ and of $ax^2 + bx + c$, for
    $a > 0$.
  \end{enumerate}
\end{pr}

\begin{solution}
  \begin{enumerate}[label=(\alph*)]
    \item Substitution immediately gives the desired result.
    \item
    \begin{equation*}
      x^2 + bx + c = x^2 + bx + \frac{b^2}{4}
      - \frac{b^2}{4} + c
    \end{equation*}
    which immediately yields
    $\left(x + \dfrac{b}{2}\right)^2
    + \dfrac{[-(b^2 - 4c)]}{4} > 0$ for all $x$ since
    $b^2 - 4c < 0$.
    \item If $y = 0$, $x^2>0$. Suppose not,
    using \ref{1.18:b}, we obtain
    $-3y^2 < 0$. Hence, $x^2 + xy + y^2 > 0$.
    \item If $y = 0$, the result follows for all
    $\alpha$. Suppose $y\neq0$,
    using \ref{1.18:b}, we obtain $\alpha^2y^2
    - 4y^2 < 0$, which is $y^2(\alpha^2 - 4) < 0$.
    It follows that $-2 < \alpha < 2$.
    \item From \ref{1.18:b}, it follows that
    the minimum value of $x^2 + bx + c$ is
    $\dfrac{[-(b^2 - 4c)]}{4}$ when $x = -b/2$.
    Since $a > 0$, with the role of $b$ is now
    $b/a$ and of $c$ is $c/a$, we easily
    derive the result.
    \begin{equation*}
      x^2 + \frac{b}{a}x + \frac{c}{a} =
      \left(x + \frac{b}{2a}\right)^2
      + \frac{[-(b^2 - 4ac)]}{4a^2}
    \end{equation*}
    So its minimum value is $\dfrac{[-(b^2 - 4ac)]}{4a^2}$
    when $x = -\dfrac{b}{2a}$.
  \end{enumerate}
\end{solution}

\begin{pr} \label{1.19}
  The fact that $a^2\geq0$ for all numbers $a$, elementary
  as it may seem, is nevertheless the fundamental idea
  upon which most important inequalities are ultimately
  based. The great-granddaddy of all inequalities is the
  \note{Schwarz inequality}:
  \begin{equation*}
    x_1 y_1 + x_2 y_2 \leq \sqrt{x_1^2+x_2^2}
    \sqrt{y_1^2+y_2^2}.
  \end{equation*}
  (A more general form occurs in Problem 2.21)\quad%ref{2.21}
  The three proofs of the Schwarz inequality outlined
  below have only one thing in common--their reliance
  on the fact that $a^2\geq0$ for all $a$.
  \begin{enumerate}[label=(\alph*)]
    \item \label{1.19:a}
    Prove that if $x_1 = \lambda y_1$ and
    $x_2 = \lambda y_2$ for some number $\lambda$, then
    equality holds in Schwarz inequality. Prove the same
    thing if $y_1 = y_2 = 0$. Now suppose that $y_1$ and
    $y_2$ are not both $0$, and that there is no number
    $\lambda$ such that $x_1 = \lambda y_1$ and
    $x_2 = \lambda y_2$. Then
    \begin{IEEEeqnarray*}{rCl}
      0 & < & (\lambda y_1 - x_1)^2 + (\lambda y_2 - x_2)^2 \\
        & = & \lambda^2(y_1^2 + y_2^2) - 2\lambda
        (x_1y_1 + x_2y_2) + (x_1^2 + x_2^2).
    \end{IEEEeqnarray*}
    Using \autoref{1.18}, complete the proof of the
    Schwarz inequality.
    \item \label{1.19:b}
    Prove the Schwarz inequality by using
    $2xy \leq x^2 + y^2$ (how is this derived?) with
    \begin{align*}
      x=\frac{x_i}{\sqrt{x_1^2+x_2^2}}, \quad
      y=\frac{y_i}{\sqrt{y_1^2+y_2^2}},
    \end{align*}
    first for $i = 1$ and then for $i = 2$.
    \item \label{1.19:c}
    Prove the Schwarz inequality by first proving
    that
    \begin{equation*}
      (x_1^2+x_2^2)(y_1^2+y_2^2)
      =(x_1y_1+x_2y_2)^2 + (x_1y_2 - x_2y_1)^2.
    \end{equation*}
    \item \label{1.19:d}
    Deduce, from each of these three proofs,
    that equality holds only when $y_1=y_2=0$ or when
    there is a number $\lambda$ such that $x_1=\lambda y_1$
    and $x_2=\lambda y_2$.
  \end{enumerate}
\end{pr}

\begin{solution} %Solution 1.19
  \begin{enumerate}[label=(\alph*)]
    \item If $x_1=\lambda y_1$ and $x_2=\lambda y_2$
    for every $\lambda \geq 0$,
    \begin{IEEEeqnarray*}{rCl}
      \lambda(y_1^2 + y_2^2) & = &
      |\lambda|\sqrt{(y_1^2 + y_2^2)^2} \\
      & = & \lambda  (y_1^2 + y_2^2)
    \end{IEEEeqnarray*}
    Or if $y_1=y_2=0$, then equality holds since both sides
    are $0$. Otherwise, suppose that $y_1$ and $y_2$ are
    not both $0$, and there is no number $\lambda$ such
    that $x_1=\lambda y_1$ and $x_2=\lambda y_2$, then
    \begin{IEEEeqnarray*}{rCl}
      0 & < & \lambda^2(y_1^2+y_2^2)
      - 2 \lambda(x_1y_1 + x_2y_2) + (x_1^2 + x_2^2) \\
        & = & \lambda^2 - 2 \lambda
        \frac{x_1y_1 + x_2y_2}{y_1^2 + y_2^2}
        + \frac{x_1^2 + x_2^2}{y_1^2 + y_2^2}
    \end{IEEEeqnarray*}
    This holds only when, by \autoref{1.18}\ref{1.18:b},
    \begin{equation*}
      \frac{4(x_1y_1 + x_2y_2)^2}{(y_1^2 + y_2^2)^2}
      + \frac{[-4(x_1^2 + x_2^2)(y_1^2 + y_2^2)]}%
      {(y_1^2 + y_2^2)^2} < 0
    \end{equation*}
    which only holds when
    \begin{equation*}
      x_1y_1 + x_2y_2 < \sqrt{x_1^2 + x_2^2}
      \sqrt{y_1^2 + y_2^2}
    \end{equation*}
    since $a\leq|a|$ for all $a$.
    \item Note that $(x-y)^2\geq0$. For $i = 1$,
    \begin{IEEEeqnarray}{rCl}
      \frac{x_1^2}{x_1^2 + x_2^2}
     +\frac{y_1^2}{y_1^2 + y_2^2} & \geq &
     2\cdot\frac{x_1 y_1}{\sqrt{x_1^2 + x_2^2}%
                          \sqrt{y_1^2 + y_2^2}} \label{1.19.1}
    \end{IEEEeqnarray}
    For $i = 2$,
    \begin{IEEEeqnarray}{rCl}
      \frac{x_2^2}{x_1^2 + x_2^2}
     +\frac{y_2^2}{y_1^2 + y_2^2} & \geq &
     2\cdot\frac{x_2 y_2}{\sqrt{x_1^2 + x_2^2}%
                          \sqrt{y_1^2 + y_2^2}} \label{1.19.2}
    \end{IEEEeqnarray}
    (\ref{1.19.1})$+$(\ref{1.19.2}), we derive
    \begin{equation*}
      x_1y_1 + x_2y_2 \leq \sqrt{x_1^2 + x_2^2}%
      \sqrt{y_1^2 + y_2^2}
    \end{equation*}
    \item
    \begin{IEEEeqnarray*}{lC}
        & (x_1^2 + x_2^2)(y_1^2 + y_2^2) \\
      = & (x_1^2 y_1^2 + 2 x_1 y_1 x_2 y_2 + x_2^2 y_2^2)
      + (x_1^2 y_2^2 - 2 x_1 y_2 x_2 y_1 + x_2^2 y_1^2) \\
      = & (x_1 y_1 + x_2 y_2)^2
      + (x_1 y_2 - x_2 y_1)^2
    \end{IEEEeqnarray*}
    Note that $(x_1 y_2 - x_2 y_1)^2 \geq 0$. Hence,
    \begin{equation*}
      x_1 y_1 + x_2 y_2 \leq \sqrt{x_1^2 + x_2^2}
                             \sqrt{y_1^2 + y_2^2}
    \end{equation*}
    since $a \leq |a|$ for all $a$.
    \item In \ref{1.19:a}, it is obvious; the proof is
    based on the separation of two cases, $a^2 = 0$ and
    $a^2 > 0$. In \ref{1.19:b}, equality occurs only when
    $x=y$; by construction, $y_1 = y_2 = 0$ or, if not,
    \begin{IEEEeqnarray*}{rCl}
      \frac{x_1}{\sqrt{x_1^2 + x_2^2}} & = &
      \frac{y_1}{\sqrt{y_1^2 + y_2^2}} \\
      \frac{x_2}{\sqrt{x_1^2 + x_2^2}} & = &
      \frac{y_2}{\sqrt{y_1^2 + y_2^2}}
    \end{IEEEeqnarray*}
    implies that for
    \begin{equation*}
      \lambda = \frac{\sqrt{x_1^2 + x_2^2}}
                     {\sqrt{y_1^2 + y_2^2}}
    \end{equation*}
    $x_1 = \lambda y_1$ and
    $x_2 = \lambda y_2$.
    \par
    In \ref{1.19:c}, equality occurs only when
    $(x_1 y_2 - x_2 y_1)^2 = 0$ and $x_1 y_1 + x_2 y_2 \geq 0$.
    These will be satisfied only when $y_1 = y_2 = 0$ or
    for $\lambda \geq 0$, $x_1 = \lambda y_1$ and
    $x_2 = \lambda y_2$.
  \end{enumerate}
\end{solution}

\begin{pr} %Problem 1.20
  Prove that if
  \begin{align*}
    |x - x_0| < \frac{\epsilon}{2} \quad \text{and} \quad
    |y - y_0| < \frac{\epsilon}{2},
  \end{align*}
  then
  \begin{IEEEeqnarray*}{rCl}
    |(x + y) - (x_0 + y_0)| & < & \epsilon,\\
    |(x - y) - (x_0 - y_0)| & < & \epsilon
  \end{IEEEeqnarray*}
\end{pr}

\begin{solution} %Solution to problem 1.20
  This problem mainly uses the results from \autoref{1.12}.
  For the first inequality, note that
  $|(x + y) - (x_0 + y_0)| = |(x - x_0) + (y - y_0)|$, and
  \begin{IEEEeqnarray*}{rCl}
    |(x - x_0) + (y - y_0)| & \leq &
    |x - x_0| + |y - y_0| \\
                            &  <  &
    \frac{\epsilon}{2} + \frac{\epsilon}{2} \\
                            &  =  &
    \epsilon
  \end{IEEEeqnarray*}
  For the second inequality, we rewrite
  $|(x - y) - (x_0 - y_0)| = |(x - x_0) - (y - y_0)|$,
  then
  \begin{IEEEeqnarray*}{rCl}
    |(x - x_0) - (y - y_0)| & \leq &
    |x - x_0| + |y - y_0| \\
                            &   <  &
    \frac{\epsilon}{2} + \frac{\epsilon}{2} \\
                            &   =  &
    \epsilon
  \end{IEEEeqnarray*}
\end{solution}

\begin{pr}[*] \label{1.21} % Problem 1.21
  Prove that if
  \begin{align*}
    |x - x_0| <
    \text{min}\left(\frac{\epsilon}{2(|y_0| + 1)},1\right)
    \quad \text{and} \quad
    |y - y_0| < \frac{\epsilon}{2(|x_0| + 1)},
  \end{align*}
  then $|xy - x_0 y_0| < \epsilon$.
\end{pr}

\begin{solution} % Solution to 1.21
  We want to utilize the more diverse cases of inequality
  expression in term of $x$, therefore
  we rewrite $|xy - x_0 y_0| =
  |xy - x y_0 + x y_0 - x_0 y_0|$. Hence,
  \begin{IEEEeqnarray*}{rCl}
    |xy - x y_0 + x y_0 - x_0 y_0| & \leq &
    |x||y - y_0| + |y_0||x - x_0| \\
                                   &  <   &
    (|x_0| + 1)\frac{\epsilon}{2(|x_0| + 1)}
    + |y_0|\frac{\epsilon}{2(|y_0| + 1)} \\
                                   &  <   &
    \frac{\epsilon}{2} + \frac{\epsilon}{2} \\
                                   &  =   &
    \epsilon
  \end{IEEEeqnarray*}
  where the first strict inequality follows from
  $|x| - |x_0| \leq |x - x_0|$ (\autoref{1.12}), and
  the second strict inequality comes from the fact
  $\dfrac{|y_0|}{|y_0| + 1} < 1$.
\end{solution}

\begin{pr}[*] \label{1.22} % Problem 1.22
  Prove that if $y_0 \neq 0$ and
  \begin{equation*}
    |y - y_0| < \text{min}\left(
    \frac{|y_0|}{2},
    \frac{\epsilon|y_0|^2}{2}
    \right),
  \end{equation*}
  then $y \neq 0$ and
  \begin{equation*}
    \left|
    \frac{1}{y} - \frac{1}{y_0}
    \right|                       < \epsilon.
  \end{equation*}
\end{pr}

\begin{solution} % Solution to problem 1.22
  Note that the assumption implies $|y|
  > \dfrac{|y_0|}{2} > 0$, which further implies
  $\dfrac{1}{|y|} < \dfrac{2}{|y_0|}$%
  ; therefore, it must be
  that $y \neq 0$. Note $\left|
  \dfrac{1}{y} - \dfrac{1}{y_0}
  \right| = \left|
  \dfrac{y - y_0}{y y_0}
  \right|$, and from that
  \begin{IEEEeqnarray*}{rCl}
    \left|
    \dfrac{y - y_0}{y y_0}
    \right|               & = &
    |y - y_0|\left|\frac{1}{y y_0}\right| \\
                          & < &
    \frac{\epsilon|y_0|^2}{2}\frac{2}{|y_0|^2} \\
                          & = &
    \epsilon
  \end{IEEEeqnarray*}
\end{solution}

\begin{pr}[*] %Problem 1.23
  Replace the question marks in the following statement
  by expressions involving $\epsilon, x_0,$ and $y_0$
  so that the conclusion will be true:

  \medskip
  If $y_0 \neq 0$ and
  \begin{align*}
    |y - y_0| < \text{?} \qquad \text{and} \qquad
    |x - x_0| < \text{?}
  \end{align*}
  then $y \neq 0$ and
  \begin{equation*}
    \left|
    \frac{x}{y} - \frac{x_0}{y_0}
    \right|
                                  < \epsilon
    .
  \end{equation*}
  This problem is trivial in the sense that its solution
  follows from \autoref{1.21} and \autoref{1.22} with
  almost no work at all (notice that $x/y=x\cdot 1/y)$).
  The crucial point is not to become confused; decide
  which of the two problems should be used first,
  and don't panic if your answer looks unlikely.
\end{pr}

\begin{solution} % Solution to 1.23
  An observation at both suggested related problems reveals
  \begin{IEEEeqnarray*}{rCl}
    |y - y_0| & < & \text{min}\left(
    \frac{|y_0|}{2}, \frac{\epsilon|y_0|^2}{4(|x_0|+1)}
    \right) \\
    |x - x_0| & < & \text{min}\left(
    \frac{\epsilon|y_0|}{2(|y_0| + 1)}, 1
    \right)
  \end{IEEEeqnarray*}
  Since $y_0 \neq 0$, we easily obtain $y\neq0$ by
  \autoref{1.22}. For the latter part of the proof,
  notice that
  \begin{IEEEeqnarray*}{rCl}
    \left|
    \frac{x}{y} - \frac{x_0}{y_0}
    \right| & = &
    \left|
    x(\frac{1}{y} - \frac{1}{y_0})
    +\frac{1}{y_0}(x - x_0)
    \right| \\
            & \leq &
    |x|\left|
    \frac{1}{y} - \frac{1}{y_0}
    \right|
    + \frac{1}{|y_0|}|x - x_0| \\
            &  <   &
    (|x_0| + 1)\frac{\epsilon}{2(|x_0| + 1)}
    +
    \frac{1}{|y_0|}\frac{\epsilon|y_0|}{2(|y_0| + 1)} \\
            &  <  &
    \frac{\epsilon}{2} + \frac{\epsilon}{2} \\
            &  =  &
    \epsilon
  \end{IEEEeqnarray*}
\end{solution}

\begin{pr}[*] \label{1.24} % Problem 1.24
  This problem shows that the actual placement of parentheses
  in a sum is irrelevant. The proof involve ``mathematical
  induction''; if you are not familiar with such proofs,
  but still want to tackle this problem, it can be saved
  until after Chapter 2, where proofs by induction are
  explained.
  \par
  \quad
  Let us agree, for definiteness, that
  $a_1 + \dots + a_n$ will denote
  \begin{equation*}
    a_1 + (a_2 + (a_3 + \dots +(a_{n-2}+(a_{n-1} + a_n)))
    \dots)
  \end{equation*}
  Thus $a_1 + a_2 + a_3$ denotes $a_1 + (a_2 + a_3)$,
  and $a_1 + a_2 + a_3 + a_4$ denotes
  $a_1 + (a_2 + (a_3 + a_4))$, etc.
  \begin{enumerate}[label=(\alph*)]
    \item \label{1.24:a}
    Prove that
    \begin{equation*}
      (a_1 + \dots + a_k) + a_{k+1} =
      a_1 + \dots + a_{k+1}.
    \end{equation*}
    Hint: Use induction on $k$.
    \item \label{1.24:b}
    Prove that if $n \geq k$, then
    \begin{equation*}
      (a_1 + \dots + a_k) + (a_{k+1} + \dots + a_n) =
      a_1 + \dots + a_n.
    \end{equation*}
    Hint: Use part \ref{1.24:a} to give proof
    by induction on $k$.
    \item \label{1.24:c}
    Let $s(a_1,\dots,a_k)$ be some sum formed from
    $a_1,\dots,a_k$. Show that
    \begin{equation*}
      s(a_1,\dots,a_k) = a_1 + \dots + a_k
    \end{equation*}
    Hint: There must be two sums $s'(a_1,\dots,a_l)$
    and $s''(a_{l+1},\dots,a_k)$ such that
    \begin{equation*}
      s(a_1,\dots,a_k) = s'(a_1,\dots,a_l)
      + s''(a_{l+1},\dots,a_k).
    \end{equation*}
  \end{enumerate}
\end{pr}

\begin{solution} % Solution to 1.24
  \begin{enumerate}[label=(\alph*)]
    \item If $k = 1$, there is nothing to prove. Suppose
    the equation holds for $k = l$, then for
    $k = l + 1$,
    \begin{IEEEeqnarray*}{rCl}
        & (a_1 + \dots + a_{l + 1}) + a_{l + 2} \\
      = & ((a_1 + \dots + a_l)+ a_{l + 1}) + a_{l + 2} \\
      = & a_1 + \dots + a_l + (a_{l + 1} + a_{l + 2})
      & \text{(since $(a+b)+c=a+(b+c)$)} \\
      = & a_1 + (a_2 + \dots + (a_{l - 1} + (a_l
      + (a_{l + 1} + a_{l + 2})))\dots) \\
      = & a_1 + \dots + a_{l + 1} + a_{l + 2}
    \end{IEEEeqnarray*}
    The proof is complete.
    \item If $n = k$, there is nothing to prove. Suppose
    the equation holds for $n \geq k$, then for
    $n \geq k + 1$,
    \begin{IEEEeqnarray*}{rCl}
       & (a_1 + \dots + a_{k + 1}) + (a_{k + 2} + \dots
       + a_n) \\
     = & ((a_1 + \dots + a_k) + a_{k + 1})
     + (a_{k + 2} + \dots + a_n)
       & \quad \text{(by \ref{1.24:a})} \\
     = & (a_1 + \dots + a_k) + (a_{k + 1} + \dots
     + a_n)   \\
     = & a_1 + \dots + a_n & \quad \text{(by assumption)}
    \end{IEEEeqnarray*}
    The proof is complete.
    \item By \ref{1.24:b}, there exists a number $l$
    such that
    \begin{equation*}
      s'(a_1,\dots,a_l) + s''(a_{l + 1},\dots,a_n)
      = a_1 + \dots + a_n
    \end{equation*}
    Hence, $s(a_1,\dots,a_n) = a_1 + \dots + a_n$.
  \end{enumerate}
\end{solution}

\begin{pr} % Problem 1.25
  Suppose that we interpret ``number'' to mean either
  $0$ or $1$, and $+$ and $\cdot$ to be the operations
  defined by the following two tables.

  \par
  \medskip
  \begin{tabular}{c|c|c|}
    \multicolumn{1}{c}{$+$}  & \multicolumn{1}{c}{$0$} &
    \multicolumn{1}{c}{$1$} \\
    \cline{2-3}
    \multicolumn{1}{c|}{$0$}  & $0$ & $1$ \\
    \cline{2-3}
    \multicolumn{1}{c|}{$1$}  & $1$ & $0$ \\
    \cline{2-3}
  \end{tabular}
  \hspace{35pt}
  \begin{tabular}{c|c|c|}
    \multicolumn{1}{c}{$\cdot$}  & \multicolumn{1}{c}{$0$} &
    \multicolumn{1}{c}{$1$} \\
    \cline{2-3}
    \multicolumn{1}{c|}{$0$}  & $0$ & $0$ \\
    \cline{2-3}
    \multicolumn{1}{c|}{$1$}  & $0$ & $1$ \\
    \cline{2-3}
  \end{tabular}

  \par
  \medskip
  Check that properties P1-P9 all hold, even though
  $1 + 1 = 0$.
\end{pr}

\begin{solution}
  It is easy to check!
\end{solution}

    \chapter{Numbers of Various Sorts}
\begin{pr} \label{2.1}% 2.1
  Prove the following formula by induction.
  \begin{enumerate}[label=(\roman*)]
    \item \label{2.1:i}
    $1^2+\dots+n^2=\dfrac{n(n+1)(2n+1)}{6}$.
    \item $1^3+\dots+n^3=(1+\dots+n)^2$.
  \end{enumerate}
\end{pr}

\begin{solution} % s2.1
  \begin{enumerate}[label=(\roman*)]
    \item If $n=1$, the equation holds. Suppose the
    equation holds for $n=k$, then for $n=k+1$,
    \begin{IEEEeqnarray*}{lC}
      & 1^2+\dots+k^2+(k+1)^2 \\
    = & \frac{k(k+1)(2k+1)}{6} + (k+1)^2 \\
    = & \frac{(k+1)[k(2k+1)+6(k+1)]}{6}  \\
    = & \frac{(k+1)(2k^2+7k+6)}{6}       \\
    = & \frac{(k+1)(2k^2+4k+3k+6)}{6}    \\
    = & \frac{(k+1)[2k(k+2)+3(k+2)]}{6}  \\
    = & \frac{(k+1)[(k+1)+1][2(k+1)+1]}{6}
    \end{IEEEeqnarray*}
    Then the formula holds for every $n$.
    \item If $n=1$, there is nothing to prove. If
    $n=k$ holds for the equation, then for $n=k+1$,
    \begin{IEEEeqnarray*}{lC}
      & [1+\dots+k+(k+1)]^2 \\
    = & (1+\dots+k)^2 + (k+1)^2 + 2(k+1)(1+\dots+k) \\
    = & 1^3+\dots+k^3 + (k+1)[(k+1)+2\frac{k(k+1)}{2}] \\
    = & 1^3+\dots+k^3 + (k+1)^3
    \end{IEEEeqnarray*}
    This finishes the proof for every $n$.
  \end{enumerate}
\end{solution}

\begin{pr} % 2.2
  Find a formula for
  \begin{enumerate}[label=(\roman*)]
    \item
    $\displaystyle\sum_{i=1}^n (2i-1) = 1+3+5+\dots+(2n-1)$.
    \item $\displaystyle\sum_{i=1}^n (2i-1)^2
    = 1^2 + 3^2 + 5^2 + \dots + (2n - 1)^2$.
  \end{enumerate}
  Hint: What do these expressions have to do with
  $1 + 2 + 3 + \dots + 2n$ and $1^2 + 2^2 + 3^2 + \dots
  + (2n)^2$?
\end{pr}

\begin{solution} % s2.2
  \begin{enumerate}[label=(\roman*)]
    \item Remind that $1+\dots+2n=n(2n+1)$ and
    that $2 + \dots + 2n = 2 \cdot \dfrac{n(n+1)}{2} = n(n+1)$.
    Hence,
    \begin{equation*}
      \sum_{i=1}^n (2i - 1) = n(2n+1) - n(n+1) = n^2
    \end{equation*}
    \item Using \autoref{2.1}\ref{2.1:i}, we easily
    derive that $\displaystyle
    \sum_{i=1}^{2n} i^2 = \dfrac{n(2n+1)(4n+1)}{3}$
    and $\displaystyle
    \sum_{i=1}^n (2i)^2 = \dfrac{2n(n+1)(2n+1)}{3}$.
    Therefore,
    \begin{equation*}
      \sum_{i=1}^n (2i - 1)^2 = \sum_{i=1}^{2n} i^2
      - \sum_{i=1}^n (2i)^2
      = \frac{n(4n^2 - 1)}{3}
    \end{equation*}
  \end{enumerate}
\end{solution}

\begin{pr} \label{2.3} % 2.3
  If $0 \leq k \leq n$, the ``binomial coefficient''
  $\displaystyle\binom{n}{k}$ is defined by
  $\displaystyle\binom{n}{k} = \dfrac{n!}{k!(n-k)!}
  = \dfrac{n(n-1)\cdots(n-k+1)}{k!}$, if $k\neq0,n$ \\
  $\displaystyle\binom{n}{0} = \binom{n}{n} = 1$.
  (This becomes a special case of the first formula
  if we define $0! = 1$.)
  \begin{enumerate}[label=(\alph*)]
    \item \label{2.3:a}
    Prove that
    \begin{equation*}
      \binom{n+1}{k} = \binom{n}{k-1} + \binom{n}{k}.
    \end{equation*}
    (The proof does not require an induction argument.)

    \medskip
    This relation gives rise to the following configuration,
    known as ``Pascal's triangle''---a number not on one of the
    sides is the sum of two numbers above it; the binomial
    coefficient $\displaystyle\binom{n}{k}$ is the
    $(k+1)$st number in the $(n+1)$st row.
    \par
    \begin{center} \label{PascalTriangle}
      \begin{tikzpicture}
        \foreach \n in {0,...,5}{
          \foreach \k in {0,...,\n}{
            \node at (\k - \n/2, -\n){$\binomialCoefficient{\n}{\k}$};
          }
        }
      \end{tikzpicture}
    \end{center}
    \item Notice that all numbers in Pascal's triangle are
    natural numbers. Use part \ref{2.3:a} to prove by
    induction that $\displaystyle\binom{n}{k}$ is always
    a natural number. (Your entire proof by induction will,
    in a sense, be summed up in a glance by Pascal's
    triangle.)
    \item Give another proof that $\displaystyle\binom{n}{k}$
    is a natural number by showing that
    $\displaystyle\binom{n}{k}$ is the number of sets of
    exactly $k$ integers each chosen from $1,\dots,n$.
    \item Prove the ``binomial theorem'': If $a$ and $b$
    are any numbers and $n$ is a natural number, then
    \begin{IEEEeqnarray*}{rCl}
      (a+b)^n & = & a^n + \binom{n}{1}a^{n-1}b
      + \binom{n}{2}a^{n-2}b^2+\cdots+\binom{n}{n-1}ab^{n-1}
      + b^n \\
              & = & \sum_{j=0}^n \binom{n}{j}a^{n-j}b^j.
    \end{IEEEeqnarray*}
    \item Prove that
    \begin{enumerate}[label=(\roman*)]
      \item \label{2.3:ei}
      $\displaystyle
      \sum_{j=0}^n \binom{n}{j}
      = \binom{n}{0} + \cdots + \binom{n}{n}
      = 2^n$.
      \item \label{2.3:eii}
      $\displaystyle
      \sum_{j=0}^n (-1)^j \binom{n}{j}
      = \binom{n}{0} - \binom{n}{1} + \cdots \pm \binom{n}{n}
      = 0$.
      \item $\displaystyle
      \sum_{l\text{ odd}} \binom{n}{l}
      = \binom{n}{1} + \binom{n}{3} + \cdots
      = 2^{n - 1}$.
      \item $\displaystyle
      \sum_{l\text{ even}} \binom{n}{l}
      = \binom{n}{0} + \binom{n}{2} + \cdots
      = 2^{n - 1}$.
    \end{enumerate}
  \end{enumerate}
\end{pr}

\begin{solution} % s2.3
  \begin{enumerate}[label=(\alph*)]
    \item Starting from the left-hand side,
    \begin{IEEEeqnarray*}{rCl}
      \binom{n + 1}{k} & = &
      \frac{n!(n - k + 1 + k)}{k!(n - k + 1)!} \\
                       & = &
      \frac{n!(n - k + 1)}{k!(n - k + 1)!}
    + \frac{n!k}{k!(n - k + 1)!}\\
                       & = &
      \frac{n!}{k!(n - k)!}
    + \frac{n!}{(k - 1)![n - (k - 1)]!} \\
                       & = &
      \binom{n}{k} + \binom{n}{k - 1}
    \end{IEEEeqnarray*}
    \item It is sufficient to prove that $\displaystyle
    \binom{n}{k}$ is a natural number for all
    $1 \leq k \leq (n - 1)$. If $n = 1$, then
    \begin{equation*}
      \binom{2}{1} = \binom{1}{0} + \binom{1}{1} = 2
    \end{equation*}
    Suppose that $\displaystyle\binom{n}{k}$ is natural
    number for any $n$ and $1 \leq k \leq n - 1$.
    Then for any $1 \leq k \leq n$, $\displaystyle
    \binom{n + 1}{k}$ is the sum of two natural numbers,
    and therefore it must be a natural number.
    \item It is sufficient to prove for the case
    $0 < k \leq n$.
    If $n=1$, the claim is trivial. Suppose that
    $\displaystyle\binom{n}{k}$ is the number of sets
    of $k$ integers each chosen from $1,\dots,n$; then
    $\displaystyle\binom{n + 1}{k}$ must include
    $\displaystyle\binom{n}{k}$ sets of $k$ integers
    \note{without} the newly added element and a number
    of sets of $k$ integers \note{with} the newly added
    element. The latter is exactly $\displaystyle
    \binom{n}{k - 1}$ and is a natural number by assumption.
    Thereby, $\displaystyle\binom{n+1}{k}$ must be
    a natural number.
    \item We prove by induction on $n$. If $n=1$, there
    is nothing to prove. Suppose the binomial theorem holds
    for $n$; then for $n+1$,
    \begin{IEEEeqnarray*}{rCl}
      (a+b)^n(a+b) & = &
      \sum_{j=0}^n \binom{n}{j} a^{n-j}b^j(a + b) \\
                   & = &
      \sum_{j=0}^n \binom{n}{j} a^{n+1-j}b^j
    + \sum_{j=0}^n \binom{n}{j} a^{n-j}b^{j+1}    \\
                   & = &
      a^{n + 1} + b^{n + 1}
    + \sum_{j=1}^n \binom{n}{j} a^{n+1-j}b^j      \\
                   & + &
      \sum_{j=1}^n \binom{n}{j-1} a^{n+1-j}b^j    \\
                   & = &
      \sum_{j=1}^n \binom{n+1}{j} a^{n+1-j}b^j
    + a^{n+1} + b^{n+1}                           \\
                   & = &
      \sum_{j=0}^{n+1} \binom{n+1}{j} a^{n+1-j}b^j
    \end{IEEEeqnarray*}
    which completes the proof.
    \item This part relies heavily on
    the binomial theorem from the above.
    \begin{enumerate}[label=(\roman*)]
      \item This is directly from the above: Applying
      the binomial theorem for $a=b=1$ yields the result.
      \item Let $a=1$ and $b=-1$ yield the result.
      \item Applying \ref{2.3:ei} $+$ \ref{2.3:eii},
      we derive that for $l$ even,
      \begin{equation*}
        \sum_{l\text{ even}} \binom{n}{l} = 2^{n - 1}
      \end{equation*}
      Thereby, $\displaystyle
      \sum_{l\text{ odd}} \binom{n}{l} = 2^n - 2^{n - 1}
      = 2^{n - 1}$.
      \item See the above.
    \end{enumerate}
    One thing to note: Both of the previous parts do not
    have a final term expressed in their sum due to the
    dependence of value $n$ (if $n$ is even or odd).
  \end{enumerate}
\end{solution}

\begin{pr} \label{2.4} % 2.4
  \begin{enumerate}[label=(\alph*)]
    \item \label{2.4:a}
    Prove that
    \begin{equation*}
      \sum_{k=0}^l \binom{n}{k} \binom{m}{l-k}
      = \binom{n+m}{l}.
    \end{equation*}
    Hint: Apply the binomial theorem to $(1+x)^n(1+x)^m$.
    \item \label{2.4:b}
    Prove that
    \begin{equation*}
      \sum_{k=0}^n \binom{n}{k}^2 = \binom{2n}{n}.
    \end{equation*}
  \end{enumerate}
\end{pr}

\begin{solution} % 2.4 solved
  \begin{enumerate}[label=(\alph*)]
    \item Remind that $(1+x)^n(1+x)^m = (1+x)^{n+m}$. Hence,
    \begin{equation*}
      \sum_{k=0}^n \binom{n}{k}x^k \cdot
      \sum_{j=0}^m \binom{m}{j}x^j
      = \sum_{l=0}^{n+m} \binom{n+m}{l}x^l
    \end{equation*}
    Observing that each term of $x^l$ is
    \begin{equation*}
      \sum_{k=0}^l \binom{n}{k}\binom{m}{l-k}
    \end{equation*}
    for every $k$ and $j$ such that $j = l - k$.
    \item From \ref{2.4:a}, let $m = l = n$ and observe
    that for all $0\leq k\leq n$,
    \begin{equation*}
      \binom{n}{k} = \binom{n}{n - k}
    \end{equation*}
    since $\dfrac{n!}{k!(n-k)!} = \dfrac{n!}{(n-k)!k!}$.
    Hence,
    $\displaystyle\sum_{k=0}^n \binom{n}{k}^2 = \binom{2n}{n}$.
  \end{enumerate}
\end{solution}

\begin{pr} \label{2.5} % 2.5
  \begin{enumerate}[label=(\alph*)]
    \item \label{2.5:a}
    Prove by induction on $n$ that
    \begin{equation*}
      1 + r + r^2 + \cdots + r^n = \frac{1-r^{n+1}}{1-r}
    \end{equation*}
    if $r\neq1$ (if $r=1$, evaluating the sum certainly
    presents no problem).
    \item \label{2.5:b}
    Derive this result by setting $S = 1 + r + \cdots + r^n$,
    multiplying this equation by $r$, and solving the
    two equations for $S$.
  \end{enumerate}
\end{pr}

\begin{solution} % 2.5 solved
  \begin{enumerate}[label=(\alph*)]
    \item If $n=1$, then the formula immediately holds. Suppose
    now that it holds for $n$, then for $n+1$,
    \begin{IEEEeqnarray*}{rCl}
      1 + r + \cdots + r^n + r^{n+1}
    & = & \frac{1 - r^{n+1}}{1-r} + r^{n+1} \\
    & = & \frac{1 - r^{n + 2}}{1 - r}
  \end{IEEEeqnarray*}
  which completes the proof.
    \item Let
    \begin{equation*}
      S = 1 + r + r^2 + \cdots + r^n
    \end{equation*}
    then
    \begin{equation*}
      rS = r + r^2 + r^3 + \cdots + r^{n+1}
    \end{equation*}
    Solving for $S$, we easily obtain
    \begin{equation*}
      S = \frac{1 - r^{n+1}}{1 - r}
    \end{equation*}
  \end{enumerate}
\end{solution}

\begin{pr} \label{2.6} % 2.6
  The formula for $1^2 + \cdots + n^2$ may be derived as
  follows. We begin with the formula
  \begin{equation*}
    (k+1)^3 - k^3 = 3k^2 + 3k + 1.
  \end{equation*}
  Writing this formula for $k=1,\dots,n$ and adding,
  we obtain

  \bigskip
  \begin{IEEEeqnarray*}{rCl}
    2^3 - 1^3 & = & 3 \cdot 1^2 + 3 \cdot 1 + 1 \\
    3^3 - 2^3 & = & 3 \cdot 2^2 + 3 \cdot 2 + 1 \\
    \vdots                                      \\
    (n + 1)^3 - n^3 & = & 3 \cdot n^2 + 3 \cdot n + 1 \\
    \hline                                            \\
    (n + 1)^3 - 1 & = & 3[1^2 + \cdots + n^2]
    + 3[1 + \cdots + n] + n.
  \end{IEEEeqnarray*}

  \bigskip
  Thus we can find
  $\displaystyle \sum_{k=1}^n k^2$ if we already know
  $\displaystyle \sum_{k=1}^n k$ (which could have been
  found in a similar way). Use this method to find
  \begin{enumerate}[label=(\roman*)]
    \item $1^3 + \cdots + n^3$.
    \item $1^4 + \cdots + n^4$.
    \item $\displaystyle
    \frac{1}{1\cdot2} + \frac{1}{2\cdot3} + \cdots
    + \frac{1}{n(n + 1)}$.
    \item $\displaystyle
    \frac{3}{1^2\cdot2^2} + \frac{5}{2^2\cdot3^2} + \cdots
    + \frac{2n + 1}{n^2(n + 1)^2}$.
  \end{enumerate}
\end{pr}

\begin{solution} % 2.6 solved
  \begin{enumerate}[label=(\roman*)]
    \item We easily derive that
    \begin{equation*}
      (k+1)^4 - k^4 = 4k^3 + 6k^2 + 4k + 1
    \end{equation*}
    Hence,
    \begin{IEEEeqnarray*}{rCl}
      2^4 - 1^4 &=& 4\cdot1^3 + 6\cdot1^2 + 4\cdot1 + 1 \\
      3^4 - 2^4 &=& 4\cdot2^3 + 6\cdot2^2 + 4\cdot2 + 1 \\
      \vdots                                            \\
      (n + 1)^4 - n^4 &=& 4\cdot n^3 + 6 \cdot n^2
      + 4 \cdot n + 1                                   \\
      \hline                                            \\
      (n + 1)^4 - 1 &=& 4[1^3 + 2^3 + \cdots + n^3]     \\
      & + & 6[1^2 + 2^2 + \cdots + n^2] + 4[1 + 2 + \cdots + n]
      + n
    \end{IEEEeqnarray*}
    It is easily derivable that
    \begin{equation*}
      1^3 + \cdots + n^3 = \frac{1}{4}n^2(n + 1)^2
    \end{equation*}
    \item The easiest way to calculate
    $(k + 1)^5 - k^5$ is to use the Pascal triangle from
    page \pageref{PascalTriangle}: We derive the result
    \begin{equation*}
      (k + 1)^5 - k^5 = 5k^4 + 10k^3 + 10k^2 + 5k + 1
    \end{equation*}
    A similar step from above shows that
    \begin{IEEEeqnarray*}{rCl}
      (n + 1)^5 - 1 & = & 5[1^4 + \cdots + n^4]
      + 10[1^3 + \cdots + n^3] \\
                      & + & 10[1^2 + \cdots + n^2]
      + 5[1 + \cdots + n] + n
    \end{IEEEeqnarray*}
    We derive from the available results,
    \begin{equation*}
      1^4 + 2^4 + \cdots + n^4 = \frac{1}{30}
      n(n+1)(2n+1)(3n^2 + 3n - 1)
    \end{equation*}
    \item Observe that
    \begin{equation*}
      \frac{1}{k} - \frac{1}{k + 1} = \frac{1}{k(k + 1)}
    \end{equation*}
    Henceforth,
    \begin{IEEEeqnarray*}{rCl}
      1 - \frac{1}{2} &=& \frac{1}{1\cdot2} \\
      \frac{1}{2} - \frac{1}{3} &=& \frac{1}{2\cdot3} \\
      \vdots                                          \\
      \frac{1}{n} - \frac{1}{n + 1}
      &=& \frac{1}{n\cdot(n+1)}                       \\
      \hline                                          \\
      1 - \frac{1}{n + 1} & = &
      \frac{1}{1\cdot2} + \frac{1}{2\cdot3} + \cdots
      + \frac{1}{n(n + 1)}                            \\
      \frac{1}{1\cdot2} + \frac{1}{2\cdot3} + \cdots
      + \frac{1}{n(n + 1)}& = & \frac{n}{n + 1}
    \end{IEEEeqnarray*}
    \item See that
    \begin{equation*}
      \frac{1}{k^2} - \frac{1}{(k+1)^2}
      = \frac{2k + 1}{k^2(k + 1)^2}
    \end{equation*}
    Henceforth,
    \begin{IEEEeqnarray*}{rCl}
      1 - \frac{1}{2^2} &=& \frac{3}{1^2\cdot2^2} \\
      \frac{1}{2^2} - \frac{1}{3^2} &=& \frac{5}{2^2\cdot3^2} \\
      \vdots \\
      \frac{1}{n^2} - \frac{1}{(n+1)^2} &=&
      \frac{2n + 1}{n^2(n + 1)^2}                  \\
      \hline                                       \\
      \frac{3}{1^2\cdot2^2} + \frac{5}{2^2\cdot3^2}
      + \cdots + \frac{2n + 1}{n^2(n + 1)^2}
      & = & \frac{n(n+2)}{(n+1)^2}
    \end{IEEEeqnarray*}
  \end{enumerate}
\end{solution}

\begin{pr} \label{2.7}% 2.7
  Use the method of \autoref{2.6} to show that $\displaystyle
  \sum_{k=1}^n k^p$ can always be written in the form
  \begin{equation*}
    \frac{n^{p+1}}{p+1} + An^{p} + Bn^{p-1}
    + Cn^{p-2} + \cdots .
  \end{equation*}
  (The first 10 such expressions are
  \begin{IEEEeqnarray*}{rCCCCCCCCCCCCCCl}
    \sum_{k=1}^n k & = &
    \frac{1}{2}n^2 &+& \frac{1}{2}n \\
    \sum_{k=1}^n k^2 &=&
    \frac{1}{3}n^3 &+& \frac{1}{2}n^2
    &+& \frac{1}{6}n                                     \\
    \sum_{k=1}^n k^3 &=&
    \frac{1}{4}n^4 &+& \frac{1}{2}n^3
    &+& \frac{1}{4}n^2                                   \\
    \sum_{k=1}^n k^4 &=&
    \frac{1}{5}n^5 &+& \frac{1}{2}n^4
    &+& \frac{1}{3}n^3 &-& \frac{1}{30}n                   \\
    \sum_{k=1}^n k^5 &=&
    \frac{1}{6}n^6 &+& \frac{1}{2}n^5
    &+& \frac{5}{12}n^4 &-& \frac{1}{12}n^2                \\
    \sum_{k=1}^n k^6 &=&
    \frac{1}{7}n^7 &+& \frac{1}{2}n^6
    &+& \frac{1}{2}n^5 &-& \frac{1}{6}n^3 &+& \frac{1}{42}n  \\
    \sum_{k=1}^n k^7 &=&
    \frac{1}{8}n^8 &+& \frac{1}{2}n^7
    &+& \frac{7}{12}n^6 &-& \frac{7}{24}n^4
    &+& \frac{1}{12}n^2 \\
    \sum_{k=1}^n k^8 &=&
    \frac{1}{9}n^9 &+& \frac{1}{2}n^8
    &+& \frac{2}{3}n^7 &-& \frac{7}{15}n^5 &+& \frac{2}{9}n^3
    &-& \frac{1}{30}n                                       \\
    \sum_{k=1}^n k^9 &=&
    \frac{1}{10}n^{10}
    &+& \frac{1}{2}n^9 &+& \frac{3}{4}n^8
    &-& \frac{7}{10}n^6 &+& \frac{1}{2}n^4
    &-& \frac{3}{20}n^2  \\
    \sum_{k=1}^n k^{10} &=&
    \frac{1}{11}n^{11} &+& \frac{1}{2}n^{10}
    &+& \frac{5}{6}n^9 &-& 1n^7 &+& 1n^5 &-& \frac{1}{2}n^3
    &+& \frac{5}{66}n.
  \end{IEEEeqnarray*}
  Notice that the coefficients in the second column are
  always $\dfrac{1}{2}$, and that after the third column
  the powers of $n$ with nonzero coefficients decrease
  by $2$ until $n$ or $n^2$ is reached. The coefficients
  in all but the first two columns seem to be rather
  haphazard, but there is actually is some sort of pattern;
  finding it may be regarded as a super-perspicacity test.
  See Problem 27.17 for the complete story.)
\end{pr}

\begin{solution} % 2.7 solved
  We prove by complete induction on $p$. Know that
  \begin{equation*}
    (k+1)^{p+2} - k^{p+2}
  = \sum_{j=2}^{p} \binom{p+2}{j}k^j
  + (p+2)k^{p+2} + (p+2)k + 1
  \end{equation*}
  If $p=1$, it is obvious. Suppose the expression holds
  for $1,\dots,p$, then for $p+1$,
  using the known method presented above, we easily obtain
  \begin{equation*}
    (n+1)^{p+2} - 1
  = \sum_{j=2}^{p} \binom{p+2}{j} \sum_{k=1}^n k^j
  + (p+2)\sum_{k=1}^n k^{p+1} + (p+2)\sum_{k=1}^n k + n
  \end{equation*}
  Notice that the left-hand side highest power of $n$ is
  $p+2$ while that on the right is $p+1$ by assumption.
  Dividing both sides by $p+2$ and solve for
  $\sum_{k=1}^n k^{p+1}$, we have
  \begin{equation*}
    \sum_{k=1}^n k^{p+1}
  = \frac{n^{p+2}}{p+2} + An^{p+1} + Bn^p + \cdots
  \end{equation*}
  for some number $A$, $B$.
\end{solution}

\begin{pr} % 2.8
  Prove that every natural number is either even or odd.
\end{pr}

\begin{solution} % 2.8 solved
  Suppose that $B$
  is the set of all natural numbers that is neither even
  nor odd. Suppose that $B\neq\emptyset$. Obviously,
  $1\notin B$ since $1$ is an odd number. If $k \notin B$,
  then $k$ is either even or odd. Then $k + 1$ is either
  odd or even, and therefore $(k+1) \notin B$. Henceforth,
  $B = \emptyset$, contradicting assumption.
\end{solution}

\begin{pr} % 2.9
  Prove that if a set $A$ of natural numbers contains $n_0$
  and contains $k+1$ whenever it contains $k$, then $A$
  contains all natural numbers $\geq n_0$.
\end{pr}

\begin{solution} % 2.9 solved
  Obivously, $n_0 \in A$. Suppose that $(n_0 + k - 1) \in A$.
  By assumption, $(n_0 + k) \in A$. Hence, $A$ contains all
  natural numbers $\geq n_0$.
\end{solution}

\begin{pr} % 2.10
  Prove the principle of mathematical induction from the
  well-ordering principle.
\end{pr}

\begin{solution} \label{2.10} % 2.10 solved
  We shall prove the theorem by contradiction,
  \begin{thm}[Principle of mathematical induction]
    If $A$ is the set of natural numbers and
    \begin{enumerate}[label=(\arabic*)]
      \item \label{2.10:1}
      $1$ is in $A$
      \item \label{2.10:2}
      $k+1$ is in $A$ wherever $k$ is in $A$,
    \end{enumerate}
    then $A$ is the set of all natural numbers.
  \end{thm}
  Let $B\neq\emptyset$
  be the set of all natural numbers \note{not}
  in $A$. By the well-ordering principle, $B$ has a least
  member $k$. By construction, $k \notin A$. By property
  \ref{2.10:2}, $(k - 1) \notin A$, which means that
  $(k - 1) \in B$, but this contradicts the fact that
  $k$ is the least member in $B$.
\end{solution}

\begin{pr} % 2.11
  Prove the principle of complete induction from the ordinary
  principle of induction. Hint: If $A$ contains $1$ and $A$
  contains $n + 1$ whenever it contains $1,\dots,n$,
  consider the set $B$ of all $k$ such that $1,\dots,k$
  are all in $A$.
\end{pr}

\begin{solution} % 2.11 solved
  We want to prove the following theorem,
  \begin{thm}[Principle of complete induction]
    If $A$ is the set of natural numbers and
    \begin{enumerate}[label=(\arabic*)]
      \item $1$ is in $A$
      \item $n+1$ is in $A$ if $1,\dots,n$ is in $A$
    \end{enumerate}
    then $A$ is the set of all natural numbers.
  \end{thm}
  Let $B$ be the set of all $k$ such that
  $1,2,3,\dots,k$ are all in $A$. Obviously, $B \subseteq A$.
  Conversely, $A \subseteq B$ because if not,
  then $1,\dots,k$ are not all in $A$ which implies
  $1 \notin A$.
  Hence,
  $A = B\neq\emptyset$
  since $1\in A$. We see that $B$ satisfies:
  \begin{enumerate}[label=(\arabic*)]
    \item $1$ is in $B$
    \item $k+1$ is in $B$ whenever $k$ is in $B$. Otherwise,
    $(k+1) \notin A$, which implies $1,\dots,k$
    are not in $A$,
    which implies $k \notin B$.
  \end{enumerate}
  By principle of mathematical induction, $B$ is the set of
  all natural numbers, which means $A$ is the same.
\end{solution}

\begin{pr} \label{2.12}% 2.12
  \begin{enumerate}[label=(\alph*)]
    \item If $a$ is rational and $b$ is irrational, is
    $a + b$ necessarily irrational? What if $a$ and $b$
    are both irrational?
    \item If $a$ is rational and $b$ is irrational,
    is $ab$ necessarily irrational? (Careful!)
    \item Is there a number $a$ such that $a^2$ is irrational,
    but $a^4$ is rational?
    \item Are there two irrational numbers whose sum and
    product are both rational?
  \end{enumerate}
\end{pr}

\begin{solution} % 2.12 solved
  \begin{enumerate}[label=(\alph*)]
    \item If $a+b$ were rational then
    $b = (a+b) - a$ would be rational! Hence,
    $a+b$ is necessarily irrational. However,
    if $a$ and $b$ are irrational, then let $b = r - a$ for
    any rational $r$, then $a+b$ is rational.
    \item $ab$ is not necessarily irrational for the case
    when $a = 0$ and $b = \sqrt{2}$. However, if $a \neq 0$,
    then if $ab$ were rational, then $\dfrac{ab}{a} = b$
    would be rational. Hence, $ab$ must be irrational.
    \item Let $a = \sqrt{\sqrt{2}}$: $a^2 = \sqrt{2}$, but
    $a^4 = 2$.
    \item Let $a = -\sqrt{2}$ and $b = \sqrt{2}$. Then
    both $a+b$ and $ab$ are rational even though $a$
    and $b$ are irrational.
  \end{enumerate}
\end{solution}

\begin{pr} \label{2.13} % 2.13
  \begin{enumerate}[label=(\alph*)]
    \item Prove that $\sqrt{3},\sqrt{5},$ and $\sqrt{6}$ are
    irrational. Hint: To treat $\sqrt{3}$, for example, use
    the fact that every integer is of the form $3n$ or
    $3n + 1$ or $3n + 2$. Why does this proof not work for
    $\sqrt{4}$?
    \item Prove that $\sqrt[3]{2}$ and $\sqrt[3]{3}$ are
    irrational.
  \end{enumerate}
\end{pr}

\begin{solution} % 2.13 solved
  \begin{enumerate}[label=(\alph*)]
    \item Since every number is written in the form of
    either $3n$, $3n+1$ or $3n+2$ for some natural $n$, then
    \begin{IEEEeqnarray*}{rClCl}
      (3n+1)^2  &=& 9n^2+6n+1 &=& 3(3n^2 + 2n) + 1 \\
      (3n+2)^2  &=& 9n^2+12n+4 &=& 3(3n^2 + 4n + 1) + 1
    \end{IEEEeqnarray*}
    This implies that if a squared integer is divisible by $3$,
    then the integer is divisible by $3$.
    Suppose now that $\sqrt{3}$ were rational. Then there
    would be
    a pair of integer $p$ and $q$ with no common divisor such
    that
    \begin{equation*}
      \sqrt{3} = \frac{p}{q}
    \end{equation*}
    This implies that $p^2 = 3 q^2$. Therefore, there exist
    a natural $k$ such that $p = 3 k$, which implies
    \begin{equation*}
      p^2=9k^2=3q^2
    \end{equation*}
    which means that $q^2 = 3 k^2$. Henceforth, $q = 3m$
    for some natural $m$, but these imply that $p$ and $q$
    have a common divisor: A contradiction.
    Similar for $\sqrt{5}$, we consider
    \begin{IEEEeqnarray*}{rClCl}
      (5n+1)^2 &=& 25n^2 + 10n + 1 &=& 5(5n^2 + 2n) + 1\\
      (5n+2)^2 &=& 25n^2 + 20n + 4 &=& 5(5n^2 + 4n) + 4\\
      (5n+3)^2 &=& 25n^2 + 30n + 9 &=& 5(5n^2 + 6n + 1) + 4\\
      (5n+4)^2 &=& 25n^2 + 40n + 16 &=&
      5(5n^2 + 8n + 3) + 1
    \end{IEEEeqnarray*}
    We see that the very same method of proof is applicable
    for $\sqrt{5}$ and $\sqrt{6}$: If $k^2$  is divisible
    by either $5$ or $6$, then $k$ must be divisible by
    $5$ or $6$, respectively. Hence, we easily conclude
    that $\sqrt{5}$ and $\sqrt{6}$ are irrational.
    \par
    Now we cannot use this method of proof for $\sqrt{4}$
    since the statement \note{if $k^2$ is divisible by $4$,
    then $k$ is divisible by $4$} is false by letting $k=2$.
    \item Let us first consider
    \begin{IEEEeqnarray*}{rClCl}
      (2n + 1)^3 &=& 8n^3 + 12n^2 + 6n + 1 &=& 2(4n^3
      + 6n^2 + 3n) + 1
    \end{IEEEeqnarray*}
    We conclude that if $k^3$ is divisible by $2$, then
    $k$ is divisible by $2$ for any natural $k$. Suppose
    $\sqrt[3]{2}$ were rational, then there would be
    integers $p$ and $q$ with no common divisor such that
    \begin{equation*}
      \sqrt[3]{2} = \frac{p}{q}
    \end{equation*}
    then $p^3 = 2 q^3$: $p$ is divisible by $2$,
    and hence $4k^3 = q^3$ for some $k$: $q$ is divisible by $2$:
    A contradiction since both have common divisor. We
    conclude that $\sqrt[3]{2}$ is irrational.
    Similarly,
    \begin{IEEEeqnarray*}{rClCl}
      (3n+1)^3 &=& 27n^3 + 27n^2 + 9n + 1 &=&
      3(9n^3 + 9n^2 + 3n) + 1                 \\
      (3n+2)^3 &=& 27n^3 + 54n^2 + 36n^2 + 8 &=&
      3(9n^3 + 18n^2 + 12n^2 + 2) + 2
    \end{IEEEeqnarray*}
    Hence, for any natural $k$ if $k^3$ is divisible by $3$,
    then $k$ is divisible by $3$. The very same method as above
    is used to prove that $\sqrt[3]{3}$ is irrational.
  \end{enumerate}
\end{solution}

\begin{pr} % 2.14
  Prove that
  \begin{enumerate}[label=(\alph*)]
    \item $\sqrt{2} + \sqrt{6}$ is irrational.
    \item $\sqrt{2} + \sqrt{3}$ is irrational.
  \end{enumerate}
\end{pr}

\begin{solution} % 2.14 solved
  \begin{enumerate}[label=(\alph*)]
    \item Suppose $\sqrt{2} + \sqrt{6}$ is rational. Then
    $(\sqrt{2} + \sqrt{6})^2 = 8 + 4\sqrt{3}$ must be rational.
    Hence, $4\sqrt{3}= 8 + 4\sqrt{3} - 8$ must be rational:
    A contradiction by since $4\neq0$ and $\sqrt{3}$ is
    irrational.
    \item Suppose that $\sqrt{2} + \sqrt{3}$ is rational.
    Then $5 + 2\sqrt{6}$ is rational, which implies
    $2\sqrt{6} = 5 + 2\sqrt{6} - 5$ is rational: A contradiction
    similar to the above.
  \end{enumerate}
\end{solution}

\begin{pr} % 2.15
  \begin{enumerate}[label=(\alph*)]
    \item Prove that if $x = p + \sqrt{q}$ where $p$ and $q$
    are rational, and $m$ is a natural number, then
    $x^m = a + b\sqrt{q}$ for some rational $a$ and $b$.
    \item Prove also that $(p - \sqrt{q})^m = a - b\sqrt{q}$.
  \end{enumerate}
\end{pr}

\begin{solution} % 2.15 solved
  \begin{enumerate}[label=(\alph*)]
    \item Proof is by induction. Let $m=1$, the statement
    holds. Suppose the statement holds for $m$, then for
    $m + 1$,
    \begin{IEEEeqnarray*}{rCl}
      x^{m+1} & = & (a + b\sqrt{q})(p + \sqrt{q}) \\
              & = & (ap + bq) + (a + bp)\sqrt{q}
    \end{IEEEeqnarray*}
    The conclusion follows.
    \item The proof is exactly the same as above except
    now that $x = p - \sqrt{q}$. Hence, for $m + 1$,
    \begin{IEEEeqnarray*}{rCl}
      x^{m+1} & = & (a - b\sqrt{q})(p - \sqrt{q}) \\
              & = & (ap + bq) - (a + bp)\sqrt{q}
    \end{IEEEeqnarray*}
  \end{enumerate}
\end{solution}

\begin{pr} % 2.16
  \begin{enumerate}[label=(\alph*)]
    \item Prove that if $m$ and $n$ are natural numbers
    and $\dfrac{m^2}{n^2} < 2$, then
    $\dfrac{(m + 2n)^2}{(m + n)^2} > 2$; show, moreover, that
    \begin{equation*}
      \frac{(m + 2n)^2}{(m + n)^2} - 2 < 2 - \frac{m^2}{n^2}
    \end{equation*}
    \item Prove the same results with all inequality signs
    reversed.
    \item Prove that if $\dfrac{m}{n} < \sqrt{2}$, then there
    is another rational number $\dfrac{m'}{n'}$ with
    $\dfrac{m}{n} < \dfrac{m'}{n'} < \sqrt{2}$.
  \end{enumerate}
\end{pr}

\begin{solution} % 2.16 solved
  \begin{enumerate}[label=(\alph*)]
    \item We first observe from $m^2/n^2 < 2$,
    \begin{equation*}
      (m + n)^2 < 3n^2 + 2nm
    \end{equation*}
    And from here,
    \begin{IEEEeqnarray*}{rCl}
      \frac{(m+2n)^2}{(m+n)^2} & = &
      \frac{(m+n)^2 + n^2 + 2n(m+n)}{(m+n)^2} \\
                               & = &
      1 + \frac{3n^2 + 2nm}{(n+m)^2}          \\
                               & > &
      1 + 1                                   \\
                              & = &
      2
    \end{IEEEeqnarray*}
    Furthermore,
    \begin{IEEEeqnarray*}{rCl}
      \frac{(m+2n)^2}{(m+n)^2} - 2 & = &
      \frac{-(m+n)^2 + n^2 + 2n(m+n)}{(m+n)^2}  \\
                                   & = &
      \frac{2n^2 - m^2}{(m+n)^2}                \\
                                   & < &
      \frac{2n^2 - m^2}{n^2}                    \\
                                   & = &
      2 - \frac{m^2}{n^2}
    \end{IEEEeqnarray*}
    \item Applying the exact method from the above, we
    see that if $m^2/n^2 > 2$, then
    $\dfrac{(m+2n)^2}{(m+n)^2} < 2$. As for the last inequality,
    note from the third line above: since $2n^2 - m^2 < 0$,
    the inequality sign is reversed. Therefore,
    $\dfrac{(m+2n)^2}{(m+n)^2} - 2 > 2 - \dfrac{m^2}{n^2}$.
    \item Let $m' = 2n^2 - m^2$ and $n' = (m+n)^2$, we obtain
    \begin{equation*}
      \frac{m^2}{n^2} < \frac{m'^2}{n'^2} < 2
    \end{equation*}
    Hence, $\dfrac{m}{n} < \dfrac{m'}{n'} < \sqrt{2}$.
  \end{enumerate}
\end{solution}

\begin{pr}[*] \label{2.17} % 2.17
  It seems likely that $\sqrt{n}$ is irrational whenever
  the natural number $n$ is not the square of another
  natural number. Although the method of \autoref{2.13}
  may actually be used to treat any particular case, it is
  not clear in advance that it will always work, and a proof
  for the general case requires some extra information.
  A natural number $p$ is called \boldText{prime number} if
  it is impossible to write $p = ab$ for natural numbers
  $a$ and $b$ unless one of these is $p$, and the other $1$;
  for convenience we also agree that $1$ is \note{not}
  a prime number. The first few prime numbers are
  $2,3,5,7,11,13,17,19$. If $n > 1$ is not a prime, then
  $n = ab$, with $a$ and $b$ both $< n$; if either $a$ or
  $b$ is not a prime it can be factored similarly; continuing
  in this way proves that we can write $n$ as a product of
  primes. For example, $28 = 4 \cdot 7 = 2\cdot2\cdot7$.
  \begin{enumerate}[label=(\alph*)]
    \item \label{2.17:a}
    Turn this argument into a rigorous proof by
    complete induction.
  \end{enumerate}
  A fundamental theorem about integers, which we will not
  prove here, states that this factorization is unique,
  except for the order of the factors. Thus, for example,
  $28$ can never be written as a product of primes one of
  which is $3$, nor can it be written in a way that involves
  $2$ only once (now you should appreciate why $1$ is not
  allowed as a prime).
  \begin{enumerate}[label=(\alph*),resume]
    \item Using this fact, prove that $\sqrt{n}$ is irrational
    unless $n = m^2$ for some natural number $m$.
    \item Prove more generally that $\sqrt[k]{n}$ is irrational
    unless $n = m^k$
    \item Prove that there cannot be only finitely many
    prime numbers $p_1, p_2, \dots , p_n$ by considering
    $p_1\cdot p_2 \cdot \ldots \cdot p_n + 1$.
  \end{enumerate}
\end{pr}

\begin{solution} % 2.17 solved
  \begin{enumerate}[label=(\alph*)]
    \item Let $n = 2$, then it is a prime itself. Suppose
    that the argument holds for $n=1,\dots,k$. Hence,
    for $n = k + 1$, if $k + 1$ is a prime, we are done. If not,
    then $k + 1$ can be written as a product
    of two numbers $n_1$ and $n_2$ smaller than $k + 1$.
    By assumption, $n_1$ and $n_2$ can be rewritten as
    the product of primes: Hence, any number $n > 1$
    can be written as a product of primes.
    \item Suppose that $\sqrt{n} = \dfrac{p}{q}$, for
    any integer $p,q$. Then $p^2 = nq^2$; because factorization
    is unique and by \ref{2.17:a}, $nq^2$ can be expressed
    as a product of primes; each of which appears
    twice, and the same is true for $q^2$.
    This implies that $n = m^2$, for some natural $m$.
    \item Suppose that $\sqrt[k]{n} = \dfrac{p}{q}$, for
    any integer $p,q$. Then $p^k = nq^k$; by the same
    reasoning as above, $nq^k$ and $q^k$ can be expressed
    as product of primes; each of which appears $k$ times.
    This implies that $n = m^k$, for some natural $m$.
    \item The proof is by contradiction. Suppose there were
    only a finite number of primes $p_1,\dots,p_n$. Then
    $p_1 \cdot p_2 \cdot \ldots \cdot p_n + 1$ could not
    be a prime, but then it is not evenly divisible by
    any of $p_1,\dots,p_n$ but $1$ and itself: This is
    then a prime number!
    \begin{ab}
      BEWARE! This does not say that
      $p_1 \cdot p_2 \cdot \ldots \cdot p_n + 1$ is
      \note{definitely} a prime number. What if it is
      evenly divisible for some primes in
      $p_{n+1},p_{n+2},\dots$? For example:
      \begin{equation*}
        2 \cdot 3 \cdot 5 \cdot 7 \cdot 11 \cdot 13 + 1
        = 30031 = 59 \cdot 509
      \end{equation*}
      Therefore, $p_{n+1} \geq
      p_1 \cdot p_2 \cdot \ldots \cdot p_n + 1$.
    \end{ab}
  \end{enumerate}
\end{solution}

\begin{pr}[*] \label{2.18} % 2.18
  Prove that
  \begin{enumerate}[label=(\alph*)]
    \item \label{2.18:a} If $x$ satisfies
    \begin{equation*}
      x^n + a_{n-1}x^{n-1} + \dots + a_0 = 0,
    \end{equation*}
    for some integers $a_{n-1},\dots,a_0$, then $x$ is
    irrational unless $x$ is an integer.(Why is this
    a generalization of \autoref{2.17}?)
    \item $\sqrt{6} - \sqrt{2} - \sqrt{3}$ is irrational.
    \item $\sqrt{2} + \sqrt[3]{2}$ is irrational. Hint: Start
    by working out the first $6$ powers of this number.
  \end{enumerate}
\end{pr}

\begin{solution} % 2.18 solved
  \begin{enumerate}[label=(\alph*)]
    \item We need first to prove the following lemma,
    \begin{id}
      For any $p > 1$, if $p^n$ is divisible by a prime,
      so is $p$.
    \end{id}
    \renewcommand{\qedsymbol}{$\square$}
    \begin{proof}
      $p^n$ can be written as a product of primes; each
      of which appears for $n$ times. Since $p^n$
      is divisible by a prime, this prime appears as a factor
      of $p^n$ for $n$ times,
      which means $p$ is divisible by the prime.
    \end{proof}
    Suppose $x$ were rational, then $x = \dfrac{p}{q}$
    such that $p$ and $q$ are integers that have no common
    factors.
    Hence, the equation satisfying $x$
    \begin{equation*}
      \left(\frac{p}{q}\right)^n + a_{n-1}\left(\frac{p}{q}
      \right)^{n-1} +
      \dots + a_0 = 0
    \end{equation*}
    is equivalent to
    \begin{equation*}
      p^n + a_{n - 1}p^{n - 1}q + \dots + a_0q^n = 0
    \end{equation*}
    if and only if $q^n = 1$. Suppose $q \neq \pm1$, then
    by \autoref{2.17}, $q$ can be factored as the product
    of primes. Hence, $a_{n - 1}p^{n - 1}q + \dots + a_0q^n$
    must be divisible by one of those primes. Since the
    result is an integer, it is clear that $p^n$ must
    be also divisible by that prime, and note that $p > 1$.
    But this implies that $p$ must also be divisible
    by the prime: A contradiction that $p$ and $q$ have
    no common factors. Therefore, $q = \pm1$, and we
    conclude that $x$ is an integer.
    \par
    Now let $a_{n-1},\dots,a_1$ be $0$ and $k = -a_0$,
    we see that $\sqrt[n]{k} = x$ is irrational unless
    $k = x^n$ for some integer $x$ since $k$ is an integer.
    \item Let $x = \sqrt{6} - \sqrt{3} - \sqrt{2}$. Then
    \begin{IEEEeqnarray*}{rCl}
      x^2 & = & 11 + 2\sqrt{6}(1 - \sqrt{2} - \sqrt{3}) \\
      (x^2 - 11)^2 & = & 24(1 - \sqrt{2} - \sqrt{3})^2  \\
      x^4 - 22x^2 + 121 & = & 24(6 + 2x)                \\
      x^4 - 22x^2 - 48x - 23 & = & 0
    \end{IEEEeqnarray*}
    By \ref{2.18:a}, $x$ is either an irrational or an
    integer. We will show that the latter is impossible.
    Note that since $\sqrt{6}$ and $\sqrt{2} + \sqrt{3}$
    are positive and,
    \begin{equation*}
      (\sqrt{2} + \sqrt{3})^2 - 6 = 2\sqrt{6} - 1 > 0
    \end{equation*}
    Moreover,
    \begin{equation*}
      (\sqrt{2} + \sqrt{3})^2 = 5 + 2\sqrt{6} < 7 + 2\sqrt{6}
      = (1 + \sqrt{6})^2
    \end{equation*}
    Therefore, we conclude
    $0 < \sqrt{2} + \sqrt{3} - \sqrt{6} < 1$, which means
    that $x$ must be irrational.
    \item Observe that $\sqrt{2} + \sqrt[3]{2}$ satisfies
    \begin{equation*}
      x^6 - 6x^4 - 4x^3 + 12x^2 - 24x - 4 = 0
    \end{equation*}
    Therefore, it is either an irrational number or an
    integer, but $2 < \sqrt{2} + \sqrt[3]{2} < 3$: Hence,
    the former must be true!
  \end{enumerate}
\end{solution}

\begin{pr} % 2.19
  Prove Bernoulli's inequality: If $h > -1$, then
  \begin{equation*}
    (1 + h)^n \geq 1 + nh.
  \end{equation*}
  Why is this trivial if $h > 0$?
\end{pr}

\begin{solution} % 2.19 solved
  The proof is by induction. When $n = 1$, $1 + h = 1 + h$.
  Suppose the inequality holds for $n$, then for $n+1$,
  \begin{IEEEeqnarray*}{rCl}
    (1+h)^{n+1} = (1+h)^n(1+h) & \geq & (1 + nh)(1 + h)     \\
                               &  =   & 1 + h + nh + nh^2   \\
                               & \geq & 1 + (n + 1)h
  \end{IEEEeqnarray*}
  Note that without the condition $h > -1$, the first inequality
  sign will be reversed!
  \par
  If $h > 0$, expanding the left-hand side using the binomial
  theorem directly yields the result since
  $\displaystyle\sum_{k=2}^n \binom{n}{k} h^k \geq 0$.
\end{solution}

\begin{pr} % 2.20
  The Fibonacci sequence $a_1, a_2, a_3, \dots$ is defined
  as follows:
  \begin{IEEEeqnarray*}{rClr}
    a_1 & = & 1 \\
    a_2 & = & 1 \\
    a_{n} & = & a_{n - 1} + a_{n - 2} & \quad
    \text{for $n \geq 3$.}
  \end{IEEEeqnarray*}
  Prove that
  \begin{equation*}
    a_n = \frac{\left(\dfrac{1 + \sqrt{5}}{2}\right)^n
    - \left(\dfrac{1 - \sqrt{5}}{2}\right)^n}%
    {\sqrt{5}}
  \end{equation*}
  One way of deriving this astonishing formula is presented
  in Problem 24.15 % \autoref{24.15}
\end{pr}

\begin{solution} % 2.20 solved
  The proof is by complete induction. Let $n=1$, then
  $a_1 = 1$. Suppose now that the formula holds for
  $1,\dots,n$, then for $n+1$,
  \begin{IEEEeqnarray*}{rCl}
    a_{n + 1} & = & a_n + a_{n - 1}               \\
    & = & \frac{\left(\dfrac{1 + \sqrt{5}}{2}\right)^n
        - \left(\dfrac{1 - \sqrt{5}}{2}\right)^n}%
        {\sqrt{5}} +
        \frac{\left(\dfrac{1 + \sqrt{5}}{2}\right)^{n - 1}
        - \left(\dfrac{1 - \sqrt{5}}{2}\right)^{n - 1}}%
        {\sqrt{5}}                                 \\
   & = &  \frac{\left(\dfrac{1 + \sqrt{5}}{2}\right)^n
        \left(\dfrac{3 + \sqrt{5}}{1 + \sqrt{5}}\right)
       - \left(\dfrac{1 - \sqrt{5}}{2}\right)^n
        \left(\dfrac{3 - \sqrt{5}}{1 - \sqrt{5}}\right)
       }%
       {\sqrt{5}}                                   \\
  & = &  \frac{\left(\dfrac{1 + \sqrt{5}}{2}\right)^{n+1}
      - \left(\dfrac{1 - \sqrt{5}}{2}\right)^{n+1}}%
      {\sqrt{5}}
  \end{IEEEeqnarray*}
  where in the second equality,
  \begin{equation*}
    \frac{(3+\sqrt{5})(1-\sqrt{5})}{1-5} =
    \frac{1 + \sqrt{5}}{2}
  \end{equation*}
  and
  \begin{equation*}
    \frac{(3-\sqrt{5})(1+\sqrt{5})}{1-5} =
    \frac{1 - \sqrt{5}}{2}
  \end{equation*}
\end{solution}

\begin{pr} % 2.21
  The Schwarz inequality (\autoref{1.19}) actually has
  a more general form:
  \begin{equation*}
    \sum_{i=1}^n x_iy_i \leq
    \sqrt{\sum_{i=1}^n x_i^2}\sqrt{\sum_{i=1}^n y_i^2}
  \end{equation*}
  Give three proofs of this, analogous to the three proofs
  presented in \autoref{1.19}.
\end{pr}

\begin{solution} % 2.21 solved
  We will give three proofs just as in \autoref{1.19}
  \begin{enumerate}[label=(\alph*)]
    \item \label{2.21:a}
    \note{The first proof}: Note that the equality
    occurs when either $y_i = 0 \quad \forall i$ or
    $x_i = \lambda y_i \quad \forall i$ for $\lambda > 0$.
    Otherwise,
    \begin{IEEEeqnarray*}{rCl}
      0 & < & \sum_{i=1}^n (x_i - \lambda y_i)^2 \\
        & = & \lambda^2(\sum_{i=1}^n y_i^2)
        - 2 \lambda (\sum_{i=1}^n x_iy_i)
        + (\sum_{i=1}^n x_i^2)                   \\
        & = & \lambda^2
        - 2 \lambda \left(\frac{\displaystyle\sum_{i=1}^n x_iy_i}%
        {\displaystyle\sum_{i=1}^n y_i^2}\right)
        + \left(\frac{\displaystyle\sum_{i=1}^n x_i^2}%
        {\displaystyle\sum_{i=1}^n y_i^2}\right)
    \end{IEEEeqnarray*}
    This is the case only if
    \begin{IEEEeqnarray*}{rCl}
      \left(\sum_{i=1}^n x_iy_i\right)^2 & \leq &
      \sum_{i=1}^n x_i^2\sum_{i=1}^n y_i^2      \\
      \sum_{i=1}^n x_iy_i \leq \left|\sum_{i=1}^n x_iy_i\right|
                                         & \leq &
      \sqrt{\sum_{i=1}^n x_i^2}\sqrt{\sum_{i=1}^n y_i^2}
    \end{IEEEeqnarray*}
    This concludes our first proof.
    \item \note{The second proof:} Let
    \begin{align*}
      x = \frac{x_i}{\sqrt{\sum_{i=1}^n x_i^2}} &&
      y = \frac{y_i}{\sqrt{\sum_{i=1}^n y_i^2}} &&
      \text{for all $i=1,\ldots,n$}
    \end{align*}
    and note that $(x - y)^2 \geq 0$. Hence,
    \begin{align*}
      \frac{x_i^2}{\sum_{i=1}^n x_i^2} +
      \frac{y_i^2}{\sum_{i=1}^n y_i^2} \geq
      2 \cdot \frac{x_i y_i}{\sqrt{\sum_{i=1}^n x_i^2}%
      \sqrt{\sum_{i=1}^n y_i^2}} &&
      \text{for all $i=1,\ldots,n$}
    \end{align*}
    Summing up all $n$ inequalities yields the result.
    Notice also that equality occurs with the same
    conditions as above.
    \item \note{The third proof:} Notice that
    \begin{equation*}
      \left(\sum_{i=1}^n x_i^2\right)
      \left(\sum_{i=1}^n y_i^2\right) =
      \left(\sum_{i=1}^n x_i y_i\right)^2 +
      \sum_{\substack{i=1 \\ i\neq j}}^n
      \left(x_i y_j - x_j y_i\right)^2
    \end{equation*}
    This essentially delivers the result since
    $\sum_{\substack{i=1 \\ i\neq j}}^n
    \left(x_i y_j - x_j y_i\right)^2 \geq 0$
    for all terms, with the
    equality condition as in \ref{2.21:a}.
  \end{enumerate}
\end{solution}

\end{document}
