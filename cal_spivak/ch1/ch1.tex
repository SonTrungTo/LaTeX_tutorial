\begin{pr}
  Prove the following:
  \begin{enumerate}[label=(\roman*)]
    \item If $ax=a$ for some number $a\neq0$, then
    $x=1$
    \item $x^2-y^2=(x-y)(x+y)$
    \item If $x^2=y^2$, then $x=y$ or $x=-y$
    \item \label{1:iv}
    $x^3-y^3=(x-y)(x^2+xy+y^2)$
    \item $x^n-y^n=(x-y)(x^{n-1}+x^{n-2}y+\dots+xy^{n-2}
    +y^{n-1})$
    \item $x^3+y^3=(x+y)(x^2-xy+y^2)$ (There is a particularly
    easy way to do this using \ref{1:iv}, and it will show
    you how to find a factorization for $x^n+y^n$ whenever
    n is odd.)
  \end{enumerate}
\end{pr}

\begin{solution}
  \begin{enumerate}[label=(\roman*)]
    \item By (P7)(Existence of multiplicative inverses),
    there exists $a^{-1}$ such that,
    \begin{IEEEeqnarray*}{rCl}
      (a^{-1}\cdot a)x & = & (a^{-1}\cdot a) \\
      x & = & 1 \\
    \end{IEEEeqnarray*}
    \item \label{(ii)} By (P9) for 2 times,
    \begin{IEEEeqnarray*}{+rCl+x*}
      (x-y)(x+y) & \stackrel{1}{=} & x\cdot(x+y)+(-y)\cdot(x+y)\\
      & \stackrel{2}{=} & x\cdot x+x\cdot y+(-y)\cdot x
      +(-y)\cdot y \\
      & = & x^2 +x\cdot y+[-(x\cdot y)]+[-(y^2)] \\
      & = & x^2-y^2 \\
    \end{IEEEeqnarray*}
    \item From \ref{(ii)} and since $x^2=y^2$,
    \begin{equation*}
      x^2-y^2=(x-y)(x+y)=0
    \end{equation*}
    This means $(x-y)=0 \lor (x+y)=0$, which is $x=y \lor x=-y$
    \item Starting with the right-hand side,
    \begin{IEEEeqnarray*}{rCl}
      (x-y)(x^2+xy+y^2) & = &
      x\cdot(x^2+xy+y^2)+(-y)\cdot(x^2+xy+y^2)\\
      & = &
      x^3+x^2y+xy^2+[-(x^2y)]+[-(xy^2)]+[-(y)^3] \\
      & = &
      x^3-y^3 \\
    \end{IEEEeqnarray*}
    \item I propose two solutions for this problem.
    The first one is the direct right-hand side manipulation,
    while the latter is done by induction.
    \renewcommand{\qedsymbol}{\textsl Q.E.D}
    \begin{proof}[The first solution]
      \begin{IEEEeqnarray*}{+l+x*}
        (x-y)(x^{n-1}+x^{n-2}y+\cdots+xy^{n-2}+y^{n-1}) \\=
        x^n+x^{n-1}y+\cdots+x^2y^{n-2}+xy^{n-1} \\
        +[-(x^{n-1}y)]+[-(x^{n-2}y^2)]+\cdots+
        [-(xy^{n-1})]+[-(y^n)] \\
        =x^n-y^n \\
        & \qedhere
      \end{IEEEeqnarray*}
    \end{proof}
    \begin{proof}[The second solution]
      Let n=1, then indeed $x-y=x-y$. Suppose the statement
      holds true for $n=k$ with $k\in \mathbb{N}$, that is
      \begin{equation*}
        x^k-y^k=(x-y)(x^{k-1}+x^{k-2}y+\cdots+xy^{k-2}+y^{k-1})
      \end{equation*}
    is true. To finish the proof, we need to prove
    \begin{equation*}
      x^{k+1}-y^{k+1}=(x-y)(x^k+x^{k-1}y+\cdots+xy^{k-1}+y^k)
    \end{equation*}
    That is, the statement holds for $n=k$. Starting from
    the left hand side,
    \begin{IEEEeqnarray*}{*x+C+x*}
      & x^{k+1}-y^{k+1} \\
      = & x^{k+1}-x^ky+x^ky-y^{k+1} \\
      = & x^k(x-y)+y(x^k-y^k) \\
      = & x^k(x-y)+y(x-y)(x^{k-1}+x^{k-2}y+\cdots+
      xy^{k-2}+y^{k-1})\\
      = & (x-y)[x^k+y(x^{k-1}+x^{k-2}y+\cdots+xy^{k-2}+y^{k-1})]
      \\
      = & (x-y)(x^k+x^{k-1}y+x^{k-2}y^2+\cdots+xy^{k-1}+
      y^k) \\
      & & \qedhere
    \end{IEEEeqnarray*}
    \end{proof}
    \item We will use \ref{1:iv} in our proof,
    \begin{IEEEeqnarray*}{*x+C+x*}
      & x^3+y^3 \\
    = & x^3-y^3+2y^3 \\
    = & (x-y)(x^2+xy+y^2)+2y[(x^2+xy+y^2)+(-x)(x+y)] \\
    = & (x+y)(x^2+xy+y^2)+2[-(xy)](x+y) \\
    = & (x+y)(x^2-xy+y^2) \\
    \end{IEEEeqnarray*}
  \end{enumerate}
\end{solution}

\begin{pr}
  What is wrong with the following ``proof''? Let $x=y$. Then
  \begin{IEEEeqnarray*}{rCl}
    x^2 & = & xy, \\
    x^2-y^2 & = & xy -y^2, \\
    (x+y)(x-y) & = & y(x-y), \\
    x+y & = & y, \\
    2y & = & y, \\
    2 & = & 1.
  \end{IEEEeqnarray*}
\end{pr}

\begin{solution}
  Note that in the transition from line 3 to line 4,
  the author ``simplifies'' $(x-y)$ by
  dividing $(x-y)$ on both sides. This is wrong since $x-y=0$,
  and hence $1/0$ is undefined as implied by (P7)
  in the textbook.
\end{solution}

\begin{pr}
  This is problem 3
\end{pr}
