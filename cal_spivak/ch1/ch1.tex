\begin{pr}
  Prove the following:
  \begin{enumerate}[label=(\roman*)]
    \item If $ax=a$ for some number $a\neq0$, then
    $x=1$
    \item $x^2-y^2=(x-y)(x+y)$
    \item If $x^2=y^2$, then $x=y$ or $x=-y$
    \item \label{1:iv}
    $x^3-y^3=(x-y)(x^2+xy+y^2)$
    \item $x^n-y^n=(x-y)(x^{n-1}+x^{n-2}y+\dots+xy^{n-2}
    +y^{n-1})$
    \item $x^3+y^3=(x+y)(x^2-xy+y^2)$ (There is a particular
    easy way to do this using \ref{1:iv}, and it will show
    you how to find a factorization for $x^n+y^n$ whenever
    n is odd.)
  \end{enumerate}
\end{pr}
\begin{solution}
  \begin{enumerate}[label=(\roman*)]
    \item By (P7)(Existence of multiplicative inverses),
    there exists $a^{-1}$ such that,
    \begin{IEEEeqnarray*}{rCl}
      (a^{-1}\cdot a)x & = & (a^{-1}\cdot a) \\
      x & = & 1 \\
    \end{IEEEeqnarray*}
    \item \label{(ii)} By (P9) for 2 times,
    \begin{IEEEeqnarray*}{+rCl+x*}
      (x-y)(x+y) & \stackrel{1}{=} & x\cdot(x+y)+(-y)\cdot(x+y)\\
      & \stackrel{2}{=} & x\cdot x+x\cdot y+(-y)\cdot x
      +(-y)\cdot y \\
      & = & x^2 +x\cdot y+[-(x\cdot y)]+[-(y^2)] \\
      & = & x^2-y^2 \\
    \end{IEEEeqnarray*}
    \item From \ref{(ii)},
    \begin{equation*}
      x^2-y^2=(x-y)(x+y)=0
    \end{equation*}
    This means $(x-y)=0 \lor (x+y)=0$, which is $x=y \lor x=-y$
  \end{enumerate}
\end{solution}

\begin{pr}
  This is problem 2.
\end{pr}
