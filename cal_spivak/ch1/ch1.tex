\begin{pr} \label{1.1}
  Prove the following:
  \begin{enumerate}[label=(\roman*)]
    \item If $ax=a$ for some number $a\neq0$, then
    $x=1$
    \item $x^2-y^2=(x-y)(x+y)$
    \item If $x^2=y^2$, then $x=y$ or $x=-y$
    \item \label{1:iv}
    $x^3-y^3=(x-y)(x^2+xy+y^2)$
    \item \label{1.1:v}
    $x^n-y^n=(x-y)(x^{n-1}+x^{n-2}y+\dots+xy^{n-2}
    +y^{n-1})$
    \item \label{1.1:vi}
    $x^3+y^3=(x+y)(x^2-xy+y^2)$ (There is a particularly
    easy way to do this using \ref{1:iv}, and it will show
    you how to find a factorization for $x^n+y^n$ whenever
    n is odd.)
  \end{enumerate}
\end{pr}

\begin{solution}
  \begin{enumerate}[label=(\roman*)]
    \item By (P7)(Existence of multiplicative inverses),
    there exists $a^{-1}$ such that,
    \begin{IEEEeqnarray*}{rCl}
      (a^{-1}\cdot a)x & = & (a^{-1}\cdot a) \\
      x & = & 1 \\
    \end{IEEEeqnarray*}
    \item \label{(ii)} By (P9) for 2 times,
    \begin{IEEEeqnarray*}{+rCl+x*}
      (x-y)(x+y) & \stackrel{1}{=} & x\cdot(x+y)+(-y)\cdot(x+y)\\
      & \stackrel{2}{=} & x\cdot x+x\cdot y+(-y)\cdot x
      +(-y)\cdot y \\
      & = & x^2 +x\cdot y+[-(x\cdot y)]+[-(y^2)] \\
      & = & x^2-y^2 \\
    \end{IEEEeqnarray*}
    \item From \ref{(ii)} and since $x^2=y^2$,
    \begin{equation*}
      x^2-y^2=(x-y)(x+y)=0
    \end{equation*}
    This means $(x-y)=0 \lor (x+y)=0$, which is $x=y \lor x=-y$
    \item Starting with the right-hand side,
    \begin{IEEEeqnarray*}{rCl}
      (x-y)(x^2+xy+y^2) & = &
      x\cdot(x^2+xy+y^2)+(-y)\cdot(x^2+xy+y^2)\\
      & = &
      x^3+x^2y+xy^2+[-(x^2y)]+[-(xy^2)]+[-(y)^3] \\
      & = &
      x^3-y^3 \\
    \end{IEEEeqnarray*}
    \item I propose two solutions for this problem.
    The first one is the direct right-hand side manipulation,
    while the latter is done by induction.
    \renewcommand{\qedsymbol}{\textsl Q.E.D}
    \begin{proof}[The first solution]
      \begin{IEEEeqnarray*}{+l+x*}
        (x-y)(x^{n-1}+x^{n-2}y+\cdots+xy^{n-2}+y^{n-1}) \\=
        x^n+x^{n-1}y+\cdots+x^2y^{n-2}+xy^{n-1} \\
        +[-(x^{n-1}y)]+[-(x^{n-2}y^2)]+\cdots+
        [-(xy^{n-1})]+[-(y^n)] \\
        =x^n-y^n \\
        & \qedhere
      \end{IEEEeqnarray*}
    \end{proof}
    \begin{proof}[The second solution]
      Let n=1, then indeed $x-y=x-y$. Suppose the statement
      holds true for $n=k$ with $k\in \mathbb{N}$, that is
      \begin{equation*}
        x^k-y^k=(x-y)(x^{k-1}+x^{k-2}y+\cdots+xy^{k-2}+y^{k-1})
      \end{equation*}
    is true. To finish the proof, we need to prove
    \begin{equation*}
      x^{k+1}-y^{k+1}=(x-y)(x^k+x^{k-1}y+\cdots+xy^{k-1}+y^k)
    \end{equation*}
    That is, the statement holds for $n=k$. Starting from
    the left hand side,
    \begin{IEEEeqnarray*}{*x+C+x*}
      & x^{k+1}-y^{k+1} \\
      = & x^{k+1}-x^ky+x^ky-y^{k+1} \\
      = & x^k(x-y)+y(x^k-y^k) \\
      = & x^k(x-y)+y(x-y)(x^{k-1}+x^{k-2}y+\cdots+
      xy^{k-2}+y^{k-1})\\
      = & (x-y)[x^k+y(x^{k-1}+x^{k-2}y+\cdots+xy^{k-2}+y^{k-1})]
      \\
      = & (x-y)(x^k+x^{k-1}y+x^{k-2}y^2+\cdots+xy^{k-1}+
      y^k) \\
      & & \qedhere
    \end{IEEEeqnarray*}
    \end{proof}
    \item We will use \ref{1:iv} in our proof,
    \begin{IEEEeqnarray*}{*x+C+x*}
      & x^3+y^3 \\
    = & x^3-y^3+2y^3 \\
    = & (x-y)(x^2+xy+y^2)+2y[(x^2+xy+y^2)+(-x)(x+y)] \\
    = & (x+y)(x^2+xy+y^2)+2[-(xy)](x+y) \\
    = & (x+y)(x^2-xy+y^2) \\
    \end{IEEEeqnarray*}
  \end{enumerate}
\end{solution}

\begin{pr}
  What is wrong with the following ``proof''? Let $x=y$. Then
  \begin{IEEEeqnarray*}{rCl}
    x^2 & = & xy, \\
    x^2-y^2 & = & xy -y^2, \\
    (x+y)(x-y) & = & y(x-y), \\
    x+y & = & y, \\
    2y & = & y, \\
    2 & = & 1.
  \end{IEEEeqnarray*}
\end{pr}

\begin{solution}
  Note that in the transition from line 3 to line 4,
  the author ``simplifies'' $(x-y)$ by
  dividing $(x-y)$ on both sides. This is wrong since $x-y=0$,
  and hence $1/0$ is undefined as implied by (P7)
  in the textbook.
\end{solution}

\begin{pr}
  Prove the following:
  \begin{enumerate}[label=(\roman*)]
    \item $\displaystyle{\frac{a}{b} = \frac{ac}{bc}}$,
    if $b,c\neq0$.
    \item
      $\displaystyle{\frac{a}{b}+\frac{c}{d}=\frac{ad+bc}{bd}}$,
    if $b,d\neq0$.
    \item
    $(ab)^{-1}=a^{-1}b^{-1},\text{ if }a,b\neq0$. (To do this
    you must remember the defining property of $(ab)^{-1}$.)
    \item
    $\dfrac{a}{b}\cdot\dfrac{c}{d}=\dfrac{ac}{db}$, if
    $b,d\neq0$.
    \item
    $\dfrac{a}{b}\Bigg/\dfrac{c}{d}=\dfrac{ad}{bc}$, if
    $b,c,d\neq0$.
    \item
    If $b,d\neq0$, then $\dfrac{a}{b}=\dfrac{c}{d}$ if and
    only if $ad=bc$. Also determine when $\dfrac{a}{b}=
    \dfrac{b}{a}$.
  \end{enumerate}
\end{pr}

\begin{solution}
  \begin{enumerate}[label=(\roman*)]
    \item \label{2:i}
    Until \ref{2:iii} is proved, the solution is to
    test the equality between two sides.
    \begin{IEEEeqnarray*}{rCl}
      a(b)^{-1} & = & (ac)(bc)^{-1}\\
      a[(b)^{-1}b] & = & (ac)(bc)^{-1}b \\
      (a^{-1}a) & = & (a^{-1}a)c(bc)^{-1}b \\
      1 & = & (bc)(bc)^{-1} = 1\\
    \end{IEEEeqnarray*}
    \item\label{p13:ii} Similar to the above,
    \begin{IEEEeqnarray*}{rCl}
      a(b)^{-1}+c(d)^{-1} & = & (ad+bc)(bd)^{-1} \\
      a(b)^{-1}bd+c(d)^{-1}bd & = & (ad+bc)[(bd)^{-1}(bd)]\\
      ad(b^{-1}b)+bc(d^{-1}d) & = & (ad+bc)\\
      ad+bc & = & ad+bc
    \end{IEEEeqnarray*}
    \item \label{2:iii}
    Since $a,b\neq0$, there exists $(ab)^{-1},a^{-1},
    b^{-1}$ such that,
    \begin{IEEEeqnarray*}{rCl}
      ab & = & ab \\
      (ab)^{-1}(ab) & = & (ab)^{-1}(ab)=1 \\
      (ab)^{-1}a(bb^{-1}) & = & b^{-1} \\
      (ab)^{-1}(aa^{-1}) & = & b^{-1}a^{-1} \\
      (ab)^{-1} & = & a^{-1}b^{-1} \\
    \end{IEEEeqnarray*}
    \item For $b,d\neq0$,
    \begin{equation*}
      \frac{a}{b}\cdot\frac{c}{d}=ab^{-1}cd^{-1}=ac(d^{-1}b^{-1})
      =ac(db)^{-1}=\frac{ac}{db}
    \end{equation*}
    where the next-to-last equality follows from \ref{2:iii}.
    \item I first establish for any number $a\neq0$,
    \begin{equation*}
      (a^{-1})^{-1}=a
    \end{equation*}
    Let $t=a^{-1}$, we want to prove $t^{-1}=a$. Observe that
    \begin{IEEEeqnarray*}{rCl}
      t & = & a^{-1} \\
      t\cdot(t)^{-1} & = & a^{-1}\cdot (t)^{-1} \\
      a\cdot 1 & = & (a\cdot a^{-1})\cdot (t)^{-1} \\
      a & = & (t)^{-1}
    \end{IEEEeqnarray*}
    From the left hand side of the statement,
    \begin{equation*}
      \frac{a}{b}\Bigg/\frac{c}{d}=a(b)^{-1}[c(d)^{-1}]^{-1}
      =a(b)^{-1}(c)^{-1}[(d)^{-1}]^{-1}
      =(ad)(bc)^{-1}=\frac{ad}{bc}
    \end{equation*}
    where the second and third equality follows both
    from \ref{2:iii}
    and the proof above.
    \item Using \ref{p13:ii},
    \begin{IEEEeqnarray*}{rCl}
      \frac{a}{b} & = & \frac{c}{d} \\
      \frac{a}{b}+(-\frac{c}{d}) & = & 0 \\
      \frac{ad-bc}{bd} & = & 0 \\
      ad & = & bc \\
    \end{IEEEeqnarray*}
    Now, put $c=b \land d=a$. It follows that
    $\dfrac{a}{b}=\dfrac{b}{a}$ if and only if
    $a^2=b^2$. It follows $(a-b)(a+b)=0$, or
    $a=b \lor a=-b$.
  \end{enumerate}
\end{solution}

\begin{pr}
  Find all numbers x for which
  \begin{enumerate}[label=(\roman*)]
    \item $4-x<3-2x$
    \item $5-x^2<8$
    \item $5-x^2<-2$
    \item $(x-1)(x-3)>0$ (When is a product of two numbers
    positive?)
    \item $x^2-2x+2>0$
    \item $x^2+x+1>2$
    \item $x^2-x+10>16$
    \item $x^2+x+1>0$
    \item $(x-\pi)(x+5)(x-3)>0$
    \item $(x-\sqrt[3]{2})(x-\sqrt{2})>0$
    \item $2^x<8$
    \item $x+3^x<4$
    \item $\dfrac{1}{x}+\dfrac{1}{1-x}>0$
    \nopagebreak[3]
    \item $\dfrac{x-1}{x+1}>0$
  \end{enumerate}
\end{pr}
\pagebreak
\begin{solution}
  \begin{enumerate}[label=(\roman*)]
    \item
    \begin{IEEEeqnarray*}{rCl}
      4-x & < & 3-2x \\
      4+(-x+2x) & < & 3+(-2x+2x) \\
      (-4+4)+x & < & -4+3 \\
      x & < & -1 \\
    \end{IEEEeqnarray*}
    \item
    \begin{IEEEeqnarray*}{rCl}
      5-x^2 & < & 8 \\
      5-8 & < & x^2 \\
      -3 & < & x^2
    \end{IEEEeqnarray*}
    Since $x^2\geq0\ \forall x\in\mathbb{R}$, the inequality
    holds $\forall x$.
    \item
    \begin{IEEEeqnarray*}{rCl}
      5-x^2 & < & -2 \\
      7 & < & x^2 \\
      0 & < & x^2-7=(x-\sqrt{7})(x+\sqrt{7})
    \end{IEEEeqnarray*}
    Hence, either $x>\sqrt{7}~\land\ x>-\sqrt{7}$
    or $x<\sqrt{7}~\land\ x<-\sqrt{7}$, which is
    $x>\sqrt{7}~\lor\ x<-\sqrt{7}$.
    \item
    \begin{IEEEeqnarray*}{rCl}
      (x-1)(x-3) & > & 0 \\
      (x>1~\land\ x>3) & \lor & (x<1~\land\ x<3)\\
      x>3 & \lor & x<1
    \end{IEEEeqnarray*}
    \item
    \begin{IEEEeqnarray*}{rCl}
      x^2-2x+2 & > & 0 \\
      (x^2-2x+1)+1 & > & 0 \\
      (x-1)^2+1 & > & 0
    \end{IEEEeqnarray*}
    Hence the inequality is satisfied $\forall\ x$.
    \item
    \begin{IEEEeqnarray*}{rCl}
      x^2+x+1 & > & 2 \\
      x^2+x-1 & > & 0 \\
      x^2+\left(\frac{1+\sqrt{5}}{2}\right)x
      +\left(\frac{1-\sqrt{5}}{2}\right)x
      +\left(\frac{(1-\sqrt{5})(1+\sqrt{5})}{4}\right)
      & > & 0 \\
      \left(x+\frac{1+\sqrt{5}}{2}\right)
      \left(x+\frac{1-\sqrt{5}}{2}\right) & > & 0 \\
      x>\left(\frac{\sqrt{5}-1}{2}\right) \lor
      x<\left(\frac{-(\sqrt{5}+1)}{2}\right)
    \end{IEEEeqnarray*}
    \item
    \begin{IEEEeqnarray*}{rCl}
      x^2-x+10 & > & 16 \\
      x^2-x-6 & > & 0 \\
      x^2-3x+2x-6 & > & 0 \\
      x(x-3)+2(x-3) & > & 0 \\
      (x+2)(x-3) & > & 0 \\
      x>3 & \lor & x<-2
    \end{IEEEeqnarray*}
    \item
    \begin{IEEEeqnarray*}{rCl}
      x^2+x+1 & > & 0 \\
      x^2+x+\frac{1}{4}-\frac{1}{4}+1 & > & 0 \\
      (x+\frac{1}{2})^2+\frac{3}{4} & > & 0
    \end{IEEEeqnarray*}
    which is true for all $x$.
    \item Divide the problem into two cases: $x>\pi$
    and $x<\pi$.
    \par
    \note{Case 1: }$x>\pi$\\
    Then $(x+5)(x-3)>0$, which is $x>3~\lor\ x<-5$.
    \par
    \note{Case 2: }$x<\pi$\\
    Then $(x+5)(x-3)<0$, which is $-5<x<3$.
    \item
    \begin{IEEEeqnarray*}{rCl}
      (x-\sqrt[3]{2})(x-\sqrt{2}) & > & 0 \\
      x>\sqrt{2} & \lor & x<\sqrt[3]{2}
    \end{IEEEeqnarray*}
    \item (Sometimes, to solve a problem, intuition
    is a necessity.)
    \begin{IEEEeqnarray*}{rCl}
      2^x & < & 8 \\
      2^x & < & 2^3 \\
      x & < & 3
    \end{IEEEeqnarray*}
    \item
    \begin{IEEEeqnarray*}{rCl}
      x+3^x & < & 4 \\
      x+3^x & < & 1+3^1 \\
      x & < & 1
    \end{IEEEeqnarray*}
    \item
    \begin{IEEEeqnarray*}{rCl}
      \frac{1}{x}+\frac{1}{1-x} & > & 0 \\
      \frac{1}{x(1-x)} & > & 0 \\
    \end{IEEEeqnarray*}
    Hence, $x(1-x)>0$. This means $0<x<1$.
    \item
    \begin{IEEEeqnarray*}{rCl}
      \frac{x-1}{x+1} & > & 0 \\
    \end{IEEEeqnarray*}
    Hence, $(x-1)(x+1)>0$, or $x>1~\lor\ x<-1$.
  \end{enumerate}
\end{solution}

\begin{pr} \label{1.5}
  Prove the following:
  \begin{enumerate}[label=(\roman*)]
    \item \label{1.5:i}
    If $a<b$ and $c<d$, then $a+c<b+d$
    \item If $a<b$, then $-b<-a$
    \item If $a<b$ and $c>d$, then $a-c<b-d$
    \item If $a<b$ and $c>0$, then $ac<bc$
    \item If $a<b$ and $c<0$, then $ac>bc$
    \item If $a>1$, then $a^2>a$
    \item If $0<a<1$, then $a^2<a$
    \item \label{1.5:viii}
    If $0\leq a<b$ and $0\leq c<d$, then $ac<bd$
    \nopagebreak[3]
    \item \label{1.5:ix}
    If $0\leq a<b$, then $a^2<b^2$. (Use \ref{1.5:viii}.)
    \item If $a,b\geq0$ and $a^2<b^2$, then $a<b$.%
    (Use \ref{1.5:ix}, backwards.)
  \end{enumerate}
\end{pr}
\pagebreak
\begin{solution}
  Let P be the set of all positive numbers.
  \begin{enumerate}[label=(\roman*)]
    \item To prove this, we apply (P11): If $a<b~\land\ c<d$,
    then $(b-a\in P)~\land\ (d-c\in P)$. Then $(b-a)+(d-c)=
    (b+d)-(a+c)\in P$. Therefore, $a+c<b+d$.
    \item \label{1.5:ii}
    We provide two solutions: The first one is
    by Trichotomy Law (P10), and the second one is
    by adding $[(-a)+(-b)]$ to both sides.
    \renewcommand{\qedsymbol}{\textsl Q.E.D}
    \begin{proof}[Proof by Trichotomy Law]
      If $a<b$, then $b-a\in P$. By Trichotomy Law,
      $a-b\notin P$ and $a-b\neq 0$. Therefore, $a-b<0$,
      which is $-b<-a$.
    \end{proof}
    \begin{proof}[Proof by adding]
      \begin{IEEEeqnarray*}{+rCl+x*}
        a & < & b \\
        a+[(-a)+(-b)] & < & b+[(-a)+(-b)] \\
        \left[a+(-a)\right]+(-b) & < & [b+(-b)]+(-a) \\
        -b & < & -a \\
        & & & \qedhere
      \end{IEEEeqnarray*}
    \end{proof}
    \item Using (P11), we have $b-a\in P~\land\ c-d\in P$.
    Then $(b-a)+(c-d)\in P$. Hence, $a-c<b-d$.
    \item \label{1.5:iv}
    Using (P12), note that $b-a\in P$. Since $c>0$,
    $c(b-a)\in P$, which means $bc-ac>0$, or $ac<bc$.
    \item By Trichotomy law(P10), $-c\in P$. Then
    by \ref{1.5:iv}, $-(ac)<-(bc)$. By \ref{1.5:ii},
    $ac>bc$.
    \item
    Since $a>1>0$, by \ref{1.5:iv}, $a^2>a$.
    \item Since $a>0$, by \ref{1.5:iv}, $a^2<a$.
    \item Because $0<b$, $bc<bd$. Furthermore, if $c\geq0$,
    $ac\leq bc$ (equality occurs if $c=0$), by \ref{1.5:iv}.
    Therefore, $ac\leq bc<bd$. Hence, $ac<bd$.
    \item From \ref{1.5:viii}, let $c=a$ and $d=b$, then
    the result follows.
    \item
    Suppose $a\geq b$. Then $a\geq b\geq 0$. By \ref{1.5:ix}
    and (P9), $a^{2}\geq b^2$. This contradicts $a^2<b^2$.
  \end{enumerate}
\end{solution}

\begin{pr} \label{1.6}
  \begin{enumerate}[label=(\alph*)]
    \item \label{1.6:a}
    Prove that if $0\leq x<y$, then $x^n<y^n$,
    $n=1,2,3,\dots$.
    \item \label{1.6:b}
    Prove that if $x<y$ and $n$ is odd,
    then $x^n<y^n$.
    \item \label{1.6:c}
    Prove that if $x^n=y^n$ and $n$ is odd,
    then $x=y$.
    \item \label{1.6:d}
    Prove that if $x^n=y^n$ and $n$ is even,
    then $x=y$ or $x=-y$.
  \end{enumerate}
\end{pr}
\pagebreak
\begin{solution}
  \begin{enumerate}[label=(\alph*)]
    \item
    Repeatedly apply problem \ref{1.5}\ref{1.5:viii} for
    $0\leq x<y$, we have $x^n<y^n$ with $n=1,2,3,\ldots$
    \item
    The statement is true for the case $0\leq x<y$. In
    the case $x<y\leq 0$, by \ref{1.5}\ref{1.5:ii},
    $(-x)>(-y)\geq 0$. By \ref{1.6:a}, $(-x)^n>(-y)^n$
    for all odd $n$. Since $n$ is odd, $-(x^n)>-(y^n)$.
    Hence, by \ref{1.5}\ref{1.5:ii}, $x^n<y^n$. In the
    case $x\leq 0<y$, since $n$ is odd, $x^n<y^n$.
    \item
    Suppose that either $x\neq y$. W.l.o.g, let $x<y$,
    by \ref{1.6:b}, $x^n<y^n$ for all odd $n$, contradicting
    $x^n=y^n$ for all odd $n$.
    \item
    Suppose that both $x\neq y$ and $x\neq -y$. Then
    $x^2-y^2\neq 0$. W.l.o.g, suppose $x^2>y^2\geq0$.
    Applying \ref{1.6:a}, this generalizes to
    $x^n>y^n$ for all even $n$, contradicting our assumption.
    Therefore, $x=y$ or $x=-y$.
    \renewcommand{\qedsymbol}{}
    \begin{proof}[The direct proof]
      In the case $x,y \geq 0$; by \ref{1.6:a}, if $x^n=y^n$
      for all even $n$, then $x=y$. In the case $x,y\leq 0$;
      if $x^n=y^n$ for all even $n$, then
      $(-x),(-y)\geq 0$ and $(-x)^n=(-y)^n$, so $-x=-y$
      and hence $x=y$.
      In the case of $x$ and $y$ have different signs, then
      $x$ and $-y$ are either two positive or two negative
      numbers. In either subcase, if $x^n=y^n$ for all even $n$,
      then $x^n=(-y)^n$, and it follows $x=-y$ from the
      previous case.
    \end{proof}
  \end{enumerate}
\end{solution}

\begin{pr}
  Prove that if $0<a<b$, then
  \begin{equation*}
    a<\sqrt{ab}<\frac{a+b}{2}<b
  \end{equation*}
  Notice that the inequality $\sqrt{ab}\leq(a+b)/2$ holds
  for all $a,b\geq0$. A generalization of this fact
  occurs in Problem 2.22. % to be \ref{2.22}.
\end{pr}
\begin{solution}
  Let us first establish that $a<\dfrac{a+b}{2}<b$. Note
  that,
  \begin{equation*}
    a+a<a+b<b+b
  \end{equation*}
  and therefore, $a<\dfrac{a+b}{2}<b$. To finish the proof,
  we need to prove ${a<\sqrt{ab}<\dfrac{a+b}{2}}$. To do this,
  let us prove that if $0<a<b$, then ${0<\sqrt{a}<\sqrt{b}}$.
  Note that since $b-a>0$,
  \begin{equation*}
    b-a=(\sqrt{b}-\sqrt{a})(\sqrt{b}+\sqrt{a})>0
  \end{equation*}
  Therefore, $\sqrt{b}>\sqrt{a}>0$. We rewrite the inequality
  as follows,
  \begin{equation*}
    \sqrt{a}\cdot(\sqrt{b}-\sqrt{a})>0
  \end{equation*}
  Then
  \begin{equation}
    a<\sqrt{ab} \label{1.7:eq1}
  \end{equation}
  We next notice that since ${\sqrt{b}-\sqrt{a}>0}$, it follows
  that \\$(\sqrt{b}-\sqrt{a})\cdot(\sqrt{b}-\sqrt{a})
  =(\sqrt{b}-\sqrt{a})^2>0$. Expand the left hand side,
  \begin{equation*}
    (\sqrt{b}-\sqrt{a})^2=a+b-2\sqrt{ab}>0
  \end{equation*}
  which implies,
  \begin{equation}
    \sqrt{ab}<\frac{a+b}{2} \label{1.7:eq2}
  \end{equation}
  From (\ref{1.7:eq1}) and (\ref{1.7:eq2}), we have
  $a<\sqrt{ab}<\dfrac{a+b}{2}$.
\end{solution}

\begin{pr}[*]
  Although the basic properties of inequalities were stated
  in terms of the collection P of all positive numbers,
  and $<$ was defined in terms of $P$, this procedure
  can be reversed. Suppose that P10--P12 are replaced by
  \begin{enumerate}
    \item[\mylabel{P'10}{(P'10)}]
    For any numbers $a$ and $b$ one, and only one, of the
    following holds:
    \begin{enumerate}[label=(\roman*)]
      \item $a=b$,
      \item $a<b$,
      \item $b<a$.
    \end{enumerate}
    \item[\mylabel{P'11}{(P'11)}]
    For any numbers $a$, $b$, and $c$, if $a<b$ and $b<c$,
    then $a<c$.
    \item[\mylabel{P'12}{(P'12)}]
    For any numbers $a$, $b$, and $c$, if $a<b$, then
    $a+c<b+c$.
    \item[\mylabel{P'13}{(P'13)}]
    For any numbers $a$, $b$, and $c$, if $a<b$ and
    $0<c$, then $ac<bc$.
  \end{enumerate}
  Show that P10--P12 can then be deduced as theorems.
\end{pr}
\begin{solution}
  Let $P$ be the set of all positive numbers.
  \begin{itemize}
    \item To prove P10, let $c=a-b$, from \ref{P'10}, P10
    follows.
    \item To prove P11, let $a,b\in P$; it is
    sufficient to prove that $a+b>0$. From \ref{P'10},
    we divide the proof into three subscases:\par
    \note{Case 1: }$a=b$ \\
    Then $a+b=b+b>0+b>0$, where the first inequality follows
    from \ref{P'12}. By \ref{P'11}, $a+b>0$.\par
    \note{Case 2: }$a<b$ \\
    Then $a+b>a+a>0+a>0$, where the first and second inequality
    follow from \ref{P'12}. By applying \ref{P'11} twice,
    $a+b>0$. \par
    \note{Case 3: }$a>b$ \\
    Interchanging the role of $a$ and $b$, we have the result.
    \item To prove P12, let $a,b\in P$; it is
    sufficient to prove that $a\cdot b>0$. From \ref{P'10},
    we divide the proof into three subcases:\par
    \note{Case 1: }$a=b$ \\
    Then $a\cdot b=b\cdot b>0\cdot b=0$, where the first
    inequality follows from \ref{P'13} and the equality after
    which is from (P9).\par
    \note{Case 2: }$a<b$ \\
    Then $b\cdot a>a\cdot a>0\cdot a=0$, where the first
    and second inequality is from \ref{P'13}. By \ref{P'11},
    $a\cdot b>0$. \par
    \note{Case 3: }$a>b$ \\
    Interchanging $a$ and $b$ returns us to case 2, which
    yields the result.
  \end{itemize}
\end{solution}

\begin{pr}
  Express each of the following with at least one less pair
  of absolute value signs.
  \begin{enumerate}[label=(\roman*)]
    \item $|\sqrt{2}+\sqrt{3}-\sqrt{5}+\sqrt{7}|$
    \item $|(|a+b|-|a|-|b|)|$
    \item $|(|a+b|+|c|-|a+b+c|)|$
    \item $|(|\sqrt{2}+\sqrt{3}|-|\sqrt{5}-\sqrt{7}|)|$
  \end{enumerate}
\end{pr}

\begin{solution}
  \begin{enumerate}[label=(\roman*)]
    \item Note $\sqrt{7}-\sqrt{5}>0$, hence
    \begin{equation*}
      |\sqrt{2}+\sqrt{3}-\sqrt{5}+\sqrt{7}|=
      \sqrt{2}+\sqrt{3}-\sqrt{5}+\sqrt{7}
    \end{equation*}
    \item Since $|a+b|-|a|-|b|\leq 0$,
    \begin{equation*}
      |(|a+b|-|a|-|b|)|=|a|+|b|-|a+b|
    \end{equation*}
    \item Since $|a+b+c|\leq|a+b|+|c|$,
    \begin{equation*}
      |(|a+b|+|c|-|a+b+c|)|=|a+b|+|c|-|a+b+c|
    \end{equation*}
    \item
    \begin{equation*}
      |(|\sqrt{2}+\sqrt{3}|-|\sqrt{5}-\sqrt{7}|)|
      =|\sqrt{2}+\sqrt{3}-\sqrt{7}+\sqrt{5}|
    \end{equation*}
  \end{enumerate}
\end{solution}

\begin{pr}
  Express each of the following without absolute value
  signs, treating various cases separately when necessary.
  \begin{enumerate}[label=(\roman*)]
    \item $|a+b|-|b|$
    \item $|(|x|-1)|$
    \item $|x|-|x^2|$
    \item $a-|(a-|a|)|$
  \end{enumerate}
\end{pr}
\pagebreak
\begin{solution}
  \begin{enumerate}[label=(\roman*)]
    \item We divide into four cases:
    \begin{align}
      a\geq0 \quad \text{and} \quad b\geq0 \tag{Case 1} \\
      a\leq0 \quad \text{and} \quad b\leq0 \tag{Case 2} \\
      a\geq0 \quad \text{and} \quad b\leq0 \tag{Case 3} \\
      a\leq0 \quad \text{and} \quad b\geq0 \tag{Case 4}
    \end{align}
    In Case 1 and Case 2, we have $|a+b|-|b|=a$
    since $|a+b|\leq|a|+|b|$.\par
    In Case 3,
    if $a+b\geq0$, then
    \begin{equation*}
      |a+b|-|b|=(a+b)-(-b)=a+b+b=2b
    \end{equation*}
    If $a+b\leq0$, then
    \begin{equation*}
      |a+b|-|b|=(-a-b)-(-b)=-a+(-b)+b=-a
    \end{equation*}
    In Case 4, if $a+b\geq0$, then
    \begin{equation*}
      |a+b|-|b|=(a+b)-(b)=a
    \end{equation*}
    If $a+b\leq0$, then
    \begin{equation*}
      |a+b|-|b|=-a+(-b)+(-b)=-a-2b
    \end{equation*}
    \item We make the problem into 4 cases.
    \begin{align}
      x\geq1 \tag{Case 1} \\
      0\leq x\leq 1 \tag{Case 2} \\
      -1\leq x\leq 0 \tag{Case 3} \\
      x\leq -1 \tag{Case 4}
    \end{align}
    In Case 1, $|(|x|-1)|=x-1$. \par
    In Case 2, $|(|x|-1)|=1-x$. \par
    In Case 3, $|(|x|-1)|=x+1$. \par
    In Case 4, $|(|x|-1)|=-(x+1)$.
    \item Since $x^2\geq0$, $|x|-|x^2|=|x|-x^2$. \par
    If $x\geq0$, then $|x|-x^2=x(1-x)$. If $x\leq0$,
    then\\ $|x|-x^2=-x+(-x^2)=-x(1+x)$.
    \item Note that $|a|\geq a$. Hence,
    \begin{equation*}
      a-|(a-|a|)|=a+a-|a|=2a-|a|
    \end{equation*}
    We have two cases,
    \par
    \note{Case 1: }$a\geq0$
    \begin{equation*}
      2a-|a|=2a-a=a
    \end{equation*}
    \note{Case 2: }$a\leq0$
    \begin{equation*}
      2a-|a|=2a+a=3a
    \end{equation*}
  \end{enumerate}
\end{solution}

\begin{pr}
  Find all numbers $x$ for which
  \begin{enumerate}[label=(\roman*)]
    \item $|x-3|=8$
    \item $|x-3|<8$
    \item $|x+4|<2$
    \item $|x-1|+|x-2|>1$
    \item $|x-1|+|x+1|<2$
    \item $|x-1|+|x+1|<1$
    \item $|x-1|\cdot|x+1|=0$
    \item $|x-1|\cdot|x+2|=3$
  \end{enumerate}
\end{pr}

\begin{solution}
  \begin{enumerate}[label=(\roman*)]
    \item \begin{IEEEeqnarray*}{rCl}
      x-3=8 & \lor & x-3=-8 \\
      x=11 & \lor & x=-5
    \end{IEEEeqnarray*}
    \item Then $-8<x-3<8$. Hence, $-5<x<11$.
    \item Then $-2<x+4<2$. Hence, $-6<x<-2$.
    \item If $1\leq x\leq 2$, then the inequality becomes
    $(x-1)+(2-x)=1$. If $x>2$, then $2x-3>1$, which is
    $x>2$. If $x<1$, then $-2x+3>1$, which is $x<1$.
    Therefore, either $x>2$ or $x<1$ satisfies the inequality.
    \item If $-1\leq x\leq 1$, then $(1-x)+(x+1)=2$. If
    $x>1$, then $x<1$, which is contradictory. If $x<-1$,
    then $(1-x)+(-x-1)=-2x<2$ only if $x>-1$, which is
    contradictory. Hence, there is no $x$ to satisfy the
    inequality.
    \item It is implied from above that
    \begin{equation*}
      |x-1|+|x+1| \geq 2
    \end{equation*}
    Therefore, there is no $x$ satisfying the inequality.
    \item Either $x=1$ or $x=-1$.
    \item If $-2\leq x\leq 1$, then we obtain $x^2+x+1>0$.
    Hence, in either $x<-2$ or $x>1$, we have to solve
    the equation $x^2+x-5=0$, whose solution is either
    $x=\dfrac{-1+\sqrt{21}}{2}$ or
    $x=\dfrac{-1-\sqrt{21}}{2}$.
  \end{enumerate}
\end{solution}

\begin{pr} \label{1.12}% Prob12
  Prove the following:
  \begin{enumerate}[label=(\roman*)]
    \item \label{1.12:i}
    $|xy|=|x|\cdot|y|$
    \item \label{1.12:ii}
    $\left|\dfrac{1}{x}\right|=\dfrac{1}{|x|}$,
    if $x\neq0$. (The
    best way to do this is to remember what\\ $|x|^{-1}$ is.)
    \item $\dfrac{|x|}{|y|}=\left|\dfrac{x}{y}\right|$,
    if $y\neq0$.
    \item $|x-y|\leq|x|+|y|$ (Give a very short proof.)
    \item \label{1.12:v}
    $|x|-|y|\leq|x-y|$ (A very short proof is possible,
    if you write things in the right way.)
    \item $|(|x|-|y|)|\leq|x-y|$ (Why does this follow
    immediately from \ref{1.12:v}?)
    \item $|x+y+z|\leq|x|+|y|+|z|$. Indicate when equality
    holds, and prove your statement.
  \end{enumerate}
\end{pr}

\begin{solution}
  \begin{enumerate}[label=(\roman*)]
    \item We have 4 cases,
    \begin{align*}
      x\geq0 \quad y\geq0 \tag{1} \\
      x\geq0 \quad y\leq0 \tag{2} \\
      x\leq0 \quad y\geq0 \tag{3} \\
      x\leq0 \quad y\leq0 \tag{4}
    \end{align*}
    In (1), $|x|\cdot|y|=xy=|xy|$ \par
    In (4), $|x|\cdot|y|=(-x)(-y)=xy=|xy|$ \par
    In (3), $|x|\cdot|y|=(-x)(y)=-(xy)=|xy|$ \par
    In (2), interchanging $x$ and $y$ leads to (3).
    \item Since $x\neq0$, there exists $|x|^{-1}$ such that
    \begin{equation*}
      |x||x|^{-1}=1=|x|\left|\frac{1}{x}\right|
    \end{equation*}
    where the second equality is by \ref{1.12:i}. Dividing
    both sides by $|x|$, we have the result.
    \item Since $y\neq0$, from \ref{1.12:ii}, we immediately
    have
    \begin{equation*}
      \left|\frac{1}{y}\right|=\frac{1}{|y|}
    \end{equation*}
    Hence, applying \ref{1.12:ii} once more,
    \begin{equation*}
      \left|\frac{x}{y}\right|=|x|\left|\frac{1}{y}\right|
      =\frac{|x|}{|y|}
    \end{equation*}
    \item Note that,
    \begin{equation*}
      |x-y|=|x+(-y)|\leq|x|+|-y|=|x|+|y|
    \end{equation*}
    where the last equality follows from \ref{1.12:i}.
    \item Note that,
    \begin{equation*}
      |x-y+y|\leq|x-y|+|y|
    \end{equation*}
    Therefore, $|x|-|y|\leq|x-y|$.
    \item Let the first term be $y$ and the second term
    be $y-x$. Applying \ref{1.12:v}, we have
    \begin{equation*}
      |y|-|y-x|\leq|x|
    \end{equation*}
    Hence, $-|x-y|\leq|x|-|y|$. Combining with \ref{1.12:v}
    gives $|(|x|-|y|)|\leq|x-y|$.
    \item Notice the pattern,
    \begin{equation*}
      |x+y+z|\leq|x+y|+|z|\leq|x|+|y|+|z|
    \end{equation*}
    the equality holds only if either $x,y,z$ have the same
    sign or at least two of them must be equal to $0$.
    It is easy to verify this.\\
    Suppose not, then both $x,y,z$ have different signs and
    at most one of them is $0$. If the latter is true,
    then, w.l.o.g, suppose $z=0$, then $x,y$ have different
    sign, and we are done. If none of them is $0$, then,
    w.l.o.g, suppose $z<0$ and pick $z$ such that
    $x+y<-z$. Then,
    \begin{equation*}
      |x+y+z|=-(x+y+z)=-x-y-z<|x|+|y|+|z|
    \end{equation*}
    where inequality must follow since $x,y\neq0$.
  \end{enumerate}
\end{solution}

\begin{pr}
  The maximum of two numbers $x$ and $y$ is denoted by
  $max(x,y)$. Thus $max(-1,3)=max(3,3)=3$ and
  $max(-1,-4)=max(-4,-1)=-1$. The minimum of $x$ and $y$
  is denoted by $min(x,y)$. Prove that
  \begin{equation*}
    max(x,y)=\frac{x+y+|y-x|}{2},
  \end{equation*}
  \begin{equation*}
    min(x,y)=\frac{x+y-|y-x|}{2}.
  \end{equation*}
  Derive the formula for $max(x,y,z)$ and $min(x,y,z)$,
  using, for example
  \begin{equation*}
    max(x,y,z)=max(x,max(y,z)).
  \end{equation*}
\end{pr}
\pagebreak
\begin{solution}
  Assume that $x\geq y$, we want to prove that $max(x,y)=x$.
  \begin{equation*}
    max(x,y)=\frac{x+y+|y-x|}{2}=\frac{x+y+x-y}{2}
    =\frac{2x}{2}=x
  \end{equation*}
  Similarly, we need $min(x,y)=y$.
  \begin{equation*}
    min(x,y)=\frac{x+y-|y-x|}{2}=\frac{x+y-(x-y)}{2}
    =\frac{x+y-x+y}{2}=\frac{2y}{2}=y
  \end{equation*}
  Let $max(x,y,z)=max(x,max(y,z))$. Then
  \begin{IEEEeqnarray*}{rCl}
    max(x,y,z) & = & \frac{x+max(y,z)+|max(y,z)-x|}{2} \\
               & = & \frac{x+\dfrac{y+z+|z-y|}{2}+\left|
    \dfrac{y+z+|z-y|}{2}-x\right|}{2} \\
               & = & \frac{2x+y+z+|z-y|+\left|
    y+z+|z-y|-2x\right|}{4}
  \end{IEEEeqnarray*}
  Similarly,
  \begin{IEEEeqnarray*}{rCl}
    min(x,y,z) & = & \frac{2x+y+z-|z-y|-\left|
    y+z-|z-y|-2x\right|}{4}
  \end{IEEEeqnarray*}
\end{solution}

\begin{pr} \label{1.14}%1.14
  \begin{enumerate}[label=(\alph*)]
    \item \label{1.14:i}
    Prove that $|a|=|-a|$. (The trick is not to become
    confused by too many cases. First prove the statement for
    $a\geq0$. Why is it then obvious for $a\leq0$?)
    \item \label{1.14:ii}
    Prove that $-b\leq a\leq b$ if and only if
    $|a|\leq b$. In particular, it follows that
    $-|a|\leq a\leq|a|$.
    \item Use this fact to give a new proof that
    $|a+b|\leq|a|+|b|$.
  \end{enumerate}
\end{pr}

\begin{solution} %1.14 solution
  \begin{enumerate}[label=(\alph*)]
    \item \autoref{1.12}\ref{1.12:i} easily tells us that
    \begin{equation*}
      |-a|=|(-1)a|=|-1||a|=1|a|=|a|
    \end{equation*}
    \item
    If $a\geq0$, then $a\leq b$. If $a\leq0$,
    $-a\leq b$ follows from $a\geq -b$.
    Therefore, $|a|\leq b$. Conversely,
    suppose $|a|\leq b$. Then it is certain $a\leq b$
    since $a\leq|a|\leq b$. From \ref{1.14:i}, $|-a|\leq b$,
    and hence $a\geq -b$. We conclude that $-b\leq a\leq b$.
    Note that since $|a|\leq|a|$, $-|a|\leq a\leq|a|$.
    \item
    Because we have $-|a|\leq a\leq|a|$
    and $-|b|\leq b\leq|b|$, by \autoref{1.5}\ref{1.5:i},
    we obtain $-(|a|+|b|)\leq a+b\leq|a|+|b|$. From
    \ref{1.14:ii}, we arrive at the
    conclusion $|a+b|\leq|a|+|b|$.
  \end{enumerate}
\end{solution}

\pagebreak

\begin{pr}[*] %1.15
  Prove that if $x$ and $y$ are not both $0$, then
  \begin{IEEEeqnarray*}{rCl}
    x^2 + xy + y^2 & > & 0 \\
    x^4+x^3y+x^2y^2+xy^3+y^4 & > & 0
  \end{IEEEeqnarray*}
  Hint: Use problem 1.
\end{pr}

\begin{solution} %1.15 solution
  For the first part, note that
  \begin{equation*}
    x^2+xy+y^2=x^2+2\cdot x\cdot \frac{1}{2}y
    +\frac{1}{4}y^2-\frac{1}{4}y^2
    +y^2=\left(x+\frac{1}{2}y\right)^2+\frac{3}{4}y^2 > 0
  \end{equation*}
  For the second part, if $x=y$, then the left-hand side
  is $5x^4>0$. Hence, suppose $x\neq y$.
  From \autoref{1.1}\ref{1.1:v},
  \begin{equation*}
    x^5-y^5=(x-y)(x^4+x^3y+x^2y^2+xy^3+y^4)\neq0
  \end{equation*}
  If $x>y$, then $x^5>y^5$ by \autoref{1.6}\ref{1.6:b}.
  This implies that the second term must be greater than $0$.
  Conversely, $x<y\Rightarrow x^5<y^5$ implies that
  it must be greater than $0$.
\end{solution}

\begin{pr}[*] \label{1.16}%Problem 1.16
  \begin{enumerate}[label=(\alph*)]
    \item \label{1.16:a}
    Show that
    \begin{IEEEeqnarray*}{rCl}
      (x+y)^2 & = & x^2+y^2 \ \text{ only when }
      x=0 \text{ or } y=0,\\
      (x+y)^3 & = & x^3+y^3 \ \text{ only when }
      x=0 \text{ or } y=0 \text{ or } x=-y.
    \end{IEEEeqnarray*}
    \item \label{1.16:b}
    Using the fact that
    \begin{equation*}
      x^2 + 2xy + y^2 = (x+y)^2\geq0,
    \end{equation*}
    show that $4x^2+6xy+4y^2>0$ unless $x$ and $y$ are
    both $0$.
    \item \label{1.16:c}
    Use part \ref{1.16:b} to find out when
    $(x+y)^4 = x^4 + y^4$.
    \item \label{1.16:d}
    Find out when $(x+y)^5=x^5+y^5$. Hint: From the assumption
    $(x+y)^5=x^5+y^5$ you should be able to derive the
    equation $x^3+2x^2y+2xy^2+y^3=0$, if $xy\neq0$. This
    implies that $(x+y)^3=x^2y+xy^2=xy(x+y)$.
  \end{enumerate}
  You should know be able to make a good guess as to when
  $(x+y)^n=x^n+y^n$; the proof is contained in Problem
  11.57 % \ref{11.57}
\end{pr}

\begin{solution}
  \begin{enumerate}[label=(\alph*)]
    \item For the first part,
    \begin{equation*}
      (x+y)^2=x^2+2xy+y^2
    \end{equation*}
    Hence, $(x+y)^2=x^2+y^2$ only when $x=0$ or $y=0$.
    For the second part, from \autoref{1.1}\ref{1.1:vi},
    \begin{IEEEeqnarray*}{rCl}
      (x+y)^3-(x+y)(x^2-xy+y^2) & = & 0 \\
      (x+y)(xy) & = & 0
    \end{IEEEeqnarray*}
    which is true only when $x=0$ or $y=0$ or $x=-y$.
    \item Note that $4x^2+6xy+4y^2=
    \underbrace{3(x+y)^2}_\text{$\geq0$}
    +\underbrace{x^2+y^2}_\text{$>0$}>0$ unless
    $x=0$ and $y=0$.
    \item Let us expand $(x+y)^4$.
    \begin{IEEEeqnarray*}{rCl}
      (x+y)^2(x+y)^2 & = & (x^2+2xy+y^2)(x^2+2xy+y^2) \\
      & = & x^4 + 4x^3y+ 6x^2y^2+ 4xy^3 + y^4 \\
      & = & x^4 + y^4 + xy(4x^2 + 6xy + 4y^2)
    \end{IEEEeqnarray*}
    Hence, $(x+y)^4=x^4+y^4$ only when $x=0$ or $y=0$,
    by part \ref{1.16:b}.
    \item Let us expand $(x+y)^5$.
    \begin{IEEEeqnarray*}{rCl}
      (x+y)^4(x+y) & = & x^5 + y^5 + xy(x+y)(4x^2 + 6xy + 4y^2)
      +xy(x^3 + y^3) \\
      & = & x^5 + y^5 + 5xy(x+y)(x^2 - xy + y^2)
    \end{IEEEeqnarray*}
    If $xy\neq0$ and $x+y\neq0$, let $z=-y$, by \autoref{1.6}%
    \ref{1.6:b}, $x^3\neq z^3$. Hence, $x^2 - xy + y^2\neq0$.
    Therefore, $(x+y)^5=x^5+y^5$ only when $x=0$ or $y=0$
    or $x=-y$.
    \begin{ab}
      Hence, for $(x+y)^n = x^n + y^n$, if $n$ is even,
      then $x=0$ or $y=0$. If $n$ is odd, then $x=0$ or
      $y=0$ or $x=-y$.
    \end{ab}
  \end{enumerate}
\end{solution}

\begin{pr} %Problem 1.17
  \begin{enumerate}[label=(\alph*)]
    \item
    Find the smallest possible value of $2x^2 - 3x + 4$.
    Hint: ``Complete the square'', i.e., write
    $2x^2 - 3x + 4 = 2(x-3/4)^2 + \text{?}$
    \item
    Find the smallest possible value of
    $x^2 - 3x + 2y^2 + 4y + 2$.
    \item
    Find the smallest possible value of
    $x^2 + 4xy + 5y^2 - 4x - 6y + 7$.
  \end{enumerate}
\end{pr}

\begin{solution} %Solution 1.17
  \begin{enumerate}[label=(\alph*)]
    \item Since $2x^2 - 3x + 4 = 2(x^2 - \dfrac{3}{2}x + 2)$,
    \begin{IEEEeqnarray*}{rCl}
      2(x^2 - 2\cdot x\frac{3}{4} + \frac{9}{16} - \frac{9}{16}
      + 2) & = & 2(x-\frac{3}{4})^2 + \frac{23}{8}
    \end{IEEEeqnarray*}
    Hence the minimum value is $\dfrac{23}{8}$ when
    $x=\dfrac{3}{4}$.
    \item \begin{equation*}
      x^2 - 3x + \frac{9}{4} - \frac{9}{4}
      + 2(y^2 + 2y + 1)
      = \left(x - \frac{3}{2}\right)^2 + 2(y + 1)^2
      - \frac{9}{4}
  \end{equation*}
  The minimum value is $-\dfrac{9}{4}$ when $x=\dfrac{3}{2}$
  and $y=-1$.
  \item
  \begin{IEEEeqnarray*}{rC}
    & \frac{1}{2}x^2 + 4xy + 8y^2 - 3y^2 - 6y + 7
    +\frac{1}{2}x^2 - 4x \\
    = & \frac{1}{2}(x^2 + 8xy + 16y^2) - 3(y^2 + 2y +1)
    + \frac{1}{2}(x^2 - 8x + 16) + 2 \\
    = & \frac{1}{2}(x + 4y)^2 - 3(y + 1)^2
    +\frac{1}{2}(x - 4)^2 + 2
  \end{IEEEeqnarray*}
  Therefore, the minimum value is $2$ when $x=4$ and
  $y=-1$.
  \end{enumerate}
\end{solution}


\begin{pr} \label{1.18}%Problem 1.18
  \begin{enumerate}[label=(\alph*)]
    \item Suppose that $b^2 - 4c \geq 0$. Show that
    the numbers
    \begin{align*}
      \frac{-b+\sqrt{b^2-4c}}{2},\quad
      \frac{-b-\sqrt{b^2-4c}}{2}
    \end{align*}
    both satisfy the equation $x^2 + bx + c = 0$.
    \item \label{1.18:b}
    Suppose that $b^2 - 4c < 0$. Show that there
    are no numbers $x$ satisfying $x^2 + bx + c = 0$;
    in fact, $x^2 + bx + c > 0$ for all $x$. Hint:
    Complete the square.
    \item Use this fact to give another proof that
    if $x$ and $y$ are not both $0$, then
    $x^2 + xy + y^2 > 0$.
    \item For which number $\alpha$ is it true that
    $x^2 + \alpha xy + y^2 > 0$ whenever $x$ and $y$
    are not both $0$?
    \item Find the smallest possible value of
    $x^2 + bx + c$ and of $ax^2 + bx + c$, for
    $a > 0$.
  \end{enumerate}
\end{pr}

\begin{solution}
  \begin{enumerate}[label=(\alph*)]
    \item Substitution immediately gives the desired result.
    \item
    \begin{equation*}
      x^2 + bx + c = x^2 + bx + \frac{b^2}{4}
      - \frac{b^2}{4} + c
    \end{equation*}
    which immediately yields
    $\left(x + \dfrac{b}{2}\right)^2
    + \dfrac{[-(b^2 - 4c)]}{4} > 0$ for all $x$ since
    $b^2 - 4c < 0$.
    \item If $y = 0$, $x^2>0$. Suppose not,
    using \ref{1.18:b}, we obtain
    $-3y^2 < 0$. Hence, $x^2 + xy + y^2 > 0$.
    \item If $y = 0$, the result follows for all
    $\alpha$. Suppose $y\neq0$,
    using \ref{1.18:b}, we obtain $\alpha^2y^2
    - 4y^2 < 0$, which is $y^2(\alpha^2 - 4) < 0$.
    It follows that $-2 < \alpha < 2$.
    \item From \ref{1.18:b}, it follows that
    the minimum value of $x^2 + bx + c$ is
    $\dfrac{[-(b^2 - 4c)]}{4}$ when $x = -b/2$.
    Since $a > 0$, with the role of $b$ is now
    $b/a$ and of $c$ is $c/a$, we easily
    derive the result.
    \begin{equation*}
      x^2 + \frac{b}{a}x + \frac{c}{a} =
      \left(x + \frac{b}{2a}\right)^2
      + \frac{[-(b^2 - 4ac)]}{4a^2}
    \end{equation*}
    So its minimum value is $\dfrac{[-(b^2 - 4ac)]}{4a^2}$
    when $x = -\dfrac{b}{2a}$.
  \end{enumerate}
\end{solution}

\begin{pr} \label{1.19}
  The fact that $a^2\geq0$ for all numbers $a$, elementary
  as it may seem, is nevertheless the fundamental idea
  upon which most important inequalities are ultimately
  based. The great-granddaddy of all inequalities is the
  \note{Schwarz inequality}:
  \begin{equation*}
    x_1 y_1 + x_2 y_2 \leq \sqrt{x_1^2+x_2^2}
    \sqrt{y_1^2+y_2^2}.
  \end{equation*}
  (A more general form occurs in Problem 2.21)\quad%ref{2.21}
  The three proofs of the Schwarz inequality outlined
  below have only one thing in common--their reliance
  on the fact that $a^2\geq0$ for all $a$.
  \begin{enumerate}[label=(\alph*)]
    \item \label{1.19:a}
    Prove that if $x_1 = \lambda y_1$ and
    $x_2 = \lambda y_2$ for some number $\lambda$, then
    equality holds in Schwarz inequality. Prove the same
    thing if $y_1 = y_2 = 0$. Now suppose that $y_1$ and
    $y_2$ are not both $0$, and that there is no number
    $\lambda$ such that $x_1 = \lambda y_1$ and
    $x_2 = \lambda y_2$. Then
    \begin{IEEEeqnarray*}{rCl}
      0 & < & (\lambda y_1 - x_1)^2 + (\lambda y_2 - x_2)^2 \\
        & = & \lambda^2(y_1^2 + y_2^2) - 2\lambda
        (x_1y_1 + x_2y_2) + (x_1^2 + x_2^2).
    \end{IEEEeqnarray*}
    Using \autoref{1.18}, complete the proof of the
    Schwarz inequality.
    \item \label{1.19:b}
    Prove the Schwarz inequality by using
    $2xy \leq x^2 + y^2$ (how is this derived?) with
    \begin{align*}
      x=\frac{x_i}{\sqrt{x_1^2+x_2^2}}, \quad
      y=\frac{y_i}{\sqrt{y_1^2+y_2^2}},
    \end{align*}
    first for $i = 1$ and then for $i = 2$.
    \item \label{1.19:c}
    Prove the Schwarz inequality by first proving
    that
    \begin{equation*}
      (x_1^2+x_2^2)(y_1^2+y_2^2)
      =(x_1y_1+x_2y_2)^2 + (x_1y_2 - x_2y_1)^2.
    \end{equation*}
    \item \label{1.19:d}
    Deduce, from each of these three proofs,
    that equality holds only when $y_1=y_2=0$ or when
    there is a number $\lambda$ such that $x_1=\lambda y_1$
    and $x_2=\lambda y_2$.
  \end{enumerate}
\end{pr}

\begin{solution} %Solution 1.19
  \begin{enumerate}[label=(\alph*)]
    \item If $x_1=\lambda y_1$ and $x_2=\lambda y_2$
    for every $\lambda \geq 0$,
    \begin{IEEEeqnarray*}{rCl}
      \lambda(y_1^2 + y_2^2) & = &
      |\lambda|\sqrt{(y_1^2 + y_2^2)^2} \\
      & = & \lambda  (y_1^2 + y_2^2)
    \end{IEEEeqnarray*}
    Or if $y_1=y_2=0$, then equality holds since both sides
    are $0$. Otherwise, suppose that $y_1$ and $y_2$ are
    not both $0$, and there is no number $\lambda$ such
    that $x_1=\lambda y_1$ and $x_2=\lambda y_2$, then
    \begin{IEEEeqnarray*}{rCl}
      0 & < & \lambda^2(y_1^2+y_2^2)
      - 2 \lambda(x_1y_1 + x_2y_2) + (x_1^2 + x_2^2) \\
        & = & \lambda^2 - 2 \lambda
        \frac{x_1y_1 + x_2y_2}{y_1^2 + y_2^2}
        + \frac{x_1^2 + x_2^2}{y_1^2 + y_2^2}
    \end{IEEEeqnarray*}
    This holds only when, by \autoref{1.18}\ref{1.18:b},
    \begin{equation*}
      \frac{4(x_1y_1 + x_2y_2)^2}{(y_1^2 + y_2^2)^2}
      + \frac{[-4(x_1^2 + x_2^2)(y_1^2 + y_2^2)]}%
      {(y_1^2 + y_2^2)^2} < 0
    \end{equation*}
    which only holds when
    \begin{equation*}
      x_1y_1 + x_2y_2 < \sqrt{x_1^2 + x_2^2}
      \sqrt{y_1^2 + y_2^2}
    \end{equation*}
    since $a\leq|a|$ for all $a$.
    \item Note that $(x-y)^2\geq0$. For $i = 1$,
    \begin{IEEEeqnarray}{rCl}
      \frac{x_1^2}{x_1^2 + x_2^2}
     +\frac{y_1^2}{y_1^2 + y_2^2} & \geq &
     2\cdot\frac{x_1 y_1}{\sqrt{x_1^2 + x_2^2}%
                          \sqrt{y_1^2 + y_2^2}} \label{1.19.1}
    \end{IEEEeqnarray}
    For $i = 2$,
    \begin{IEEEeqnarray}{rCl}
      \frac{x_2^2}{x_1^2 + x_2^2}
     +\frac{y_2^2}{y_1^2 + y_2^2} & \geq &
     2\cdot\frac{x_2 y_2}{\sqrt{x_1^2 + x_2^2}%
                          \sqrt{y_1^2 + y_2^2}} \label{1.19.2}
    \end{IEEEeqnarray}
    (\ref{1.19.1})$+$(\ref{1.19.2}), we derive
    \begin{equation*}
      x_1y_1 + x_2y_2 \leq \sqrt{x_1^2 + x_2^2}%
      \sqrt{y_1^2 + y_2^2}
    \end{equation*}
    \item
    \begin{IEEEeqnarray*}{lC}
        & (x_1^2 + x_2^2)(y_1^2 + y_2^2) \\
      = & (x_1^2 y_1^2 + 2 x_1 y_1 x_2 y_2 + x_2^2 y_2^2)
      + (x_1^2 y_2^2 - 2 x_1 y_2 x_2 y_1 + x_2^2 y_1^2) \\
      = & (x_1 y_1 + x_2 y_2)^2
      + (x_1 y_2 - x_2 y_1)^2
    \end{IEEEeqnarray*}
    Note that $(x_1 y_2 - x_2 y_1)^2 \geq 0$. Hence,
    \begin{equation*}
      x_1 y_1 + x_2 y_2 \leq \sqrt{x_1^2 + x_2^2}
                             \sqrt{y_1^2 + y_2^2}
    \end{equation*}
    for $a \leq |a|$ for all $a$.
    \item In \ref{1.19:a}, it is obvious; the proof is
    based on the separation of two cases, $a^2 = 0$ and
    $a^2 > 0$. In \ref{1.19:b}, equality occurs only when
    $x=y$; by construction, $y_1 = y_2 = 0$ or, if not,
    \begin{IEEEeqnarray*}{rCl}
      \frac{x_1}{\sqrt{x_1^2 + x_2^2}} & = &
      \frac{y_1}{\sqrt{y_1^2 + y_2^2}} \\
      \frac{x_2}{\sqrt{x_1^2 + x_2^2}} & = &
      \frac{y_2}{\sqrt{y_1^2 + y_2^2}}
    \end{IEEEeqnarray*}
    implies that for
    \begin{equation*}
      \lambda = \frac{\sqrt{x_1^2 + x_2^2}}
                     {\sqrt{y_1^2 + y_2^2}}
    \end{equation*}
    $x_1 = \lambda y_1$ and
    $x_2 = \lambda y_2$.
    \par
    In \ref{1.19:c}, equality occurs only when
    $(x_1 y_2 - x_2 y_1)^2 = 0$.
  \end{enumerate}
\end{solution}
