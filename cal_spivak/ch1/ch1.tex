\begin{pr}
  Prove the following:
  \begin{enumerate}[label=(\roman*)]
    \item If $ax=a$ for some number $a\neq0$, then
    $x=1$
    \item $x^2-y^2=(x-y)(x+y)$
    \item If $x^2=y^2$, then $x=y$ or $x=-y$
    \item \label{1:iv}
    $x^3-y^3=(x-y)(x^2+xy+y^2)$
    \item $x^n-y^n=(x-y)(x^{n-1}+x^{n-2}y+\dots+xy^{n-2}
    +y^{n-1})$
    \item $x^3+y^3=(x+y)(x^2-xy+y^2)$ (There is a particularly
    easy way to do this using \ref{1:iv}, and it will show
    you how to find a factorization for $x^n+y^n$ whenever
    n is odd.)
  \end{enumerate}
\end{pr}

\begin{solution}
  \begin{enumerate}[label=(\roman*)]
    \item By (P7)(Existence of multiplicative inverses),
    there exists $a^{-1}$ such that,
    \begin{IEEEeqnarray*}{rCl}
      (a^{-1}\cdot a)x & = & (a^{-1}\cdot a) \\
      x & = & 1 \\
    \end{IEEEeqnarray*}
    \item \label{(ii)} By (P9) for 2 times,
    \begin{IEEEeqnarray*}{+rCl+x*}
      (x-y)(x+y) & \stackrel{1}{=} & x\cdot(x+y)+(-y)\cdot(x+y)\\
      & \stackrel{2}{=} & x\cdot x+x\cdot y+(-y)\cdot x
      +(-y)\cdot y \\
      & = & x^2 +x\cdot y+[-(x\cdot y)]+[-(y^2)] \\
      & = & x^2-y^2 \\
    \end{IEEEeqnarray*}
    \item From \ref{(ii)} and since $x^2=y^2$,
    \begin{equation*}
      x^2-y^2=(x-y)(x+y)=0
    \end{equation*}
    This means $(x-y)=0 \lor (x+y)=0$, which is $x=y \lor x=-y$
    \item Starting with the right-hand side,
    \begin{IEEEeqnarray*}{rCl}
      (x-y)(x^2+xy+y^2) & = &
      x\cdot(x^2+xy+y^2)+(-y)\cdot(x^2+xy+y^2)\\
      & = &
      x^3+x^2y+xy^2+[-(x^2y)]+[-(xy^2)]+[-(y)^3] \\
      & = &
      x^3-y^3 \\
    \end{IEEEeqnarray*}
    \item I propose two solutions for this problem.
    The first one is the direct right-hand side manipulation,
    while the latter is done by induction.
    \renewcommand{\qedsymbol}{\textsl Q.E.D}
    \begin{proof}[The first solution]
      \begin{IEEEeqnarray*}{+l+x*}
        (x-y)(x^{n-1}+x^{n-2}y+\cdots+xy^{n-2}+y^{n-1}) \\=
        x^n+x^{n-1}y+\cdots+x^2y^{n-2}+xy^{n-1} \\
        +[-(x^{n-1}y)]+[-(x^{n-2}y^2)]+\cdots+
        [-(xy^{n-1})]+[-(y^n)] \\
        =x^n-y^n \\
        & \qedhere
      \end{IEEEeqnarray*}
    \end{proof}
    \begin{proof}[The second solution]
      Let n=1, then indeed $x-y=x-y$. Suppose the statement
      holds true for $n=k$ with $k\in \mathbb{N}$, that is
      \begin{equation*}
        x^k-y^k=(x-y)(x^{k-1}+x^{k-2}y+\cdots+xy^{k-2}+y^{k-1})
      \end{equation*}
    is true. To finish the proof, we need to prove
    \begin{equation*}
      x^{k+1}-y^{k+1}=(x-y)(x^k+x^{k-1}y+\cdots+xy^{k-1}+y^k)
    \end{equation*}
    That is, the statement holds for $n=k$. Starting from
    the left hand side,
    \begin{IEEEeqnarray*}{*x+C+x*}
      & x^{k+1}-y^{k+1} \\
      = & x^{k+1}-x^ky+x^ky-y^{k+1} \\
      = & x^k(x-y)+y(x^k-y^k) \\
      = & x^k(x-y)+y(x-y)(x^{k-1}+x^{k-2}y+\cdots+
      xy^{k-2}+y^{k-1})\\
      = & (x-y)[x^k+y(x^{k-1}+x^{k-2}y+\cdots+xy^{k-2}+y^{k-1})]
      \\
      = & (x-y)(x^k+x^{k-1}y+x^{k-2}y^2+\cdots+xy^{k-1}+
      y^k) \\
      & & \qedhere
    \end{IEEEeqnarray*}
    \end{proof}
    \item We will use \ref{1:iv} in our proof,
    \begin{IEEEeqnarray*}{*x+C+x*}
      & x^3+y^3 \\
    = & x^3-y^3+2y^3 \\
    = & (x-y)(x^2+xy+y^2)+2y[(x^2+xy+y^2)+(-x)(x+y)] \\
    = & (x+y)(x^2+xy+y^2)+2[-(xy)](x+y) \\
    = & (x+y)(x^2-xy+y^2) \\
    \end{IEEEeqnarray*}
  \end{enumerate}
\end{solution}

\begin{pr}
  What is wrong with the following ``proof''? Let $x=y$. Then
  \begin{IEEEeqnarray*}{rCl}
    x^2 & = & xy, \\
    x^2-y^2 & = & xy -y^2, \\
    (x+y)(x-y) & = & y(x-y), \\
    x+y & = & y, \\
    2y & = & y, \\
    2 & = & 1.
  \end{IEEEeqnarray*}
\end{pr}

\begin{solution}
  Note that in the transition from line 3 to line 4,
  the author ``simplifies'' $(x-y)$ by
  dividing $(x-y)$ on both sides. This is wrong since $x-y=0$,
  and hence $1/0$ is undefined as implied by (P7)
  in the textbook.
\end{solution}

\begin{pr}
  Prove the following:
  \begin{enumerate}[label=(\roman*)]
    \item $\displaystyle{\frac{a}{b} = \frac{ac}{bc}}$,
    if $b,c\neq0$.
    \item
      $\displaystyle{\frac{a}{b}+\frac{c}{d}=\frac{ad+bc}{bd}}$,
    if $b,d\neq0$.
    \item
    $(ab)^{-1}=a^{-1}b^{-1},\text{ if }a,b\neq0$. (To do this
    you must remember the defining property of $(ab)^{-1}$.)
    \item
    $\dfrac{a}{b}\cdot\dfrac{c}{d}=\dfrac{ac}{db}$, if
    $b,d\neq0$.
    \item
    $\dfrac{a}{b}\Bigg/\dfrac{c}{d}=\dfrac{ad}{bc}$, if
    $b,c,d\neq0$.
    \item
    If $b,d\neq0$, then $\dfrac{a}{b}=\dfrac{c}{d}$ if and
    only if $ad=bc$. Also determine when $\dfrac{a}{b}=
    \dfrac{b}{a}$.
  \end{enumerate}
\end{pr}

\begin{solution}
  \begin{enumerate}[label=(\roman*)]
    \item \label{2:i}
    Until \ref{2:iii} is proved, the solution is to
    test the equality between two sides.
    \begin{IEEEeqnarray*}{rCl}
      a(b)^{-1} & = & (ac)(bc)^{-1}\\
      a[(b)^{-1}b] & = & (ac)(bc)^{-1}b \\
      (a^{-1}a) & = & (a^{-1}a)c(bc)^{-1}b \\
      1 & = & (bc)(bc)^{-1} = 1\\
    \end{IEEEeqnarray*}
    \item\label{p13:ii} Similar to the above,
    \begin{IEEEeqnarray*}{rCl}
      a(b)^{-1}+c(d)^{-1} & = & (ad+bc)(bd)^{-1} \\
      a(b)^{-1}bd+c(d)^{-1}bd & = & (ad+bc)[(bd)^{-1}(bd)]\\
      ad(b^{-1}b)+bc(d^{-1}d) & = & (ad+bc)\\
      ad+bc & = & ad+bc
    \end{IEEEeqnarray*}
    \item \label{2:iii}
    Since $a,b\neq0$, there exists $(ab)^{-1},a^{-1},
    b^{-1}$ such that,
    \begin{IEEEeqnarray*}{rCl}
      ab & = & ab \\
      (ab)^{-1}(ab) & = & (ab)^{-1}(ab)=1 \\
      (ab)^{-1}a(bb^{-1}) & = & b^{-1} \\
      (ab)^{-1}(aa^{-1}) & = & b^{-1}a^{-1} \\
      (ab)^{-1} & = & a^{-1}b^{-1} \\
    \end{IEEEeqnarray*}
    \item For $b,d\neq0$,
    \begin{equation*}
      \frac{a}{b}\cdot\frac{c}{d}=ab^{-1}cd^{-1}=ac(d^{-1}b^{-1})
      =ac(db)^{-1}=\frac{ac}{db}
    \end{equation*}
    where the next-to-last equality follows from \ref{2:iii}.
    \item I first establish for any number $a\neq0$,
    \begin{equation*}
      (a^{-1})^{-1}=a
    \end{equation*}
    Let $t=a^{-1}$, we want to prove $t^{-1}=a$. Observe that
    \begin{IEEEeqnarray*}{rCl}
      t & = & a^{-1} \\
      t\cdot(t)^{-1} & = & a^{-1}\cdot (t)^{-1} \\
      a\cdot 1 & = & (a\cdot a^{-1})\cdot (t)^{-1} \\
      a & = & (t)^{-1}
    \end{IEEEeqnarray*}
    From the left hand side of the statement,
    \begin{equation*}
      \frac{a}{b}\Bigg/\frac{c}{d}=a(b)^{-1}[c(d)^{-1}]^{-1}
      =a(b)^{-1}(c)^{-1}[(d)^{-1}]^{-1}
      =(ad)(bc)^{-1}=\frac{ad}{bc}
    \end{equation*}
    where the second and third equality follows both
    from \ref{2:iii}
    and the proof above.
    \item Using \ref{p13:ii},
    \begin{IEEEeqnarray*}{rCl}
      \frac{a}{b} & = & \frac{c}{d} \\
      \frac{a}{b}+(-\frac{c}{d}) & = & 0 \\
      \frac{ad-bc}{bd} & = & 0 \\
      ad & = & bc \\
    \end{IEEEeqnarray*}
    Now, put $c=b \land d=a$. It follows that
    $\dfrac{a}{b}=\dfrac{b}{a}$ if and only if
    $a^2=b^2$. It follows $(a-b)(a+b)=0$, or
    $a=b \lor a=-b$.
  \end{enumerate}
\end{solution}

\begin{pr}
  Find all numbers x for which
  \begin{enumerate}[label=(\roman*)]
    \item $4-x<3-2x$
    \item $5-x^2<8$
    \item $5-x^2<-2$
    \item $(x-1)(x-3)>0$ (When is a product of two numbers
    positive?)
    \item $x^2-2x+2>0$
    \item $x^2+x+1>2$
    \item $x^2-x+10>16$
    \item $x^2+x+1>0$
    \item $(x-\pi)(x+5)(x-3)>0$
    \item $(x-\sqrt[3]{2})(x-\sqrt{2})>0$
    \item $2^x<8$
    \item $x+3^x<4$
    \item $\dfrac{1}{x}+\dfrac{1}{1-x}>0$
    \nopagebreak[3]
    \item $\dfrac{x-1}{x+1}>0$
  \end{enumerate}
\end{pr}
\pagebreak
\begin{solution}
  \begin{enumerate}[label=(\roman*)]
    \item
    \begin{IEEEeqnarray*}{rCl}
      4-x & < & 3-2x \\
      4+(-x+2x) & < & 3+(-2x+2x) \\
      (-4+4)+x & < & -4+3 \\
      x & < & -1 \\
    \end{IEEEeqnarray*}
    \item
    \begin{IEEEeqnarray*}{rCl}
      5-x^2 & < & 8 \\
      5-8 & < & x^2 \\
      -3 & < & x^2
    \end{IEEEeqnarray*}
    Since $x^2\geq0\ \forall x\in\mathbb{R}$, the inequality
    holds $\forall x$.
    \item
    \begin{IEEEeqnarray*}{rCl}
      5-x^2 & < & -2 \\
      7 & < & x^2 \\
      0 & < & x^2-7=(x-\sqrt{7})(x+\sqrt{7})
    \end{IEEEeqnarray*}
    Hence, either $x>\sqrt{7}~\land\ x>-\sqrt{7}$
    or $x<\sqrt{7}~\land\ x<-\sqrt{7}$, which is
    $x>\sqrt{7}~\lor\ x<-\sqrt{7}$.
    \item
    \begin{IEEEeqnarray*}{rCl}
      (x-1)(x-3) & > & 0 \\
      (x>1~\land\ x>3) & \lor & (x<1~\land\ x<3)\\
      x>3 & \lor & x<1
    \end{IEEEeqnarray*}
    \item
    \begin{IEEEeqnarray*}{rCl}
      x^2-2x+2 & > & 0 \\
      (x^2-2x+1)+1 & > & 0 \\
      (x-1)^2+1 & > & 0
    \end{IEEEeqnarray*}
    Hence the inequality is satisfied $\forall\ x$.
    \item
    \begin{IEEEeqnarray*}{rCl}
      x^2+x+1 & > & 2 \\
      x^2+x-1 & > & 0 \\
      x^2+\left(\frac{1+\sqrt{5}}{2}\right)x
      +\left(\frac{1-\sqrt{5}}{2}\right)x
      +\left(\frac{(1-\sqrt{5})(1+\sqrt{5})}{4}\right)
      & > & 0 \\
      \left(x+\frac{1+\sqrt{5}}{2}\right)
      \left(x+\frac{1-\sqrt{5}}{2}\right) & > & 0 \\
      x>\left(\frac{\sqrt{5}-1}{2}\right) \lor
      x<\left(\frac{-(\sqrt{5}+1)}{2}\right)
    \end{IEEEeqnarray*}
    \item
    \begin{IEEEeqnarray*}{rCl}
      x^2-x+10 & > & 16 \\
      x^2-x-6 & > & 0 \\
      x^2-3x+2x-6 & > & 0 \\
      x(x-3)+2(x-3) & > & 0 \\
      (x+2)(x-3) & > & 0 \\
      x>3 & \lor & x<-2
    \end{IEEEeqnarray*}
    \item
    \begin{IEEEeqnarray*}{rCl}
      x^2+x+1 & > & 0 \\
      x^2+x+\frac{1}{4}-\frac{1}{4}+1 & > & 0 \\
      (x+\frac{1}{2})^2+\frac{3}{4} & > & 0
    \end{IEEEeqnarray*}
    which is true for all $x$.
    \item Divide the problem into two cases: $x>\pi$
    and $x<\pi$.
    \par
    \note{Case 1: }$x>\pi$\\
    Then $(x+5)(x-3)>0$, which is $x>3~\lor\ x<-5$.
    \par
    \note{Case 2: }$x<\pi$\\
    Then $(x+5)(x-3)<0$, which is $-5<x<3$.
    \item
    \begin{IEEEeqnarray*}{rCl}
      (x-\sqrt[3]{2})(x-\sqrt{2}) & > & 0 \\
      x>\sqrt{2} & \lor & x<\sqrt[3]{2}
    \end{IEEEeqnarray*}
    \item (Sometimes, to solve a problem, intuition
    is a necessity.)
    \begin{IEEEeqnarray*}{rCl}
      2^x & < & 8 \\
      2^x & < & 2^3 \\
      x & < & 3
    \end{IEEEeqnarray*}
    \item
    \begin{IEEEeqnarray*}{rCl}
      x+3^x & < & 4 \\
      x+3^x & < & 1+3^1 \\
      x & < & 1
    \end{IEEEeqnarray*}
    \item
    \begin{IEEEeqnarray*}{rCl}
      \frac{1}{x}+\frac{1}{1-x} & > & 0 \\
      \frac{1}{x(1-x)} & > & 0 \\
    \end{IEEEeqnarray*}
    Hence, $x(1-x)>0$. This means $0<x<1$.
    \item
    \begin{IEEEeqnarray*}{rCl}
      \frac{x-1}{x+1} & > & 0 \\
    \end{IEEEeqnarray*}
    Hence, $(x-1)(x+1)>0$, or $x>1~\lor\ x<-1$.
  \end{enumerate}
\end{solution}

\begin{pr} \label{1.5}
  Prove the following:
  \begin{enumerate}[label=(\roman*)]
    \item If $a<b$ and $c<d$, then $a+c<b+d$
    \item If $a<b$, then $-b<-a$
    \item If $a<b$ and $c>d$, then $a-c<b-d$
    \item If $a<b$ and $c>0$, then $ac<bc$
    \item If $a<b$ and $c<0$, then $ac>bc$
    \item If $a>1$, then $a^2>a$
    \item If $0<a<1$, then $a^2<a$
    \item \label{1.5:viii}
    If $0\leq a<b$ and $0\leq c<d$, then $ac<bd$
    \nopagebreak[3]
    \item \label{1.5:ix}
    If $0\leq a<b$, then $a^2<b^2$. (Use \ref{1.5:viii}.)
    \item If $a,b\geq0$ and $a^2<b^2$, then $a<b$.%
    (Use \ref{1.5:ix}, backwards.)
  \end{enumerate}
\end{pr}
\pagebreak
\begin{solution}
  Let P be the set of all positive numbers.
  \begin{enumerate}[label=(\roman*)]
    \item To prove this, we apply (P11): If $a<b~\land\ c<d$,
    then $(b-a\in P)~\land\ (d-c\in P)$. Then $(b-a)+(d-c)=
    (b+d)-(a+c)\in P$. Therefore, $a+c<b+d$.
    \item \label{1.5:ii}
    We provide two solutions: The first one is
    by Trichotomy Law (P10), and the second one is
    by adding $[(-a)+(-b)]$ to both sides.
    \renewcommand{\qedsymbol}{\textsl Q.E.D}
    \begin{proof}[Proof by Trichotomy Law]
      If $a<b$, then $b-a\in P$. By Trichotomy Law,
      $a-b\notin P$ and $a-b\neq 0$. Therefore, $a-b<0$,
      which is $-b<-a$.
    \end{proof}
    \begin{proof}[Proof by adding]
      \begin{IEEEeqnarray*}{+rCl+x*}
        a & < & b \\
        a+[(-a)+(-b)] & < & b+[(-a)+(-b)] \\
        \left[a+(-a)\right]+(-b) & < & [b+(-b)]+(-a) \\
        -b & < & -a \\
        & & & \qedhere
      \end{IEEEeqnarray*}
    \end{proof}
    \item Using (P11), we have $b-a\in P~\land\ c-d\in P$.
    Then $(b-a)+(c-d)\in P$. Hence, $a-c<b-d$.
    \item \label{1.5:iv}
    Using (P12), note that $b-a\in P$. Since $c>0$,
    $c(b-a)\in P$, which means $bc-ac>0$, or $ac<bc$.
    \item By Trichotomy law(P10), $-c\in P$. Then
    by \ref{1.5:iv}, $-(ac)<-(bc)$. By \ref{1.5:ii},
    $ac>bc$.
    \item
    Since $a>1>0$, by \ref{1.5:iv}, $a^2>a$.
    \item Since $a>0$, by \ref{1.5:iv}, $a^2<a$.
    \item Because $0<b$, $bc<bd$. Furthermore, if $c\geq0$,
    $ac\leq bc$ (equality occurs if $c=0$), by \ref{1.5:iv}.
    Therefore, $ac\leq bc<bd$. Hence, $ac<bd$.
    \item From \ref{1.5:viii}, let $c=a$ and $d=b$, then
    the result follows.
    \item
    Suppose $a\geq b$. Then $a\geq b\geq 0$. By \ref{1.5:ix}
    and (P9), $a^{2}\geq b^2$. This contradicts $a^2<b^2$.
  \end{enumerate}
\end{solution}

\begin{pr} \label{1.6}
  \begin{enumerate}[label=(\alph*)]
    \item \label{1.6:a}
    Prove that if $0\leq x<y$, then $x^n<y^n$,
    $n=1,2,3,\dots$.
    \item \label{1.6:b}
    Prove that if $x<y$ and $n$ is odd,
    then $x^n<y^n$.
    \item \label{1.6:c}
    Prove that if $x^n=y^n$ and $n$ is odd,
    then $x=y$.
    \item \label{1.6:d}
    Prove that if $x^n=y^n$ and $n$ is even,
    then $x=y$ or $x=-y$.
  \end{enumerate}
\end{pr}
\pagebreak
\begin{solution}
  \begin{enumerate}[label=(\alph*)]
    \item
    Repeatedly apply problem \ref{1.5}\ref{1.5:viii} for
    $0\leq x<y$, we have $x^n<y^n$ with $n=1,2,3,\ldots$
    \item
    The statement is true for the case $0\leq x<y$. In
    the case $x<y\leq 0$, by \ref{1.5}\ref{1.5:ii},
    $(-x)>(-y)\geq 0$. By \ref{1.6:a}, $(-x)^n>(-y)^n$
    for all odd $n$. Since $n$ is odd, $-(x^n)>-(y^n)$.
    Hence, by \ref{1.5}\ref{1.5:ii}, $x^n<y^n$. In the
    case $x\leq 0<y$, since $n$ is odd, $x^n<y^n$.
    \item
    Suppose that either $x\neq y$. W.l.o.g, let $x<y$,
    by \ref{1.6:b}, $x^n<y^n$ for all odd $n$, contradicting
    $x^n=y^n$ for all odd $n$.
    \item
    Suppose that both $x\neq y$ and $x\neq -y$. Then
    $x^2-y^2\neq 0$. W.l.o.g, suppose $x^2>y^2\geq0$.
    Applying \ref{1.6:a}, this generalizes to
    $x^n>y^n$ for all even $n$, contradicting our assumption.
    Therefore, $x=y$ or $x=-y$.
    \renewcommand{\qedsymbol}{}
    \begin{proof}[The direct proof]
      In the case $x,y \geq 0$; by \ref{1.6:a}, if $x^n=y^n$
      for all even $n$, then $x=y$. In the case $x,y\leq 0$;
      if $x^n=y^n$ for all even $n$, then
      $(-x),(-y)\geq 0$ and $(-x)^n=(-y)^n$, so $-x=-y$
      and hence $x=y$.
      In the case of $x$ and $y$ have different signs, then
      $x$ and $-y$ are either two positive or two negative
      numbers. In either subcase, if $x^n=y^n$ for all even $n$,
      then $x^n=(-y)^n$, and it follows $x=-y$ from the
      previous case.
    \end{proof}
  \end{enumerate}
\end{solution}

\begin{pr}
  Prove that if $0<a<b$, then
  \begin{equation*}
    a<\sqrt{ab}<\frac{a+b}{2}<b
  \end{equation*}
  Notice that the inequality $\sqrt{ab}\leq(a+b)/2$ holds
  for all $a,b\geq0$. A generalization of this fact
  occurs in Problem 2.22. % to be \ref{2.22}.
\end{pr}
\begin{solution}
  Let us first establish that $a<\dfrac{a+b}{2}<b$. Note
  that,
  \begin{equation*}
    a+a<a+b<b+b
  \end{equation*}
  and therefore, $a<\dfrac{a+b}{2}<b$. To finish the proof,
  we need to prove ${a<\sqrt{ab}<\dfrac{a+b}{2}}$. To do this,
  let us prove that if $0<a<b$, then ${0<\sqrt{a}<\sqrt{b}}$.
  Note that since $b-a>0$,
  \begin{equation*}
    b-a=(\sqrt{b}-\sqrt{a})(\sqrt{b}+\sqrt{a})>0
  \end{equation*}
  Therefore, $\sqrt{b}>\sqrt{a}>0$. We rewrite the inequality
  as follows,
  \begin{equation*}
    \sqrt{a}\cdot(\sqrt{b}-\sqrt{a})>0
  \end{equation*}
  Then
  \begin{equation}
    a<\sqrt{ab} \label{1.7:eq1}
  \end{equation}
  We next notice that since ${\sqrt{b}-\sqrt{a}>0}$, it follows
  that \\$(\sqrt{b}-\sqrt{a})\cdot(\sqrt{b}-\sqrt{a})
  =(\sqrt{b}-\sqrt{a})^2>0$. Expand the left hand side,
  \begin{equation*}
    (\sqrt{b}-\sqrt{a})^2=a+b-2\sqrt{ab}>0
  \end{equation*}
  which implies,
  \begin{equation}
    \sqrt{ab}<\frac{a+b}{2} \label{1.7:eq2}
  \end{equation}
  From (\ref{1.7:eq1}) and (\ref{1.7:eq2}), we have
  $a<\sqrt{ab}<\dfrac{a+b}{2}$.
\end{solution}

\begin{pr}[*]
  Although the basic properties of inequalities were stated
  in terms of the collection P of all positive numbers,
  and $<$ was defined in terms of $P$, this procedure
  can be reversed. Suppose that P10--P12 are replaced by
  \begin{enumerate}
    \item[\mylabel{P'10}{(P'10)}]
    For any numbers $a$ and $b$ one, and only one, of the
    following holds:
    \begin{enumerate}[label=(\roman*)]
      \item $a=b$,
      \item $a<b$,
      \item $b<a$.
    \end{enumerate}
    \item[\mylabel{P'11}{(P'11)}]
    For any numbers $a$, $b$, and $c$, if $a<b$ and $b<c$,
    then $a<c$.
    \item[\mylabel{P'12}{(P'12)}]
    For any numbers $a$, $b$, and $c$, if $a<b$, then
    $a+c<b+c$.
    \item[\mylabel{P'13}{(P'13)}]
    For any numbers $a$, $b$, and $c$, if $a<b$ and
    $0<c$, then $ac<bc$.
  \end{enumerate}
  Show that P10--P12 can then be deduced as theorems.
\end{pr}
\begin{solution}
  Let $P$ be the set of all positive numbers.
  \begin{itemize}
    \item To prove P10, let $c=a-b$, from \ref{P'10}, P10
    follows.
    \item To prove P11, let $a,b\in P$; it is
    sufficient to prove that $a+b>0$. From \ref{P'10},
    we divide the proof into three subscases:\par
    \note{Case 1: }$a=b$ \\
    Then $a+b=b+b>0+b>0$, where the first inequality follows
    from \ref{P'12}. By \ref{P'11}, $a+b>0$.\par
    \note{Case 2: }$a<b$ \\
    Then $a+b>a+a>0+a>0$, where the first and second inequality
    follow from \ref{P'12}. By applying \ref{P'11} twice,
    $a+b>0$. \par
    \note{Case 3: }$a>b$ \\
    Interchanging the role of $a$ and $b$, we have the result.
    \item To prove P12, let $a,b\in P$; it is
    sufficient to prove that $a\cdot b>0$. From \ref{P'10},
    we divide the proof into three subcases:\par
    \note{Case 1: }$a=b$ \\
    Then $a\cdot b=b\cdot b>0\cdot b=0$, where the first
    inequality follows from \ref{P'13} and the equality after
    which is from (P9).\par
    \note{Case 2: }$a<b$ \\
    Then $b\cdot a>a\cdot a>0\cdot a=0$, where the first
    and second inequality is from \ref{P'13}. By \ref{P'11},
    $a\cdot b>0$. \par
    \note{Case 3: }$a>b$ \\
    Interchanging $a$ and $b$ returns us to case 2, which
    yields the result.
  \end{itemize}
\end{solution}
