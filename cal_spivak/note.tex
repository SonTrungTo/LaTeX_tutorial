\documentclass[a4paper,11pt,twoside]{book}
\usepackage[affil-it]{authblk}
\usepackage{amsmath,amssymb,amsthm,fancyhdr,enumitem,%
hyperref}
\usepackage[retainorgcmds]{IEEEtrantools}
\newcommand{\note}[1]{\emph{#1}}
%--Possibly use this inside IEEEeqnarray environment
\newcommand{\norm}[1]{\noalign{\noindent #1}}
%---
\theoremstyle{plain} \newtheorem{id}{Lemma}
\newcommand{\idautorefname}{Lemma}
\pagestyle{fancy}

\begin{document}

\title{A Note of Calculus-Michael Spivak}
\author{Son To
  \thanks{Contact me at: \texttt{son.trung.to@gmail.com}}}
\date{23rd June, 2017}
\affil{Ravintola Kiltakellari}

\maketitle
\makeatletter
\def\cleardoublepage{\clearpage\if@twoside
\ifodd\c@page\else
  \hbox{}
  \vspace*{\fill}
  \begin{center}
    This page is intentionally left blank.
  \end{center}
  \vspace*{\fill}
  \thispagestyle{empty}
  \newpage
  \if@twocolumn\hbox{}\newpage\fi\fi\fi
}
\newcommand{\mylabel}[2]{#2\def\@currentlabel{#2}\label{#1}}
\makeatother

\frontmatter
    \chapter*{Preface}
      This is the note for the book Calculus
      writtten by Michael Spivak,
      citing what I think the most interesting
      and important subjects
      mentioned in the book.
    \addcontentsline{toc}{chapter}{Preface}
    \tableofcontents
    \thispagestyle{empty}
\mainmatter
  \part{Prologue}
    \chapter{Basic properties of number}
    \begin{itemize}
      \item[(P1)] If $a$, $b$, and $c$ are any numbers, then
      \begin{equation*}
        a+(b+c)=(a+b)+c
      \end{equation*}
    \end{itemize}
    See \note{problem 24} for the generalization of
    $a_1+a_2+a_3+\dots+a_n$ for (P1).

    The number 0 has important properties.
    \begin{itemize}
      \item[(P2)] If $a$ is any number, then
      \begin{equation*}
        a+0=0+a=a
      \end{equation*}
      \item[(P3)] For every number $a$, there is
      also a number $-a$ such that
      \begin{equation*}
        a+(-a)=(-a)+a=0
      \end{equation*}
    \end{itemize}
    We now prove \autoref{lemm:1}.
    \begin{id}\label{lemm:1}
      If $a+x=a$, then $x=0$
    \end{id}
    \begin{proof}
      \begin{IEEEeqnarray*}{x+rClr+x*}
        \text{If} & a+x & = & a \\
        \text{then} & (-a)+(a+x) & = & (-a)+a=0 & (by~(P3))\\
        \text{hence} & ((-a)+a)+x & = & 0 & (by~(P1))\\
        \text{hence} & 0+x & = & 0 & (by~(P3)~again)\\
        \text{therefore,} & x & = & 0 & (by~(P2)) \\
        &&&&& \qedhere
      \end{IEEEeqnarray*}
    \end{proof}
  Also, remember that the order of addition does not
  matter.
  \begin{itemize}
    \item[(P4)] If $a$ and $b$ are any numbers, then
    \begin{equation*}
      a+b=b+a
    \end{equation*}
  \end{itemize}
  However, with only (P1)-(P4), we are powerless to
  figure out what conditions needed to have
  $a-b=b-a$. Therefore, we need to set new properties, and,
  oddly, they involve multiplication.
  \begin{itemize}[label=\textnormal{(\arabic*)}]
    \item[\mylabel{itm:P5}{(P5)}]
    If $a$, $b$ and $c$ are any numbers, then
    \begin{equation*}
      a\cdot(b\cdot c)=(a\cdot b)\cdot c
    \end{equation*}
    \item[\mylabel{itm:P6}{(P6)}]
    If $a$ is any number, then
    \begin{equation*}
      a\cdot 1=1\cdot a=a
    \end{equation*}
    Moreover, $1\neq0$ (This cannot be proved by other
    properties listed!)
    \item[\mylabel{itm:P7}{(P7)}]
    For every number $a\neq0$, there is a number
    $a^{-1}$ such that
    \begin{equation*}
      a\cdot a^{-1}=a^{-1}\cdot a=1 (\Leftarrow 0\cdot
      b=0~\forall b)
    \end{equation*}
    \note{This is why $1/0$ is meaningless!}
    \item[\mylabel{itm:P8}{(P8)}]
    If $a$ and $b$ are any numbers, then
    \begin{equation*}
      a\cdot b=b\cdot a
    \end{equation*}
  \end{itemize}
  From \ref{itm:P5}, \ref{itm:P6} and \ref{itm:P7},
  we have two lemmas:
    \begin{id} \label{lemm:2}
      If $a\cdot b=a\cdot c$ then $a=0~\lor~b=c$
    \end{id}
    \begin{proof}
      If $a=0$ then the lemma is trivial. Suppose now
      $a\neq0$,
      \begin{IEEEeqnarray*}{*x+rCl+x*}
        \text{Multiply $a^{-1}$ to both sides,} & (a^{-1})
        \cdot (a\cdot b) & = & (a^{-1})\cdot (a\cdot c)\\
        \text{By \ref{itm:P5},} &
        (a^{-1}\cdot a)\cdot b & = &
        (a^{-1}\cdot a)\cdot c\\
        \text{By \ref{itm:P7},} &
        1\cdot b & = &
        1\cdot c\\
        \text{By \ref{itm:P6},} &
        b & = &
        c\\
        &&&& \qedhere
      \end{IEEEeqnarray*}
    \end{proof}
  \begin{id} \label{lemm:3}
    If $a\cdot b=0$ then $a=0~\lor~b=0$
  \end{id}
  \begin{proof}
    If $a=0$, there is nothing to prove. Suppose now
    $a\neq0$, follow the proof of \autoref{lemm:2} by
    consecutively applying \ref{itm:P5}, \ref{itm:P7} and
    \ref{itm:P6} in that order to finish the proof.
  \end{proof}
\end{document}
