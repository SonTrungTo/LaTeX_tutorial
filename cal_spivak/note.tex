\documentclass[a4paper,11pt,twoside]{book}
\usepackage[affil-it]{authblk}
\usepackage{amsmath,amssymb,amsthm}
\usepackage[retainorgcmds]{IEEEtrantools}
\newcommand{\note}[1]{\textbf{#1}}
%--Possibly use this inside IEEEeqnarray environment
\newcommand{\norm}[1]{\noalign{\noindent #1}}
%---
\theoremstyle{plain} \newtheorem{id}{Lemma}

\begin{document}

\title{A Note of Calculus-Michael Spivak}
\author{Son To
  \thanks{Contact me at: \texttt{son.trung.to@gmail.com}}}
\date{23rd June, 2017}
\affil{Ravintola Kiltakellari}

\maketitle
\makeatletter
\def\cleardoublepage{\clearpage\if@twoside
\ifodd\c@page\else
  \hbox{}
  \vspace*{\fill}
  \begin{center}
    This page is intentionally left blank.
  \end{center}
  \vspace*{\fill}
  \thispagestyle{empty}
  \newpage
  \if@twocolumn\hbox{}\newpage\fi\fi\fi
}
\makeatother

\frontmatter
    \chapter*{Preface}
      This is the note for the book Calculus
      writtten by Michael Spivak,
      citing what I think the most interesting
      and important subjects
      mentioned in the book.
    \addcontentsline{toc}{chapter}{Preface}
    \tableofcontents
\mainmatter
  \part{Prologue}
    \chapter{Basic properties of number}
    \begin{itemize}
      \item[(P1)] If $a$, $b$, and $c$ are any numbers, then
      \begin{equation*}
        a+(b+c)=(a+b)+c
      \end{equation*}
    \end{itemize}
    See \note{problem 24} for the generalization of
    $a_1+a_2+a_3+\dots+a_n$ for (P1).

    The number 0 has important properties.
    \begin{itemize}
      \item[(P2)] If $a$ is any number, then
      \begin{equation*}
        a+0=0+a=a
      \end{equation*}
      \item[(P3)] For every number $a$, there is
      also a number $-a$ such that
      \begin{equation*}
        a+(-a)=(-a)+a=0
      \end{equation*}
    \end{itemize}
    We now prove lemma~\ref{id}.
    \begin{id}\label{id}
      If $a+x=a$, then $x=0$
    \end{id}
    \begin{proof}
      \begin{IEEEeqnarray*}{x+rClr+x*}
        \text{If} & a+x & = & a \\
        \text{then} & (-a)+(a+x) & = & (-a)+a=0 & (by~(P3))\\
        \text{hence} & ((-a)+a)+x & = & 0 & (by~(P1))\\
        \text{hence} & 0+x & = & 0 & (by~(P3)~again)\\
        \text{therefore,} & x & = & 0 & (by~(P2)) \\
        &&&&& \qedhere
      \end{IEEEeqnarray*}
    \end{proof}
\end{document}
