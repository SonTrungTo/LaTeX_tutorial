\chapter{Functions}
\begin{pr} \label{pr3.1}
  Let $f(x) = 1/(1+x)$. What is
  \begin{enumerate}[label = (\roman*)]
    \item $f(f(x))$ (for which $x$ does this make sense?).
    \item $f\left(\displaystyle\frac{1}{x}\right)$.
    \item $f(cx)$.
    \item $f(x + y)$.
    \item $f(x) + f(y)$.
    \item For which numbers $c$ is there a number $x$ such that $f(cx) = f(x)$.
    Hint: There are a lot more than you might think at first glance.
    \item For which numbers $c$ is it true that $f(cx) = f(x)$ for two different
    numbers $x$
  \end{enumerate}
\end{pr}

\begin{solution}
  \begin{enumerate}[label = (\roman*)]
    \item $(f \circ f)(x) = \displaystyle\frac{1}{1 + \displaystyle\frac{1}{1 + x} }%
    = \displaystyle\frac{1 + x}{2 + x}$ for all $x \neq -1,-2$.
    \item For $\{x: x \neq 0,-1\}$,
    \begin{equation*}
      f(1/x) = \frac{1}{1 + \displaystyle\frac{1}{x}} = \frac{x}{x + 1}
    \end{equation*}
    \item $f(cx) = \displaystyle\frac{1}{1 + cx}$. If $c = 0$, then $f(0) = 1$
    for all $x$. If $c\neq0$, then $x = -\frac{1}{c}$.
    \item $f(x+y) = \displaystyle\frac{1}{1 + x + y}$ for $\{x: x\neq -(y + 1)\}$
    \item $f(x) + f(y) = \displaystyle\frac{1}{1+x} + \displaystyle\frac{1}{1+y}%
    = \displaystyle\frac{2+x+y}{(1+x)(1+y)}$ for $\{x,y: x\neq -1 \text{ and } y\neq - 1\}$.
    \item Observe that
    \begin{IEEEeqnarray*}{rCl}
      \frac{1}{1+cx} &=& \frac{1}{1+x} \\
      1 + x          &=& 1 + cx \\
      x(c - 1)       &=& 0
    \end{IEEEeqnarray*}
    If $c = 0$ then $x = 0$. If $c\neq0$, then if $c = 1$, there is infinite $x$
    satisfying our requirement; if not, then $x = 0$. Therefore, for all $c\in \mathbb{R}$,
    there is a number $x$ such that $f(cx) = f(x)$.
    \item From the above, when $c = 1$.
  \end{enumerate}
\end{solution}
