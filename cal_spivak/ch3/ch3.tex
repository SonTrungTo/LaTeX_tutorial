\chapter{Functions}
\begin{pr} \label{pr3.1}
  Let $f(x) = 1/(1+x)$. What is
  \begin{enumerate}[label = (\roman*)]
    \item $f(f(x))$ (for which $x$ does this make sense?).
    \item $f\left(\displaystyle\frac{1}{x}\right)$.
    \item $f(cx)$.
    \item $f(x + y)$.
    \item $f(x) + f(y)$.
    \item For which numbers $c$ is there a number $x$ such that $f(cx) = f(x)$.
    Hint: There are a lot more than you might think at first glance.
    \item For which numbers $c$ is it true that $f(cx) = f(x)$ for two different
    numbers $x$
  \end{enumerate}
\end{pr}

\begin{solution}
  \begin{enumerate}[label = (\roman*)]
    \item $(f \circ f)(x) = \displaystyle\frac{1}{1 + \displaystyle\frac{1}{1 + x} }%
    = \displaystyle\frac{1 + x}{2 + x}$ for all $x \neq -1,-2$.
    \item For $\{x: x \neq 0,-1\}$,
    \begin{equation*}
      f(1/x) = \frac{1}{1 + \displaystyle\frac{1}{x}} = \frac{x}{x + 1}
    \end{equation*}
    \item $f(cx) = \displaystyle\frac{1}{1 + cx}$. If $c = 0$, then $f(0) = 1$
    for all $x$. If $c\neq0$, then $x = -\frac{1}{c}$.
    \item $f(x+y) = \displaystyle\frac{1}{1 + x + y}$ for $\{x: x\neq -(y + 1)\}$
    \item $f(x) + f(y) = \displaystyle\frac{1}{1+x} + \displaystyle\frac{1}{1+y}%
    = \displaystyle\frac{2+x+y}{(1+x)(1+y)}$ for $\{x,y: x\neq -1 \text{ and } y\neq - 1\}$.
    \item Observe that
    \begin{IEEEeqnarray*}{rCl}
      \frac{1}{1+cx} &=& \frac{1}{1+x} \\
      1 + x          &=& 1 + cx \\
      x(c - 1)       &=& 0
    \end{IEEEeqnarray*}
    If $c = 0$ then $x = 0$. If $c\neq0$, then if $c = 1$, there is infinite $x$
    satisfying our requirement; if not, then $x = 0$. Therefore, for all $c\in \mathbb{R}$,
    there is a number $x$ such that $f(cx) = f(x)$.
    \item From the above, when $c = 1$.
  \end{enumerate}
\end{solution}

\begin{pr} \label{pr3.2}
  Let $g(x) = x^2$, and let
  \begin{equation*}
    h(x) = \left\{
    \begin{array}{rl}
      0, & x \text{ rational} \\
      1, & x \text{ irrational}
    \end{array}
    \right.
  \end{equation*}
  \begin{enumerate}[label = (\roman*)]
    \item For which $y$ is $h(y) \leq y$?
    \item For which $y$ is $h(y) \leq g(y)$?
    \item What is $g(h(z)) - h(z)$?
    \item For which $w$ is $g(w) \leq w$?
    \item For which $\epsilon$ is $g(g(\epsilon)) = g(\epsilon)$?
  \end{enumerate}
\end{pr}

\begin{solution}
  \begin{enumerate}[label = (\roman*)]
    \item $\{ y: y\geq0 \text{ and } y \text{ is rational} \} \cup
    \{ y: y > 1 \text{ and } y \text{ is irrational} \}$.
    \item $ \{y: y \text{ is rational} \} \cup
    \{ y: y > 1 \text{ or } y < -1 \text{ and } y \text{ is irrational} \} $.
    \item $0$, for $z$ is not changed by the composite function.
    \item $w^2 \leq w$ if and only if $0 \leq w \leq 1$.
    \item $\epsilon^4 = \epsilon^2$ only when $\epsilon = 0,\pm1$.
  \end{enumerate}
\end{solution}

\begin{pr} \label{pr3.3}
  Find the domain of the functions defined by the following formulas.
  \begin{enumerate}[label=(\roman*)]
    \item $f(x) = \sqrt{1 - x^2}$
    \item $f(x) = \sqrt{1 - \sqrt{1 - x^2}}$
    \item $f(x) = \displaystyle\frac{1}{x - 1} + \displaystyle\frac{1}{x - 2}$
    \item $f(x) = \sqrt{1 - x^2} + \sqrt{x^2 - 1}$
    \item $f(x) = \sqrt{1 - x} + \sqrt{x - 2}$
  \end{enumerate}
\end{pr}

\begin{solution}
  \begin{enumerate}[label=(\roman*)]
    \item $-1 \leq x \leq 1$.
    \item Same as above, for $1 - x^2 \leq 1$ is true for all $x$ in the domain.
    \item $\{ x: x\neq1 \text{ and } x\neq2 \}$
    \item $\{ x: x \geq 1 \text{ or } x \leq -1 \} \cap
    \{ x: -1 \leq x \leq 1 \} = \{x: x = -1,1\}$
    \item Domain does not exist since $\{ x: x \leq 1 \text{ and } x \geq 2 \}
    = \emptyset$
  \end{enumerate}
\end{solution}

\begin{pr} \label{3.4}
  Let $S(x) = x^2$, let $P(x) = 2^x$, and let $s(x) = \sin x$. Find each of the
  following. In each case your answer should be a \note{number}.
  \begin{enumerate}[label = (\roman*)]
    \item $(S \circ P)(y)$.
    \item $(S \circ s)(y)$.
    \item $(S \circ P \circ s)(t) + (s \circ P)(t)$.
    \item $s(t^3)$.
  \end{enumerate}
\end{pr}

\begin{solution}
  \begin{enumerate}[label=(\roman*)]
    \item $(2^y)^2 = 2^{2y}$.
    \item $\sin^2 y$.
    \item $2^{2\sin t} + \sin 2^t$.
    \item $\sin t^3$.
  \end{enumerate}
\end{solution}

\begin{pr} \label{3.5}
  Express each of the following functions in terms of $S, P, s$, using only
  $+,\cdot,\circ$ (for example, the answer to \ref{3.5.i} is $P \circ s$). In each
  case your answer should be a \note{function}.
  \begin{enumerate}[label = (\roman*)]
    \item \label{3.5.i} $f(x) = 2^{\sin x}$.
    \item $f(x) = \sin 2^x$.
    \item $f(x) = \sin x^2$.
    \item $f(x) = \sin^2 x$.
    \item $f(t) = 2^{2^t}$.
    \item $f(u) = \sin (2^u + 2^{u^2})$.
    \item $f(y) = \sin (\sin (\sin (2^{2^{2^{\sin y}}})))$.
    \item $f(a) = 2^{\sin^2 a} + \sin (a^2) + 2^{\sin (a^2 + \sin a)}$.
  \end{enumerate}
\end{pr}

\begin{solution}
  \begin{enumerate}[label = (\roman*)]
    \item $(P \circ s)(x)$.
    \item $(s \circ P)(x)$.
    \item $(s \circ S)(x)$.
    \item $(S \circ s)(x)$.
    \item $(S \circ S)(t)$.
    \item $(s \circ (P + P \circ S))(u)$.
    \item $(s \circ s \circ s \circ P \circ P \circ P \circ s)(y)$.
    \item $(P \circ S \circ s + s \circ S + P \circ (s \circ (S + s)))(a)$.
  \end{enumerate}
\end{solution}

\flushleft
Polynomial functions, because they are simple, yet flexible, occupy a favored
role in most investigations of functions. The following two problems illustrate
their flexibility, and guide you through a derivation of their most important
elementary properties.

\begin{pr} \label{3.6}
  \begin{enumerate}[label = (\alph*)]
    \item If $x_1,\ldots,x_n$ are distinct numbers, find a polynomial function $f_i$
    of degree $n - 1$ which is $1$ at $x_i$ and $0$ at $x_j$ for $j \neq i$. Hint:
    the product of all $(x - x_j)$ for $j \neq i$ is $0$ at $x_j$ if $j \neq i$.
    (This product is usually denoted by
    \begin{equation*}
      \prod_{\substack{j = 1 \\ j \neq i}}^{n} (x - x_j),
    \end{equation*}
    the symbol $\prod$ (capital pi) playing the same role for products that $\sum$
    plays for sums.)
    \item Now find a polynomial function $f$ of degree $n - 1$ such that $f(x_i) = a_i$,
    where $a_1,\ldots,a_n$ are given numbers. The formula obtained is called
    the ``Lagrange interpolation formula''.
  \end{enumerate}
\end{pr}

\begin{solution}
  \begin{enumerate}[label = (\alph*)]
    \item Let
    \begin{equation*}
      f_i(x) = \frac{\displaystyle\prod_{\substack{j = 1 \\ j \neq i}}^{n} (x - x_j)}%
      {\displaystyle\prod_{\substack{j = 1 \\ j \neq i}}^{n} (x_i - x_j)}
    \end{equation*}
    Observe that $f_i$ is a polynomial of degree $n - 1$, equal to $1$ if $x = x_i$
    and to $0$ if $x = x_j$ for $j \neq i$.
    \item Let
    \begin{equation*}
      f(x) = \sum_{i = 1}^{n} (f_i(x) \cdot a_i)
    \end{equation*}
    where $f_i(x)$ is the above-constructed function.
  \end{enumerate}
\end{solution}
