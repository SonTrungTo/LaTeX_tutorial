\chapter{Numbers of Various Sorts}
\begin{pr} \label{2.1}% 2.1
  Prove the following formula by induction.
  \begin{enumerate}[label=(\roman*)]
    \item \label{2.1:i}
    $1^2+\dots+n^2=\dfrac{n(n+1)(2n+1)}{6}$.
    \item $1^3+\dots+n^3=(1+\dots+n)^2$.
  \end{enumerate}
\end{pr}

\begin{solution} % s2.1
  \begin{enumerate}[label=(\roman*)]
    \item If $n=1$, the equation holds. Suppose the
    equation holds for $n=k$, then for $n=k+1$,
    \begin{IEEEeqnarray*}{lC}
      & 1^2+\dots+k^2+(k+1)^2 \\
    = & \frac{k(k+1)(2k+1)}{6} + (k+1)^2 \\
    = & \frac{(k+1)[k(2k+1)+6(k+1)]}{6}  \\
    = & \frac{(k+1)(2k^2+7k+6)}{6}       \\
    = & \frac{(k+1)(2k^2+4k+3k+6)}{6}    \\
    = & \frac{(k+1)[2k(k+2)+3(k+2)]}{6}  \\
    = & \frac{(k+1)[(k+1)+1][2(k+1)+1]}{6}
    \end{IEEEeqnarray*}
    Then the formula holds for every $n$.
    \item If $n=1$, there is nothing to prove. If
    $n=k$ holds for the equation, then for $n=k+1$,
    \begin{IEEEeqnarray*}{lC}
      & [1+\dots+k+(k+1)]^2 \\
    = & (1+\dots+k)^2 + (k+1)^2 + 2(k+1)(1+\dots+k) \\
    = & 1^3+\dots+k^3 + (k+1)[(k+1)+2\frac{k(k+1)}{2}] \\
    = & 1^3+\dots+k^3 + (k+1)^3
    \end{IEEEeqnarray*}
    This finishes the proof for every $n$.
  \end{enumerate}
\end{solution}

\begin{pr} % 2.2
  Find a formula for
  \begin{enumerate}[label=(\roman*)]
    \item
    $\displaystyle\sum_{i=1}^n (2i-1) = 1+3+5+\dots+(2n-1)$.
    \item $\displaystyle\sum_{i=1}^n (2i-1)^2
    = 1^2 + 3^2 + 5^2 + \dots + (2n - 1)^2$.
  \end{enumerate}
  Hint: What do these expressions have to do with
  $1 + 2 + 3 + \dots + 2n$ and $1^2 + 2^2 + 3^2 + \dots
  + (2n)^2$?
\end{pr}

\begin{solution} % s2.2
  \begin{enumerate}[label=(\roman*)]
    \item Remind that $1+\dots+2n=n(2n+1)$ and
    that $2 + \dots + 2n = 2 \cdot \dfrac{n(n+1)}{2} = n(n+1)$.
    Hence,
    \begin{equation*}
      \sum_{i=1}^n (2i - 1) = n(2n+1) - n(n+1) = n^2
    \end{equation*}
    \item Using \autoref{2.1}\ref{2.1:i}, we easily
    derive that $\displaystyle
    \sum_{i=1}^{2n} i^2 = \dfrac{n(2n+1)(4n+1)}{3}$
    and $\displaystyle
    \sum_{i=1}^n (2i)^2 = \dfrac{2n(n+1)(2n+1)}{3}$.
    Therefore,
    \begin{equation*}
      \sum_{i=1}^n (2i - 1)^2 = \sum_{i=1}^{2n} i^2
      - \sum_{i=1}^n (2i)^2
      = \frac{n(4n^2 - 1)}{3}
    \end{equation*}
  \end{enumerate}
\end{solution}

\begin{pr} \label{2.3} % 2.3
  If $0 \leq k \leq n$, the ``binomial coefficient''
  $\displaystyle\binom{n}{k}$ is defined by
  $\displaystyle\binom{n}{k} = \dfrac{n!}{k!(n-k)!}
  = \dfrac{n(n-1)\cdots(n-k+1)}{k!}$, if $k\neq0,n$ \\
  $\displaystyle\binom{n}{0} = \binom{n}{n} = 1$.
  (This becomes a special case of the first formula
  if we define $0! = 1$.)
  \begin{enumerate}[label=(\alph*)]
    \item \label{2.3:a}
    Prove that
    \begin{equation*}
      \binom{n+1}{k} = \binom{n}{k-1} + \binom{n}{k}.
    \end{equation*}
    (The proof does not require an induction argument.)

    \medskip
    This relation gives rise to the following configuration,
    known as ``Pascal's triangle''---a number not on one of the
    sides is the sum of two numbers above it; the binomial
    coefficient $\displaystyle\binom{n}{k}$ is the
    $(k+1)$st number in the $(n+1)$st row.
    \par
    \begin{center}
      \begin{tikzpicture}
        \foreach \n in {0,...,5}{
          \foreach \k in {0,...,\n}{
            \node at (\k - \n/2, -\n){$\binomialCoefficient{\n}{\k}$};
          }
        }
      \end{tikzpicture}
    \end{center}
    \item Notice that all numbers in Pascal's triangle are
    natural numbers. Use part \ref{2.3:a} to prove by
    induction that $\displaystyle\binom{n}{k}$ is always
    a natural number. (Your entire proof by induction will,
    in a sense, be summed up in a glance by Pascal's
    triangle.)
    \item Give another proof that $\displaystyle\binom{n}{k}$
    is a natural number by showing that
    $\displaystyle\binom{n}{k}$ is the number of sets of
    exactly $k$ integers each chosen from $1,\dots,n$.
    \item Prove the ``binomial theorem'': If $a$ and $b$
    are any numbers and $n$ is a natural number, then
    \begin{IEEEeqnarray*}{rCl}
      (a+b)^n & = & a^n + \binom{n}{1}a^{n-1}b
      + \binom{n}{2}a^{n-2}b^2+\cdots+\binom{n}{n-1}ab^{n-1}
      + b^n \\
              & = & \sum_{j=0}^n \binom{n}{j}a^{n-j}b^j.
    \end{IEEEeqnarray*}
    \item Prove that
    \begin{enumerate}[label=(\roman*)]
      \item \label{2.3:ei}
      $\displaystyle
      \sum_{j=0}^n \binom{n}{j}
      = \binom{n}{0} + \cdots + \binom{n}{n}
      = 2^n$.
      \item \label{2.3:eii}
      $\displaystyle
      \sum_{j=0}^n (-1)^j \binom{n}{j}
      = \binom{n}{0} - \binom{n}{1} + \cdots \pm \binom{n}{n}
      = 0$.
      \item $\displaystyle
      \sum_{l\text{ odd}} \binom{n}{l}
      = \binom{n}{1} + \binom{n}{3} + \cdots
      = 2^{n - 1}$.
      \item $\displaystyle
      \sum_{l\text{ even}} \binom{n}{l}
      = \binom{n}{0} + \binom{n}{2} + \cdots
      = 2^{n - 1}$.
    \end{enumerate}
  \end{enumerate}
\end{pr}

\begin{solution} % s2.3
  \begin{enumerate}[label=(\alph*)]
    \item Starting from the left-hand side,
    \begin{IEEEeqnarray*}{rCl}
      \binom{n + 1}{k} & = &
      \frac{n!(n - k + 1 + k)}{k!(n - k + 1)!} \\
                       & = &
      \frac{n!(n - k + 1)}{k!(n - k + 1)!}
    + \frac{n!k}{k!(n - k + 1)!}\\
                       & = &
      \frac{n!}{k!(n - k)!}
    + \frac{n!}{(k - 1)![n - (k - 1)]!} \\
                       & = &
      \binom{n}{k} + \binom{n}{k - 1}
    \end{IEEEeqnarray*}
    \item It is sufficient to prove that $\displaystyle
    \binom{n}{k}$ is a natural number for all
    $1 \leq k \leq (n - 1)$. If $n = 1$, then
    \begin{equation*}
      \binom{2}{1} = \binom{1}{0} + \binom{1}{1} = 2
    \end{equation*}
    Suppose that $\displaystyle\binom{n}{k}$ is natural
    number for any $n$ and $1 \leq k \leq n - 1$.
    Then for any $1 \leq k \leq n$, $\displaystyle
    \binom{n + 1}{k}$ is the sum of two natural numbers,
    and therefore it must be a natural number.
    \item It is sufficient to prove for the case
    $0 < k \leq n$.
    If $n=1$, the claim is trivial. Suppose that
    $\displaystyle\binom{n}{k}$ is the number of sets
    of $k$ integers each chosen from $1,\dots,n$; then
    $\displaystyle\binom{n + 1}{k}$ must include
    $\displaystyle\binom{n}{k}$ sets of $k$ integers
    \note{without} the newly added element and a number
    of sets of $k$ integers \note{with} the newly added
    element. The latter is exactly $\displaystyle
    \binom{n}{k - 1}$ and is a natural number by assumption.
    Thereby, $\displaystyle\binom{n+1}{k}$ must be
    a natural number.
    \item We prove by induction on $n$. If $n=1$, there
    is nothing to prove. Suppose the binomial theorem holds
    for $n$; then for $n+1$,
    \begin{IEEEeqnarray*}{rCl}
      (a+b)^n(a+b) & = &
      \sum_{j=0}^n \binom{n}{j} a^{n-j}b^j(a + b) \\
                   & = &
      \sum_{j=0}^n \binom{n}{j} a^{n+1-j}b^j
    + \sum_{j=0}^n \binom{n}{j} a^{n-j}b^{j+1}    \\
                   & = &
      a^{n + 1} + b^{n + 1}
    + \sum_{j=1}^n \binom{n}{j} a^{n+1-j}b^j      \\
                   & + &
      \sum_{j=1}^n \binom{n}{j-1} a^{n+1-j}b^j    \\
                   & = &
      \sum_{j=1}^n \binom{n+1}{j} a^{n+1-j}b^j
    + a^{n+1} + b^{n+1}                           \\
                   & = &
      \sum_{j=0}^{n+1} \binom{n+1}{j} a^{n+1-j}b^j
    \end{IEEEeqnarray*}
    which completes the proof.
    \item This part relies heavily on
    the binomial theorem from the above.
    \begin{enumerate}[label=(\roman*)]
      \item This is directly from the above: Applying
      the binomial theorem for $a=b=1$ yields the result.
      \item Let $a=1$ and $b=-1$ yield the result.
      \item Applying \ref{2.3:ei} $+$ \ref{2.3:eii},
      we derive that for $l$ even,
      \begin{equation*}
        \sum_{l\text{ even}} \binom{n}{l} = 2^{n - 1}
      \end{equation*}
      Thereby, $\displaystyle
      \sum_{l\text{ odd}} \binom{n}{l} = 2^n - 2^{n - 1}
      = 2^{n - 1}$.
      \item See the above.
    \end{enumerate}
    One thing to note: Both of the previous parts do not
    have a final term expressed in their sum due to the
    dependence of value $n$ (if $n$ is even or odd).
  \end{enumerate}
\end{solution}
