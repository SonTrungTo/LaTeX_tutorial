\chapter{Numbers of Various Sorts}
\begin{pr} \label{2.1}% 2.1
  Prove the following formula by induction.
  \begin{enumerate}[label=(\roman*)]
    \item \label{2.1:i}
    $1^2+\dots+n^2=\dfrac{n(n+1)(2n+1)}{6}$.
    \item $1^3+\dots+n^3=(1+\dots+n)^2$.
  \end{enumerate}
\end{pr}

\begin{solution} % s2.1
  \begin{enumerate}[label=(\roman*)]
    \item If $n=1$, the equation holds. Suppose the
    equation holds for $n=k$, then for $n=k+1$,
    \begin{IEEEeqnarray*}{lC}
      & 1^2+\dots+k^2+(k+1)^2 \\
    = & \frac{k(k+1)(2k+1)}{6} + (k+1)^2 \\
    = & \frac{(k+1)[k(2k+1)+6(k+1)]}{6}  \\
    = & \frac{(k+1)(2k^2+7k+6)}{6}       \\
    = & \frac{(k+1)(2k^2+4k+3k+6)}{6}    \\
    = & \frac{(k+1)[2k(k+2)+3(k+2)]}{6}  \\
    = & \frac{(k+1)[(k+1)+1][2(k+1)+1]}{6}
    \end{IEEEeqnarray*}
    Then the formula holds for every $n$.
    \item If $n=1$, there is nothing to prove. If
    $n=k$ holds for the equation, then for $n=k+1$,
    \begin{IEEEeqnarray*}{lC}
      & [1+\dots+k+(k+1)]^2 \\
    = & (1+\dots+k)^2 + (k+1)^2 + 2(k+1)(1+\dots+k) \\
    = & 1^3+\dots+k^3 + (k+1)[(k+1)+2\frac{k(k+1)}{2}] \\
    = & 1^3+\dots+k^3 + (k+1)^3
    \end{IEEEeqnarray*}
    This finishes the proof for every $n$.
  \end{enumerate}
\end{solution}

\begin{pr} % 2.2
  Find a formula for
  \begin{enumerate}[label=(\roman*)]
    \item
    $\displaystyle\sum_{i=1}^n (2i-1) = 1+3+5+\dots+(2n-1)$.
    \item $\displaystyle\sum_{i=1}^n (2i-1)^2
    = 1^2 + 3^2 + 5^2 + \dots + (2n - 1)^2$.
  \end{enumerate}
  Hint: What do these expressions have to do with
  $1 + 2 + 3 + \dots + 2n$ and $1^2 + 2^2 + 3^2 + \dots
  + (2n)^2$?
\end{pr}

\begin{solution} % s2.2
  \begin{enumerate}[label=(\roman*)]
    \item Remind that $1+\dots+2n=n(2n+1)$ and
    that $2 + \dots + 2n = 2 \cdot \dfrac{n(n+1)}{2} = n(n+1)$.
    Hence,
    \begin{equation*}
      \sum_{i=1}^n (2i - 1) = n(2n+1) - n(n+1) = n^2
    \end{equation*}
    \item Using \autoref{2.1}\ref{2.1:i}, we easily
    derive that $\displaystyle
    \sum_{i=1}^{2n} i^2 = \dfrac{n(2n+1)(4n+1)}{3}$
    and $\displaystyle
    \sum_{i=1}^n (2i)^2 = \dfrac{2n(n+1)(2n+1)}{3}$.
    Therefore,
    \begin{equation*}
      \sum_{i=1}^n (2i - 1)^2 = \sum_{i=1}^{2n} i^2
      - \sum_{i=1}^n (2i)^2
      = \frac{n(4n^2 - 1)}{3}
    \end{equation*}
  \end{enumerate}
\end{solution}

\begin{pr} \label{2.3} % 2.3
  If $0 \leq k \leq n$, the ``binomial coefficient''
  $\displaystyle\binom{n}{k}$ is defined by
  $\displaystyle\binom{n}{k} = \dfrac{n!}{k!(n-k)!}
  = \dfrac{n(n-1)\cdots(n-k+1)}{k!}$, if $k\neq0,n$ \\
  $\displaystyle\binom{n}{0} = \binom{n}{n} = 1$.
  (This becomes a special case of the first formula
  if we define $0! = 1$.)
  \begin{enumerate}[label=(\alph*)]
    \item \label{2.3:a}
    Prove that
    \begin{equation*}
      \binom{n+1}{k} = \binom{n}{k-1} + \binom{n}{k}.
    \end{equation*}
    (The proof does not require an induction argument.)

    \medskip
    This relation gives rise to the following configuration,
    known as ``Pascal's triangle''---a number not on one of the
    sides is the sum of two numbers above it; the binomial
    coefficient $\displaystyle\binom{n}{k}$ is the
    $(k+1)$st number in the $(n+1)$st row.
    \par
    \begin{center} \label{PascalTriangle}
      \begin{tikzpicture}
        \foreach \n in {0,...,5}{
          \foreach \k in {0,...,\n}{
            \node at (\k - \n/2, -\n){$\binomialCoefficient{\n}{\k}$};
          }
        }
      \end{tikzpicture}
    \end{center}
    \item Notice that all numbers in Pascal's triangle are
    natural numbers. Use part \ref{2.3:a} to prove by
    induction that $\displaystyle\binom{n}{k}$ is always
    a natural number. (Your entire proof by induction will,
    in a sense, be summed up in a glance by Pascal's
    triangle.)
    \item Give another proof that $\displaystyle\binom{n}{k}$
    is a natural number by showing that
    $\displaystyle\binom{n}{k}$ is the number of sets of
    exactly $k$ integers each chosen from $1,\dots,n$.
    \item Prove the ``binomial theorem'': If $a$ and $b$
    are any numbers and $n$ is a natural number, then
    \begin{IEEEeqnarray*}{rCl}
      (a+b)^n & = & a^n + \binom{n}{1}a^{n-1}b
      + \binom{n}{2}a^{n-2}b^2+\cdots+\binom{n}{n-1}ab^{n-1}
      + b^n \\
              & = & \sum_{j=0}^n \binom{n}{j}a^{n-j}b^j.
    \end{IEEEeqnarray*}
    \item Prove that
    \begin{enumerate}[label=(\roman*)]
      \item \label{2.3:ei}
      $\displaystyle
      \sum_{j=0}^n \binom{n}{j}
      = \binom{n}{0} + \cdots + \binom{n}{n}
      = 2^n$.
      \item \label{2.3:eii}
      $\displaystyle
      \sum_{j=0}^n (-1)^j \binom{n}{j}
      = \binom{n}{0} - \binom{n}{1} + \cdots \pm \binom{n}{n}
      = 0$.
      \item $\displaystyle
      \sum_{l\text{ odd}} \binom{n}{l}
      = \binom{n}{1} + \binom{n}{3} + \cdots
      = 2^{n - 1}$.
      \item $\displaystyle
      \sum_{l\text{ even}} \binom{n}{l}
      = \binom{n}{0} + \binom{n}{2} + \cdots
      = 2^{n - 1}$.
    \end{enumerate}
  \end{enumerate}
\end{pr}

\begin{solution} % s2.3
  \begin{enumerate}[label=(\alph*)]
    \item Starting from the left-hand side,
    \begin{IEEEeqnarray*}{rCl}
      \binom{n + 1}{k} & = &
      \frac{n!(n - k + 1 + k)}{k!(n - k + 1)!} \\
                       & = &
      \frac{n!(n - k + 1)}{k!(n - k + 1)!}
    + \frac{n!k}{k!(n - k + 1)!}\\
                       & = &
      \frac{n!}{k!(n - k)!}
    + \frac{n!}{(k - 1)![n - (k - 1)]!} \\
                       & = &
      \binom{n}{k} + \binom{n}{k - 1}
    \end{IEEEeqnarray*}
    \item It is sufficient to prove that $\displaystyle
    \binom{n}{k}$ is a natural number for all
    $1 \leq k \leq (n - 1)$. If $n = 1$, then
    \begin{equation*}
      \binom{2}{1} = \binom{1}{0} + \binom{1}{1} = 2
    \end{equation*}
    Suppose that $\displaystyle\binom{n}{k}$ is natural
    number for any $n$ and $1 \leq k \leq n - 1$.
    Then for any $1 \leq k \leq n$, $\displaystyle
    \binom{n + 1}{k}$ is the sum of two natural numbers,
    and therefore it must be a natural number.
    \item It is sufficient to prove for the case
    $0 < k \leq n$.
    If $n=1$, the claim is trivial. Suppose that
    $\displaystyle\binom{n}{k}$ is the number of sets
    of $k$ integers each chosen from $1,\dots,n$; then
    $\displaystyle\binom{n + 1}{k}$ must include
    $\displaystyle\binom{n}{k}$ sets of $k$ integers
    \note{without} the newly added element and a number
    of sets of $k$ integers \note{with} the newly added
    element. The latter is exactly $\displaystyle
    \binom{n}{k - 1}$ and is a natural number by assumption.
    Thereby, $\displaystyle\binom{n+1}{k}$ must be
    a natural number.
    \item We prove by induction on $n$. If $n=1$, there
    is nothing to prove. Suppose the binomial theorem holds
    for $n$; then for $n+1$,
    \begin{IEEEeqnarray*}{rCl}
      (a+b)^n(a+b) & = &
      \sum_{j=0}^n \binom{n}{j} a^{n-j}b^j(a + b) \\
                   & = &
      \sum_{j=0}^n \binom{n}{j} a^{n+1-j}b^j
    + \sum_{j=0}^n \binom{n}{j} a^{n-j}b^{j+1}    \\
                   & = &
      a^{n + 1} + b^{n + 1}
    + \sum_{j=1}^n \binom{n}{j} a^{n+1-j}b^j      \\
                   & + &
      \sum_{j=1}^n \binom{n}{j-1} a^{n+1-j}b^j    \\
                   & = &
      \sum_{j=1}^n \binom{n+1}{j} a^{n+1-j}b^j
    + a^{n+1} + b^{n+1}                           \\
                   & = &
      \sum_{j=0}^{n+1} \binom{n+1}{j} a^{n+1-j}b^j
    \end{IEEEeqnarray*}
    which completes the proof.
    \item This part relies heavily on
    the binomial theorem from the above.
    \begin{enumerate}[label=(\roman*)]
      \item This is directly from the above: Applying
      the binomial theorem for $a=b=1$ yields the result.
      \item Let $a=1$ and $b=-1$ yield the result.
      \item Applying \ref{2.3:ei} $+$ \ref{2.3:eii},
      we derive that for $l$ even,
      \begin{equation*}
        \sum_{l\text{ even}} \binom{n}{l} = 2^{n - 1}
      \end{equation*}
      Thereby, $\displaystyle
      \sum_{l\text{ odd}} \binom{n}{l} = 2^n - 2^{n - 1}
      = 2^{n - 1}$.
      \item See the above.
    \end{enumerate}
    One thing to note: Both of the previous parts do not
    have a final term expressed in their sum due to the
    dependence of value $n$ (if $n$ is even or odd).
  \end{enumerate}
\end{solution}

\begin{pr} \label{2.4} % 2.4
  \begin{enumerate}[label=(\alph*)]
    \item \label{2.4:a}
    Prove that
    \begin{equation*}
      \sum_{k=0}^l \binom{n}{k} \binom{m}{l-k}
      = \binom{n+m}{l}.
    \end{equation*}
    Hint: Apply the binomial theorem to $(1+x)^n(1+x)^m$.
    \item \label{2.4:b}
    Prove that
    \begin{equation*}
      \sum_{k=0}^n \binom{n}{k}^2 = \binom{2n}{n}.
    \end{equation*}
  \end{enumerate}
\end{pr}

\begin{solution} % 2.4 solved
  \begin{enumerate}[label=(\alph*)]
    \item Remind that $(1+x)^n(1+x)^m = (1+x)^{n+m}$. Hence,
    \begin{equation*}
      \sum_{k=0}^n \binom{n}{k}x^k \cdot
      \sum_{j=0}^m \binom{m}{j}x^j
      = \sum_{l=0}^{n+m} \binom{n+m}{l}x^l
    \end{equation*}
    Observing that each term of $x^l$ is
    \begin{equation*}
      \sum_{k=0}^l \binom{n}{k}\binom{m}{l-k}
    \end{equation*}
    for every $k$ and $j$ such that $j = l - k$.
    \item From \ref{2.4:a}, let $m = l = n$ and observe
    that for all $0\leq k\leq n$,
    \begin{equation*}
      \binom{n}{k} = \binom{n}{n - k}
    \end{equation*}
    since $\dfrac{n!}{k!(n-k)!} = \dfrac{n!}{(n-k)!k!}$.
    Hence,
    $\displaystyle\sum_{k=0}^n \binom{n}{k}^2 = \binom{2n}{n}$.
  \end{enumerate}
\end{solution}

\begin{pr} \label{2.5} % 2.5
  \begin{enumerate}[label=(\alph*)]
    \item \label{2.5:a}
    Prove by induction on $n$ that
    \begin{equation*}
      1 + r + r^2 + \cdots + r^n = \frac{1-r^{n+1}}{1-r}
    \end{equation*}
    if $r\neq1$ (if $r=1$, evaluating the sum certainly
    presents no problem).
    \item \label{2.5:b}
    Derive this result by setting $S = 1 + r + \cdots + r^n$,
    multiplying this equation by $r$, and solving the
    two equations for $S$.
  \end{enumerate}
\end{pr}

\begin{solution} % 2.5 solved
  \begin{enumerate}[label=(\alph*)]
    \item If $n=1$, then the formula immediately holds. Suppose
    now that it holds for $n$, then for $n+1$,
    \begin{IEEEeqnarray*}{rCl}
      1 + r + \cdots + r^n + r^{n+1}
    & = & \frac{1 - r^{n+1}}{1-r} + r^{n+1} \\
    & = & \frac{1 - r^{n + 2}}{1 - r}
  \end{IEEEeqnarray*}
  which completes the proof.
    \item Let
    \begin{equation*}
      S = 1 + r + r^2 + \cdots + r^n
    \end{equation*}
    then
    \begin{equation*}
      rS = r + r^2 + r^3 + \cdots + r^{n+1}
    \end{equation*}
    Solving for $S$, we easily obtain
    \begin{equation*}
      S = \frac{1 - r^{n+1}}{1 - r}
    \end{equation*}
  \end{enumerate}
\end{solution}

\begin{pr} \label{2.6} % 2.6
  The formula for $1^2 + \cdots + n^2$ may be derived as
  follows. We begin with the formula
  \begin{equation*}
    (k+1)^3 - k^3 = 3k^2 + 3k + 1.
  \end{equation*}
  Writing this formula for $k=1,\dots,n$ and adding,
  we obtain

  \bigskip
  \begin{IEEEeqnarray*}{rCl}
    2^3 - 1^3 & = & 3 \cdot 1^2 + 3 \cdot 1 + 1 \\
    3^3 - 2^3 & = & 3 \cdot 2^2 + 3 \cdot 2 + 1 \\
    \vdots                                      \\
    (n + 1)^3 - n^3 & = & 3 \cdot n^2 + 3 \cdot n + 1 \\
    \hline                                            \\
    (n + 1)^3 - 1 & = & 3[1^2 + \cdots + n^2]
    + 3[1 + \cdots + n] + n.
  \end{IEEEeqnarray*}

  \bigskip
  Thus we can find
  $\displaystyle \sum_{k=1}^n k^2$ if we already know
  $\displaystyle \sum_{k=1}^n k$ (which could have been
  found in a similar way). Use this method to find
  \begin{enumerate}[label=(\roman*)]
    \item $1^3 + \cdots + n^3$.
    \item $1^4 + \cdots + n^4$.
    \item $\displaystyle
    \frac{1}{1\cdot2} + \frac{1}{2\cdot3} + \cdots
    + \frac{1}{n(n + 1)}$.
    \item $\displaystyle
    \frac{3}{1^2\cdot2^2} + \frac{5}{2^2\cdot3^2} + \cdots
    + \frac{2n + 1}{n^2(n + 1)^2}$.
  \end{enumerate}
\end{pr}

\begin{solution} % 2.6 solved
  \begin{enumerate}[label=(\roman*)]
    \item We easily derive that
    \begin{equation*}
      (k+1)^4 - k^4 = 4k^3 + 6k^2 + 4k + 1
    \end{equation*}
    Hence,
    \begin{IEEEeqnarray*}{rCl}
      2^4 - 1^4 &=& 4\cdot1^3 + 6\cdot1^2 + 4\cdot1 + 1 \\
      3^4 - 2^4 &=& 4\cdot2^3 + 6\cdot2^2 + 4\cdot2 + 1 \\
      \vdots                                            \\
      (n + 1)^4 - n^4 &=& 4\cdot n^3 + 6 \cdot n^2
      + 4 \cdot n + 1                                   \\
      \hline                                            \\
      (n + 1)^4 - 1 &=& 4[1^3 + 2^3 + \cdots + n^3]     \\
      & + & 6[1^2 + 2^2 + \cdots + n^2] + 4[1 + 2 + \cdots + n]
      + n
    \end{IEEEeqnarray*}
    It is easily derivable that
    \begin{equation*}
      1^3 + \cdots + n^3 = \frac{1}{4}n^2(n + 1)^2
    \end{equation*}
    \item The easiest way to calculate
    $(k + 1)^5 - k^5$ is to use the Pascal triangle from
    page \pageref{PascalTriangle}: We derive the result
    \begin{equation*}
      (k + 1)^5 - k^5 = 5k^4 + 10k^3 + 10k^2 + 5k + 1
    \end{equation*}
    A similar step from above shows that
    \begin{IEEEeqnarray*}{rCl}
      (n + 1)^5 - 1 & = & 5[1^4 + \cdots + n^4]
      + 10[1^3 + \cdots + n^3] \\
                      & + & 10[1^2 + \cdots + n^2]
      + 5[1 + \cdots + n] + n
    \end{IEEEeqnarray*}
    We derive from the available results,
    \begin{equation*}
      1^4 + 2^4 + \cdots + n^4 = \frac{1}{30}
      n(n+1)(2n+1)(3n^2 + 3n - 1)
    \end{equation*}
    \item Observe that
    \begin{equation*}
      \frac{1}{k} - \frac{1}{k + 1} = \frac{1}{k(k + 1)}
    \end{equation*}
    Henceforth,
    \begin{IEEEeqnarray*}{rCl}
      1 - \frac{1}{2} &=& \frac{1}{1\cdot2} \\
      \frac{1}{2} - \frac{1}{3} &=& \frac{1}{2\cdot3} \\
      \vdots                                          \\
      \frac{1}{n} - \frac{1}{n + 1}
      &=& \frac{1}{n\cdot(n+1)}                       \\
      \hline                                          \\
      1 - \frac{1}{n + 1} & = &
      \frac{1}{1\cdot2} + \frac{1}{2\cdot3} + \cdots
      + \frac{1}{n(n + 1)}                            \\
      \frac{1}{1\cdot2} + \frac{1}{2\cdot3} + \cdots
      + \frac{1}{n(n + 1)}& = & \frac{n}{n + 1}
    \end{IEEEeqnarray*}
    \item See that
    \begin{equation*}
      \frac{1}{k^2} - \frac{1}{(k+1)^2}
      = \frac{2k + 1}{k^2(k + 1)^2}
    \end{equation*}
    Henceforth,
    \begin{IEEEeqnarray*}{rCl}
      1 - \frac{1}{2^2} &=& \frac{3}{1^2\cdot2^2} \\
      \frac{1}{2^2} - \frac{1}{3^2} &=& \frac{5}{2^2\cdot3^2} \\
      \vdots \\
      \frac{1}{n^2} - \frac{1}{(n+1)^2} &=&
      \frac{2n + 1}{n^2(n + 1)^2}                  \\
      \hline                                       \\
      \frac{3}{1^2\cdot2^2} + \frac{5}{2^2\cdot3^2}
      + \cdots + \frac{2n + 1}{n^2(n + 1)^2}
      & = & \frac{n(n+2)}{(n+1)^2}
    \end{IEEEeqnarray*}
  \end{enumerate}
\end{solution}

\begin{pr} \label{2.7}% 2.7
  Use the method of \autoref{2.6} to show that $\displaystyle
  \sum_{k=1}^n k^p$ can always be written in the form
  \begin{equation*}
    \frac{n^{p+1}}{p+1} + An^{p} + Bn^{p-1}
    + Cn^{p-2} + \cdots .
  \end{equation*}
  (The first 10 such expressions are
  \begin{IEEEeqnarray*}{rCCCCCCCCCCCCCCl}
    \sum_{k=1}^n k & = &
    \frac{1}{2}n^2 &+& \frac{1}{2}n \\
    \sum_{k=1}^n k^2 &=&
    \frac{1}{3}n^3 &+& \frac{1}{2}n^2
    &+& \frac{1}{6}n                                     \\
    \sum_{k=1}^n k^3 &=&
    \frac{1}{4}n^4 &+& \frac{1}{2}n^3
    &+& \frac{1}{4}n^2                                   \\
    \sum_{k=1}^n k^4 &=&
    \frac{1}{5}n^5 &+& \frac{1}{2}n^4
    &+& \frac{1}{3}n^3 &-& \frac{1}{30}n                   \\
    \sum_{k=1}^n k^5 &=&
    \frac{1}{6}n^6 &+& \frac{1}{2}n^5
    &+& \frac{5}{12}n^4 &-& \frac{1}{12}n^2                \\
    \sum_{k=1}^n k^6 &=&
    \frac{1}{7}n^7 &+& \frac{1}{2}n^6
    &+& \frac{1}{2}n^5 &-& \frac{1}{6}n^3 &+& \frac{1}{42}n  \\
    \sum_{k=1}^n k^7 &=&
    \frac{1}{8}n^8 &+& \frac{1}{2}n^7
    &+& \frac{7}{12}n^6 &-& \frac{7}{24}n^4
    &+& \frac{1}{12}n^2 \\
    \sum_{k=1}^n k^8 &=&
    \frac{1}{9}n^9 &+& \frac{1}{2}n^8
    &+& \frac{2}{3}n^7 &-& \frac{7}{15}n^5 &+& \frac{2}{9}n^3
    &-& \frac{1}{30}n                                       \\
    \sum_{k=1}^n k^9 &=&
    \frac{1}{10}n^{10}
    &+& \frac{1}{2}n^9 &+& \frac{3}{4}n^8
    &-& \frac{7}{10}n^6 &+& \frac{1}{2}n^4
    &-& \frac{3}{20}n^2  \\
    \sum_{k=1}^n k^{10} &=&
    \frac{1}{11}n^{11} &+& \frac{1}{2}n^{10}
    &+& \frac{5}{6}n^9 &-& 1n^7 &+& 1n^5 &-& \frac{1}{2}n^3
    &+& \frac{5}{66}n.
  \end{IEEEeqnarray*}
  Notice that the coefficients in the second column are
  always $\dfrac{1}{2}$, and that after the third column
  the powers of $n$ with nonzero coefficients decrease
  by $2$ until $n$ or $n^2$ is reached. The coefficients
  in all but the first two columns seem to be rather
  haphazard, but there is actually is some sort of pattern;
  finding it may be regarded as a super-perspicacity test.
  See Problem 27.17 for the complete story.)
\end{pr}

\begin{solution} % 2.7 solved
  We prove by complete induction on $p$. Know that
  \begin{equation*}
    (k+1)^{p+2} - k^{p+2}
  = \sum_{j=2}^{p} \binom{p+2}{j}k^j
  + (p+2)k^{p+2} + (p+2)k + 1
  \end{equation*}
  If $p=1$, it is obvious. Suppose the expression holds
  for $1,\dots,p$, then for $p+1$,
  using the known method presented above, we easily obtain
  \begin{equation*}
    (n+1)^{p+2} - 1
  = \sum_{j=2}^{p} \binom{p+2}{j} \sum_{k=1}^n k^j
  + (p+2)\sum_{k=1}^n k^{p+1} + (p+2)\sum_{k=1}^n k + n
  \end{equation*}
  Notice that the left-hand side highest power of $n$ is
  $p+2$ while that on the right is $p+1$ by assumption.
  Dividing both sides by $p+2$ and solve for
  $\sum_{k=1}^n k^{p+1}$, we have
  \begin{equation*}
    \sum_{k=1}^n k^{p+1}
  = \frac{n^{p+2}}{p+2} + An^{p+1} + Bn^p + \cdots
  \end{equation*}
  for some number $A$, $B$.
\end{solution}

\begin{pr} % 2.8
  Prove that every natural number is either even or odd.
\end{pr}

\begin{solution} % 2.8 solved
  Suppose that $B$
  is the set of all natural numbers that is neither even
  nor odd. Suppose that $B\neq\emptyset$. Obviously,
  $1\notin B$ since $1$ is an odd number. If $k \notin B$,
  then $k$ is either even or odd. Then $k + 1$ is either
  odd or even, and therefore $(k+1) \notin B$. Henceforth,
  $B = \emptyset$, contradicting assumption.
\end{solution}

\begin{pr} % 2.9
  Prove that if a set $A$ of natural numbers contains $n_0$
  and contains $k+1$ whenever it contains $k$, then $A$
  contains all natural numbers $\geq n_0$.
\end{pr}

\begin{solution} % 2.9 solved
  Obivously, $n_0 \in A$. Suppose that $(n_0 + k - 1) \in A$.
  By assumption, $(n_0 + k) \in A$. Hence, $A$ contains all
  natural numbers $\geq n_0$.
\end{solution}

\begin{pr} % 2.10
  Prove the principle of mathematical induction from the
  well-ordering principle.
\end{pr}

\begin{solution} \label{2.10} % 2.10 solved
  We shall prove the theorem by contradiction,
  \begin{thm}[Principle of mathematical induction]
    If $A$ is the set of natural numbers and
    \begin{enumerate}[label=(\arabic*)]
      \item \label{2.10:1}
      $1$ is in $A$
      \item \label{2.10:2}
      $k+1$ is in $A$ wherever $k$ is in $A$,
    \end{enumerate}
    then $A$ is the set of all natural numbers.
  \end{thm}
  Let $B\neq\emptyset$
  be the set of all natural numbers \note{not}
  in $A$. By the well-ordering principle, $B$ has a least
  member $k$. By construction, $k \notin A$. By property
  \ref{2.10:2}, $(k - 1) \notin A$, which means that
  $(k - 1) \in B$, but this contradicts the fact that
  $k$ is the least member in $B$.
\end{solution}

\begin{pr} % 2.11
  Prove the principle of complete induction from the ordinary
  principle of induction. Hint: If $A$ contains $1$ and $A$
  contains $n + 1$ whenever it contains $1,\dots,n$,
  consider the set $B$ of all $k$ such that $1,\dots,k$
  are all in $A$.
\end{pr}

\begin{solution} % 2.11 solved
  We want to prove the following theorem,
  \begin{thm}[Principle of complete induction]
    If $A$ is the set of natural numbers and
    \begin{enumerate}[label=(\arabic*)]
      \item $1$ is in $A$
      \item $n+1$ is in $A$ if $1,\dots,n$ is in $A$
    \end{enumerate}
    then $A$ is the set of all natural numbers.
  \end{thm}
  Let $B$ be the set of all $k$ such that
  $1,2,3,\dots,k$ are all in $A$. Obviously, $B \subseteq A$.
  Conversely, $A \subseteq B$ because if not,
  then $1,\dots,k$ are not all in $A$ which implies
  $1 \notin A$.
  Hence,
  $A = B\neq\emptyset$
  since $1\in A$. We see that $B$ satisfies:
  \begin{enumerate}[label=(\arabic*)]
    \item $1$ is in $B$
    \item $k+1$ is in $B$ whenever $k$ is in $B$. Otherwise,
    $(k+1) \notin A$, which implies $1,\dots,k$
    are not in $A$,
    which implies $k \notin B$.
  \end{enumerate}
  By principle of mathematical induction, $B$ is the set of
  all natural numbers, which means $A$ is the same.
\end{solution}

\begin{pr} \label{2.12}% 2.12
  \begin{enumerate}[label=(\alph*)]
    \item If $a$ is rational and $b$ is irrational, is
    $a + b$ necessarily irrational? What if $a$ and $b$
    are both irrational?
    \item If $a$ is rational and $b$ is irrational,
    is $ab$ necessarily irrational? (Careful!)
    \item Is there a number $a$ such that $a^2$ is irrational,
    but $a^4$ is rational?
    \item Are there two irrational numbers whose sum and
    product are both rational?
  \end{enumerate}
\end{pr}

\begin{solution} % 2.12 solved
  \begin{enumerate}[label=(\alph*)]
    \item If $a+b$ were rational then
    $b = (a+b) - a$ would be rational! Hence,
    $a+b$ is necessarily irrational. However,
    if $a$ and $b$ are irrational, then let $b = r - a$ for
    any rational $r$, then $a+b$ is rational.
    \item $ab$ is not necessarily irrational for the case
    when $a = 0$ and $b = \sqrt{2}$. However, if $a \neq 0$,
    then if $ab$ were rational, then $\dfrac{ab}{a} = b$
    would be rational. Hence, $ab$ must be irrational.
    \item Let $a = \sqrt{\sqrt{2}}$: $a^2 = \sqrt{2}$, but
    $a^4 = 2$.
    \item Let $a = -\sqrt{2}$ and $b = \sqrt{2}$. Then
    both $a+b$ and $ab$ are rational even though $a$
    and $b$ are irrational.
  \end{enumerate}
\end{solution}

\begin{pr} \label{2.13} % 2.13
  \begin{enumerate}[label=(\alph*)]
    \item Prove that $\sqrt{3},\sqrt{5},$ and $\sqrt{6}$ are
    irrational. Hint: To treat $\sqrt{3}$, for example, use
    the fact that every integer is of the form $3n$ or
    $3n + 1$ or $3n + 2$. Why does this proof not work for
    $\sqrt{4}$?
    \item Prove that $\sqrt[3]{2}$ and $\sqrt[3]{3}$ are
    irrational.
  \end{enumerate}
\end{pr}

\begin{solution} % 2.13 solved
  \begin{enumerate}[label=(\alph*)]
    \item Since every number is written in the form of
    either $3n$, $3n+1$ or $3n+2$ for some natural $n$, then
    \begin{IEEEeqnarray*}{rClCl}
      (3n+1)^2  &=& 9n^2+6n+1 &=& 3(3n^2 + 2n) + 1 \\
      (3n+2)^2  &=& 9n^2+12n+4 &=& 3(3n^2 + 4n + 1) + 1
    \end{IEEEeqnarray*}
    This implies that if a squared integer is divisible by $3$,
    then the integer is divisible by $3$.
    Suppose now that $\sqrt{3}$ were rational. Then there
    would be
    a pair of integer $p$ and $q$ with no common divisor such
    that
    \begin{equation*}
      \sqrt{3} = \frac{p}{q}
    \end{equation*}
    This implies that $p^2 = 3 q^2$. Therefore, there exist
    a natural $k$ such that $p = 3 k$, which implies
    \begin{equation*}
      p^2=9k^2=3q^2
    \end{equation*}
    which means that $q^2 = 3 k^2$. Henceforth, $q = 3m$
    for some natural $m$, but these imply that $p$ and $q$
    have a common divisor: A contradiction.
    Similar for $\sqrt{5}$, we consider
    \begin{IEEEeqnarray*}{rClCl}
      (5n+1)^2 &=& 25n^2 + 10n + 1 &=& 5(5n^2 + 2n) + 1\\
      (5n+2)^2 &=& 25n^2 + 20n + 4 &=& 5(5n^2 + 4n) + 4\\
      (5n+3)^2 &=& 25n^2 + 30n + 9 &=& 5(5n^2 + 6n + 1) + 4\\
      (5n+4)^2 &=& 25n^2 + 40n + 16 &=&
      5(5n^2 + 8n + 3) + 1
    \end{IEEEeqnarray*}
    We see that the very same method of proof is applicable
    for $\sqrt{5}$ and $\sqrt{6}$: If $k^2$  is divisible
    by either $5$ or $6$, then $k$ must be divisible by
    $5$ or $6$, respectively. Hence, we easily conclude
    that $\sqrt{5}$ and $\sqrt{6}$ are irrational.
    \par
    Now we cannot use this method of proof for $\sqrt{4}$
    since the statement \note{if $k^2$ is divisible by $4$,
    then $k$ is divisible by $4$} is false by letting $k=2$.
    \item Let us first consider
    \begin{IEEEeqnarray*}{rClCl}
      (2n + 1)^3 &=& 8n^3 + 12n^2 + 6n + 1 &=& 2(4n^3
      + 6n^2 + 3n) + 1
    \end{IEEEeqnarray*}
    We conclude that if $k^3$ is divisible by $2$, then
    $k$ is divisible by $2$ for any natural $k$. Suppose
    $\sqrt[3]{2}$ were rational, then there would be
    integers $p$ and $q$ with no common divisor such that
    \begin{equation*}
      \sqrt[3]{2} = \frac{p}{q}
    \end{equation*}
    then $p^3 = 2 q^3$: $p$ is divisible by $2$,
    and hence $4k^3 = q^3$ for some $k$: $q$ is divisible by $2$:
    A contradiction since both have common divisor. We
    conclude that $\sqrt[3]{2}$ is irrational.
    Similarly,
    \begin{IEEEeqnarray*}{rClCl}
      (3n+1)^3 &=& 27n^3 + 27n^2 + 9n + 1 &=&
      3(9n^3 + 9n^2 + 3n) + 1                 \\
      (3n+2)^3 &=& 27n^3 + 54n^2 + 36n^2 + 8 &=&
      3(9n^3 + 18n^2 + 12n^2 + 2) + 2
    \end{IEEEeqnarray*}
    Hence, for any natural $k$ if $k^3$ is divisible by $3$,
    then $k$ is divisible by $3$. The very same method as above
    is used to prove that $\sqrt[3]{3}$ is irrational.
  \end{enumerate}
\end{solution}

\begin{pr} % 2.14
  Prove that
  \begin{enumerate}[label=(\alph*)]
    \item $\sqrt{2} + \sqrt{6}$ is irrational.
    \item $\sqrt{2} + \sqrt{3}$ is irrational.
  \end{enumerate}
\end{pr}

\begin{solution} % 2.14 solved
  \begin{enumerate}[label=(\alph*)]
    \item Suppose $\sqrt{2} + \sqrt{6}$ is rational. Then
    $(\sqrt{2} + \sqrt{6})^2 = 8 + 4\sqrt{3}$ must be rational.
    Hence, $4\sqrt{3}= 8 + 4\sqrt{3} - 8$ must be rational:
    A contradiction by since $4\neq0$ and $\sqrt{3}$ is
    irrational.
    \item Suppose that $\sqrt{2} + \sqrt{3}$ is rational.
    Then $5 + 2\sqrt{6}$ is rational, which implies
    $2\sqrt{6} = 5 + 2\sqrt{6} - 5$ is rational: A contradiction
    similar to the above.
  \end{enumerate}
\end{solution}

\begin{pr} % 2.15
  \begin{enumerate}[label=(\alph*)]
    \item Prove that if $x = p + \sqrt{q}$ where $p$ and $q$
    are rational, and $m$ is a natural number, then
    $x^m = a + b\sqrt{q}$ for some rational $a$ and $b$.
    \item Prove also that $(p - \sqrt{q})^m = a - b\sqrt{q}$.
  \end{enumerate}
\end{pr}

\begin{solution} % 2.15 solved
  \begin{enumerate}[label=(\alph*)]
    \item Proof is by induction. Let $m=1$, the statement
    holds. Suppose the statement holds for $m$, then for
    $m + 1$,
    \begin{IEEEeqnarray*}{rCl}
      x^{m+1} & = & (a + b\sqrt{q})(p + \sqrt{q}) \\
              & = & (ap + bq) + (a + bp)\sqrt{q}
    \end{IEEEeqnarray*}
    The conclusion follows.
    \item The proof is exactly the same as above except
    now that $x = p - \sqrt{q}$. Hence, for $m + 1$,
    \begin{IEEEeqnarray*}{rCl}
      x^{m+1} & = & (a - b\sqrt{q})(p - \sqrt{q}) \\
              & = & (ap + bq) - (a + bp)\sqrt{q}
    \end{IEEEeqnarray*}
  \end{enumerate}
\end{solution}

\begin{pr} % 2.16
  \begin{enumerate}[label=(\alph*)]
    \item Prove that if $m$ and $n$ are natural numbers
    and $\dfrac{m^2}{n^2} < 2$, then
    $\dfrac{(m + 2n)^2}{(m + n)^2} > 2$; show, moreover, that
    \begin{equation*}
      \frac{(m + 2n)^2}{(m + n)^2} - 2 < 2 - \frac{m^2}{n^2}
    \end{equation*}
    \item Prove the same results with all inequality signs
    reversed.
    \item Prove that if $\dfrac{m}{n} < \sqrt{2}$, then there
    is another rational number $\dfrac{m'}{n'}$ with
    $\dfrac{m}{n} < \dfrac{m'}{n'} < \sqrt{2}$.
  \end{enumerate}
\end{pr}

\begin{solution} % 2.16 solved
  \begin{enumerate}[label=(\alph*)]
    \item We first observe from $m^2/n^2 < 2$,
    \begin{equation*}
      (m + n)^2 < 3n^2 + 2nm
    \end{equation*}
    And from here,
    \begin{IEEEeqnarray*}{rCl}
      \frac{(m+2n)^2}{(m+n)^2} & = &
      \frac{(m+n)^2 + n^2 + 2n(m+n)}{(m+n)^2} \\
                               & = &
      1 + \frac{3n^2 + 2nm}{(n+m)^2}          \\
                               & > &
      1 + 1                                   \\
                              & = &
      2
    \end{IEEEeqnarray*}
    Furthermore,
    \begin{IEEEeqnarray*}{rCl}
      \frac{(m+2n)^2}{(m+n)^2} - 2 & = &
      \frac{-(m+n)^2 + n^2 + 2n(m+n)}{(m+n)^2}  \\
                                   & = &
      \frac{2n^2 - m^2}{(m+n)^2}                \\
                                   & < &
      \frac{2n^2 - m^2}{n^2}                    \\
                                   & = &
      2 - \frac{m^2}{n^2}
    \end{IEEEeqnarray*}
    \item Applying the exact method from the above, we
    see that if $m^2/n^2 > 2$, then
    $\dfrac{(m+2n)^2}{(m+n)^2} < 2$. As for the last inequality,
    note from the third line above: since $2n^2 - m^2 < 0$,
    the inequality sign is reversed. Therefore,
    $\dfrac{(m+2n)^2}{(m+n)^2} - 2 > 2 - \dfrac{m^2}{n^2}$.
    \item Let $m' = 2n^2 - m^2$ and $n' = (m+n)^2$, we obtain
    \begin{equation*}
      \frac{m^2}{n^2} < \frac{m'^2}{n'^2} < 2
    \end{equation*}
    Hence, $\dfrac{m}{n} < \dfrac{m'}{n'} < \sqrt{2}$.
  \end{enumerate}
\end{solution}

\begin{pr}[*] % 2.17
  It seems likely that $\sqrt{n}$ is irrational whenever
  the natural number $n$ is not the square of another
  natural number. Although the method of \autoref{2.13}
  may actually be used to treat any particular case, it is
  not clear in advance that it will always work, and a proof
  for the general case requires some extra information.
  A natural number $p$ is called \boldText{prime number} if
  it is impossible to write $p = ab$ for natural numbers
  $a$ and $b$ unless one of these is $p$, and the other $1$;
  for convenience we also agree that $1$ is \note{not}
  a prime number. The first few prime numbers are
  $2,3,5,7,11,13,17,19$. If $n > 1$ is not a prime, then
  $n = ab$, with $a$ and $b$ both $< n$; if either $a$ or
  $b$ is not a prime it can be factored similarly; continuing
  in this way proves that we can write $n$ as a product of
  primes. For example, $28 = 4 \cdot 7 = 2\cdot2\cdot7$.
  \begin{enumerate}[label=(\alph*)]
    \item Turn this argument into a rigorous proof by
    complete induction.
  \end{enumerate}
  A fundamental theorem about integers, which we will not
  prove here, states that this factorization is unique,
  except for the order of the factors. Thus, for example,
  $28$ can never be written as a product of primes one of
  which is $3$, nor can it be written in a way that involves
  $2$ only once (now you should appreciate why $1$ is not
  allowed as a prime).
  \begin{enumerate}[label=(\alph*),resume]
    \item Using this fact, prove that $\sqrt{n}$ is irrational
    unless $n = m^2$ for some natural number $m$.
    \item Prove more generally that $\sqrt[k]{n}$ is irrational
    unless $n = m^k$
    \item Prove that there cannot be only finitely many
    prime numbers $p_1, p_2, \dots , p_n$ by considering
    $p_1\cdot p_2 \cdot \ldots \cdot p_n + 1$.
  \end{enumerate}
\end{pr}

\begin{solution} % 2.17 solved
  \begin{enumerate}[label=(\alph*)]
    \item Let $n = 2$, then it is a prime itself. Suppose
    that the argument holds for $n=1,\dots,k$. Hence,
    for $n = k + 1$, if $k + 1$ is a prime, we are done. If not,
    then $k + 1$ can be written as a product
    of two numbers $n_1$ and $n_2$ smaller than $k + 1$.
    By assumption, $n_1$ and $n_2$ can be rewritten as
    the product of primes: Hence, any number $n > 1$
    can be written as a product of primes.
  \end{enumerate}
\end{solution}
