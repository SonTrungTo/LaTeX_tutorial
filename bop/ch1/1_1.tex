\section{Introduction to sets}
   \subsection{} \{\ldots $-16,-11,-6,-1,4,9,14,$\ldots \}.
   \subsection{} \{\ldots $-7,-4,-1,2,5,8,11,$\ldots \}.
   \subsection{} \{$-2,-1,\ldots,6$\}.
   \subsection{} \{$1,2,\ldots,7$\}.
   \subsection{} \{$\pm\sqrt{3}$\}.
   \subsection{} \{$\pm 3$\}.
   \subsection{} \{$-2,-3$\}.
   \subsection{} \{$0,-2,-3$\}.
   \subsection{} $\mathbb{Z}$.
   \subsection{} \{$2\pi x : x \in \mathbb{Z}$\}.
   \subsection{} $\{-4,-3,\ldots,4\}$.
   \subsection{} $\{-2,-1,\ldots,2\}$.
   \subsection{} $\{0\}$.
   \subsection{} $\{-20,-15,-10,\ldots,10,15,20\}$.
   \subsection{} Let's call the set $S$. It's clear that every member of $S$ is
   an integer. Conversely, note that $n=5n+2(-2n)$, $n \in \mathbb{Z}$. Therefore,
   $S=\integerset$.
   \subsection{} The reasoning is similar, but note that there exists no $a,b \in \integerset$
   such that either $n=6n+2b$ or $n=6a+2b$, $n \in \integerset$. Also, note that
   $6a+2b=2(3a+b)$, in which $n=3n-2n$. Therefore, $S$ is the set of even integers in $\integerset$.
   \begin{equation}
     S = \{2n: n \in \integerset\} \subset \integerset
   \end{equation}

   \subsection{} $\{2^n: n\in\naturalset\}$.
   \subsection{}
   \subsection{}
