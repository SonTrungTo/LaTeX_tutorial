\documentclass[a4paper,11pt]{memoir}
\usepackage{natbib}
\usepackage[affil-it]{authblk}
\usepackage{amsmath,amssymb,amsthm,fancyhdr,enumitem,%
microtype,pgfplots,tikz,enumerate}
\usepackage[hyperindex=true,%
unicode]{hyperref}
\pgfplotsset{compat=1.15}
\usepackage[retainorgcmds]{IEEEtrantools}
\newcommand{\note}[1]{\emph{#1}}
\newcommand{\boldText}[1]{\textbf{#1}}

\theoremstyle{plain}        \newtheorem{id}{Lemma}[chapter]
                            \newtheorem{thm}{Theorem}[chapter]
\theoremstyle{definition}   \newtheorem{ex}{Exercise}[section]
\theoremstyle{remark}       \newtheorem{ab}{Conjecture}[section]
                            \newtheorem{cd}{Remark}[section]

\newenvironment{exx}[1]{\begin{ex}[#1]$ $ \newline\nobreak\ignorespaces}%
{\end{ex}}
\setsecnumdepth{subsection}
\setcounter{tocdepth}{2}

\title{Solutions to Book Of Proof}
\author{Son To\\
$<$\href{mailto:son.trung.to@gmail.com}{son.trung.to@gmail.com}$>$}
\affil{Fazer Oy, Arcada Ammattikorkeakoulu}

\begin{document}
  \maketitle
  \thispagestyle{empty}

  \frontmatter
   \chapter{Preface}
     An attempt at solving all exercises by Richard Hammack.
     \begin{flushright}
       Helsinki, Finland \\$11^{\text{th}}$ January, $2020$
     \end{flushright}
   \clearpage
   \tableofcontents

   \mainmatter
   \part{Fundamentals}
   \chapter{Sets}
   \section{Introduction to sets}
      \subsection{} \{\ldots $-16,-11,-6,-1,4,9,14,$\ldots \}.
      \subsection{} \{\ldots $-7,-4,-1,2,5,8,11,$\ldots \}.





\end{document}
