\documentclass[a4paper,11pt]{article}
\pagestyle{headings}
\title{Gnuplot and LaTeX}
\author{Son To}
\date{June 6,2017}
\begin{document}
\maketitle
\tableofcontents
\newpage
\section{Using gnuplot for LaTeX: A tutorial}
Figure~\ref{fig:eg1} is a basic plot, using with `set key'/`unset key'
command to add/remove legends.
  \begin{figure}
    \begin{center}
      % GNUPLOT: LaTeX picture
\setlength{\unitlength}{0.240900pt}
\ifx\plotpoint\undefined\newsavebox{\plotpoint}\fi
\sbox{\plotpoint}{\rule[-0.200pt]{0.400pt}{0.400pt}}%
\begin{picture}(1500,900)(0,0)
\sbox{\plotpoint}{\rule[-0.200pt]{0.400pt}{0.400pt}}%
\put(130.0,82.0){\rule[-0.200pt]{4.818pt}{0.400pt}}
\put(110,82){\makebox(0,0)[r]{$-1$}}
\put(1419.0,82.0){\rule[-0.200pt]{4.818pt}{0.400pt}}
\put(130.0,160.0){\rule[-0.200pt]{4.818pt}{0.400pt}}
\put(110,160){\makebox(0,0)[r]{$-0.8$}}
\put(1419.0,160.0){\rule[-0.200pt]{4.818pt}{0.400pt}}
\put(130.0,237.0){\rule[-0.200pt]{4.818pt}{0.400pt}}
\put(110,237){\makebox(0,0)[r]{$-0.6$}}
\put(1419.0,237.0){\rule[-0.200pt]{4.818pt}{0.400pt}}
\put(130.0,315.0){\rule[-0.200pt]{4.818pt}{0.400pt}}
\put(110,315){\makebox(0,0)[r]{$-0.4$}}
\put(1419.0,315.0){\rule[-0.200pt]{4.818pt}{0.400pt}}
\put(130.0,393.0){\rule[-0.200pt]{4.818pt}{0.400pt}}
\put(110,393){\makebox(0,0)[r]{$-0.2$}}
\put(1419.0,393.0){\rule[-0.200pt]{4.818pt}{0.400pt}}
\put(130.0,471.0){\rule[-0.200pt]{4.818pt}{0.400pt}}
\put(110,471){\makebox(0,0)[r]{$0$}}
\put(1419.0,471.0){\rule[-0.200pt]{4.818pt}{0.400pt}}
\put(130.0,548.0){\rule[-0.200pt]{4.818pt}{0.400pt}}
\put(110,548){\makebox(0,0)[r]{$0.2$}}
\put(1419.0,548.0){\rule[-0.200pt]{4.818pt}{0.400pt}}
\put(130.0,626.0){\rule[-0.200pt]{4.818pt}{0.400pt}}
\put(110,626){\makebox(0,0)[r]{$0.4$}}
\put(1419.0,626.0){\rule[-0.200pt]{4.818pt}{0.400pt}}
\put(130.0,704.0){\rule[-0.200pt]{4.818pt}{0.400pt}}
\put(110,704){\makebox(0,0)[r]{$0.6$}}
\put(1419.0,704.0){\rule[-0.200pt]{4.818pt}{0.400pt}}
\put(130.0,781.0){\rule[-0.200pt]{4.818pt}{0.400pt}}
\put(110,781){\makebox(0,0)[r]{$0.8$}}
\put(1419.0,781.0){\rule[-0.200pt]{4.818pt}{0.400pt}}
\put(130.0,859.0){\rule[-0.200pt]{4.818pt}{0.400pt}}
\put(110,859){\makebox(0,0)[r]{$1$}}
\put(1419.0,859.0){\rule[-0.200pt]{4.818pt}{0.400pt}}
\put(159.0,82.0){\rule[-0.200pt]{0.400pt}{4.818pt}}
\put(159,41){\makebox(0,0){$-3$}}
\put(159.0,839.0){\rule[-0.200pt]{0.400pt}{4.818pt}}
\put(368.0,82.0){\rule[-0.200pt]{0.400pt}{4.818pt}}
\put(368,41){\makebox(0,0){$-2$}}
\put(368.0,839.0){\rule[-0.200pt]{0.400pt}{4.818pt}}
\put(576.0,82.0){\rule[-0.200pt]{0.400pt}{4.818pt}}
\put(576,41){\makebox(0,0){$-1$}}
\put(576.0,839.0){\rule[-0.200pt]{0.400pt}{4.818pt}}
\put(785.0,82.0){\rule[-0.200pt]{0.400pt}{4.818pt}}
\put(785,41){\makebox(0,0){$0$}}
\put(785.0,839.0){\rule[-0.200pt]{0.400pt}{4.818pt}}
\put(993.0,82.0){\rule[-0.200pt]{0.400pt}{4.818pt}}
\put(993,41){\makebox(0,0){$1$}}
\put(993.0,839.0){\rule[-0.200pt]{0.400pt}{4.818pt}}
\put(1201.0,82.0){\rule[-0.200pt]{0.400pt}{4.818pt}}
\put(1201,41){\makebox(0,0){$2$}}
\put(1201.0,839.0){\rule[-0.200pt]{0.400pt}{4.818pt}}
\put(1410.0,82.0){\rule[-0.200pt]{0.400pt}{4.818pt}}
\put(1410,41){\makebox(0,0){$3$}}
\put(1410.0,839.0){\rule[-0.200pt]{0.400pt}{4.818pt}}
\put(130.0,82.0){\rule[-0.200pt]{0.400pt}{187.179pt}}
\put(130.0,82.0){\rule[-0.200pt]{315.338pt}{0.400pt}}
\put(1439.0,82.0){\rule[-0.200pt]{0.400pt}{187.179pt}}
\put(130.0,859.0){\rule[-0.200pt]{315.338pt}{0.400pt}}
\put(1279,818){\makebox(0,0)[r]{cos(x)}}
\put(1299.0,818.0){\rule[-0.200pt]{24.090pt}{0.400pt}}
\put(130,82){\usebox{\plotpoint}}
\put(130,81.67){\rule{3.132pt}{0.400pt}}
\multiput(130.00,81.17)(6.500,1.000){2}{\rule{1.566pt}{0.400pt}}
\put(143,83.17){\rule{2.700pt}{0.400pt}}
\multiput(143.00,82.17)(7.396,2.000){2}{\rule{1.350pt}{0.400pt}}
\multiput(156.00,85.60)(1.943,0.468){5}{\rule{1.500pt}{0.113pt}}
\multiput(156.00,84.17)(10.887,4.000){2}{\rule{0.750pt}{0.400pt}}
\multiput(170.00,89.59)(1.123,0.482){9}{\rule{0.967pt}{0.116pt}}
\multiput(170.00,88.17)(10.994,6.000){2}{\rule{0.483pt}{0.400pt}}
\multiput(183.00,95.59)(0.950,0.485){11}{\rule{0.843pt}{0.117pt}}
\multiput(183.00,94.17)(11.251,7.000){2}{\rule{0.421pt}{0.400pt}}
\multiput(196.00,102.59)(0.824,0.488){13}{\rule{0.750pt}{0.117pt}}
\multiput(196.00,101.17)(11.443,8.000){2}{\rule{0.375pt}{0.400pt}}
\multiput(209.00,110.58)(0.704,0.491){17}{\rule{0.660pt}{0.118pt}}
\multiput(209.00,109.17)(12.630,10.000){2}{\rule{0.330pt}{0.400pt}}
\multiput(223.00,120.58)(0.590,0.492){19}{\rule{0.573pt}{0.118pt}}
\multiput(223.00,119.17)(11.811,11.000){2}{\rule{0.286pt}{0.400pt}}
\multiput(236.00,131.58)(0.497,0.493){23}{\rule{0.500pt}{0.119pt}}
\multiput(236.00,130.17)(11.962,13.000){2}{\rule{0.250pt}{0.400pt}}
\multiput(249.58,144.00)(0.493,0.536){23}{\rule{0.119pt}{0.531pt}}
\multiput(248.17,144.00)(13.000,12.898){2}{\rule{0.400pt}{0.265pt}}
\multiput(262.58,158.00)(0.493,0.576){23}{\rule{0.119pt}{0.562pt}}
\multiput(261.17,158.00)(13.000,13.834){2}{\rule{0.400pt}{0.281pt}}
\multiput(275.58,173.00)(0.494,0.607){25}{\rule{0.119pt}{0.586pt}}
\multiput(274.17,173.00)(14.000,15.784){2}{\rule{0.400pt}{0.293pt}}
\multiput(289.58,190.00)(0.493,0.655){23}{\rule{0.119pt}{0.623pt}}
\multiput(288.17,190.00)(13.000,15.707){2}{\rule{0.400pt}{0.312pt}}
\multiput(302.58,207.00)(0.493,0.734){23}{\rule{0.119pt}{0.685pt}}
\multiput(301.17,207.00)(13.000,17.579){2}{\rule{0.400pt}{0.342pt}}
\multiput(315.58,226.00)(0.493,0.734){23}{\rule{0.119pt}{0.685pt}}
\multiput(314.17,226.00)(13.000,17.579){2}{\rule{0.400pt}{0.342pt}}
\multiput(328.58,245.00)(0.494,0.754){25}{\rule{0.119pt}{0.700pt}}
\multiput(327.17,245.00)(14.000,19.547){2}{\rule{0.400pt}{0.350pt}}
\multiput(342.58,266.00)(0.493,0.814){23}{\rule{0.119pt}{0.746pt}}
\multiput(341.17,266.00)(13.000,19.451){2}{\rule{0.400pt}{0.373pt}}
\multiput(355.58,287.00)(0.493,0.853){23}{\rule{0.119pt}{0.777pt}}
\multiput(354.17,287.00)(13.000,20.387){2}{\rule{0.400pt}{0.388pt}}
\multiput(368.58,309.00)(0.493,0.893){23}{\rule{0.119pt}{0.808pt}}
\multiput(367.17,309.00)(13.000,21.324){2}{\rule{0.400pt}{0.404pt}}
\multiput(381.58,332.00)(0.493,0.893){23}{\rule{0.119pt}{0.808pt}}
\multiput(380.17,332.00)(13.000,21.324){2}{\rule{0.400pt}{0.404pt}}
\multiput(394.58,355.00)(0.494,0.864){25}{\rule{0.119pt}{0.786pt}}
\multiput(393.17,355.00)(14.000,22.369){2}{\rule{0.400pt}{0.393pt}}
\multiput(408.58,379.00)(0.493,0.933){23}{\rule{0.119pt}{0.838pt}}
\multiput(407.17,379.00)(13.000,22.260){2}{\rule{0.400pt}{0.419pt}}
\multiput(421.58,403.00)(0.493,0.972){23}{\rule{0.119pt}{0.869pt}}
\multiput(420.17,403.00)(13.000,23.196){2}{\rule{0.400pt}{0.435pt}}
\multiput(434.58,428.00)(0.493,0.933){23}{\rule{0.119pt}{0.838pt}}
\multiput(433.17,428.00)(13.000,22.260){2}{\rule{0.400pt}{0.419pt}}
\multiput(447.58,452.00)(0.494,0.901){25}{\rule{0.119pt}{0.814pt}}
\multiput(446.17,452.00)(14.000,23.310){2}{\rule{0.400pt}{0.407pt}}
\multiput(461.58,477.00)(0.493,0.972){23}{\rule{0.119pt}{0.869pt}}
\multiput(460.17,477.00)(13.000,23.196){2}{\rule{0.400pt}{0.435pt}}
\multiput(474.58,502.00)(0.493,0.933){23}{\rule{0.119pt}{0.838pt}}
\multiput(473.17,502.00)(13.000,22.260){2}{\rule{0.400pt}{0.419pt}}
\multiput(487.58,526.00)(0.493,0.933){23}{\rule{0.119pt}{0.838pt}}
\multiput(486.17,526.00)(13.000,22.260){2}{\rule{0.400pt}{0.419pt}}
\multiput(500.58,550.00)(0.493,0.933){23}{\rule{0.119pt}{0.838pt}}
\multiput(499.17,550.00)(13.000,22.260){2}{\rule{0.400pt}{0.419pt}}
\multiput(513.58,574.00)(0.494,0.864){25}{\rule{0.119pt}{0.786pt}}
\multiput(512.17,574.00)(14.000,22.369){2}{\rule{0.400pt}{0.393pt}}
\multiput(527.58,598.00)(0.493,0.893){23}{\rule{0.119pt}{0.808pt}}
\multiput(526.17,598.00)(13.000,21.324){2}{\rule{0.400pt}{0.404pt}}
\multiput(540.58,621.00)(0.493,0.853){23}{\rule{0.119pt}{0.777pt}}
\multiput(539.17,621.00)(13.000,20.387){2}{\rule{0.400pt}{0.388pt}}
\multiput(553.58,643.00)(0.493,0.853){23}{\rule{0.119pt}{0.777pt}}
\multiput(552.17,643.00)(13.000,20.387){2}{\rule{0.400pt}{0.388pt}}
\multiput(566.58,665.00)(0.494,0.754){25}{\rule{0.119pt}{0.700pt}}
\multiput(565.17,665.00)(14.000,19.547){2}{\rule{0.400pt}{0.350pt}}
\multiput(580.58,686.00)(0.493,0.774){23}{\rule{0.119pt}{0.715pt}}
\multiput(579.17,686.00)(13.000,18.515){2}{\rule{0.400pt}{0.358pt}}
\multiput(593.58,706.00)(0.493,0.734){23}{\rule{0.119pt}{0.685pt}}
\multiput(592.17,706.00)(13.000,17.579){2}{\rule{0.400pt}{0.342pt}}
\multiput(606.58,725.00)(0.493,0.695){23}{\rule{0.119pt}{0.654pt}}
\multiput(605.17,725.00)(13.000,16.643){2}{\rule{0.400pt}{0.327pt}}
\multiput(619.58,743.00)(0.493,0.655){23}{\rule{0.119pt}{0.623pt}}
\multiput(618.17,743.00)(13.000,15.707){2}{\rule{0.400pt}{0.312pt}}
\multiput(632.58,760.00)(0.494,0.570){25}{\rule{0.119pt}{0.557pt}}
\multiput(631.17,760.00)(14.000,14.844){2}{\rule{0.400pt}{0.279pt}}
\multiput(646.58,776.00)(0.493,0.576){23}{\rule{0.119pt}{0.562pt}}
\multiput(645.17,776.00)(13.000,13.834){2}{\rule{0.400pt}{0.281pt}}
\multiput(659.00,791.58)(0.497,0.493){23}{\rule{0.500pt}{0.119pt}}
\multiput(659.00,790.17)(11.962,13.000){2}{\rule{0.250pt}{0.400pt}}
\multiput(672.00,804.58)(0.539,0.492){21}{\rule{0.533pt}{0.119pt}}
\multiput(672.00,803.17)(11.893,12.000){2}{\rule{0.267pt}{0.400pt}}
\multiput(685.00,816.58)(0.704,0.491){17}{\rule{0.660pt}{0.118pt}}
\multiput(685.00,815.17)(12.630,10.000){2}{\rule{0.330pt}{0.400pt}}
\multiput(699.00,826.58)(0.652,0.491){17}{\rule{0.620pt}{0.118pt}}
\multiput(699.00,825.17)(11.713,10.000){2}{\rule{0.310pt}{0.400pt}}
\multiput(712.00,836.59)(0.950,0.485){11}{\rule{0.843pt}{0.117pt}}
\multiput(712.00,835.17)(11.251,7.000){2}{\rule{0.421pt}{0.400pt}}
\multiput(725.00,843.59)(1.123,0.482){9}{\rule{0.967pt}{0.116pt}}
\multiput(725.00,842.17)(10.994,6.000){2}{\rule{0.483pt}{0.400pt}}
\multiput(738.00,849.59)(1.378,0.477){7}{\rule{1.140pt}{0.115pt}}
\multiput(738.00,848.17)(10.634,5.000){2}{\rule{0.570pt}{0.400pt}}
\multiput(751.00,854.61)(2.918,0.447){3}{\rule{1.967pt}{0.108pt}}
\multiput(751.00,853.17)(9.918,3.000){2}{\rule{0.983pt}{0.400pt}}
\put(765,857.17){\rule{2.700pt}{0.400pt}}
\multiput(765.00,856.17)(7.396,2.000){2}{\rule{1.350pt}{0.400pt}}
\put(791,857.17){\rule{2.700pt}{0.400pt}}
\multiput(791.00,858.17)(7.396,-2.000){2}{\rule{1.350pt}{0.400pt}}
\multiput(804.00,855.95)(2.918,-0.447){3}{\rule{1.967pt}{0.108pt}}
\multiput(804.00,856.17)(9.918,-3.000){2}{\rule{0.983pt}{0.400pt}}
\multiput(818.00,852.93)(1.378,-0.477){7}{\rule{1.140pt}{0.115pt}}
\multiput(818.00,853.17)(10.634,-5.000){2}{\rule{0.570pt}{0.400pt}}
\multiput(831.00,847.93)(1.123,-0.482){9}{\rule{0.967pt}{0.116pt}}
\multiput(831.00,848.17)(10.994,-6.000){2}{\rule{0.483pt}{0.400pt}}
\multiput(844.00,841.93)(0.950,-0.485){11}{\rule{0.843pt}{0.117pt}}
\multiput(844.00,842.17)(11.251,-7.000){2}{\rule{0.421pt}{0.400pt}}
\multiput(857.00,834.92)(0.652,-0.491){17}{\rule{0.620pt}{0.118pt}}
\multiput(857.00,835.17)(11.713,-10.000){2}{\rule{0.310pt}{0.400pt}}
\multiput(870.00,824.92)(0.704,-0.491){17}{\rule{0.660pt}{0.118pt}}
\multiput(870.00,825.17)(12.630,-10.000){2}{\rule{0.330pt}{0.400pt}}
\multiput(884.00,814.92)(0.539,-0.492){21}{\rule{0.533pt}{0.119pt}}
\multiput(884.00,815.17)(11.893,-12.000){2}{\rule{0.267pt}{0.400pt}}
\multiput(897.00,802.92)(0.497,-0.493){23}{\rule{0.500pt}{0.119pt}}
\multiput(897.00,803.17)(11.962,-13.000){2}{\rule{0.250pt}{0.400pt}}
\multiput(910.58,788.67)(0.493,-0.576){23}{\rule{0.119pt}{0.562pt}}
\multiput(909.17,789.83)(13.000,-13.834){2}{\rule{0.400pt}{0.281pt}}
\multiput(923.58,773.69)(0.494,-0.570){25}{\rule{0.119pt}{0.557pt}}
\multiput(922.17,774.84)(14.000,-14.844){2}{\rule{0.400pt}{0.279pt}}
\multiput(937.58,757.41)(0.493,-0.655){23}{\rule{0.119pt}{0.623pt}}
\multiput(936.17,758.71)(13.000,-15.707){2}{\rule{0.400pt}{0.312pt}}
\multiput(950.58,740.29)(0.493,-0.695){23}{\rule{0.119pt}{0.654pt}}
\multiput(949.17,741.64)(13.000,-16.643){2}{\rule{0.400pt}{0.327pt}}
\multiput(963.58,722.16)(0.493,-0.734){23}{\rule{0.119pt}{0.685pt}}
\multiput(962.17,723.58)(13.000,-17.579){2}{\rule{0.400pt}{0.342pt}}
\multiput(976.58,703.03)(0.493,-0.774){23}{\rule{0.119pt}{0.715pt}}
\multiput(975.17,704.52)(13.000,-18.515){2}{\rule{0.400pt}{0.358pt}}
\multiput(989.58,683.09)(0.494,-0.754){25}{\rule{0.119pt}{0.700pt}}
\multiput(988.17,684.55)(14.000,-19.547){2}{\rule{0.400pt}{0.350pt}}
\multiput(1003.58,661.77)(0.493,-0.853){23}{\rule{0.119pt}{0.777pt}}
\multiput(1002.17,663.39)(13.000,-20.387){2}{\rule{0.400pt}{0.388pt}}
\multiput(1016.58,639.77)(0.493,-0.853){23}{\rule{0.119pt}{0.777pt}}
\multiput(1015.17,641.39)(13.000,-20.387){2}{\rule{0.400pt}{0.388pt}}
\multiput(1029.58,617.65)(0.493,-0.893){23}{\rule{0.119pt}{0.808pt}}
\multiput(1028.17,619.32)(13.000,-21.324){2}{\rule{0.400pt}{0.404pt}}
\multiput(1042.58,594.74)(0.494,-0.864){25}{\rule{0.119pt}{0.786pt}}
\multiput(1041.17,596.37)(14.000,-22.369){2}{\rule{0.400pt}{0.393pt}}
\multiput(1056.58,570.52)(0.493,-0.933){23}{\rule{0.119pt}{0.838pt}}
\multiput(1055.17,572.26)(13.000,-22.260){2}{\rule{0.400pt}{0.419pt}}
\multiput(1069.58,546.52)(0.493,-0.933){23}{\rule{0.119pt}{0.838pt}}
\multiput(1068.17,548.26)(13.000,-22.260){2}{\rule{0.400pt}{0.419pt}}
\multiput(1082.58,522.52)(0.493,-0.933){23}{\rule{0.119pt}{0.838pt}}
\multiput(1081.17,524.26)(13.000,-22.260){2}{\rule{0.400pt}{0.419pt}}
\multiput(1095.58,498.39)(0.493,-0.972){23}{\rule{0.119pt}{0.869pt}}
\multiput(1094.17,500.20)(13.000,-23.196){2}{\rule{0.400pt}{0.435pt}}
\multiput(1108.58,473.62)(0.494,-0.901){25}{\rule{0.119pt}{0.814pt}}
\multiput(1107.17,475.31)(14.000,-23.310){2}{\rule{0.400pt}{0.407pt}}
\multiput(1122.58,448.52)(0.493,-0.933){23}{\rule{0.119pt}{0.838pt}}
\multiput(1121.17,450.26)(13.000,-22.260){2}{\rule{0.400pt}{0.419pt}}
\multiput(1135.58,424.39)(0.493,-0.972){23}{\rule{0.119pt}{0.869pt}}
\multiput(1134.17,426.20)(13.000,-23.196){2}{\rule{0.400pt}{0.435pt}}
\multiput(1148.58,399.52)(0.493,-0.933){23}{\rule{0.119pt}{0.838pt}}
\multiput(1147.17,401.26)(13.000,-22.260){2}{\rule{0.400pt}{0.419pt}}
\multiput(1161.58,375.74)(0.494,-0.864){25}{\rule{0.119pt}{0.786pt}}
\multiput(1160.17,377.37)(14.000,-22.369){2}{\rule{0.400pt}{0.393pt}}
\multiput(1175.58,351.65)(0.493,-0.893){23}{\rule{0.119pt}{0.808pt}}
\multiput(1174.17,353.32)(13.000,-21.324){2}{\rule{0.400pt}{0.404pt}}
\multiput(1188.58,328.65)(0.493,-0.893){23}{\rule{0.119pt}{0.808pt}}
\multiput(1187.17,330.32)(13.000,-21.324){2}{\rule{0.400pt}{0.404pt}}
\multiput(1201.58,305.77)(0.493,-0.853){23}{\rule{0.119pt}{0.777pt}}
\multiput(1200.17,307.39)(13.000,-20.387){2}{\rule{0.400pt}{0.388pt}}
\multiput(1214.58,283.90)(0.493,-0.814){23}{\rule{0.119pt}{0.746pt}}
\multiput(1213.17,285.45)(13.000,-19.451){2}{\rule{0.400pt}{0.373pt}}
\multiput(1227.58,263.09)(0.494,-0.754){25}{\rule{0.119pt}{0.700pt}}
\multiput(1226.17,264.55)(14.000,-19.547){2}{\rule{0.400pt}{0.350pt}}
\multiput(1241.58,242.16)(0.493,-0.734){23}{\rule{0.119pt}{0.685pt}}
\multiput(1240.17,243.58)(13.000,-17.579){2}{\rule{0.400pt}{0.342pt}}
\multiput(1254.58,223.16)(0.493,-0.734){23}{\rule{0.119pt}{0.685pt}}
\multiput(1253.17,224.58)(13.000,-17.579){2}{\rule{0.400pt}{0.342pt}}
\multiput(1267.58,204.41)(0.493,-0.655){23}{\rule{0.119pt}{0.623pt}}
\multiput(1266.17,205.71)(13.000,-15.707){2}{\rule{0.400pt}{0.312pt}}
\multiput(1280.58,187.57)(0.494,-0.607){25}{\rule{0.119pt}{0.586pt}}
\multiput(1279.17,188.78)(14.000,-15.784){2}{\rule{0.400pt}{0.293pt}}
\multiput(1294.58,170.67)(0.493,-0.576){23}{\rule{0.119pt}{0.562pt}}
\multiput(1293.17,171.83)(13.000,-13.834){2}{\rule{0.400pt}{0.281pt}}
\multiput(1307.58,155.80)(0.493,-0.536){23}{\rule{0.119pt}{0.531pt}}
\multiput(1306.17,156.90)(13.000,-12.898){2}{\rule{0.400pt}{0.265pt}}
\multiput(1320.00,142.92)(0.497,-0.493){23}{\rule{0.500pt}{0.119pt}}
\multiput(1320.00,143.17)(11.962,-13.000){2}{\rule{0.250pt}{0.400pt}}
\multiput(1333.00,129.92)(0.590,-0.492){19}{\rule{0.573pt}{0.118pt}}
\multiput(1333.00,130.17)(11.811,-11.000){2}{\rule{0.286pt}{0.400pt}}
\multiput(1346.00,118.92)(0.704,-0.491){17}{\rule{0.660pt}{0.118pt}}
\multiput(1346.00,119.17)(12.630,-10.000){2}{\rule{0.330pt}{0.400pt}}
\multiput(1360.00,108.93)(0.824,-0.488){13}{\rule{0.750pt}{0.117pt}}
\multiput(1360.00,109.17)(11.443,-8.000){2}{\rule{0.375pt}{0.400pt}}
\multiput(1373.00,100.93)(0.950,-0.485){11}{\rule{0.843pt}{0.117pt}}
\multiput(1373.00,101.17)(11.251,-7.000){2}{\rule{0.421pt}{0.400pt}}
\multiput(1386.00,93.93)(1.123,-0.482){9}{\rule{0.967pt}{0.116pt}}
\multiput(1386.00,94.17)(10.994,-6.000){2}{\rule{0.483pt}{0.400pt}}
\multiput(1399.00,87.94)(1.943,-0.468){5}{\rule{1.500pt}{0.113pt}}
\multiput(1399.00,88.17)(10.887,-4.000){2}{\rule{0.750pt}{0.400pt}}
\put(1413,83.17){\rule{2.700pt}{0.400pt}}
\multiput(1413.00,84.17)(7.396,-2.000){2}{\rule{1.350pt}{0.400pt}}
\put(1426,81.67){\rule{3.132pt}{0.400pt}}
\multiput(1426.00,82.17)(6.500,-1.000){2}{\rule{1.566pt}{0.400pt}}
\put(778.0,859.0){\rule[-0.200pt]{3.132pt}{0.400pt}}
\put(130.0,82.0){\rule[-0.200pt]{0.400pt}{187.179pt}}
\put(130.0,82.0){\rule[-0.200pt]{315.338pt}{0.400pt}}
\put(1439.0,82.0){\rule[-0.200pt]{0.400pt}{187.179pt}}
\put(130.0,859.0){\rule[-0.200pt]{315.338pt}{0.400pt}}
\end{picture}

      \caption{A basic gnuplot}
      \label{fig:eg1}
    \end{center}
  \end{figure}

Figure~\ref{fig:eg2} embellishes the plot.
\begin{figure}
  \begin{center}
    % GNUPLOT: LaTeX picture
\setlength{\unitlength}{0.240900pt}
\ifx\plotpoint\undefined\newsavebox{\plotpoint}\fi
\begin{picture}(1500,1200)(0,0)
\sbox{\plotpoint}{\rule[-0.200pt]{0.400pt}{0.400pt}}%
\put(151.0,131.0){\rule[-0.200pt]{4.818pt}{0.400pt}}
\put(131,131){\makebox(0,0)[r]{$0$}}
\put(1419.0,131.0){\rule[-0.200pt]{4.818pt}{0.400pt}}
\put(151.0,320.0){\rule[-0.200pt]{4.818pt}{0.400pt}}
\put(131,320){\makebox(0,0)[r]{$0.2$}}
\put(1419.0,320.0){\rule[-0.200pt]{4.818pt}{0.400pt}}
\put(151.0,509.0){\rule[-0.200pt]{4.818pt}{0.400pt}}
\put(131,509){\makebox(0,0)[r]{$0.4$}}
\put(1419.0,509.0){\rule[-0.200pt]{4.818pt}{0.400pt}}
\put(151.0,698.0){\rule[-0.200pt]{4.818pt}{0.400pt}}
\put(131,698){\makebox(0,0)[r]{$0.6$}}
\put(1419.0,698.0){\rule[-0.200pt]{4.818pt}{0.400pt}}
\put(151.0,887.0){\rule[-0.200pt]{4.818pt}{0.400pt}}
\put(131,887){\makebox(0,0)[r]{$0.8$}}
\put(1419.0,887.0){\rule[-0.200pt]{4.818pt}{0.400pt}}
\put(151.0,1076.0){\rule[-0.200pt]{4.818pt}{0.400pt}}
\put(1419.0,1076.0){\rule[-0.200pt]{4.818pt}{0.400pt}}
\put(151.0,131.0){\rule[-0.200pt]{0.400pt}{4.818pt}}
\put(151,90){\makebox(0,0){$0$}}
\put(151.0,1056.0){\rule[-0.200pt]{0.400pt}{4.818pt}}
\put(356.0,131.0){\rule[-0.200pt]{0.400pt}{4.818pt}}
\put(356,90){\makebox(0,0){$1$}}
\put(356.0,1056.0){\rule[-0.200pt]{0.400pt}{4.818pt}}
\put(561.0,131.0){\rule[-0.200pt]{0.400pt}{4.818pt}}
\put(561,90){\makebox(0,0){$2$}}
\put(561.0,1056.0){\rule[-0.200pt]{0.400pt}{4.818pt}}
\put(766.0,131.0){\rule[-0.200pt]{0.400pt}{4.818pt}}
\put(766,90){\makebox(0,0){$3$}}
\put(766.0,1056.0){\rule[-0.200pt]{0.400pt}{4.818pt}}
\put(971.0,131.0){\rule[-0.200pt]{0.400pt}{4.818pt}}
\put(971,90){\makebox(0,0){$4$}}
\put(971.0,1056.0){\rule[-0.200pt]{0.400pt}{4.818pt}}
\put(1176.0,131.0){\rule[-0.200pt]{0.400pt}{4.818pt}}
\put(1176,90){\makebox(0,0){$5$}}
\put(1176.0,1056.0){\rule[-0.200pt]{0.400pt}{4.818pt}}
\put(1382.0,131.0){\rule[-0.200pt]{0.400pt}{4.818pt}}
\put(1382,90){\makebox(0,0){$6$}}
\put(1382.0,1056.0){\rule[-0.200pt]{0.400pt}{4.818pt}}
\put(151.0,131.0){\rule[-0.200pt]{0.400pt}{227.650pt}}
\put(151.0,131.0){\rule[-0.200pt]{310.279pt}{0.400pt}}
\put(1439.0,131.0){\rule[-0.200pt]{0.400pt}{227.650pt}}
\put(151.0,1076.0){\rule[-0.200pt]{310.279pt}{0.400pt}}
\put(30,603){\makebox(0,0){\shortstack{This is\\ the\\ $y$ axis}}}
\put(795,29){\makebox(0,0){This is the $x$ axis}}
\put(831,866){\makebox(0,0)[r]{sin(x)}}
\put(851.0,866.0){\rule[-0.200pt]{24.090pt}{0.400pt}}
\put(151,131){\usebox{\plotpoint}}
\multiput(151.58,131.00)(0.493,2.360){23}{\rule{0.119pt}{1.946pt}}
\multiput(150.17,131.00)(13.000,55.961){2}{\rule{0.400pt}{0.973pt}}
\multiput(164.58,191.00)(0.493,2.360){23}{\rule{0.119pt}{1.946pt}}
\multiput(163.17,191.00)(13.000,55.961){2}{\rule{0.400pt}{0.973pt}}
\multiput(177.58,251.00)(0.493,2.320){23}{\rule{0.119pt}{1.915pt}}
\multiput(176.17,251.00)(13.000,55.025){2}{\rule{0.400pt}{0.958pt}}
\multiput(190.58,310.00)(0.493,2.281){23}{\rule{0.119pt}{1.885pt}}
\multiput(189.17,310.00)(13.000,54.088){2}{\rule{0.400pt}{0.942pt}}
\multiput(203.58,368.00)(0.493,2.281){23}{\rule{0.119pt}{1.885pt}}
\multiput(202.17,368.00)(13.000,54.088){2}{\rule{0.400pt}{0.942pt}}
\multiput(216.58,426.00)(0.493,2.201){23}{\rule{0.119pt}{1.823pt}}
\multiput(215.17,426.00)(13.000,52.216){2}{\rule{0.400pt}{0.912pt}}
\multiput(229.58,482.00)(0.493,2.162){23}{\rule{0.119pt}{1.792pt}}
\multiput(228.17,482.00)(13.000,51.280){2}{\rule{0.400pt}{0.896pt}}
\multiput(242.58,537.00)(0.493,2.083){23}{\rule{0.119pt}{1.731pt}}
\multiput(241.17,537.00)(13.000,49.408){2}{\rule{0.400pt}{0.865pt}}
\multiput(255.58,590.00)(0.493,2.043){23}{\rule{0.119pt}{1.700pt}}
\multiput(254.17,590.00)(13.000,48.472){2}{\rule{0.400pt}{0.850pt}}
\multiput(268.58,642.00)(0.493,1.924){23}{\rule{0.119pt}{1.608pt}}
\multiput(267.17,642.00)(13.000,45.663){2}{\rule{0.400pt}{0.804pt}}
\multiput(281.58,691.00)(0.493,1.845){23}{\rule{0.119pt}{1.546pt}}
\multiput(280.17,691.00)(13.000,43.791){2}{\rule{0.400pt}{0.773pt}}
\multiput(294.58,738.00)(0.493,1.765){23}{\rule{0.119pt}{1.485pt}}
\multiput(293.17,738.00)(13.000,41.919){2}{\rule{0.400pt}{0.742pt}}
\multiput(307.58,783.00)(0.493,1.646){23}{\rule{0.119pt}{1.392pt}}
\multiput(306.17,783.00)(13.000,39.110){2}{\rule{0.400pt}{0.696pt}}
\multiput(320.58,825.00)(0.493,1.527){23}{\rule{0.119pt}{1.300pt}}
\multiput(319.17,825.00)(13.000,36.302){2}{\rule{0.400pt}{0.650pt}}
\multiput(333.58,864.00)(0.493,1.448){23}{\rule{0.119pt}{1.238pt}}
\multiput(332.17,864.00)(13.000,34.430){2}{\rule{0.400pt}{0.619pt}}
\multiput(346.58,901.00)(0.493,1.290){23}{\rule{0.119pt}{1.115pt}}
\multiput(345.17,901.00)(13.000,30.685){2}{\rule{0.400pt}{0.558pt}}
\multiput(359.58,934.00)(0.493,1.171){23}{\rule{0.119pt}{1.023pt}}
\multiput(358.17,934.00)(13.000,27.877){2}{\rule{0.400pt}{0.512pt}}
\multiput(372.58,964.00)(0.493,1.012){23}{\rule{0.119pt}{0.900pt}}
\multiput(371.17,964.00)(13.000,24.132){2}{\rule{0.400pt}{0.450pt}}
\multiput(385.58,990.00)(0.493,0.933){23}{\rule{0.119pt}{0.838pt}}
\multiput(384.17,990.00)(13.000,22.260){2}{\rule{0.400pt}{0.419pt}}
\multiput(398.58,1014.00)(0.493,0.734){23}{\rule{0.119pt}{0.685pt}}
\multiput(397.17,1014.00)(13.000,17.579){2}{\rule{0.400pt}{0.342pt}}
\multiput(411.58,1033.00)(0.493,0.616){23}{\rule{0.119pt}{0.592pt}}
\multiput(410.17,1033.00)(13.000,14.771){2}{\rule{0.400pt}{0.296pt}}
\multiput(424.00,1049.58)(0.497,0.493){23}{\rule{0.500pt}{0.119pt}}
\multiput(424.00,1048.17)(11.962,13.000){2}{\rule{0.250pt}{0.400pt}}
\multiput(437.00,1062.59)(0.824,0.488){13}{\rule{0.750pt}{0.117pt}}
\multiput(437.00,1061.17)(11.443,8.000){2}{\rule{0.375pt}{0.400pt}}
\multiput(450.00,1070.59)(1.378,0.477){7}{\rule{1.140pt}{0.115pt}}
\multiput(450.00,1069.17)(10.634,5.000){2}{\rule{0.570pt}{0.400pt}}
\put(463,1074.67){\rule{3.132pt}{0.400pt}}
\multiput(463.00,1074.17)(6.500,1.000){2}{\rule{1.566pt}{0.400pt}}
\multiput(476.00,1074.95)(2.695,-0.447){3}{\rule{1.833pt}{0.108pt}}
\multiput(476.00,1075.17)(9.195,-3.000){2}{\rule{0.917pt}{0.400pt}}
\multiput(489.00,1071.93)(0.950,-0.485){11}{\rule{0.843pt}{0.117pt}}
\multiput(489.00,1072.17)(11.251,-7.000){2}{\rule{0.421pt}{0.400pt}}
\multiput(502.00,1064.92)(0.652,-0.491){17}{\rule{0.620pt}{0.118pt}}
\multiput(502.00,1065.17)(11.713,-10.000){2}{\rule{0.310pt}{0.400pt}}
\multiput(515.58,1053.80)(0.493,-0.536){23}{\rule{0.119pt}{0.531pt}}
\multiput(514.17,1054.90)(13.000,-12.898){2}{\rule{0.400pt}{0.265pt}}
\multiput(528.58,1039.29)(0.493,-0.695){23}{\rule{0.119pt}{0.654pt}}
\multiput(527.17,1040.64)(13.000,-16.643){2}{\rule{0.400pt}{0.327pt}}
\multiput(541.58,1020.90)(0.493,-0.814){23}{\rule{0.119pt}{0.746pt}}
\multiput(540.17,1022.45)(13.000,-19.451){2}{\rule{0.400pt}{0.373pt}}
\multiput(554.58,999.39)(0.493,-0.972){23}{\rule{0.119pt}{0.869pt}}
\multiput(553.17,1001.20)(13.000,-23.196){2}{\rule{0.400pt}{0.435pt}}
\multiput(567.58,974.01)(0.493,-1.091){23}{\rule{0.119pt}{0.962pt}}
\multiput(566.17,976.00)(13.000,-26.004){2}{\rule{0.400pt}{0.481pt}}
\multiput(580.58,945.50)(0.493,-1.250){23}{\rule{0.119pt}{1.085pt}}
\multiput(579.17,947.75)(13.000,-29.749){2}{\rule{0.400pt}{0.542pt}}
\multiput(593.58,913.24)(0.493,-1.329){23}{\rule{0.119pt}{1.146pt}}
\multiput(592.17,915.62)(13.000,-31.621){2}{\rule{0.400pt}{0.573pt}}
\multiput(606.58,878.73)(0.493,-1.488){23}{\rule{0.119pt}{1.269pt}}
\multiput(605.17,881.37)(13.000,-35.366){2}{\rule{0.400pt}{0.635pt}}
\multiput(619.58,840.35)(0.493,-1.607){23}{\rule{0.119pt}{1.362pt}}
\multiput(618.17,843.17)(13.000,-38.174){2}{\rule{0.400pt}{0.681pt}}
\multiput(632.58,799.09)(0.493,-1.686){23}{\rule{0.119pt}{1.423pt}}
\multiput(631.17,802.05)(13.000,-40.046){2}{\rule{0.400pt}{0.712pt}}
\multiput(645.58,755.71)(0.493,-1.805){23}{\rule{0.119pt}{1.515pt}}
\multiput(644.17,758.85)(13.000,-42.855){2}{\rule{0.400pt}{0.758pt}}
\multiput(658.58,709.45)(0.493,-1.884){23}{\rule{0.119pt}{1.577pt}}
\multiput(657.17,712.73)(13.000,-44.727){2}{\rule{0.400pt}{0.788pt}}
\multiput(671.58,661.07)(0.493,-2.003){23}{\rule{0.119pt}{1.669pt}}
\multiput(670.17,664.54)(13.000,-47.535){2}{\rule{0.400pt}{0.835pt}}
\multiput(684.58,609.94)(0.493,-2.043){23}{\rule{0.119pt}{1.700pt}}
\multiput(683.17,613.47)(13.000,-48.472){2}{\rule{0.400pt}{0.850pt}}
\multiput(697.58,557.69)(0.493,-2.122){23}{\rule{0.119pt}{1.762pt}}
\multiput(696.17,561.34)(13.000,-50.344){2}{\rule{0.400pt}{0.881pt}}
\multiput(710.58,503.43)(0.493,-2.201){23}{\rule{0.119pt}{1.823pt}}
\multiput(709.17,507.22)(13.000,-52.216){2}{\rule{0.400pt}{0.912pt}}
\multiput(723.58,447.43)(0.493,-2.201){23}{\rule{0.119pt}{1.823pt}}
\multiput(722.17,451.22)(13.000,-52.216){2}{\rule{0.400pt}{0.912pt}}
\multiput(736.58,391.18)(0.493,-2.281){23}{\rule{0.119pt}{1.885pt}}
\multiput(735.17,395.09)(13.000,-54.088){2}{\rule{0.400pt}{0.942pt}}
\multiput(749.58,333.05)(0.493,-2.320){23}{\rule{0.119pt}{1.915pt}}
\multiput(748.17,337.02)(13.000,-55.025){2}{\rule{0.400pt}{0.958pt}}
\multiput(762.58,273.92)(0.493,-2.360){23}{\rule{0.119pt}{1.946pt}}
\multiput(761.17,277.96)(13.000,-55.961){2}{\rule{0.400pt}{0.973pt}}
\multiput(775.58,213.92)(0.493,-2.360){23}{\rule{0.119pt}{1.946pt}}
\multiput(774.17,217.96)(13.000,-55.961){2}{\rule{0.400pt}{0.973pt}}
\multiput(788.59,154.23)(0.485,-2.323){11}{\rule{0.117pt}{1.871pt}}
\multiput(787.17,158.12)(7.000,-27.116){2}{\rule{0.400pt}{0.936pt}}
\put(831,825){\makebox(0,0)[r]{cos(x)}}
\multiput(851,825)(20.756,0.000){5}{\usebox{\plotpoint}}
\put(951,825){\usebox{\plotpoint}}
\put(151,1076){\usebox{\plotpoint}}
\put(151.00,1076.00){\usebox{\plotpoint}}
\put(170.90,1070.81){\usebox{\plotpoint}}
\put(188.54,1060.01){\usebox{\plotpoint}}
\multiput(203,1046)(12.608,-16.487){2}{\usebox{\plotpoint}}
\put(226.90,1011.39){\usebox{\plotpoint}}
\put(236.99,993.26){\usebox{\plotpoint}}
\put(246.44,974.78){\usebox{\plotpoint}}
\multiput(255,957)(8.027,-19.141){2}{\usebox{\plotpoint}}
\multiput(268,926)(7.413,-19.387){2}{\usebox{\plotpoint}}
\multiput(281,892)(6.880,-19.582){2}{\usebox{\plotpoint}}
\multiput(294,855)(6.415,-19.739){2}{\usebox{\plotpoint}}
\multiput(307,815)(6.006,-19.867){2}{\usebox{\plotpoint}}
\multiput(320,772)(5.760,-19.940){2}{\usebox{\plotpoint}}
\multiput(333,727)(5.533,-20.004){3}{\usebox{\plotpoint}}
\multiput(346,680)(5.223,-20.088){2}{\usebox{\plotpoint}}
\multiput(359,630)(5.034,-20.136){3}{\usebox{\plotpoint}}
\multiput(372,578)(4.858,-20.179){2}{\usebox{\plotpoint}}
\multiput(385,524)(4.774,-20.199){3}{\usebox{\plotpoint}}
\multiput(398,469)(4.615,-20.236){3}{\usebox{\plotpoint}}
\multiput(411,412)(4.539,-20.253){3}{\usebox{\plotpoint}}
\multiput(424,354)(4.539,-20.253){3}{\usebox{\plotpoint}}
\multiput(437,296)(4.395,-20.285){2}{\usebox{\plotpoint}}
\multiput(450,236)(4.466,-20.269){3}{\usebox{\plotpoint}}
\multiput(463,177)(4.409,-20.282){3}{\usebox{\plotpoint}}
\put(473,131){\usebox{\plotpoint}}
\multiput(1117,131)(4.701,20.216){3}{\usebox{\plotpoint}}
\multiput(1127,174)(4.466,20.269){3}{\usebox{\plotpoint}}
\multiput(1140,233)(4.395,20.285){2}{\usebox{\plotpoint}}
\multiput(1153,293)(4.539,20.253){3}{\usebox{\plotpoint}}
\multiput(1166,351)(4.539,20.253){3}{\usebox{\plotpoint}}
\multiput(1179,409)(4.615,20.236){3}{\usebox{\plotpoint}}
\multiput(1192,466)(4.774,20.199){3}{\usebox{\plotpoint}}
\multiput(1205,521)(4.858,20.179){2}{\usebox{\plotpoint}}
\multiput(1218,575)(5.034,20.136){3}{\usebox{\plotpoint}}
\multiput(1231,627)(5.223,20.088){3}{\usebox{\plotpoint}}
\multiput(1244,677)(5.426,20.034){2}{\usebox{\plotpoint}}
\multiput(1257,725)(5.760,19.940){2}{\usebox{\plotpoint}}
\multiput(1270,770)(6.006,19.867){2}{\usebox{\plotpoint}}
\multiput(1283,813)(6.415,19.739){2}{\usebox{\plotpoint}}
\multiput(1296,853)(6.880,19.582){2}{\usebox{\plotpoint}}
\multiput(1309,890)(7.227,19.457){2}{\usebox{\plotpoint}}
\multiput(1322,925)(8.253,19.044){2}{\usebox{\plotpoint}}
\put(1342.71,971.61){\usebox{\plotpoint}}
\put(1351.90,990.21){\usebox{\plotpoint}}
\multiput(1361,1007)(10.925,17.648){2}{\usebox{\plotpoint}}
\put(1385.22,1042.67){\usebox{\plotpoint}}
\put(1399.60,1057.60){\usebox{\plotpoint}}
\put(1416.45,1069.59){\usebox{\plotpoint}}
\put(1436.11,1075.56){\usebox{\plotpoint}}
\put(1439,1076){\usebox{\plotpoint}}
\put(151,1076){\makebox(0,0){$\times$}}
\put(164,1074){\makebox(0,0){$\times$}}
\put(177,1068){\makebox(0,0){$\times$}}
\put(190,1059){\makebox(0,0){$\times$}}
\put(203,1046){\makebox(0,0){$\times$}}
\put(216,1029){\makebox(0,0){$\times$}}
\put(229,1008){\makebox(0,0){$\times$}}
\put(242,984){\makebox(0,0){$\times$}}
\put(255,957){\makebox(0,0){$\times$}}
\put(268,926){\makebox(0,0){$\times$}}
\put(281,892){\makebox(0,0){$\times$}}
\put(294,855){\makebox(0,0){$\times$}}
\put(307,815){\makebox(0,0){$\times$}}
\put(320,772){\makebox(0,0){$\times$}}
\put(333,727){\makebox(0,0){$\times$}}
\put(346,680){\makebox(0,0){$\times$}}
\put(359,630){\makebox(0,0){$\times$}}
\put(372,578){\makebox(0,0){$\times$}}
\put(385,524){\makebox(0,0){$\times$}}
\put(398,469){\makebox(0,0){$\times$}}
\put(411,412){\makebox(0,0){$\times$}}
\put(424,354){\makebox(0,0){$\times$}}
\put(437,296){\makebox(0,0){$\times$}}
\put(450,236){\makebox(0,0){$\times$}}
\put(463,177){\makebox(0,0){$\times$}}
\put(1127,174){\makebox(0,0){$\times$}}
\put(1140,233){\makebox(0,0){$\times$}}
\put(1153,293){\makebox(0,0){$\times$}}
\put(1166,351){\makebox(0,0){$\times$}}
\put(1179,409){\makebox(0,0){$\times$}}
\put(1192,466){\makebox(0,0){$\times$}}
\put(1205,521){\makebox(0,0){$\times$}}
\put(1218,575){\makebox(0,0){$\times$}}
\put(1231,627){\makebox(0,0){$\times$}}
\put(1244,677){\makebox(0,0){$\times$}}
\put(1257,725){\makebox(0,0){$\times$}}
\put(1270,770){\makebox(0,0){$\times$}}
\put(1283,813){\makebox(0,0){$\times$}}
\put(1296,853){\makebox(0,0){$\times$}}
\put(1309,890){\makebox(0,0){$\times$}}
\put(1322,925){\makebox(0,0){$\times$}}
\put(1335,955){\makebox(0,0){$\times$}}
\put(1348,983){\makebox(0,0){$\times$}}
\put(1361,1007){\makebox(0,0){$\times$}}
\put(1374,1028){\makebox(0,0){$\times$}}
\put(1387,1045){\makebox(0,0){$\times$}}
\put(1400,1058){\makebox(0,0){$\times$}}
\put(1413,1068){\makebox(0,0){$\times$}}
\put(1426,1074){\makebox(0,0){$\times$}}
\put(1439,1076){\makebox(0,0){$\times$}}
\put(901,825){\makebox(0,0){$\times$}}
\put(151.0,131.0){\rule[-0.200pt]{0.400pt}{227.650pt}}
\put(151.0,131.0){\rule[-0.200pt]{310.279pt}{0.400pt}}
\put(1439.0,131.0){\rule[-0.200pt]{0.400pt}{227.650pt}}
\put(151.0,1076.0){\rule[-0.200pt]{310.279pt}{0.400pt}}
\end{picture}

    \caption{An embellished gnuplot}
    \label{fig:eg2}
  \end{center}
\end{figure}

Figure~\ref{fig:eg3} has multiples curves on the same plot.
\begin{figure}
  \begin{center}
    % GNUPLOT: LaTeX picture
\setlength{\unitlength}{0.240900pt}
\ifx\plotpoint\undefined\newsavebox{\plotpoint}\fi
\begin{picture}(1500,1200)(0,0)
\sbox{\plotpoint}{\rule[-0.200pt]{0.400pt}{0.400pt}}%
\put(151.0,131.0){\rule[-0.200pt]{4.818pt}{0.400pt}}
\put(131,131){\makebox(0,0)[r]{$-10$}}
\put(1419.0,131.0){\rule[-0.200pt]{4.818pt}{0.400pt}}
\put(151.0,320.0){\rule[-0.200pt]{4.818pt}{0.400pt}}
\put(131,320){\makebox(0,0)[r]{$-5$}}
\put(1419.0,320.0){\rule[-0.200pt]{4.818pt}{0.400pt}}
\put(151.0,509.0){\rule[-0.200pt]{4.818pt}{0.400pt}}
\put(131,509){\makebox(0,0)[r]{$0$}}
\put(1419.0,509.0){\rule[-0.200pt]{4.818pt}{0.400pt}}
\put(151.0,698.0){\rule[-0.200pt]{4.818pt}{0.400pt}}
\put(131,698){\makebox(0,0)[r]{$5$}}
\put(1419.0,698.0){\rule[-0.200pt]{4.818pt}{0.400pt}}
\put(151.0,887.0){\rule[-0.200pt]{4.818pt}{0.400pt}}
\put(131,887){\makebox(0,0)[r]{$10$}}
\put(1419.0,887.0){\rule[-0.200pt]{4.818pt}{0.400pt}}
\put(151.0,1076.0){\rule[-0.200pt]{4.818pt}{0.400pt}}
\put(1419.0,1076.0){\rule[-0.200pt]{4.818pt}{0.400pt}}
\put(151.0,131.0){\rule[-0.200pt]{0.400pt}{4.818pt}}
\put(151,90){\makebox(0,0){$-10$}}
\put(151.0,1056.0){\rule[-0.200pt]{0.400pt}{4.818pt}}
\put(473.0,131.0){\rule[-0.200pt]{0.400pt}{4.818pt}}
\put(473,90){\makebox(0,0){$-5$}}
\put(473.0,1056.0){\rule[-0.200pt]{0.400pt}{4.818pt}}
\put(795.0,131.0){\rule[-0.200pt]{0.400pt}{4.818pt}}
\put(795,90){\makebox(0,0){$0$}}
\put(795.0,1056.0){\rule[-0.200pt]{0.400pt}{4.818pt}}
\put(1117.0,131.0){\rule[-0.200pt]{0.400pt}{4.818pt}}
\put(1117,90){\makebox(0,0){$5$}}
\put(1117.0,1056.0){\rule[-0.200pt]{0.400pt}{4.818pt}}
\put(1439.0,131.0){\rule[-0.200pt]{0.400pt}{4.818pt}}
\put(1439,90){\makebox(0,0){$10$}}
\put(1439.0,1056.0){\rule[-0.200pt]{0.400pt}{4.818pt}}
\put(151.0,131.0){\rule[-0.200pt]{0.400pt}{227.650pt}}
\put(151.0,131.0){\rule[-0.200pt]{310.279pt}{0.400pt}}
\put(1439.0,131.0){\rule[-0.200pt]{0.400pt}{227.650pt}}
\put(151.0,1076.0){\rule[-0.200pt]{310.279pt}{0.400pt}}
\put(30,603){\makebox(0,0){$y$}}
\put(795,29){\makebox(0,0){$x$}}
\put(977,866){\makebox(0,0)[r]{x}}
\put(997.0,866.0){\rule[-0.200pt]{24.090pt}{0.400pt}}
\put(151,131){\usebox{\plotpoint}}
\multiput(151.00,131.59)(0.824,0.488){13}{\rule{0.750pt}{0.117pt}}
\multiput(151.00,130.17)(11.443,8.000){2}{\rule{0.375pt}{0.400pt}}
\multiput(164.00,139.59)(0.950,0.485){11}{\rule{0.843pt}{0.117pt}}
\multiput(164.00,138.17)(11.251,7.000){2}{\rule{0.421pt}{0.400pt}}
\multiput(177.00,146.59)(0.824,0.488){13}{\rule{0.750pt}{0.117pt}}
\multiput(177.00,145.17)(11.443,8.000){2}{\rule{0.375pt}{0.400pt}}
\multiput(190.00,154.59)(0.824,0.488){13}{\rule{0.750pt}{0.117pt}}
\multiput(190.00,153.17)(11.443,8.000){2}{\rule{0.375pt}{0.400pt}}
\multiput(203.00,162.59)(0.950,0.485){11}{\rule{0.843pt}{0.117pt}}
\multiput(203.00,161.17)(11.251,7.000){2}{\rule{0.421pt}{0.400pt}}
\multiput(216.00,169.59)(0.824,0.488){13}{\rule{0.750pt}{0.117pt}}
\multiput(216.00,168.17)(11.443,8.000){2}{\rule{0.375pt}{0.400pt}}
\multiput(229.00,177.59)(0.950,0.485){11}{\rule{0.843pt}{0.117pt}}
\multiput(229.00,176.17)(11.251,7.000){2}{\rule{0.421pt}{0.400pt}}
\multiput(242.00,184.59)(0.824,0.488){13}{\rule{0.750pt}{0.117pt}}
\multiput(242.00,183.17)(11.443,8.000){2}{\rule{0.375pt}{0.400pt}}
\multiput(255.00,192.59)(0.824,0.488){13}{\rule{0.750pt}{0.117pt}}
\multiput(255.00,191.17)(11.443,8.000){2}{\rule{0.375pt}{0.400pt}}
\multiput(268.00,200.59)(0.950,0.485){11}{\rule{0.843pt}{0.117pt}}
\multiput(268.00,199.17)(11.251,7.000){2}{\rule{0.421pt}{0.400pt}}
\multiput(281.00,207.59)(0.824,0.488){13}{\rule{0.750pt}{0.117pt}}
\multiput(281.00,206.17)(11.443,8.000){2}{\rule{0.375pt}{0.400pt}}
\multiput(294.00,215.59)(0.824,0.488){13}{\rule{0.750pt}{0.117pt}}
\multiput(294.00,214.17)(11.443,8.000){2}{\rule{0.375pt}{0.400pt}}
\multiput(307.00,223.59)(0.950,0.485){11}{\rule{0.843pt}{0.117pt}}
\multiput(307.00,222.17)(11.251,7.000){2}{\rule{0.421pt}{0.400pt}}
\multiput(320.00,230.59)(0.824,0.488){13}{\rule{0.750pt}{0.117pt}}
\multiput(320.00,229.17)(11.443,8.000){2}{\rule{0.375pt}{0.400pt}}
\multiput(333.00,238.59)(0.824,0.488){13}{\rule{0.750pt}{0.117pt}}
\multiput(333.00,237.17)(11.443,8.000){2}{\rule{0.375pt}{0.400pt}}
\multiput(346.00,246.59)(0.950,0.485){11}{\rule{0.843pt}{0.117pt}}
\multiput(346.00,245.17)(11.251,7.000){2}{\rule{0.421pt}{0.400pt}}
\multiput(359.00,253.59)(0.824,0.488){13}{\rule{0.750pt}{0.117pt}}
\multiput(359.00,252.17)(11.443,8.000){2}{\rule{0.375pt}{0.400pt}}
\multiput(372.00,261.59)(0.950,0.485){11}{\rule{0.843pt}{0.117pt}}
\multiput(372.00,260.17)(11.251,7.000){2}{\rule{0.421pt}{0.400pt}}
\multiput(385.00,268.59)(0.824,0.488){13}{\rule{0.750pt}{0.117pt}}
\multiput(385.00,267.17)(11.443,8.000){2}{\rule{0.375pt}{0.400pt}}
\multiput(398.00,276.59)(0.824,0.488){13}{\rule{0.750pt}{0.117pt}}
\multiput(398.00,275.17)(11.443,8.000){2}{\rule{0.375pt}{0.400pt}}
\multiput(411.00,284.59)(0.950,0.485){11}{\rule{0.843pt}{0.117pt}}
\multiput(411.00,283.17)(11.251,7.000){2}{\rule{0.421pt}{0.400pt}}
\multiput(424.00,291.59)(0.824,0.488){13}{\rule{0.750pt}{0.117pt}}
\multiput(424.00,290.17)(11.443,8.000){2}{\rule{0.375pt}{0.400pt}}
\multiput(437.00,299.59)(0.824,0.488){13}{\rule{0.750pt}{0.117pt}}
\multiput(437.00,298.17)(11.443,8.000){2}{\rule{0.375pt}{0.400pt}}
\multiput(450.00,307.59)(0.950,0.485){11}{\rule{0.843pt}{0.117pt}}
\multiput(450.00,306.17)(11.251,7.000){2}{\rule{0.421pt}{0.400pt}}
\multiput(463.00,314.59)(0.824,0.488){13}{\rule{0.750pt}{0.117pt}}
\multiput(463.00,313.17)(11.443,8.000){2}{\rule{0.375pt}{0.400pt}}
\multiput(476.00,322.59)(0.824,0.488){13}{\rule{0.750pt}{0.117pt}}
\multiput(476.00,321.17)(11.443,8.000){2}{\rule{0.375pt}{0.400pt}}
\multiput(489.00,330.59)(0.950,0.485){11}{\rule{0.843pt}{0.117pt}}
\multiput(489.00,329.17)(11.251,7.000){2}{\rule{0.421pt}{0.400pt}}
\multiput(502.00,337.59)(0.824,0.488){13}{\rule{0.750pt}{0.117pt}}
\multiput(502.00,336.17)(11.443,8.000){2}{\rule{0.375pt}{0.400pt}}
\multiput(515.00,345.59)(0.950,0.485){11}{\rule{0.843pt}{0.117pt}}
\multiput(515.00,344.17)(11.251,7.000){2}{\rule{0.421pt}{0.400pt}}
\multiput(528.00,352.59)(0.824,0.488){13}{\rule{0.750pt}{0.117pt}}
\multiput(528.00,351.17)(11.443,8.000){2}{\rule{0.375pt}{0.400pt}}
\multiput(541.00,360.59)(0.824,0.488){13}{\rule{0.750pt}{0.117pt}}
\multiput(541.00,359.17)(11.443,8.000){2}{\rule{0.375pt}{0.400pt}}
\multiput(554.00,368.59)(0.950,0.485){11}{\rule{0.843pt}{0.117pt}}
\multiput(554.00,367.17)(11.251,7.000){2}{\rule{0.421pt}{0.400pt}}
\multiput(567.00,375.59)(0.824,0.488){13}{\rule{0.750pt}{0.117pt}}
\multiput(567.00,374.17)(11.443,8.000){2}{\rule{0.375pt}{0.400pt}}
\multiput(580.00,383.59)(0.824,0.488){13}{\rule{0.750pt}{0.117pt}}
\multiput(580.00,382.17)(11.443,8.000){2}{\rule{0.375pt}{0.400pt}}
\multiput(593.00,391.59)(0.950,0.485){11}{\rule{0.843pt}{0.117pt}}
\multiput(593.00,390.17)(11.251,7.000){2}{\rule{0.421pt}{0.400pt}}
\multiput(606.00,398.59)(0.824,0.488){13}{\rule{0.750pt}{0.117pt}}
\multiput(606.00,397.17)(11.443,8.000){2}{\rule{0.375pt}{0.400pt}}
\multiput(619.00,406.59)(0.824,0.488){13}{\rule{0.750pt}{0.117pt}}
\multiput(619.00,405.17)(11.443,8.000){2}{\rule{0.375pt}{0.400pt}}
\multiput(632.00,414.59)(0.950,0.485){11}{\rule{0.843pt}{0.117pt}}
\multiput(632.00,413.17)(11.251,7.000){2}{\rule{0.421pt}{0.400pt}}
\multiput(645.00,421.59)(0.824,0.488){13}{\rule{0.750pt}{0.117pt}}
\multiput(645.00,420.17)(11.443,8.000){2}{\rule{0.375pt}{0.400pt}}
\multiput(658.00,429.59)(0.950,0.485){11}{\rule{0.843pt}{0.117pt}}
\multiput(658.00,428.17)(11.251,7.000){2}{\rule{0.421pt}{0.400pt}}
\multiput(671.00,436.59)(0.824,0.488){13}{\rule{0.750pt}{0.117pt}}
\multiput(671.00,435.17)(11.443,8.000){2}{\rule{0.375pt}{0.400pt}}
\multiput(684.00,444.59)(0.824,0.488){13}{\rule{0.750pt}{0.117pt}}
\multiput(684.00,443.17)(11.443,8.000){2}{\rule{0.375pt}{0.400pt}}
\multiput(697.00,452.59)(0.950,0.485){11}{\rule{0.843pt}{0.117pt}}
\multiput(697.00,451.17)(11.251,7.000){2}{\rule{0.421pt}{0.400pt}}
\multiput(710.00,459.59)(0.824,0.488){13}{\rule{0.750pt}{0.117pt}}
\multiput(710.00,458.17)(11.443,8.000){2}{\rule{0.375pt}{0.400pt}}
\multiput(723.00,467.59)(0.824,0.488){13}{\rule{0.750pt}{0.117pt}}
\multiput(723.00,466.17)(11.443,8.000){2}{\rule{0.375pt}{0.400pt}}
\multiput(736.00,475.59)(0.950,0.485){11}{\rule{0.843pt}{0.117pt}}
\multiput(736.00,474.17)(11.251,7.000){2}{\rule{0.421pt}{0.400pt}}
\multiput(749.00,482.59)(0.824,0.488){13}{\rule{0.750pt}{0.117pt}}
\multiput(749.00,481.17)(11.443,8.000){2}{\rule{0.375pt}{0.400pt}}
\multiput(762.00,490.59)(0.824,0.488){13}{\rule{0.750pt}{0.117pt}}
\multiput(762.00,489.17)(11.443,8.000){2}{\rule{0.375pt}{0.400pt}}
\multiput(775.00,498.59)(0.950,0.485){11}{\rule{0.843pt}{0.117pt}}
\multiput(775.00,497.17)(11.251,7.000){2}{\rule{0.421pt}{0.400pt}}
\multiput(788.00,505.59)(0.890,0.488){13}{\rule{0.800pt}{0.117pt}}
\multiput(788.00,504.17)(12.340,8.000){2}{\rule{0.400pt}{0.400pt}}
\multiput(802.00,513.59)(0.950,0.485){11}{\rule{0.843pt}{0.117pt}}
\multiput(802.00,512.17)(11.251,7.000){2}{\rule{0.421pt}{0.400pt}}
\multiput(815.00,520.59)(0.824,0.488){13}{\rule{0.750pt}{0.117pt}}
\multiput(815.00,519.17)(11.443,8.000){2}{\rule{0.375pt}{0.400pt}}
\multiput(828.00,528.59)(0.824,0.488){13}{\rule{0.750pt}{0.117pt}}
\multiput(828.00,527.17)(11.443,8.000){2}{\rule{0.375pt}{0.400pt}}
\multiput(841.00,536.59)(0.950,0.485){11}{\rule{0.843pt}{0.117pt}}
\multiput(841.00,535.17)(11.251,7.000){2}{\rule{0.421pt}{0.400pt}}
\multiput(854.00,543.59)(0.824,0.488){13}{\rule{0.750pt}{0.117pt}}
\multiput(854.00,542.17)(11.443,8.000){2}{\rule{0.375pt}{0.400pt}}
\multiput(867.00,551.59)(0.824,0.488){13}{\rule{0.750pt}{0.117pt}}
\multiput(867.00,550.17)(11.443,8.000){2}{\rule{0.375pt}{0.400pt}}
\multiput(880.00,559.59)(0.950,0.485){11}{\rule{0.843pt}{0.117pt}}
\multiput(880.00,558.17)(11.251,7.000){2}{\rule{0.421pt}{0.400pt}}
\multiput(893.00,566.59)(0.824,0.488){13}{\rule{0.750pt}{0.117pt}}
\multiput(893.00,565.17)(11.443,8.000){2}{\rule{0.375pt}{0.400pt}}
\multiput(906.00,574.59)(0.824,0.488){13}{\rule{0.750pt}{0.117pt}}
\multiput(906.00,573.17)(11.443,8.000){2}{\rule{0.375pt}{0.400pt}}
\multiput(919.00,582.59)(0.950,0.485){11}{\rule{0.843pt}{0.117pt}}
\multiput(919.00,581.17)(11.251,7.000){2}{\rule{0.421pt}{0.400pt}}
\multiput(932.00,589.59)(0.824,0.488){13}{\rule{0.750pt}{0.117pt}}
\multiput(932.00,588.17)(11.443,8.000){2}{\rule{0.375pt}{0.400pt}}
\multiput(945.00,597.59)(0.950,0.485){11}{\rule{0.843pt}{0.117pt}}
\multiput(945.00,596.17)(11.251,7.000){2}{\rule{0.421pt}{0.400pt}}
\multiput(958.00,604.59)(0.824,0.488){13}{\rule{0.750pt}{0.117pt}}
\multiput(958.00,603.17)(11.443,8.000){2}{\rule{0.375pt}{0.400pt}}
\multiput(971.00,612.59)(0.824,0.488){13}{\rule{0.750pt}{0.117pt}}
\multiput(971.00,611.17)(11.443,8.000){2}{\rule{0.375pt}{0.400pt}}
\multiput(984.00,620.59)(0.950,0.485){11}{\rule{0.843pt}{0.117pt}}
\multiput(984.00,619.17)(11.251,7.000){2}{\rule{0.421pt}{0.400pt}}
\multiput(997.00,627.59)(0.824,0.488){13}{\rule{0.750pt}{0.117pt}}
\multiput(997.00,626.17)(11.443,8.000){2}{\rule{0.375pt}{0.400pt}}
\multiput(1010.00,635.59)(0.824,0.488){13}{\rule{0.750pt}{0.117pt}}
\multiput(1010.00,634.17)(11.443,8.000){2}{\rule{0.375pt}{0.400pt}}
\multiput(1023.00,643.59)(0.950,0.485){11}{\rule{0.843pt}{0.117pt}}
\multiput(1023.00,642.17)(11.251,7.000){2}{\rule{0.421pt}{0.400pt}}
\multiput(1036.00,650.59)(0.824,0.488){13}{\rule{0.750pt}{0.117pt}}
\multiput(1036.00,649.17)(11.443,8.000){2}{\rule{0.375pt}{0.400pt}}
\multiput(1049.00,658.59)(0.824,0.488){13}{\rule{0.750pt}{0.117pt}}
\multiput(1049.00,657.17)(11.443,8.000){2}{\rule{0.375pt}{0.400pt}}
\multiput(1062.00,666.59)(0.950,0.485){11}{\rule{0.843pt}{0.117pt}}
\multiput(1062.00,665.17)(11.251,7.000){2}{\rule{0.421pt}{0.400pt}}
\multiput(1075.00,673.59)(0.824,0.488){13}{\rule{0.750pt}{0.117pt}}
\multiput(1075.00,672.17)(11.443,8.000){2}{\rule{0.375pt}{0.400pt}}
\multiput(1088.00,681.59)(0.950,0.485){11}{\rule{0.843pt}{0.117pt}}
\multiput(1088.00,680.17)(11.251,7.000){2}{\rule{0.421pt}{0.400pt}}
\multiput(1101.00,688.59)(0.824,0.488){13}{\rule{0.750pt}{0.117pt}}
\multiput(1101.00,687.17)(11.443,8.000){2}{\rule{0.375pt}{0.400pt}}
\multiput(1114.00,696.59)(0.824,0.488){13}{\rule{0.750pt}{0.117pt}}
\multiput(1114.00,695.17)(11.443,8.000){2}{\rule{0.375pt}{0.400pt}}
\multiput(1127.00,704.59)(0.950,0.485){11}{\rule{0.843pt}{0.117pt}}
\multiput(1127.00,703.17)(11.251,7.000){2}{\rule{0.421pt}{0.400pt}}
\multiput(1140.00,711.59)(0.824,0.488){13}{\rule{0.750pt}{0.117pt}}
\multiput(1140.00,710.17)(11.443,8.000){2}{\rule{0.375pt}{0.400pt}}
\multiput(1153.00,719.59)(0.824,0.488){13}{\rule{0.750pt}{0.117pt}}
\multiput(1153.00,718.17)(11.443,8.000){2}{\rule{0.375pt}{0.400pt}}
\multiput(1166.00,727.59)(0.950,0.485){11}{\rule{0.843pt}{0.117pt}}
\multiput(1166.00,726.17)(11.251,7.000){2}{\rule{0.421pt}{0.400pt}}
\multiput(1179.00,734.59)(0.824,0.488){13}{\rule{0.750pt}{0.117pt}}
\multiput(1179.00,733.17)(11.443,8.000){2}{\rule{0.375pt}{0.400pt}}
\multiput(1192.00,742.59)(0.824,0.488){13}{\rule{0.750pt}{0.117pt}}
\multiput(1192.00,741.17)(11.443,8.000){2}{\rule{0.375pt}{0.400pt}}
\multiput(1205.00,750.59)(0.950,0.485){11}{\rule{0.843pt}{0.117pt}}
\multiput(1205.00,749.17)(11.251,7.000){2}{\rule{0.421pt}{0.400pt}}
\multiput(1218.00,757.59)(0.824,0.488){13}{\rule{0.750pt}{0.117pt}}
\multiput(1218.00,756.17)(11.443,8.000){2}{\rule{0.375pt}{0.400pt}}
\multiput(1231.00,765.59)(0.950,0.485){11}{\rule{0.843pt}{0.117pt}}
\multiput(1231.00,764.17)(11.251,7.000){2}{\rule{0.421pt}{0.400pt}}
\multiput(1244.00,772.59)(0.824,0.488){13}{\rule{0.750pt}{0.117pt}}
\multiput(1244.00,771.17)(11.443,8.000){2}{\rule{0.375pt}{0.400pt}}
\multiput(1257.00,780.59)(0.824,0.488){13}{\rule{0.750pt}{0.117pt}}
\multiput(1257.00,779.17)(11.443,8.000){2}{\rule{0.375pt}{0.400pt}}
\multiput(1270.00,788.59)(0.950,0.485){11}{\rule{0.843pt}{0.117pt}}
\multiput(1270.00,787.17)(11.251,7.000){2}{\rule{0.421pt}{0.400pt}}
\multiput(1283.00,795.59)(0.824,0.488){13}{\rule{0.750pt}{0.117pt}}
\multiput(1283.00,794.17)(11.443,8.000){2}{\rule{0.375pt}{0.400pt}}
\multiput(1296.00,803.59)(0.824,0.488){13}{\rule{0.750pt}{0.117pt}}
\multiput(1296.00,802.17)(11.443,8.000){2}{\rule{0.375pt}{0.400pt}}
\multiput(1309.00,811.59)(0.950,0.485){11}{\rule{0.843pt}{0.117pt}}
\multiput(1309.00,810.17)(11.251,7.000){2}{\rule{0.421pt}{0.400pt}}
\multiput(1322.00,818.59)(0.824,0.488){13}{\rule{0.750pt}{0.117pt}}
\multiput(1322.00,817.17)(11.443,8.000){2}{\rule{0.375pt}{0.400pt}}
\multiput(1335.00,826.59)(0.824,0.488){13}{\rule{0.750pt}{0.117pt}}
\multiput(1335.00,825.17)(11.443,8.000){2}{\rule{0.375pt}{0.400pt}}
\multiput(1348.00,834.59)(0.950,0.485){11}{\rule{0.843pt}{0.117pt}}
\multiput(1348.00,833.17)(11.251,7.000){2}{\rule{0.421pt}{0.400pt}}
\multiput(1361.00,841.59)(0.824,0.488){13}{\rule{0.750pt}{0.117pt}}
\multiput(1361.00,840.17)(11.443,8.000){2}{\rule{0.375pt}{0.400pt}}
\multiput(1374.00,849.59)(0.950,0.485){11}{\rule{0.843pt}{0.117pt}}
\multiput(1374.00,848.17)(11.251,7.000){2}{\rule{0.421pt}{0.400pt}}
\multiput(1387.00,856.59)(0.824,0.488){13}{\rule{0.750pt}{0.117pt}}
\multiput(1387.00,855.17)(11.443,8.000){2}{\rule{0.375pt}{0.400pt}}
\multiput(1400.00,864.59)(0.824,0.488){13}{\rule{0.750pt}{0.117pt}}
\multiput(1400.00,863.17)(11.443,8.000){2}{\rule{0.375pt}{0.400pt}}
\multiput(1413.00,872.59)(0.950,0.485){11}{\rule{0.843pt}{0.117pt}}
\multiput(1413.00,871.17)(11.251,7.000){2}{\rule{0.421pt}{0.400pt}}
\multiput(1426.00,879.59)(0.824,0.488){13}{\rule{0.750pt}{0.117pt}}
\multiput(1426.00,878.17)(11.443,8.000){2}{\rule{0.375pt}{0.400pt}}
\put(151,131){\makebox(0,0){$+$}}
\put(164,139){\makebox(0,0){$+$}}
\put(177,146){\makebox(0,0){$+$}}
\put(190,154){\makebox(0,0){$+$}}
\put(203,162){\makebox(0,0){$+$}}
\put(216,169){\makebox(0,0){$+$}}
\put(229,177){\makebox(0,0){$+$}}
\put(242,184){\makebox(0,0){$+$}}
\put(255,192){\makebox(0,0){$+$}}
\put(268,200){\makebox(0,0){$+$}}
\put(281,207){\makebox(0,0){$+$}}
\put(294,215){\makebox(0,0){$+$}}
\put(307,223){\makebox(0,0){$+$}}
\put(320,230){\makebox(0,0){$+$}}
\put(333,238){\makebox(0,0){$+$}}
\put(346,246){\makebox(0,0){$+$}}
\put(359,253){\makebox(0,0){$+$}}
\put(372,261){\makebox(0,0){$+$}}
\put(385,268){\makebox(0,0){$+$}}
\put(398,276){\makebox(0,0){$+$}}
\put(411,284){\makebox(0,0){$+$}}
\put(424,291){\makebox(0,0){$+$}}
\put(437,299){\makebox(0,0){$+$}}
\put(450,307){\makebox(0,0){$+$}}
\put(463,314){\makebox(0,0){$+$}}
\put(476,322){\makebox(0,0){$+$}}
\put(489,330){\makebox(0,0){$+$}}
\put(502,337){\makebox(0,0){$+$}}
\put(515,345){\makebox(0,0){$+$}}
\put(528,352){\makebox(0,0){$+$}}
\put(541,360){\makebox(0,0){$+$}}
\put(554,368){\makebox(0,0){$+$}}
\put(567,375){\makebox(0,0){$+$}}
\put(580,383){\makebox(0,0){$+$}}
\put(593,391){\makebox(0,0){$+$}}
\put(606,398){\makebox(0,0){$+$}}
\put(619,406){\makebox(0,0){$+$}}
\put(632,414){\makebox(0,0){$+$}}
\put(645,421){\makebox(0,0){$+$}}
\put(658,429){\makebox(0,0){$+$}}
\put(671,436){\makebox(0,0){$+$}}
\put(684,444){\makebox(0,0){$+$}}
\put(697,452){\makebox(0,0){$+$}}
\put(710,459){\makebox(0,0){$+$}}
\put(723,467){\makebox(0,0){$+$}}
\put(736,475){\makebox(0,0){$+$}}
\put(749,482){\makebox(0,0){$+$}}
\put(762,490){\makebox(0,0){$+$}}
\put(775,498){\makebox(0,0){$+$}}
\put(788,505){\makebox(0,0){$+$}}
\put(802,513){\makebox(0,0){$+$}}
\put(815,520){\makebox(0,0){$+$}}
\put(828,528){\makebox(0,0){$+$}}
\put(841,536){\makebox(0,0){$+$}}
\put(854,543){\makebox(0,0){$+$}}
\put(867,551){\makebox(0,0){$+$}}
\put(880,559){\makebox(0,0){$+$}}
\put(893,566){\makebox(0,0){$+$}}
\put(906,574){\makebox(0,0){$+$}}
\put(919,582){\makebox(0,0){$+$}}
\put(932,589){\makebox(0,0){$+$}}
\put(945,597){\makebox(0,0){$+$}}
\put(958,604){\makebox(0,0){$+$}}
\put(971,612){\makebox(0,0){$+$}}
\put(984,620){\makebox(0,0){$+$}}
\put(997,627){\makebox(0,0){$+$}}
\put(1010,635){\makebox(0,0){$+$}}
\put(1023,643){\makebox(0,0){$+$}}
\put(1036,650){\makebox(0,0){$+$}}
\put(1049,658){\makebox(0,0){$+$}}
\put(1062,666){\makebox(0,0){$+$}}
\put(1075,673){\makebox(0,0){$+$}}
\put(1088,681){\makebox(0,0){$+$}}
\put(1101,688){\makebox(0,0){$+$}}
\put(1114,696){\makebox(0,0){$+$}}
\put(1127,704){\makebox(0,0){$+$}}
\put(1140,711){\makebox(0,0){$+$}}
\put(1153,719){\makebox(0,0){$+$}}
\put(1166,727){\makebox(0,0){$+$}}
\put(1179,734){\makebox(0,0){$+$}}
\put(1192,742){\makebox(0,0){$+$}}
\put(1205,750){\makebox(0,0){$+$}}
\put(1218,757){\makebox(0,0){$+$}}
\put(1231,765){\makebox(0,0){$+$}}
\put(1244,772){\makebox(0,0){$+$}}
\put(1257,780){\makebox(0,0){$+$}}
\put(1270,788){\makebox(0,0){$+$}}
\put(1283,795){\makebox(0,0){$+$}}
\put(1296,803){\makebox(0,0){$+$}}
\put(1309,811){\makebox(0,0){$+$}}
\put(1322,818){\makebox(0,0){$+$}}
\put(1335,826){\makebox(0,0){$+$}}
\put(1348,834){\makebox(0,0){$+$}}
\put(1361,841){\makebox(0,0){$+$}}
\put(1374,849){\makebox(0,0){$+$}}
\put(1387,856){\makebox(0,0){$+$}}
\put(1400,864){\makebox(0,0){$+$}}
\put(1413,872){\makebox(0,0){$+$}}
\put(1426,879){\makebox(0,0){$+$}}
\put(1439,887){\makebox(0,0){$+$}}
\put(1047,866){\makebox(0,0){$+$}}
\put(977,825){\makebox(0,0)[r]{x+1}}
\put(151,169){\makebox(0,0){$\times$}}
\put(164,176){\makebox(0,0){$\times$}}
\put(177,184){\makebox(0,0){$\times$}}
\put(190,192){\makebox(0,0){$\times$}}
\put(203,199){\makebox(0,0){$\times$}}
\put(216,207){\makebox(0,0){$\times$}}
\put(229,215){\makebox(0,0){$\times$}}
\put(242,222){\makebox(0,0){$\times$}}
\put(255,230){\makebox(0,0){$\times$}}
\put(268,238){\makebox(0,0){$\times$}}
\put(281,245){\makebox(0,0){$\times$}}
\put(294,253){\makebox(0,0){$\times$}}
\put(307,260){\makebox(0,0){$\times$}}
\put(320,268){\makebox(0,0){$\times$}}
\put(333,276){\makebox(0,0){$\times$}}
\put(346,283){\makebox(0,0){$\times$}}
\put(359,291){\makebox(0,0){$\times$}}
\put(372,299){\makebox(0,0){$\times$}}
\put(385,306){\makebox(0,0){$\times$}}
\put(398,314){\makebox(0,0){$\times$}}
\put(411,322){\makebox(0,0){$\times$}}
\put(424,329){\makebox(0,0){$\times$}}
\put(437,337){\makebox(0,0){$\times$}}
\put(450,344){\makebox(0,0){$\times$}}
\put(463,352){\makebox(0,0){$\times$}}
\put(476,360){\makebox(0,0){$\times$}}
\put(489,367){\makebox(0,0){$\times$}}
\put(502,375){\makebox(0,0){$\times$}}
\put(515,383){\makebox(0,0){$\times$}}
\put(528,390){\makebox(0,0){$\times$}}
\put(541,398){\makebox(0,0){$\times$}}
\put(554,406){\makebox(0,0){$\times$}}
\put(567,413){\makebox(0,0){$\times$}}
\put(580,421){\makebox(0,0){$\times$}}
\put(593,428){\makebox(0,0){$\times$}}
\put(606,436){\makebox(0,0){$\times$}}
\put(619,444){\makebox(0,0){$\times$}}
\put(632,451){\makebox(0,0){$\times$}}
\put(645,459){\makebox(0,0){$\times$}}
\put(658,467){\makebox(0,0){$\times$}}
\put(671,474){\makebox(0,0){$\times$}}
\put(684,482){\makebox(0,0){$\times$}}
\put(697,490){\makebox(0,0){$\times$}}
\put(710,497){\makebox(0,0){$\times$}}
\put(723,505){\makebox(0,0){$\times$}}
\put(736,512){\makebox(0,0){$\times$}}
\put(749,520){\makebox(0,0){$\times$}}
\put(762,528){\makebox(0,0){$\times$}}
\put(775,535){\makebox(0,0){$\times$}}
\put(788,543){\makebox(0,0){$\times$}}
\put(802,551){\makebox(0,0){$\times$}}
\put(815,558){\makebox(0,0){$\times$}}
\put(828,566){\makebox(0,0){$\times$}}
\put(841,574){\makebox(0,0){$\times$}}
\put(854,581){\makebox(0,0){$\times$}}
\put(867,589){\makebox(0,0){$\times$}}
\put(880,596){\makebox(0,0){$\times$}}
\put(893,604){\makebox(0,0){$\times$}}
\put(906,612){\makebox(0,0){$\times$}}
\put(919,619){\makebox(0,0){$\times$}}
\put(932,627){\makebox(0,0){$\times$}}
\put(945,635){\makebox(0,0){$\times$}}
\put(958,642){\makebox(0,0){$\times$}}
\put(971,650){\makebox(0,0){$\times$}}
\put(984,658){\makebox(0,0){$\times$}}
\put(997,665){\makebox(0,0){$\times$}}
\put(1010,673){\makebox(0,0){$\times$}}
\put(1023,680){\makebox(0,0){$\times$}}
\put(1036,688){\makebox(0,0){$\times$}}
\put(1049,696){\makebox(0,0){$\times$}}
\put(1062,703){\makebox(0,0){$\times$}}
\put(1075,711){\makebox(0,0){$\times$}}
\put(1088,719){\makebox(0,0){$\times$}}
\put(1101,726){\makebox(0,0){$\times$}}
\put(1114,734){\makebox(0,0){$\times$}}
\put(1127,742){\makebox(0,0){$\times$}}
\put(1140,749){\makebox(0,0){$\times$}}
\put(1153,757){\makebox(0,0){$\times$}}
\put(1166,764){\makebox(0,0){$\times$}}
\put(1179,772){\makebox(0,0){$\times$}}
\put(1192,780){\makebox(0,0){$\times$}}
\put(1205,787){\makebox(0,0){$\times$}}
\put(1218,795){\makebox(0,0){$\times$}}
\put(1231,803){\makebox(0,0){$\times$}}
\put(1244,810){\makebox(0,0){$\times$}}
\put(1257,818){\makebox(0,0){$\times$}}
\put(1270,826){\makebox(0,0){$\times$}}
\put(1283,833){\makebox(0,0){$\times$}}
\put(1296,841){\makebox(0,0){$\times$}}
\put(1309,848){\makebox(0,0){$\times$}}
\put(1322,856){\makebox(0,0){$\times$}}
\put(1335,864){\makebox(0,0){$\times$}}
\put(1348,871){\makebox(0,0){$\times$}}
\put(1361,879){\makebox(0,0){$\times$}}
\put(1374,887){\makebox(0,0){$\times$}}
\put(1387,894){\makebox(0,0){$\times$}}
\put(1400,902){\makebox(0,0){$\times$}}
\put(1413,910){\makebox(0,0){$\times$}}
\put(1426,917){\makebox(0,0){$\times$}}
\put(1439,925){\makebox(0,0){$\times$}}
\put(1047,825){\makebox(0,0){$\times$}}
\put(151.0,131.0){\rule[-0.200pt]{0.400pt}{227.650pt}}
\put(151.0,131.0){\rule[-0.200pt]{310.279pt}{0.400pt}}
\put(1439.0,131.0){\rule[-0.200pt]{0.400pt}{227.650pt}}
\put(151.0,1076.0){\rule[-0.200pt]{310.279pt}{0.400pt}}
\end{picture}

    \caption{Another plot}
    \label{fig:eg3}
  \end{center}
\end{figure}

Figure~\ref{fig:eg4} has the axis labeled in $\pi$
and provides manual tic marks.
\begin{figure}
\begin{center}
  % GNUPLOT: LaTeX picture
\setlength{\unitlength}{0.240900pt}
\ifx\plotpoint\undefined\newsavebox{\plotpoint}\fi
\begin{picture}(1500,1200)(0,0)
\sbox{\plotpoint}{\rule[-0.200pt]{0.400pt}{0.400pt}}%
\put(171.0,131.0){\rule[-0.200pt]{4.818pt}{0.400pt}}
\put(151,131){\makebox(0,0)[r]{$-1$}}
\put(1419.0,131.0){\rule[-0.200pt]{4.818pt}{0.400pt}}
\put(171.0,367.0){\rule[-0.200pt]{4.818pt}{0.400pt}}
\put(151,367){\makebox(0,0)[r]{$-0.5$}}
\put(1419.0,367.0){\rule[-0.200pt]{4.818pt}{0.400pt}}
\put(171.0,604.0){\rule[-0.200pt]{4.818pt}{0.400pt}}
\put(151,604){\makebox(0,0)[r]{$0$}}
\put(1419.0,604.0){\rule[-0.200pt]{4.818pt}{0.400pt}}
\put(171.0,840.0){\rule[-0.200pt]{4.818pt}{0.400pt}}
\put(151,840){\makebox(0,0)[r]{$0.5$}}
\put(1419.0,840.0){\rule[-0.200pt]{4.818pt}{0.400pt}}
\put(171.0,1076.0){\rule[-0.200pt]{4.818pt}{0.400pt}}
\put(1419.0,1076.0){\rule[-0.200pt]{4.818pt}{0.400pt}}
\put(171.0,131.0){\rule[-0.200pt]{0.400pt}{4.818pt}}
\put(171,90){\makebox(0,0){$-3.14$}}
\put(171.0,1056.0){\rule[-0.200pt]{0.400pt}{4.818pt}}
\put(330.0,131.0){\rule[-0.200pt]{0.400pt}{4.818pt}}
\put(330,90){\makebox(0,0){$-2.36$}}
\put(330.0,1056.0){\rule[-0.200pt]{0.400pt}{4.818pt}}
\put(488.0,131.0){\rule[-0.200pt]{0.400pt}{4.818pt}}
\put(488,90){\makebox(0,0){$-1.57$}}
\put(488.0,1056.0){\rule[-0.200pt]{0.400pt}{4.818pt}}
\put(647.0,131.0){\rule[-0.200pt]{0.400pt}{4.818pt}}
\put(647,90){\makebox(0,0){$-0.79$}}
\put(647.0,1056.0){\rule[-0.200pt]{0.400pt}{4.818pt}}
\put(805.0,131.0){\rule[-0.200pt]{0.400pt}{4.818pt}}
\put(805,90){\makebox(0,0){$0.00$}}
\put(805.0,1056.0){\rule[-0.200pt]{0.400pt}{4.818pt}}
\put(964.0,131.0){\rule[-0.200pt]{0.400pt}{4.818pt}}
\put(964,90){\makebox(0,0){$0.79$}}
\put(964.0,1056.0){\rule[-0.200pt]{0.400pt}{4.818pt}}
\put(1122.0,131.0){\rule[-0.200pt]{0.400pt}{4.818pt}}
\put(1122,90){\makebox(0,0){$1.57$}}
\put(1122.0,1056.0){\rule[-0.200pt]{0.400pt}{4.818pt}}
\put(1281.0,131.0){\rule[-0.200pt]{0.400pt}{4.818pt}}
\put(1281,90){\makebox(0,0){$2.36$}}
\put(1281.0,1056.0){\rule[-0.200pt]{0.400pt}{4.818pt}}
\put(1439.0,131.0){\rule[-0.200pt]{0.400pt}{4.818pt}}
\put(1439,90){\makebox(0,0){$3.14$}}
\put(1439.0,1056.0){\rule[-0.200pt]{0.400pt}{4.818pt}}
\put(171.0,131.0){\rule[-0.200pt]{0.400pt}{227.650pt}}
\put(171.0,131.0){\rule[-0.200pt]{305.461pt}{0.400pt}}
\put(1439.0,131.0){\rule[-0.200pt]{0.400pt}{227.650pt}}
\put(171.0,1076.0){\rule[-0.200pt]{305.461pt}{0.400pt}}
\put(30,603){\makebox(0,0){$\sin(x)$}}
\put(805,29){\makebox(0,0){$x$}}
\put(171,604){\usebox{\plotpoint}}
\multiput(171.58,599.75)(0.493,-1.171){23}{\rule{0.119pt}{1.023pt}}
\multiput(170.17,601.88)(13.000,-27.877){2}{\rule{0.400pt}{0.512pt}}
\multiput(184.58,569.75)(0.493,-1.171){23}{\rule{0.119pt}{1.023pt}}
\multiput(183.17,571.88)(13.000,-27.877){2}{\rule{0.400pt}{0.512pt}}
\multiput(197.58,539.43)(0.492,-1.272){21}{\rule{0.119pt}{1.100pt}}
\multiput(196.17,541.72)(12.000,-27.717){2}{\rule{0.400pt}{0.550pt}}
\multiput(209.58,509.88)(0.493,-1.131){23}{\rule{0.119pt}{0.992pt}}
\multiput(208.17,511.94)(13.000,-26.940){2}{\rule{0.400pt}{0.496pt}}
\multiput(222.58,480.88)(0.493,-1.131){23}{\rule{0.119pt}{0.992pt}}
\multiput(221.17,482.94)(13.000,-26.940){2}{\rule{0.400pt}{0.496pt}}
\multiput(235.58,452.01)(0.493,-1.091){23}{\rule{0.119pt}{0.962pt}}
\multiput(234.17,454.00)(13.000,-26.004){2}{\rule{0.400pt}{0.481pt}}
\multiput(248.58,424.01)(0.493,-1.091){23}{\rule{0.119pt}{0.962pt}}
\multiput(247.17,426.00)(13.000,-26.004){2}{\rule{0.400pt}{0.481pt}}
\multiput(261.58,395.99)(0.492,-1.099){21}{\rule{0.119pt}{0.967pt}}
\multiput(260.17,397.99)(12.000,-23.994){2}{\rule{0.400pt}{0.483pt}}
\multiput(273.58,370.26)(0.493,-1.012){23}{\rule{0.119pt}{0.900pt}}
\multiput(272.17,372.13)(13.000,-24.132){2}{\rule{0.400pt}{0.450pt}}
\multiput(286.58,344.39)(0.493,-0.972){23}{\rule{0.119pt}{0.869pt}}
\multiput(285.17,346.20)(13.000,-23.196){2}{\rule{0.400pt}{0.435pt}}
\multiput(299.58,319.65)(0.493,-0.893){23}{\rule{0.119pt}{0.808pt}}
\multiput(298.17,321.32)(13.000,-21.324){2}{\rule{0.400pt}{0.404pt}}
\multiput(312.58,296.65)(0.493,-0.893){23}{\rule{0.119pt}{0.808pt}}
\multiput(311.17,298.32)(13.000,-21.324){2}{\rule{0.400pt}{0.404pt}}
\multiput(325.58,273.90)(0.493,-0.814){23}{\rule{0.119pt}{0.746pt}}
\multiput(324.17,275.45)(13.000,-19.451){2}{\rule{0.400pt}{0.373pt}}
\multiput(338.58,252.96)(0.492,-0.798){21}{\rule{0.119pt}{0.733pt}}
\multiput(337.17,254.48)(12.000,-17.478){2}{\rule{0.400pt}{0.367pt}}
\multiput(350.58,234.29)(0.493,-0.695){23}{\rule{0.119pt}{0.654pt}}
\multiput(349.17,235.64)(13.000,-16.643){2}{\rule{0.400pt}{0.327pt}}
\multiput(363.58,216.41)(0.493,-0.655){23}{\rule{0.119pt}{0.623pt}}
\multiput(362.17,217.71)(13.000,-15.707){2}{\rule{0.400pt}{0.312pt}}
\multiput(376.58,199.67)(0.493,-0.576){23}{\rule{0.119pt}{0.562pt}}
\multiput(375.17,200.83)(13.000,-13.834){2}{\rule{0.400pt}{0.281pt}}
\multiput(389.00,185.92)(0.497,-0.493){23}{\rule{0.500pt}{0.119pt}}
\multiput(389.00,186.17)(11.962,-13.000){2}{\rule{0.250pt}{0.400pt}}
\multiput(402.00,172.92)(0.496,-0.492){21}{\rule{0.500pt}{0.119pt}}
\multiput(402.00,173.17)(10.962,-12.000){2}{\rule{0.250pt}{0.400pt}}
\multiput(414.00,160.92)(0.652,-0.491){17}{\rule{0.620pt}{0.118pt}}
\multiput(414.00,161.17)(11.713,-10.000){2}{\rule{0.310pt}{0.400pt}}
\multiput(427.00,150.93)(0.824,-0.488){13}{\rule{0.750pt}{0.117pt}}
\multiput(427.00,151.17)(11.443,-8.000){2}{\rule{0.375pt}{0.400pt}}
\multiput(440.00,142.93)(1.123,-0.482){9}{\rule{0.967pt}{0.116pt}}
\multiput(440.00,143.17)(10.994,-6.000){2}{\rule{0.483pt}{0.400pt}}
\multiput(453.00,136.94)(1.797,-0.468){5}{\rule{1.400pt}{0.113pt}}
\multiput(453.00,137.17)(10.094,-4.000){2}{\rule{0.700pt}{0.400pt}}
\put(466,132.17){\rule{2.500pt}{0.400pt}}
\multiput(466.00,133.17)(6.811,-2.000){2}{\rule{1.250pt}{0.400pt}}
\put(478,130.67){\rule{3.132pt}{0.400pt}}
\multiput(478.00,131.17)(6.500,-1.000){2}{\rule{1.566pt}{0.400pt}}
\put(491,130.67){\rule{3.132pt}{0.400pt}}
\multiput(491.00,130.17)(6.500,1.000){2}{\rule{1.566pt}{0.400pt}}
\multiput(504.00,132.60)(1.797,0.468){5}{\rule{1.400pt}{0.113pt}}
\multiput(504.00,131.17)(10.094,4.000){2}{\rule{0.700pt}{0.400pt}}
\multiput(517.00,136.59)(1.378,0.477){7}{\rule{1.140pt}{0.115pt}}
\multiput(517.00,135.17)(10.634,5.000){2}{\rule{0.570pt}{0.400pt}}
\multiput(530.00,141.59)(0.874,0.485){11}{\rule{0.786pt}{0.117pt}}
\multiput(530.00,140.17)(10.369,7.000){2}{\rule{0.393pt}{0.400pt}}
\multiput(542.00,148.59)(0.728,0.489){15}{\rule{0.678pt}{0.118pt}}
\multiput(542.00,147.17)(11.593,9.000){2}{\rule{0.339pt}{0.400pt}}
\multiput(555.00,157.58)(0.590,0.492){19}{\rule{0.573pt}{0.118pt}}
\multiput(555.00,156.17)(11.811,11.000){2}{\rule{0.286pt}{0.400pt}}
\multiput(568.00,168.58)(0.539,0.492){21}{\rule{0.533pt}{0.119pt}}
\multiput(568.00,167.17)(11.893,12.000){2}{\rule{0.267pt}{0.400pt}}
\multiput(581.58,180.00)(0.493,0.536){23}{\rule{0.119pt}{0.531pt}}
\multiput(580.17,180.00)(13.000,12.898){2}{\rule{0.400pt}{0.265pt}}
\multiput(594.58,194.00)(0.492,0.669){21}{\rule{0.119pt}{0.633pt}}
\multiput(593.17,194.00)(12.000,14.685){2}{\rule{0.400pt}{0.317pt}}
\multiput(606.58,210.00)(0.493,0.695){23}{\rule{0.119pt}{0.654pt}}
\multiput(605.17,210.00)(13.000,16.643){2}{\rule{0.400pt}{0.327pt}}
\multiput(619.58,228.00)(0.493,0.695){23}{\rule{0.119pt}{0.654pt}}
\multiput(618.17,228.00)(13.000,16.643){2}{\rule{0.400pt}{0.327pt}}
\multiput(632.58,246.00)(0.493,0.814){23}{\rule{0.119pt}{0.746pt}}
\multiput(631.17,246.00)(13.000,19.451){2}{\rule{0.400pt}{0.373pt}}
\multiput(645.58,267.00)(0.493,0.814){23}{\rule{0.119pt}{0.746pt}}
\multiput(644.17,267.00)(13.000,19.451){2}{\rule{0.400pt}{0.373pt}}
\multiput(658.58,288.00)(0.493,0.893){23}{\rule{0.119pt}{0.808pt}}
\multiput(657.17,288.00)(13.000,21.324){2}{\rule{0.400pt}{0.404pt}}
\multiput(671.58,311.00)(0.492,1.056){21}{\rule{0.119pt}{0.933pt}}
\multiput(670.17,311.00)(12.000,23.063){2}{\rule{0.400pt}{0.467pt}}
\multiput(683.58,336.00)(0.493,0.972){23}{\rule{0.119pt}{0.869pt}}
\multiput(682.17,336.00)(13.000,23.196){2}{\rule{0.400pt}{0.435pt}}
\multiput(696.58,361.00)(0.493,1.012){23}{\rule{0.119pt}{0.900pt}}
\multiput(695.17,361.00)(13.000,24.132){2}{\rule{0.400pt}{0.450pt}}
\multiput(709.58,387.00)(0.493,1.052){23}{\rule{0.119pt}{0.931pt}}
\multiput(708.17,387.00)(13.000,25.068){2}{\rule{0.400pt}{0.465pt}}
\multiput(722.58,414.00)(0.493,1.091){23}{\rule{0.119pt}{0.962pt}}
\multiput(721.17,414.00)(13.000,26.004){2}{\rule{0.400pt}{0.481pt}}
\multiput(735.58,442.00)(0.492,1.186){21}{\rule{0.119pt}{1.033pt}}
\multiput(734.17,442.00)(12.000,25.855){2}{\rule{0.400pt}{0.517pt}}
\multiput(747.58,470.00)(0.493,1.131){23}{\rule{0.119pt}{0.992pt}}
\multiput(746.17,470.00)(13.000,26.940){2}{\rule{0.400pt}{0.496pt}}
\multiput(760.58,499.00)(0.493,1.171){23}{\rule{0.119pt}{1.023pt}}
\multiput(759.17,499.00)(13.000,27.877){2}{\rule{0.400pt}{0.512pt}}
\multiput(773.58,529.00)(0.493,1.171){23}{\rule{0.119pt}{1.023pt}}
\multiput(772.17,529.00)(13.000,27.877){2}{\rule{0.400pt}{0.512pt}}
\multiput(786.58,559.00)(0.493,1.171){23}{\rule{0.119pt}{1.023pt}}
\multiput(785.17,559.00)(13.000,27.877){2}{\rule{0.400pt}{0.512pt}}
\multiput(799.58,589.00)(0.492,1.229){21}{\rule{0.119pt}{1.067pt}}
\multiput(798.17,589.00)(12.000,26.786){2}{\rule{0.400pt}{0.533pt}}
\multiput(811.58,618.00)(0.493,1.171){23}{\rule{0.119pt}{1.023pt}}
\multiput(810.17,618.00)(13.000,27.877){2}{\rule{0.400pt}{0.512pt}}
\multiput(824.58,648.00)(0.493,1.171){23}{\rule{0.119pt}{1.023pt}}
\multiput(823.17,648.00)(13.000,27.877){2}{\rule{0.400pt}{0.512pt}}
\multiput(837.58,678.00)(0.493,1.171){23}{\rule{0.119pt}{1.023pt}}
\multiput(836.17,678.00)(13.000,27.877){2}{\rule{0.400pt}{0.512pt}}
\multiput(850.58,708.00)(0.493,1.131){23}{\rule{0.119pt}{0.992pt}}
\multiput(849.17,708.00)(13.000,26.940){2}{\rule{0.400pt}{0.496pt}}
\multiput(863.58,737.00)(0.492,1.186){21}{\rule{0.119pt}{1.033pt}}
\multiput(862.17,737.00)(12.000,25.855){2}{\rule{0.400pt}{0.517pt}}
\multiput(875.58,765.00)(0.493,1.091){23}{\rule{0.119pt}{0.962pt}}
\multiput(874.17,765.00)(13.000,26.004){2}{\rule{0.400pt}{0.481pt}}
\multiput(888.58,793.00)(0.493,1.052){23}{\rule{0.119pt}{0.931pt}}
\multiput(887.17,793.00)(13.000,25.068){2}{\rule{0.400pt}{0.465pt}}
\multiput(901.58,820.00)(0.493,1.012){23}{\rule{0.119pt}{0.900pt}}
\multiput(900.17,820.00)(13.000,24.132){2}{\rule{0.400pt}{0.450pt}}
\multiput(914.58,846.00)(0.493,0.972){23}{\rule{0.119pt}{0.869pt}}
\multiput(913.17,846.00)(13.000,23.196){2}{\rule{0.400pt}{0.435pt}}
\multiput(927.58,871.00)(0.492,1.056){21}{\rule{0.119pt}{0.933pt}}
\multiput(926.17,871.00)(12.000,23.063){2}{\rule{0.400pt}{0.467pt}}
\multiput(939.58,896.00)(0.493,0.893){23}{\rule{0.119pt}{0.808pt}}
\multiput(938.17,896.00)(13.000,21.324){2}{\rule{0.400pt}{0.404pt}}
\multiput(952.58,919.00)(0.493,0.814){23}{\rule{0.119pt}{0.746pt}}
\multiput(951.17,919.00)(13.000,19.451){2}{\rule{0.400pt}{0.373pt}}
\multiput(965.58,940.00)(0.493,0.814){23}{\rule{0.119pt}{0.746pt}}
\multiput(964.17,940.00)(13.000,19.451){2}{\rule{0.400pt}{0.373pt}}
\multiput(978.58,961.00)(0.493,0.695){23}{\rule{0.119pt}{0.654pt}}
\multiput(977.17,961.00)(13.000,16.643){2}{\rule{0.400pt}{0.327pt}}
\multiput(991.58,979.00)(0.493,0.695){23}{\rule{0.119pt}{0.654pt}}
\multiput(990.17,979.00)(13.000,16.643){2}{\rule{0.400pt}{0.327pt}}
\multiput(1004.58,997.00)(0.492,0.669){21}{\rule{0.119pt}{0.633pt}}
\multiput(1003.17,997.00)(12.000,14.685){2}{\rule{0.400pt}{0.317pt}}
\multiput(1016.58,1013.00)(0.493,0.536){23}{\rule{0.119pt}{0.531pt}}
\multiput(1015.17,1013.00)(13.000,12.898){2}{\rule{0.400pt}{0.265pt}}
\multiput(1029.00,1027.58)(0.539,0.492){21}{\rule{0.533pt}{0.119pt}}
\multiput(1029.00,1026.17)(11.893,12.000){2}{\rule{0.267pt}{0.400pt}}
\multiput(1042.00,1039.58)(0.590,0.492){19}{\rule{0.573pt}{0.118pt}}
\multiput(1042.00,1038.17)(11.811,11.000){2}{\rule{0.286pt}{0.400pt}}
\multiput(1055.00,1050.59)(0.728,0.489){15}{\rule{0.678pt}{0.118pt}}
\multiput(1055.00,1049.17)(11.593,9.000){2}{\rule{0.339pt}{0.400pt}}
\multiput(1068.00,1059.59)(0.874,0.485){11}{\rule{0.786pt}{0.117pt}}
\multiput(1068.00,1058.17)(10.369,7.000){2}{\rule{0.393pt}{0.400pt}}
\multiput(1080.00,1066.59)(1.378,0.477){7}{\rule{1.140pt}{0.115pt}}
\multiput(1080.00,1065.17)(10.634,5.000){2}{\rule{0.570pt}{0.400pt}}
\multiput(1093.00,1071.60)(1.797,0.468){5}{\rule{1.400pt}{0.113pt}}
\multiput(1093.00,1070.17)(10.094,4.000){2}{\rule{0.700pt}{0.400pt}}
\put(1106,1074.67){\rule{3.132pt}{0.400pt}}
\multiput(1106.00,1074.17)(6.500,1.000){2}{\rule{1.566pt}{0.400pt}}
\put(1119,1074.67){\rule{3.132pt}{0.400pt}}
\multiput(1119.00,1075.17)(6.500,-1.000){2}{\rule{1.566pt}{0.400pt}}
\put(1132,1073.17){\rule{2.500pt}{0.400pt}}
\multiput(1132.00,1074.17)(6.811,-2.000){2}{\rule{1.250pt}{0.400pt}}
\multiput(1144.00,1071.94)(1.797,-0.468){5}{\rule{1.400pt}{0.113pt}}
\multiput(1144.00,1072.17)(10.094,-4.000){2}{\rule{0.700pt}{0.400pt}}
\multiput(1157.00,1067.93)(1.123,-0.482){9}{\rule{0.967pt}{0.116pt}}
\multiput(1157.00,1068.17)(10.994,-6.000){2}{\rule{0.483pt}{0.400pt}}
\multiput(1170.00,1061.93)(0.824,-0.488){13}{\rule{0.750pt}{0.117pt}}
\multiput(1170.00,1062.17)(11.443,-8.000){2}{\rule{0.375pt}{0.400pt}}
\multiput(1183.00,1053.92)(0.652,-0.491){17}{\rule{0.620pt}{0.118pt}}
\multiput(1183.00,1054.17)(11.713,-10.000){2}{\rule{0.310pt}{0.400pt}}
\multiput(1196.00,1043.92)(0.496,-0.492){21}{\rule{0.500pt}{0.119pt}}
\multiput(1196.00,1044.17)(10.962,-12.000){2}{\rule{0.250pt}{0.400pt}}
\multiput(1208.00,1031.92)(0.497,-0.493){23}{\rule{0.500pt}{0.119pt}}
\multiput(1208.00,1032.17)(11.962,-13.000){2}{\rule{0.250pt}{0.400pt}}
\multiput(1221.58,1017.67)(0.493,-0.576){23}{\rule{0.119pt}{0.562pt}}
\multiput(1220.17,1018.83)(13.000,-13.834){2}{\rule{0.400pt}{0.281pt}}
\multiput(1234.58,1002.41)(0.493,-0.655){23}{\rule{0.119pt}{0.623pt}}
\multiput(1233.17,1003.71)(13.000,-15.707){2}{\rule{0.400pt}{0.312pt}}
\multiput(1247.58,985.29)(0.493,-0.695){23}{\rule{0.119pt}{0.654pt}}
\multiput(1246.17,986.64)(13.000,-16.643){2}{\rule{0.400pt}{0.327pt}}
\multiput(1260.58,966.96)(0.492,-0.798){21}{\rule{0.119pt}{0.733pt}}
\multiput(1259.17,968.48)(12.000,-17.478){2}{\rule{0.400pt}{0.367pt}}
\multiput(1272.58,947.90)(0.493,-0.814){23}{\rule{0.119pt}{0.746pt}}
\multiput(1271.17,949.45)(13.000,-19.451){2}{\rule{0.400pt}{0.373pt}}
\multiput(1285.58,926.65)(0.493,-0.893){23}{\rule{0.119pt}{0.808pt}}
\multiput(1284.17,928.32)(13.000,-21.324){2}{\rule{0.400pt}{0.404pt}}
\multiput(1298.58,903.65)(0.493,-0.893){23}{\rule{0.119pt}{0.808pt}}
\multiput(1297.17,905.32)(13.000,-21.324){2}{\rule{0.400pt}{0.404pt}}
\multiput(1311.58,880.39)(0.493,-0.972){23}{\rule{0.119pt}{0.869pt}}
\multiput(1310.17,882.20)(13.000,-23.196){2}{\rule{0.400pt}{0.435pt}}
\multiput(1324.58,855.26)(0.493,-1.012){23}{\rule{0.119pt}{0.900pt}}
\multiput(1323.17,857.13)(13.000,-24.132){2}{\rule{0.400pt}{0.450pt}}
\multiput(1337.58,828.99)(0.492,-1.099){21}{\rule{0.119pt}{0.967pt}}
\multiput(1336.17,830.99)(12.000,-23.994){2}{\rule{0.400pt}{0.483pt}}
\multiput(1349.58,803.01)(0.493,-1.091){23}{\rule{0.119pt}{0.962pt}}
\multiput(1348.17,805.00)(13.000,-26.004){2}{\rule{0.400pt}{0.481pt}}
\multiput(1362.58,775.01)(0.493,-1.091){23}{\rule{0.119pt}{0.962pt}}
\multiput(1361.17,777.00)(13.000,-26.004){2}{\rule{0.400pt}{0.481pt}}
\multiput(1375.58,746.88)(0.493,-1.131){23}{\rule{0.119pt}{0.992pt}}
\multiput(1374.17,748.94)(13.000,-26.940){2}{\rule{0.400pt}{0.496pt}}
\multiput(1388.58,717.88)(0.493,-1.131){23}{\rule{0.119pt}{0.992pt}}
\multiput(1387.17,719.94)(13.000,-26.940){2}{\rule{0.400pt}{0.496pt}}
\multiput(1401.58,688.43)(0.492,-1.272){21}{\rule{0.119pt}{1.100pt}}
\multiput(1400.17,690.72)(12.000,-27.717){2}{\rule{0.400pt}{0.550pt}}
\multiput(1413.58,658.75)(0.493,-1.171){23}{\rule{0.119pt}{1.023pt}}
\multiput(1412.17,660.88)(13.000,-27.877){2}{\rule{0.400pt}{0.512pt}}
\multiput(1426.58,628.88)(0.493,-1.131){23}{\rule{0.119pt}{0.992pt}}
\multiput(1425.17,630.94)(13.000,-26.940){2}{\rule{0.400pt}{0.496pt}}
\put(171.0,131.0){\rule[-0.200pt]{0.400pt}{227.650pt}}
\put(171.0,131.0){\rule[-0.200pt]{305.461pt}{0.400pt}}
\put(1439.0,131.0){\rule[-0.200pt]{0.400pt}{227.650pt}}
\put(171.0,1076.0){\rule[-0.200pt]{305.461pt}{0.400pt}}
\end{picture}

  \caption{An example of the \emph{set xtics} command.}
  \label{fig:eg4}
\end{center}
\end{figure}

Figure~\ref{fig:eg5} introduces an alternative form of
figure~\ref{fig:eg4}, allowing us to specify the label
and position of each tic individually.
\begin{figure}
\begin{center}
  \input{eg5}
  \caption{A refined version of figure~\ref{fig:eg4}}
  \label{fig:eg5}
\end{center}
\end{figure}

A reference is in Figure~\ref{fig:ref}
\begin{figure}
  \begin{center}
  % GNUPLOT: LaTeX picture
\setlength{\unitlength}{0.240900pt}
\ifx\plotpoint\undefined\newsavebox{\plotpoint}\fi
\begin{picture}(1500,1200)(0,0)
\sbox{\plotpoint}{\rule[-0.200pt]{0.400pt}{0.400pt}}%
\put(0.0,0.0){\rule[-0.200pt]{361.109pt}{0.400pt}}
\put(1499.0,0.0){\rule[-0.200pt]{0.400pt}{288.839pt}}
\put(0.0,1199.0){\rule[-0.200pt]{361.109pt}{0.400pt}}
\put(40,1159){\makebox(0,0)[l]{latex  terminal test}}
\put(40,1107){\makebox(0,0)[l]{gnuplot version 5.0.3  }}
\put(0.0,0.0){\rule[-0.200pt]{0.400pt}{288.839pt}}
\put(750.0,0.0){\rule[-0.200pt]{0.400pt}{288.839pt}}
\put(0.0,600.0){\rule[-0.200pt]{361.109pt}{0.400pt}}
\put(550.0,620.0){\rule[-0.200pt]{96.360pt}{0.400pt}}
\put(950.0,580.0){\rule[-0.200pt]{0.400pt}{9.636pt}}
\put(550.0,580.0){\rule[-0.200pt]{96.360pt}{0.400pt}}
\put(550,600){\makebox(0,0)[l]{12345678901234567890}}
\put(550,657){\makebox(0,0)[l]{test of character width:}}
\put(550.0,580.0){\rule[-0.200pt]{0.400pt}{9.636pt}}
\put(750,846){\makebox(0,0)[l]{left justified}}
\put(750,805){\makebox(0,0){centre+d text}}
\put(750,764){\makebox(0,0)[r]{right justified}}
\put(41,600){\makebox(0,0){rotated ce+ntred text}}
\put(123,600){\makebox(0,0)[l]{ rotated by +45 deg}}
\put(82,600){\makebox(0,0)[l]{ rotated by -45 deg}}
\sbox{\plotpoint}{\rule[-0.400pt]{0.800pt}{0.800pt}}%
\put(790.0,1180.0){\rule[-0.400pt]{0.800pt}{4.577pt}}
\put(730,1159){\makebox(0,0)[r]{show ticscale}}
\put(750.0,1160.0){\rule[-0.400pt]{4.818pt}{0.800pt}}
\sbox{\plotpoint}{\rule[-0.200pt]{0.400pt}{0.400pt}}%
\put(1340,1159){\makebox(0,0)[r]{-1}}
\put(1360.0,1159.0){\rule[-0.200pt]{19.272pt}{0.400pt}}
\put(1340,1118){\makebox(0,0)[r]{0}}
\put(1470,1118){\rule{1pt}{1pt}}
\put(1360.0,1118.0){\rule[-0.200pt]{19.272pt}{0.400pt}}
\put(1340,1077){\makebox(0,0)[r]{1}}
\put(1470,1077){\makebox(0,0){$+$}}
\put(1360.0,1077.0){\rule[-0.200pt]{19.272pt}{0.400pt}}
\put(1340,1036){\makebox(0,0)[r]{2}}
\multiput(1360,1036)(20.756,0.000){4}{\usebox{\plotpoint}}
\put(1440,1036){\usebox{\plotpoint}}
\put(1470,1036){\makebox(0,0){$\times$}}
\sbox{\plotpoint}{\rule[-0.400pt]{0.800pt}{0.800pt}}%
\put(1340,995){\makebox(0,0)[r]{3}}
\put(1470,995){\makebox(0,0){$\ast$}}
\put(1360.0,995.0){\rule[-0.400pt]{19.272pt}{0.800pt}}
\sbox{\plotpoint}{\rule[-0.500pt]{1.000pt}{1.000pt}}%
\put(1340,954){\makebox(0,0)[r]{4}}
\multiput(1360,954)(20.756,0.000){4}{\usebox{\plotpoint}}
\put(1440,954){\usebox{\plotpoint}}
\put(1470,954){\raisebox{-.8pt}{\makebox(0,0){$\Box$}}}
\sbox{\plotpoint}{\rule[-0.600pt]{1.200pt}{1.200pt}}%
\put(1340,913){\makebox(0,0)[r]{5}}
\put(1470,913){\makebox(0,0){$\blacksquare$}}
\put(1360.0,913.0){\rule[-0.600pt]{19.272pt}{1.200pt}}
\sbox{\plotpoint}{\rule[-0.500pt]{1.000pt}{1.000pt}}%
\put(1340,872){\makebox(0,0)[r]{6}}
\multiput(1360,872)(41.511,0.000){2}{\usebox{\plotpoint}}
\put(1440,872){\usebox{\plotpoint}}
\put(1470,872){\makebox(0,0){$\circ$}}
\sbox{\plotpoint}{\rule[-0.200pt]{0.400pt}{0.400pt}}%
\put(1340,831){\makebox(0,0)[r]{7}}
\put(1470,831){\makebox(0,0){$\bullet$}}
\put(1360.0,831.0){\rule[-0.200pt]{19.272pt}{0.400pt}}
\put(1340,790){\makebox(0,0)[r]{8}}
\multiput(1360,790)(20.756,0.000){4}{\usebox{\plotpoint}}
\put(1440,790){\usebox{\plotpoint}}
\put(1470,790){\makebox(0,0){$\triangle$}}
\put(1340,749){\makebox(0,0)[r]{9}}
\put(1470,749){\makebox(0,0){$\blacktriangle$}}
\put(1360.0,749.0){\rule[-0.200pt]{19.272pt}{0.400pt}}
\put(1340,708){\makebox(0,0)[r]{10}}
\multiput(1360,708)(20.756,0.000){4}{\usebox{\plotpoint}}
\put(1440,708){\usebox{\plotpoint}}
\put(1470,708){\makebox(0,0){$\triangledown$}}
\sbox{\plotpoint}{\rule[-0.400pt]{0.800pt}{0.800pt}}%
\put(1340,667){\makebox(0,0)[r]{11}}
\put(1470,667){\makebox(0,0){$\blacktriangledown$}}
\put(1360.0,667.0){\rule[-0.400pt]{19.272pt}{0.800pt}}
\sbox{\plotpoint}{\rule[-0.500pt]{1.000pt}{1.000pt}}%
\put(1340,626){\makebox(0,0)[r]{12}}
\multiput(1360,626)(20.756,0.000){4}{\usebox{\plotpoint}}
\put(1440,626){\usebox{\plotpoint}}
\put(1470,626){\makebox(0,0){$\lozenge$}}
\sbox{\plotpoint}{\rule[-0.600pt]{1.200pt}{1.200pt}}%
\put(1340,585){\makebox(0,0)[r]{13}}
\put(1470,585){\makebox(0,0){$\blacklozenge$}}
\put(1360.0,585.0){\rule[-0.600pt]{19.272pt}{1.200pt}}
\sbox{\plotpoint}{\rule[-0.500pt]{1.000pt}{1.000pt}}%
\put(1340,544){\makebox(0,0)[r]{14}}
\multiput(1360,544)(41.511,0.000){2}{\usebox{\plotpoint}}
\put(1440,544){\usebox{\plotpoint}}
\put(1470,544){\makebox(0,0){$\heartsuit$}}
\sbox{\plotpoint}{\rule[-0.200pt]{0.400pt}{0.400pt}}%
\put(1340,503){\makebox(0,0)[r]{15}}
\put(1470,503){\makebox(0,0){$\spadesuit$}}
\put(1360.0,503.0){\rule[-0.200pt]{19.272pt}{0.400pt}}
\put(1340,462){\makebox(0,0)[r]{16}}
\multiput(1360,462)(20.756,0.000){4}{\usebox{\plotpoint}}
\put(1440,462){\usebox{\plotpoint}}
\put(1470,462){\makebox(0,0){$+$}}
\put(1340,421){\makebox(0,0)[r]{17}}
\put(1470,421){\makebox(0,0){$\times$}}
\put(1360.0,421.0){\rule[-0.200pt]{19.272pt}{0.400pt}}
\put(1340,380){\makebox(0,0)[r]{18}}
\multiput(1360,380)(20.756,0.000){4}{\usebox{\plotpoint}}
\put(1440,380){\usebox{\plotpoint}}
\put(1470,380){\makebox(0,0){$\ast$}}
\sbox{\plotpoint}{\rule[-0.400pt]{0.800pt}{0.800pt}}%
\put(1340,339){\makebox(0,0)[r]{19}}
\put(1470,339){\raisebox{-.8pt}{\makebox(0,0){$\Box$}}}
\put(1360.0,339.0){\rule[-0.400pt]{19.272pt}{0.800pt}}
\sbox{\plotpoint}{\rule[-0.500pt]{1.000pt}{1.000pt}}%
\put(1340,298){\makebox(0,0)[r]{20}}
\multiput(1360,298)(20.756,0.000){4}{\usebox{\plotpoint}}
\put(1440,298){\usebox{\plotpoint}}
\put(1470,298){\makebox(0,0){$\blacksquare$}}
\sbox{\plotpoint}{\rule[-0.600pt]{1.200pt}{1.200pt}}%
\put(1340,257){\makebox(0,0)[r]{21}}
\put(1470,257){\makebox(0,0){$\circ$}}
\put(1360.0,257.0){\rule[-0.600pt]{19.272pt}{1.200pt}}
\sbox{\plotpoint}{\rule[-0.500pt]{1.000pt}{1.000pt}}%
\put(1340,216){\makebox(0,0)[r]{22}}
\multiput(1360,216)(41.511,0.000){2}{\usebox{\plotpoint}}
\put(1440,216){\usebox{\plotpoint}}
\put(1470,216){\makebox(0,0){$\bullet$}}
\sbox{\plotpoint}{\rule[-0.200pt]{0.400pt}{0.400pt}}%
\put(1340,175){\makebox(0,0)[r]{23}}
\put(1470,175){\makebox(0,0){$\triangle$}}
\put(1360.0,175.0){\rule[-0.200pt]{19.272pt}{0.400pt}}
\put(1340,134){\makebox(0,0)[r]{24}}
\multiput(1360,134)(20.756,0.000){4}{\usebox{\plotpoint}}
\put(1440,134){\usebox{\plotpoint}}
\put(1470,134){\makebox(0,0){$\blacktriangle$}}
\put(1340,93){\makebox(0,0)[r]{25}}
\put(1470,93){\makebox(0,0){$\triangledown$}}
\put(1360.0,93.0){\rule[-0.200pt]{19.272pt}{0.400pt}}
\put(1340,52){\makebox(0,0)[r]{26}}
\multiput(1360,52)(20.756,0.000){4}{\usebox{\plotpoint}}
\put(1440,52){\usebox{\plotpoint}}
\put(1470,52){\makebox(0,0){$\blacktriangledown$}}
\put(420,600){\vector(1,0){140}}
\put(420,600){\vector(-1,0){140}}
\put(420,600){\vector(0,1){140}}
\put(420,600){\vector(0,-1){140}}
\put(320,500){\vector(1,1){200}}
\put(320,700){\line(1,-1){100}}
\put(420,600){\vector(1,-1){100}}
\put(262,48){\makebox(0,0)[l]{  lw 1}}
\put(112.0,48.0){\rule[-0.200pt]{36.135pt}{0.400pt}}
\put(262,96){\makebox(0,0)[l]{  lw 2}}
\put(112.0,96.0){\rule[-0.200pt]{36.135pt}{0.400pt}}
\put(262,144){\makebox(0,0)[l]{  lw 3}}
\put(112.0,144.0){\rule[-0.200pt]{36.135pt}{0.400pt}}
\put(262,192){\makebox(0,0)[l]{  lw 4}}
\put(112.0,192.0){\rule[-0.200pt]{36.135pt}{0.400pt}}
\put(262,240){\makebox(0,0)[l]{  lw 5}}
\put(112.0,240.0){\rule[-0.200pt]{36.135pt}{0.400pt}}
\put(262,288){\makebox(0,0)[l]{  lw 6}}
\put(112,336){\makebox(0,0)[l]{linewidth}}
\put(112.0,288.0){\rule[-0.200pt]{36.135pt}{0.400pt}}
\put(600,48){\makebox(0,0)[l]{  dt 1}}
\put(450.0,48.0){\rule[-0.200pt]{36.135pt}{0.400pt}}
\put(600,96){\makebox(0,0)[l]{  dt 2}}
\put(450.0,96.0){\rule[-0.200pt]{36.135pt}{0.400pt}}
\sbox{\plotpoint}{\rule[-0.400pt]{0.800pt}{0.800pt}}%
\sbox{\plotpoint}{\rule[-0.200pt]{0.400pt}{0.400pt}}%
\put(600,144){\makebox(0,0)[l]{  dt 3}}
\put(450.0,144.0){\rule[-0.200pt]{36.135pt}{0.400pt}}
\sbox{\plotpoint}{\rule[-0.500pt]{1.000pt}{1.000pt}}%
\sbox{\plotpoint}{\rule[-0.200pt]{0.400pt}{0.400pt}}%
\put(600,192){\makebox(0,0)[l]{  dt 4}}
\put(450.0,192.0){\rule[-0.200pt]{36.135pt}{0.400pt}}
\sbox{\plotpoint}{\rule[-0.600pt]{1.200pt}{1.200pt}}%
\sbox{\plotpoint}{\rule[-0.200pt]{0.400pt}{0.400pt}}%
\put(600,240){\makebox(0,0)[l]{  dt 5}}
\put(450,288){\makebox(0,0)[l]{dashtype}}
\put(450.0,240.0){\rule[-0.200pt]{36.135pt}{0.400pt}}
\put(1009,211){\makebox(0,0){pattern fill}}
\put(750.0,0.0){\rule[-0.200pt]{0.400pt}{36.135pt}}
\put(750.0,150.0){\rule[-0.200pt]{8.913pt}{0.400pt}}
\put(787.0,0.0){\rule[-0.200pt]{0.400pt}{36.135pt}}
\put(768,170){\makebox(0,0){ 0}}
\put(805,0){\rule{8.9133pt}{36.135pt}}
\put(750.0,0.0){\rule[-0.200pt]{8.913pt}{0.400pt}}
\put(805.0,0.0){\rule[-0.200pt]{0.400pt}{36.135pt}}
\put(805.0,150.0){\rule[-0.200pt]{8.913pt}{0.400pt}}
\put(842.0,0.0){\rule[-0.200pt]{0.400pt}{36.135pt}}
\put(823,170){\makebox(0,0){ 1}}
\put(805.0,0.0){\rule[-0.200pt]{8.913pt}{0.400pt}}
\put(860.0,0.0){\rule[-0.200pt]{0.400pt}{36.135pt}}
\put(860.0,150.0){\rule[-0.200pt]{8.913pt}{0.400pt}}
\put(897.0,0.0){\rule[-0.200pt]{0.400pt}{36.135pt}}
\put(878,170){\makebox(0,0){ 2}}
\put(915,0){\rule{8.9133pt}{36.135pt}}
\put(860.0,0.0){\rule[-0.200pt]{8.913pt}{0.400pt}}
\put(915.0,0.0){\rule[-0.200pt]{0.400pt}{36.135pt}}
\put(915.0,150.0){\rule[-0.200pt]{8.913pt}{0.400pt}}
\put(952.0,0.0){\rule[-0.200pt]{0.400pt}{36.135pt}}
\put(933,170){\makebox(0,0){ 3}}
\put(915.0,0.0){\rule[-0.200pt]{8.913pt}{0.400pt}}
\put(970.0,0.0){\rule[-0.200pt]{0.400pt}{36.135pt}}
\put(970.0,150.0){\rule[-0.200pt]{8.913pt}{0.400pt}}
\put(1007.0,0.0){\rule[-0.200pt]{0.400pt}{36.135pt}}
\put(988,170){\makebox(0,0){ 4}}
\put(1025,0){\rule{8.9133pt}{36.135pt}}
\put(970.0,0.0){\rule[-0.200pt]{8.913pt}{0.400pt}}
\put(1025.0,0.0){\rule[-0.200pt]{0.400pt}{36.135pt}}
\put(1025.0,150.0){\rule[-0.200pt]{8.913pt}{0.400pt}}
\put(1062.0,0.0){\rule[-0.200pt]{0.400pt}{36.135pt}}
\put(1043,170){\makebox(0,0){ 5}}
\put(1025.0,0.0){\rule[-0.200pt]{8.913pt}{0.400pt}}
\put(1080.0,0.0){\rule[-0.200pt]{0.400pt}{36.135pt}}
\put(1080.0,150.0){\rule[-0.200pt]{8.913pt}{0.400pt}}
\put(1117.0,0.0){\rule[-0.200pt]{0.400pt}{36.135pt}}
\put(1098,170){\makebox(0,0){ 6}}
\put(1135,0){\rule{8.9133pt}{36.135pt}}
\put(1080.0,0.0){\rule[-0.200pt]{8.913pt}{0.400pt}}
\put(1135.0,0.0){\rule[-0.200pt]{0.400pt}{36.135pt}}
\put(1135.0,150.0){\rule[-0.200pt]{8.913pt}{0.400pt}}
\put(1172.0,0.0){\rule[-0.200pt]{0.400pt}{36.135pt}}
\put(1153,170){\makebox(0,0){ 7}}
\put(1135.0,0.0){\rule[-0.200pt]{8.913pt}{0.400pt}}
\put(1190.0,0.0){\rule[-0.200pt]{0.400pt}{36.135pt}}
\put(1190.0,150.0){\rule[-0.200pt]{8.913pt}{0.400pt}}
\put(1227.0,0.0){\rule[-0.200pt]{0.400pt}{36.135pt}}
\put(1208,170){\makebox(0,0){ 8}}
\put(1190.0,0.0){\rule[-0.200pt]{8.913pt}{0.400pt}}
\put(1050,1091){\makebox(0,0){No filled polygons}}
\end{picture}

  \caption{A reference figure}
  \label{fig:ref}
\end{center}
\end{figure}
\end{document}
