\documentclass[a4paper,11pt]{article}
\usepackage[affil-it]{authblk}
\usepackage{microtype,fancyhdr}
\usepackage[pdftex,hyperindex=false%
colorlinks,unicode]{hyperref}
\pagestyle{fancy}
\newcommand{\note}[1]{\textsl{#1}}
\newcommand{\bold}[1]{\textbf{#1}}

\begin{document}

  \title{Note for the paper ``Disagreement
  Behind the Veil of Ignorance''}
  \author{Son To}
  \affil{Ravintola Kiltakellari}
  \date{$18$th August, $2017$}

  \maketitle

  \note{Contribution:} A kind of moral disagreement
  that survives Rawlsian veil of ignorance: \bold{%
  disinterested} disagreement.
  \section{Moral Disagreement and Self-Interest}
  Sen's flute problem(Sen, 2009): A,B and C argues over a
  flute: A makes it, only B can play it, and C deserves it
  since A and B has many other toys and he has none.
  \par
  \smallskip
  A: libertarianism, B: utilitarianism and C: egalitarianism.

  \medskip
  \bold{My opinion} \rule{10cm}{0.4pt}
  Human's nature is fickle and interest-dependent; they
  only care for what they want to hear.
  This is a very well-known fact that does not need big research
  (Babcock and Loewenstein, 1997, 1995)
  to begin with because this is very superfluous and totally
  waste of time. Simply playing video games like Metal Gear
  Solid or Mafia series will make you realize this without
  reading the research papers.
  \\
  \bold{Opinion ends here.}\rule{10cm}{0.4pt}
  \par
  \medskip
  \note{Rawlsian veil of ignorance:}
  Before the social contract is
  imposed on, except for basic scientific facts, none has
  any idea what the correct positions are.
  $\Rightarrow$ change the problem of interested disagreement
  to disinterested disagreement.\\
  Interested disagreement: disagreement that participants
  have a stake in, unequal footing.
  \\
  Disinterested disagreement: No stakes, equal footing.

  \medskip
  \bold{My opinion}\rule{10cm}{0.4pt}
  The argument that if Original Position, unless it is stipulated
  to not only remove agents' interests, but also \bold{force agents
  to conform to a particular categorization scheme for
  the (politcal) world}, leaves open the possibility of agents
  finding themselves in a state of meta-disagreement.%
  \footnote{the last line of page 4 and opening line of page 5.}
  This is a \note{very bad} assumption because
  \begin{itemize}
    \item If agents were forced to conform with the world,
    there would probably be no disagreement to begin with.
    \item However, if agents were to disagree with the
    conformity, there would probably NO CONSENSUS AT ALL
    to begin with; Hence, further analysis of disinterested
    disagreement is useless at this point (therefore,
    the mutual respect using Lehrer-Wagner model in section
    $5$ is a big piece of garbage.)
  \end{itemize}
  \\
  \bold{Opinion ends here.}\rule{10cm}{0.4pt}

  \medskip
  \section{Modeling Disagreement: Nash bargaining}
  This is how the paper resolved interested disagreement
  using Nash bargaining solution concepts for Sen's flute
  problem: The one has more bargaining powers is the one
  who will obtain the flute.

  \medskip
  \bold{My opinion}\rule{10cm}{0.4pt}
  The solution concepts for Nash bargaining problem is,
  without a doubt, beautiful. However, the question: Is
  the implementation of this solution successful in real life
  situation? Unless the bargaining occurs under professional
  business, it may be really hard to believe that this is
  possible due to human stubborness and social views on
  what should be the correct distribution. Even if there would
  be an institution that induces the Nash bargaining solution
  to solve the problem, it would be really hard to measure
  the disagreement point $d$ for each individual $\Rightarrow$
  hard to measure the bargaining power.
  \\
  \bold{Opinion ends here.}\rule{10cm}{0.4pt}
  \section{Modeling Disagreement: Consesus through
  Mutual Respect}
  Disinterested disagreement are to be reached consensus
  through Lehrer-Wagner model, with the assumptions,
  \begin{itemize}
    \item $w_{ij} \neq 0$ for at least one $j\neq i$.
    Hence, if the individual aggressively believe that
    his principles is correct with $100\%$, the model
    is doomed! Is it hard to find such individuals in
    real-life situation? Piece of cake!
    (Linus Torvalds,
    Steve Jobs,\dots etc.)
    \item No formation of subgroups is allowed.
  \end{itemize}

  \medskip
  \bold{My opinion}\rule{10cm}{0.4pt}
  This is by far, the most damning section that makes me
  angry. Compared to the elegant solution made by Nash
  in the interested disagreement, this ``consensus''
  does not say anything except for the sake of having
  reached a consensus. And then the rational ground,
  the implausibility of assumption,\dots \footnote{
  line $9$ from the bottom of page $12$.}
  are seemingly for the sake of the theory itself
  rather than the practical aspects of life.

  This paper is really not my taste \ldots REALLY NOT!

  In term of intellectual masturbation, this paper scores
  $4/5$, but in term of usefulness, this paper is $1/5$.
  \\
  \bold{Opinion ends here.}\rule{10cm}{0.4pt}
\end{document}
