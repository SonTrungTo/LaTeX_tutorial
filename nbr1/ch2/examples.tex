\documentclass[a4paper]{report}
\title{Examples}
\author{Son To}
\usepackage{textcomp}
\begin{document}
\maketitle
  % Example 1
  \ldots when Einstein introduced his formula
  \begin{equation}
    e = m \cdot c^2 \; ,
  \end{equation}
  which is the same time the most widely known
  and the least well understood physical formula.

  % Example 2
  \ldots from which follows Kirchhoff's current law:
  \begin{equation}
    \sum_{k=1}^{n} I_k = 0 \; .
  \end{equation}

  Kirchhoff's voltage law can be derived \ldots

  % Example 3
  \ldots which has several advantages.

  \begin{equation}
    I_D = I_F - I_R
  \end{equation}
  is the core of a very different transitor model. \ldots  I think this is:
  su\-per\-cal\-%
  i\-frag\-i\-lis\-tic\-ex\-pi\-%
  al\-i\-do\-cious

  \hyphenation{FORTRAN Hy-phen-a-tion}
  Hyphenation

  My phone number will be \fbox{032423 53543}

  The parameter \mbox{\emph{filename}} should contain
  the name of the file.

  ``Please press the `x' key ''

  $5^2-1=24$

  $\sim$ is the tilde character. \\
  \~{} is not what we want.

  read\slash write % Allow hyphenation, but / is ok for ratio.

  It's $-30\,^{\circ}\mathrm{C}$. I will start to
  super-conduct.\\
  30 \textcelsius{} is 86 \textdegree{}F.

  \begin{tabular}{|r|l|}
    \hline
    70DC & hexadecimal \\
    SleepingDogs & Bull \\ \cline{2-2} % Beginning in column i and ending in j
    Jack The Ripper & Aalto Uni \\
    \hline \hline
    1984 & decimal \\
    \hline
  \end{tabular}

  \begin{tabular}{@{} l @{}}
    \hline
    No leading space\\
    \hline
  \end{tabular}

  \begin{tabular}{l}
    \hline
    With leading space left and right \\
    \hline
  \end{tabular}
\end{document}
