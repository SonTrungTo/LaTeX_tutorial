\documentclass[a4paper,11pt]{report}
\pagestyle{headings}
\usepackage[retainorgcmds]{verbatim,amsmath,amssymb,IEEEtrantools}
\author{Son To}
\title{Sample math symbols}
\date{May 28, 2017}
\DeclareMathOperator{\co}{James Comey}
\DeclareMathOperator{\nut}{nut job}
\begin{document}
\maketitle
\tableofcontents
\chapter{Single equations}
Add $a$ squared and $b$ squared to get $c$ squared. Or,
using the more mathematical approach:
$a^2 + b^2 = c^2$ % text style

\TeX{} is pronouned as $\tau\epsilon\chi$\\[5pt] % Manual line break
100~m$^3$ of water\\[5pt]
This comes from my $heartsuit$

% Add reference and name to the equation
Add $a$ squared and $b$ squared to get $c$ squared. Or,
using the more mathematical approach:
\begin{equation}
  a^2 + b^2 = c^2 % display style
\end{equation}
Einstein says
\begin{equation}
  E = mc^2 \label{clever}
\end{equation}
He didn't say
\begin{equation}
  1 + 1 = 3 \tag{bollocks}
\end{equation}
This is a reference to \eqref{clever}.

Add $a$ squared to $b$ squared to get $c$ squared. Or, using
a more mathematical approach
\begin{equation*}
  a^2 + b^2 = c^2
\end{equation*}
or you can type less for the same effect
\[ a^2 + b^2 = c^2 \]

This is text style: $\lim_{n \to \infty}
\sum_{k=1}^n \frac{1}{k^2} = \frac{\pi^2}{6}$.
And this is the display style:
\begin{equation}
  \lim_{n \to \infty} \sum_{k=1}^n \frac{1}{k^2}
  = \frac{\pi^2}{6}
\end{equation}

% Add \smash to deep-math expressions to space evenly
A $d_{e_{e_p}}$ mathematical expression followed
by a $h_{i_{g_h}}$ expression. As opposed to a smashed
\smash{$d_{e_{e_p}}$} expression followed by a
\smash{$h_{i_{g_h}}$} expression.

$\forall x \in \mathbf{R}: \qquad x^2 \geq 0$

$x^{2} \geq 0 \qquad \text{for all } x \in \mathbf{R}$

$x^{2} \geq 0 \qquad \text{for all } x \in \mathbb{R}$

\begin{tabular}{|p{8.7cm}|}
\hline
  $p^3_{ij} \qquad m_\text{Knuth} \qquad \sum_{k=1}^n k
  \\[5pt]
  a^x + y \neq a^{x + y} \qquad e^{x^{2}} \neq e^x^2$ \\
\hline
\end{tabular}

$\sqrt{2} \Leftrightarrow x^{1/2} \quad \sqrt[3]{2}
\quad \sqrt{x^2 + \sqrt{y}}
\quad \surd{x^2 + y^2}$

$\Psi = v_1 \cdot v_2 \cdot \ldots \qquad
n! = 1 \cdot 2 \cdots (n-1) \cdot n$

$0.\overline{3} = \underline{\underline{1/3}}$

\begin{tabular}{|p{8.7cm}|}
\hline
$\underbrace{\overbrace{a+b+c}^6 \cdot \overbrace{d+e+f}^9}
_\text{Advanced Calculus} = 54$ \\
\hline
\end{tabular}

$f(x) = x^2 \quad f'(x) = 2x \quad f''(x) = 2 \\[5pt]
\hat{XY} \quad \widehat{XY} \quad \bar{x}_0
\quad \bar{x_0}$

$\vec{a} \qquad \vec{AB} \qquad
\overrightarrow{AB}$

\begin{equation*}
  lim_{x \rightarrow 0} \frac{\sin x}{x} = 1
\end{equation*}

% See the preamble for \DeclareMathOperator
\begin{equation*}
  \co = \nut_{x=D.Trump}
\end{equation*}

$a \bmod b \\
 x \equiv a \pmod b$

In in-line equations, the fraction $\tfrac{1}{2}$(text style)
is shrunk to fit the line. The reverse of which is
$\dfrac{1}{2}$(display style). A built-in fraction is
$\frac{1}{2}$

% \partial
\begin{equation*}
  \sqrt{\frac{x^2}{k+1}} \qquad
  x^\frac{2}{k+1} \qquad
  \frac{\partial^2f}{\partial x^2}
\end{equation*}

% \binom from amsmath
Pascal's rule is
\begin{equation*}
  \binom{n}{k} = \binom{n-1}{k}
  + \binom{n-1}{k-1}
\end{equation*}


% \stackrel{#1}{#2}, #1 on #2
\begin{equation*}
  f_n(x) \stackrel{d}{\succ} f_m(x)
\end{equation*}

% \int, \sum, \prod
\begin{equation*}
  \int_0^{\frac{\pi}{2}} x^2 \, \mathrm{d}x \qquad
  \sum_{i=1}^n i \qquad
  \prod_\epsilon
\end{equation*}

% \substack from amsmath
\begin{equation*}
  \sum^n_{\substack{0<i<n \\
                  j\subseteq i}}
  P(i,j) = Q(i,j)
\end{equation*}

% other delimiters
\begin{equation*}
  {a,b,c} \neq \{a,b,c\}
\end{equation*}

% \left, \right with delimiters. \left. = nothing on the left, similar to \right.
\begin{equation*}
  1 + \left(\frac{1}{1-x^2}\right)^3 \qquad
  \left. \ddagger \frac{~}{~} \right)
\end{equation*}

% manually, \big < \Big < \bigg < \Bigg
$\Big((x+1)(x-1)\Big)^3$ \\
$\big(\Big(\bigg(\Bigg( \quad
 \big\}\Big\}\bigg\}\Bigg\} \quad
 \big\|\Big\|\bigg\|\Bigg\| \quad
 \big\Updownarrow \Big\Updownarrow \bigg\Updownarrow
 \Bigg\Updownarrow$

% multline from amsmath, can be added with an *, and \\
\begin{multline}
  a+b+c+d+e+f+g+h+i+z+x+v+n+m+1+2+3+4\\
  =j+k+l+m+n
\end{multline}

\chapter{Multiple equations}
align env
\begin{align}
  a = b + c \\
 = d + e
\end{align}
\begin{align}
  a & = b + c \\
  & = d+e
\end{align}
Interpretation: \& is more standard in the use of system
of equations. \\
Its downfall:
\begin{align}
  a & = b + c \\
  & = d+e+f+g+h+j+j+u+j+k+s+c\nonumber\\
  & +c+r+e+g+t+y+z \\
  & = p+q+r+s
\end{align}
A better solution:
\begin{eqnarray}
  a & = & b+c\\
  & = & d+e+g+r+h+j+j+k\nonumber\\
  & & + \: l+b+m+v+v+c+f+h\\
  & = & p+q+r+s
\end{eqnarray}
There are two troubles:\\
Trouble I:
\begin{eqnarray}
  a & = & a=a
\end{eqnarray}
Trouble II:(the spacing between $j^2$ is big!)
\begin{eqnarray}
  a & = & b+c\\
  & = & z+x+v+n+o+m+n+b+t+r+e+t+i^2+j^2+l
  \label{eq:trouble2}
\end{eqnarray}
In addtionally, we are provided with \textbackslash
lefteqn when the LHS is too long:
\begin{eqnarray}
  \lefteqn{a+b+c+r+e+d+f+g+d+e+t+f+g+h+d} \nonumber\\
  & = & a+b+c+m+j+k \\
  & = & n+o+p+q+r+s
\end{eqnarray}
However, this still sucks as
the RHS is too short and the array is not properly centered:
\begin{eqnarray}
  \lefteqn{a+b+c+e+f+g+h+j+k+l}\\
  & = & r+s
  \label{eq:lefteqn}
\end{eqnarray}
Our new remedy will be \ldots
\section{IEEEeqnarray Environment}
\begin{IEEEeqnarray}{rCl}
  a & = & b+c \\
  & = & d+e+f+b+t+g+h \nonumber\\
  & & + \: j+k+l \\
  & = & p+q+r+s
\end{IEEEeqnarray}
Addtional spaces can be added with . and /and  ?
in an increasing order. We now show how IEEEeqnarray
solves (\ref{eq:trouble2}) and (\ref{eq:lefteqn}).
\end{document}
