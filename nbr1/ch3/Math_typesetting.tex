\documentclass[a4paper,11pt]{report}
\usepackage{verbatim,amsmath,amssymb}
\author{Son To}
\title{Sample math symbols}
\begin{document}
\maketitle
\tableofcontents
\chapter{Single equations}
Add $a$ squared and $b$ squared to get $c$ squared. Or,
using the more mathematical approach:
$a^2 + b^2 = c^2$ % text style

\TeX{} is pronouned as $\tau\epsilon\chi$\\[5pt] % Manual line break
100~m$^3$ of water\\[5pt]
This comes from my $heartsuit$

% Add reference and name to the equation
Add $a$ squared and $b$ squared to get $c$ squared. Or,
using the more mathematical approach:
\begin{equation}
  a^2 + b^2 = c^2 % display style
\end{equation}
Einstein says
\begin{equation}
  E = mc^2 \label{clever}
\end{equation}
He didn't say
\begin{equation}
  1 + 1 = 3 \tag{bollocks}
\end{equation}
This is a reference to \eqref{clever}.

Add $a$ squared to $b$ squared to get $c$ squared. Or, using
a more mathematical approach
\begin{equation*}
  a^2 + b^2 = c^2
\end{equation*}
or you can type less for the same effect
\[ a^2 + b^2 = c^2 \]

This is text style: $\lim_{n \to \infty}
\sum_{k=1}^n \frac{1}{k^2} = \frac{\pi^2}{6}$.
And this is the display style:
\begin{equation}
  \lim_{n \to \infty} \sum_{k=1}^n \frac{1}{k^2}
  = \frac{\pi^2}{6}
\end{equation}

% Add \smash to deep-math expressions to space evenly
A $d_{e_{e_p}}$ mathematical expression followed
by a $h_{i_{g_h}}$ expression. As opposed to a smashed
\smash{$d_{e_{e_p}}$} expression followed by a
\smash{$h_{i_{g_h}}$} expression.

$\forall x \in \mathbf{R}: \qquad x^2 \geq 0$

$x^{2} \geq 0 \qquad \text{for all } x \in \mathbf{R}$

$x^{2} \geq 0 \qquad \text{for all } x \in \mathbb{R}$

\begin{tabular}{|p{8.7cm}|}
\hline
  $p^3_{ij} \qquad m_\text{Knuth} \qquad \sum_{k=1}^n k
  \\[5pt]
  a^x + y \neq a^{x + y} \qquad e^{x^{2}} \neq e^x^2$ \\
\hline
\end{tabular}

$\sqrt{2} \Leftrightarrow x^{1/2} \quad \sqrt[3]{2}
\quad \sqrt{x^2 + \sqrt{y}}
\quad \surd{x^2 + y^2}$

$\Psi = v_1 \cdot v_2 \cdot \ldots \qquad
n! = 1 \cdot 2 \cdots (n-1) \cdot n$

$0.\overline{3} = \underline{\underline{1/3}}$

\begin{tabular}{|p{8.7cm}|}
\hline
$\underbrace{\overbrace{a+b+c}^6 \cdot \overbrace{d+e+f}^9}
_\text{Advanced Calculus} = 54$ \\
\hline
\end{tabular}

$f(x) = x^2 \quad f'(x) = 2x \quad f''(x) = 2 \\[5pt]
\hat{XY} \quad \widehat{XY} \quad \bar{x}_0
\quad \bar{x_0}$

\end{document}
