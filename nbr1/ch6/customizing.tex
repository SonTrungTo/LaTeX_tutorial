\documentclass[a4paper,11pt]{article}
\usepackage[affil-it]{authblk}
\usepackage{verbatim,demo,layout}

\begin{document}

\title{Customizing \LaTeX}
\author{Son To
  \thanks{Electronic address: \texttt{son.trung.to@gmail.com}}}
\date{18th June,2017}
\affil{Ravintola Kiltakellari}

  \maketitle
  \tableofcontents
  \clearpage

\section{New commands,Environments and Packages}
\subsection{New commands}
The format for the new commands is:
\vskip 20pt
\begin{tabular}{|l|}
\hline
  \verb+\newcommand{name}[num]{definition}+ \\
\hline
\end{tabular}
\vskip 20pt
\flushleft
An example of this is \tnss, and hey, we have \tnss.\\
Another example is:
\begin{itemize}
  \item \txsit{not so}{short}
  \item \txsit{very}{long}
\end{itemize}
\LaTeX~will allow overwriting available commands only if
we use \verb+\renewcommand+, with the same syntax as
\verb+\newcommand+. Also, the \verb+\providecommand+ works
the same as \verb+\newcommand+, but \LaTeX will silently
ignore it if the command exists!

\subsection{New environments}
\vskip 20pt
\begin{tabular}{|l|}
  \hline
    \verb+newenvironment{name}[num]{before}{after}+ \\
  \hline
\end{tabular}
\vskip 20pt
\flushleft
Let's try an example.\\
\begin{king}
  I am Edward Longsharks\ldots
\end{king}
\\
Similar to new commands, we have \verb+\renewenvironment+

\subsection{Extra space}
\begin{simple}
  I lied.\\ I have an healthy baby boy.
\end{simple}
Same\\here.

When creating a new environment, extra spaces can easily creep
in. Fix this problem with \verb+\ignorespaces+
and \verb+\ignorespacesafterend+

\begin{correct}
  I lied.\\ I have an healthy baby boy.
\end{correct}
Same\\here.
\subsection{Own package}
Use \verb+\ProvidesPackage{package name}+
on .sty files containing new modifications.

\section{Fonts and Sizes}
\subsection{Changing Font commands}
\begin{itemize}
  \item Groups are important for limit the effect of font size(%
  and of most \LaTeX commands)
  \item Font size changes line spacing only if the paragraph
  ends within the group. See \newpage
  {\Large My name is Darth Vader.\par} not Palpatine\\
  and\\
  {\Large My name is Darth Vader.}\par not Palpatine
\end{itemize}

\subsection{\LaTeX, the basic idea}
The basic idea of \LaTeX is to separate the logical from
visual markup presentation.
\begin{quote}
  Do not \oops{enter} this room, it's occupied
  by \oops{machines} of unknown origin and purpose.
\end{quote}
\verb+\emph{}+ is context aware while changing fonts is absolute.
\begin{quote}
  \textit{You can also \emph{emphasize} text
  if it is set in italics,}
  \textsf{in a \emph{sans-serif} font,}
  \texttt{or in \emph{typewriter} style.}
\end{quote}

\section{Spacing}
\subsection{Line Spacing}
\vskip 20pt
\verb+\linespread{factor}+
\vskip 20pt
\flushleft
to change line inter-line spacing in a document. Use
\verb+\linespread{1.3}+ for ``one and a half'' line spacing.
Use \verb+\linespread{1.6}+ for ``double'' line spacing.

Use the command
\vskip 20pt
\verb+\setlength{\baselineskip}{1.5\baselineskip}+
\vskip 20pt
\flushleft
if you really want to space lines for good reason.

{\setlength{\baselineskip}%
{1.5\baselineskip}
This paragraph is typeset with the baseline skip set to
1.5 of what it was before. Note the par command at the end
of the paragraph. \par}
This paragraph has a clear purpose: it shows that after
the curly brace has been closed, everything is back to
normal.

\subsection{Paragraph Formatting}
Use
\vskip 20pt
\verb+\setlength{\parindent}{0pt}+
\verb+\setlength{\parskip}{1ex plus 0.5ex minus 0.2ex}+
\vskip 20pt
\flushleft
to influence paragraph formatting in \LaTeX. Addtionally,
we can put
\vskip 20pt
\verb+\indent+ and \verb+\noindent+
\vskip 20pt
\flushleft
at the beginning of the paragraph.

\subsection{Horizontal space}
\verb+\hspace{length}+ and \verb+\stretch{n}+
If a space should be kept no matter what, use
\verb+\hspace*+
\subsection{Vertical space}
\verb+\vspace{length}+\\
\verb+\vspace{\stretch{1}}+ with \verb+\pagebreak+ typesets
the last line of the page.\\
Also, if two lines in the same paragraph, use
\verb+\\{length}+. Moreover, \verb+\bigskip+ and
\verb+\smallskip+ come in handy vertically.

\section{Page Layout}
Pay attention to these two commands, especially the second one.
And look at the layout in the next page.

\smallskip
\verb+\setlength{parameter}{length}+\\
\verb+\addtolength{parameter}{length}+
\layout

\section{More Fun With Lengths}
A modification of format based on the size of
other page elements is more useful.

\smallskip
\verb+\settoheight{variable}{text}+\\
\verb+\settowidth{variable}{text}+\\
\verb+\settodepth{variable}{text}+
\smallskip
\flushleft
Now, an example with \verb+\nicespace{}+\ldots
\begin{displaymath}
  a^2+b^2=c^2
\end{displaymath}

\begin{nicespace}{Where}$a$,
  $b$ -- are adjoin to the right angle
  of a right-angled triangle.

  $c$ -- is the hypotenuse of the triangle and
  feels lonely.

  $d$ -- finally does not show up here at all.
  Isn't that puzzling?
\end{nicespace}

\section{Boxes}
\oops{Paragraph}:
Pack a paragraph into a box with either
\verb+\parbox[pos]{width}{text}+ or\\
\verb+\begin{minipage}[pos]{width}text\end{minipage}+\\
with [pos]=c,t,b

\oops{Horizontally aligned box}: \verb+\mbox+ to prevent
separation of a series of boxes. Also,
\verb+\makebox[width][pos]{text}+ in which
\begin{itemize}
  \item $[pos]=c,l,r,s$
  \item $[width]$ can be \verb+\width+, \verb+\height+,
  \verb+\depth+, \verb+\totalheight+ w.r.t \emph{text}
  width.
  \item \verb+\framebox+ works exactly the same with
  the addition of having a drawn box around.
\end{itemize}
Example:
\par
\makebox[\textwidth]{%
c e n t r a l}\par
\makebox[\textwidth][s]{%
c e n t r a l}\par
\framebox[1.1\width]{%
Guess I'm framed now!}\par
\framebox[0.8\width][r]{Bummer,
I am too wide}\par
\framebox[1cm][l]{%
never say never, so am I}
Can you read this?

\smallskip
\oops{Vertically aligned box}:
\begin{verbatim}
\raisebox{lift}[extend-above-baseline][extend-below-baseline]{text}
\end{verbatim}

Example:\\
\raisebox{0pt}[0pt][0pt]{%
\textbf{Aaaa\raisebox{-0.3ex}{a}%
\raisebox{-0.7ex}{aa}%
\raisebox{-1.2ex}{r}%
\raisebox{-2.2ex}{g}%
\raisebox{-4.5ex}{h}}}
she shouted, but not even the next one in line noticed
that something terrible had happened to her.
\section{Rules}

\verb+\rule[lift]{width}{height}+\\
is used to,usually, draw a box.

Examples:
\rule{3mm}{.1pt}%
\rule[-1mm]{5mm}{1cm}%
\rule{3mm}{.1pt}%
\rule[1mm]{1cm}{5mm}%
\rule{3mm}{.1pt}
\flushright
Sayonara.
\end{document}
